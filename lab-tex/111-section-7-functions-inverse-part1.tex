%============================================================
% MTH 111Z Project - Template File
% Section 3.7 from OpenStax OER
%	Updated 202302
%============================================================

\section{Inverse Functions} \label{functions-inverse}

In this section, we'll see what happens if you turn a function inside-out and make the output become the input and the input become the output.  We'll also explore when doing this will result in a function and what it means if it does.   \\[0.5em]
\textbook{3.7}


\subsection*{Preparation Exercises} \label{prep-functions-inverse}

\begin{myPrep}
Consider the following set of ordered pairs:\\
$$ \{ (\text{Alaska}, 1), ~ (\text{Washington},10), ~ (\text{Oregon},6), ~ (\text{Idaho}, 2),~ (\text{Nevada}, 4),~ (\text{Hawaii}, 2),~ (\text{California}, 52)\} $$
\begin{enumerate}
	\item Does this set represent a function?  Answer by referencing the definition of a function.
	\vfill
	\item If we were to swap the $x$ and $y$-values, would this new set be a function? Explain your answer.\\ 
$ \{ (1, \text{Alaska}), ~ (10, \text{Washington}), ~ (6, \text{Oregon}), ~ (2, \text{Idaho}),~ (4, \text{Nevada}),~ (2, \text{Hawaii}),~ (52, \text{California})\} $
	\vfill
\end{enumerate}
\end{myPrep}

\begin{myPrep}
The formula to convert a temperature $F$ in degrees Fahrenheit to a temperature $C$ in degrees Celsius is given by the function
\[ C = g(F) = \frac{5}{9}(F-32)  \]

\begin{enumerate}
	\item For each temperature in degrees Fahrenheit, how many temperatures in degrees Celsius are produced by this formula?
	\vfill
	\item For each temperature in degrees Celsius, how many temperatures in degrees Fahrenheit are produced by this formula?
	\vfill
	\item Would it be true to state that $C$ is a function of $F$ and also that $F$ is a function of $C$?  Why or why not?
	\vfill

\end{enumerate}
\end{myPrep}



