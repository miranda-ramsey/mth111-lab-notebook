
%======================================================
\newpage
%======================================================


\subsection*{Practice Exercises} \label{practice-exponential-and-logarithmic-equations}


\begin{myPractice}
Solve each of the following algebraically.  Use a calculator to approximate any irrational solutions.
	\begin{enumerate}
		\begin{multicols}{2}
			\item	$ 3^{5x-6} = 81$
			\item $9^{x-11} = 7 $
		\end{multicols}
		\vfill
		\vfill
		\begin{multicols}{2}
			\item	$ e^{x+3} +4= 19$
			\item	$ 5^{x-6}= 3^{2x+7}$
		\end{multicols}
		\vfill
		\vfill
		\vfill
	\end{enumerate}
\end{myPractice}

\newpage
\begin{myPractice}
Solve each of the following algebraically.  Be sure to confirm any solutions are not extraneous.
	\begin{enumerate}
		\begin{multicols}{2}
			\item $\log_6(3x+1) =\log_6(x-9) $
			\item	$\log_7(x-4)+3=5$
		\end{multicols}
		\vfill
		\vfill
		\item $\log_3(x-2)= 1-\log_3(x-4)$
		\vfill
		\vfill
		\vfill
	\end{enumerate}
\end{myPractice}


%======================================================
\newpage
%======================================================

\subsection*{Definitions and Properties} \label{def-exponential-and-logarithmic-equations}

\begin{myDefinition}[Logarithm:]~\\[0.5mm]
For any real number $x>0$, the \textbf{logarithm with base $\boldsymbol{b}$ of $\boldsymbol{x}$}, where $b>0$ and $b\neq 1$, is denoted by $\boldsymbol{\log_b (x)}$ and is defined by
			$$
					y = \log_b (x) \ \  \textrm{if and only if} \ \  x = b^y
			$$
\end{myDefinition}


\begin{myDefinition}[One-to-One Property of Exponential Functions]~\\[0.5mm]
For any algebraic expressions $S$ and $T$, and any positive real number $b$, with $b\neq1$,
$$b^S=b^T \text{ if and only if } S=T$$
\end{myDefinition}


\begin{myDefinition}[One-to-One Property of Logarithmic Functions]~\\[0.5mm]
For any algebraic expressions $S>0$, $T>0$, and any positive real number $b$, where $b\neq1$,
$$\log_b(S)=\log_b(T) \text{ if and only if } S=T$$ 
Note: Because $\log_b(x)$ has the domain $(0,\infty)$ for all $b>0, b\neq1$, when we solve an equation involving logarithms, {\bf we must always check} to see if the solution we've found is valid or if it is an extraneous solution. 
\end{myDefinition}








%======================================================
 \newpage
%======================================================

\subsection*{Exit Exercises} \label{exit-exponential-and-logarithmic-equations}



\begin{myExit}
	\begin{enumerate}
		\item What's the general process for solving exponential equations that have one exponential expression in them?
		\vfill
		\item Why can logarithmic equations have extraneous solutions and how can an extraneous solution be recognized?
		\vfill
		\begin{multicols}{2}
			\item Solve $4e^{2k+1}+3=27$.
			\item Solve $\log_8(5x+12)-\log_8(x)=\log_8(2)$.
		\end{multicols}
		\vfill
		\vfill
		\vfill
	\end{enumerate}
\end{myExit}
\vfill


\exitlikert{exponential and logarithmic equations}











