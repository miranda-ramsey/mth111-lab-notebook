%============================================================
% MTH 111Z Project - Template File
% Section 3.4 from OpenStax OER
%	Updated 202302
%============================================================

\section{Algebraic Combinations of Functions and Function Composition} \label{functions-algebra-and-composition}

In this section, we'll learn about several ways we can combine functions algebraically, as well as use the output from one function as the input for another.  As we've done before, we'll investigate these ideas with functions presented as graphs, as tables, and as formulas.  \\[0.5em]
\textbook{3.4}



\subsection*{Preparation Exercises } \label{prep-functions-algebra-and-composition}

\begin{myPrep}
Suppose it costs a bakery \$3,000 for rent and utilities each month and each loaf of bread costs \$2.15 to produce.  The bakery sells each loaf of bread for \$6.49.
	\begin{enumerate}
		\item Find a function $C$ that calculates the total cost (in dollars) each month to produce $x$ loaves of bread.
		\vfill
		\item Find a function $R$ that calculates the total revenue (in dollars) each month from selling $x$ loaves of bread.
		\vfill
		\item Find a function $P$ that calculates the profit (in dollars) from producing and selling $x$ loaves of bread each month.  (Note: The profit is the difference between the revenue and costs.)
		\vfill
	\end{enumerate}
\end{myPrep}

\begin{myPrep}
The function $r=f(t)=0.35t$ gives the radius (in inches) of the circular pattern formed when a drop of water hits a pond $t$ seconds after the drop of water hits the pond's surface.  The function $a = g(r) = \pi r^2$ gives the area of a circle (in square inches) when the circle has a radius of $r$ inches.\\[0.5em]
How would you determine the area of the circle 9 seconds after a water drop lands on the pond?
\vfill
\end{myPrep}

