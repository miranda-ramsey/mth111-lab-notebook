

%======================================================
\newpage
%======================================================





\subsection*{Practice Exercises} \label{practice-functions-and-notation}

\begin{myPractice}
Determine if each of the following relations show $y$ as a function of $x$.  \\
Explain your reasoning by referencing the definition of a function.  If a relation is not a function, provide a specific example why the definition was not satisfied.\\
Assume all ordered pairs are of the form $(x,y)$.

	\begin{enumerate}
		\item  $\{ (\text{red},\text{pepper}), ~ (\text{green},\text{pear}), ~ (\text{purple},\text{grape}), ~ (\text{orange},\text{orange}), ~ (\text{yellow},\text{pepper}), ~ (\text{red},\text{onion})\}$
		\vfill
		\item	
		\begin{minipage}{0.1\linewidth}
		\captionof{table}{}
		\end{minipage}~\\[-0.8em]
			\renewcommand\arraystretch{1.5}
			\begin{tabular}{|c|cccccc|}
			\hline
			$x$ & ~$-5$~ & ~$3$~ & ~$-1$~ & ~$2$~ & ~$4$~ & ~$6$~\\
			\hline
			$y$ & $9$ & ~$-4$~ & $2$ & ~$-4$~ & ~$-5$~ & ~$-6$~\\
			\hline
			\end{tabular}		
			\label{tab:fan-prex1}
		\vfill
		\begin{multicols}{2}
			\item
				\begin{minipage}{0.15\linewidth}
				\captionof{figure}{}
				\end{minipage}~\\[-0.8em]
				\begin{tikzpicture}
			 		\begin{axis}[
		 				framedaxes,
		 				width=5.5cm,height=5.5cm,
		 				xlabel={$x$},
		 				ylabel={$y$},
		 				xmin=-6,xmax=6,
		 				ymin=-6,ymax=6,
						xtick={-4,-2,...,4},
							minor xtick={-7,-5,...,7},
						ytick={-4,-2,...,4},
							minor ytick={-7,-5,...,7},
					         grid=both
		 				]
 				% use TeX as calculator:
%	 					\addplot[smooth,mark=*,first,line width=1.0pt,fill=white]coordinates{	(-4,2)	};
%	 					\addplot[smooth,mark=*,first,line width=1.0pt]coordinates{	(5,-4)	};
		 				\addplot[first,line width=1.5pt,samples=200,<->]expression[domain=-4.826:4.59036]{0.05*(x+4)*(x+2)*(x-1)*(x-4)};
			 		\end{axis}
		 		\end{tikzpicture}
				\label{fig:fan-prex1}
			\item	
				\begin{minipage}{0.15\linewidth}
				\captionof{figure}{}
				\end{minipage}~\\[-0.8em]
				\begin{tikzpicture}
			 		\begin{axis}[
		 				framedaxes,
		 				width=5.5cm,height=5.5cm,
		 				xlabel={$x$},
		 				ylabel={$y$},
		 				xmin=-6,xmax=6,
		 				ymin=-6,ymax=6,
						xtick={-4,-2,...,4},
							minor xtick={-7,-5,...,7},
						ytick={-4,-2,...,4},
							minor ytick={-7,-5,...,7},
					         grid=both
		 				]
 				% use TeX as calculator:
%	 					\addplot[smooth,mark=*,first,line width=1.0pt,fill=white]coordinates{	(-4,2)	};
%	 					\addplot[smooth,mark=*,first,line width=1.0pt]coordinates{	(5,-4)	};
		 				\addplot[first,line width=1.5pt,samples=400,->]expression[domain=-1:6]{(x+1)^0.5+2};
		 				\addplot[first,line width=1.5pt,samples=400]expression[domain=-1:1]{-0.5*(x+1)^0.5+2};
		 				\addplot[first,line width=1.5pt,samples=400]expression[domain=1:3]{0.5*(-(x-3))^0.5+0.586};
		 				\addplot[first,line width=1.5pt,samples=400]expression[domain=0:3]{-0.5*(-(x-3))^0.5+0.586};
		 				\addplot[first,line width=1.5pt,samples=400]expression[domain=-3:0]{0.5*((x+3))^0.5-1.146};
		 				\addplot[first,line width=1.5pt,->,samples=400]expression[domain=-3:6.01]{-1*((x+3))^0.5-1.146};
						\addplot[first,line width=1.5pt,-,forget plot] coordinates {(-1,1.98) (-1,2.02)};
						\addplot[first,line width=1.5pt,-,forget plot] coordinates {(3,0.566) (3,0.606)};
						\addplot[first,line width=1.5pt,-,forget plot] coordinates {(-3,-1.166) (-3,-1.126)};
						
			 		\end{axis}
		 		\end{tikzpicture}
				\label{fig:fan-prex2}
		\end{multicols}
		\vfill
	\end{enumerate}
\end{myPractice}


%======================================================
\newpage
%======================================================


\begin{myPractice}
	\begin{enumerate}
		\item Let $y=g(x)$ be defined by the set of $(x,y)$ ordered pairs:\\
		 $\{ (-60,5), ~ (-23,4), ~ (-4,3), ~ (3,2), ~ (4,1), ~ (5,0), ~ (12,-1)\}$.
			\begin{enumerate}[label=\roman*.]
				\begin{multicols}{2}
				\item Find $g(5)$.
				\item Solve $g(x)=3$.
				\end{multicols}
			\end{enumerate}
		\vfill

		\item Let $y=h(x)$ be defined by the graph in Figure~\ref{fig:fan-ex1} below.
		
		\begin{minipage}{0.4\linewidth}
					\captionof{figure}{$y=h(x)$}~\\[-0.8em]
					\begin{tikzpicture}
			 		\begin{axis}[
		 				framedaxes,
		 				width=7cm,height=7cm,
		 				xlabel={$x$},
		 				ylabel={$y$},
		 				xmin=-8,xmax=8,
		 				ymin=-8,ymax=8,
						xtick={-6,-4,...,6},
							minor xtick={-7,-5,...,7},
						ytick={-6,-4,...,6},
							minor ytick={-7,-5,...,7},
					         grid=both
		 				]
 				% use TeX as calculator:
%	 					\addplot[smooth,mark=*,first,line width=1.0pt,fill=white]coordinates{	(-4,2)	};
%	 					\addplot[smooth,mark=*,first,line width=1.0pt]coordinates{	(5,-4)	};
						\addplot[first,line width=1.5pt,samples=400,-]expression[domain=-7:-5.99]{x+1};
						\addplot[first,smooth,mark=*,line width=1pt,fill=white]coordinates{	(-7,-6)	};
						\addplot[first,line width=1.5pt,samples=400,-]expression[domain=-6.01:-3.99]{2*(x)+7};
						\addplot[first,line width=1.5pt,samples=400,-]expression[domain=-4.01:-2.99]{(x)+3};
						\addplot[first,line width=1.5pt,samples=400,-]expression[domain=-3.01:-0.99]{-1/2*(x)-1.5};
						\addplot[first,line width=1.5pt,samples=400,-]expression[domain=-1.01:0]{2*(x)^2-3};
						\addplot[first,line width=1.5pt,samples=400,-]expression[domain=0:1.01]{-(x)^2-3};
						\addplot[first,line width=1.5pt,samples=400,-]expression[domain=0.99:2]{-(x-2)^2-3};
						\addplot[first,line width=1.5pt,samples=400,-]expression[domain=2:3.01]{2*(x-2)^2-3};
						\addplot[first,line width=1.5pt,samples=400,-]expression[domain=2.99:5.01]{(x-3)^2-1};
						\addplot[first,line width=1.5pt,samples=400,-]expression[domain=4.99:6]{3};
						\addplot[first,smooth,mark=*,line width=1.5pt]coordinates{	(6,3)	};
			 		\end{axis}
		 		\end{tikzpicture}
				\label{fig:fan-ex1}
		\end{minipage}
		\begin{minipage}{0.4\linewidth}
			\begin{enumerate}[label=\roman*.]
				\item Find $h(-4)$.\vspace{2cm}
				\item Solve $h(x)=-3$.\vspace{2cm}
			\end{enumerate}
		\end{minipage}
		\item Let $f(x) = x^2-3$.
		\begin{enumerate}[label=\roman*.]
			\begin{multicols}{2}
				\item Find $f(-4)$.
				\item Solve $f(x)=46$.
			\end{multicols}
		\end{enumerate}
		\vfill
		\vfill
	

	\end{enumerate}
\end{myPractice}



%======================================================
\newpage
%======================================================

\subsection*{Definitions} \label{def-functions-and-notation}

\begin{myDefinition}[{Relation:}]~\\[0.5mm]
A {\bf relation} is a set of $(x,y)$ ordered pairs.  \\[0.5em]
The variable $x$ is called the {\bf independent variable} or {\bf input variable}.  \\
Each individual $x$-value is referred to as an {\bf input} or {\bf input value}.\\[0.5em]
The variable $y$ is called the {\bf dependent variable} or {\bf output variable}.  \\
Each individual $y$-value is referred to as an {\bf output} or {\bf output value}. \\[0.5em]
\defexample The set $\{ (0,-2), ~ (1, -1), ~ (2, 0), ~ (1, -3),~ (3,-4) \}$ is a relation.

\end{myDefinition}

\begin{myDefinition}[Function:]~\\[0.5mm]
A {\bf function} is a relation where each possible input value is paired with {\it exactly one} output value.\\
We say, ``The output is a function of the input,'' and often write this algebraically as $y = f(x)$. \\[0.5em]
\defexample The set $\{ (0,-2), ~ (1, -1), ~ (4, 0), ~ (9,1),~ (16,2) \}$ is a function.\\
\defexample The set $\{ (0,-2), ~ (\boldsymbol{1}, -1), ~ (2, 0), ~ (\boldsymbol{1}, -3),~ (3,-4) \}$ is a relation, but {\it not} a function.

\end{myDefinition}

\begin{myDefinition}[Domain and Range:]~\\[0.5mm]
The {\bf domain} of a relation or function is the set of all possible input values.\\
The {\bf range} of a relation or function is the set of all possible output values.\\[0.5em]
\defexample Given the function $\{ (0,-2), ~ (4, 0), ~ (1, -1), ~ (9,1),~ (16,2) \}$,\\
the domain of the function is $\{ 0,\,4,\,1,\,9,\,16\}$ and the range of the function is $\{ -2,\,0,\,-1,\,1,\,2\}$.

\end{myDefinition}


\begin{myDefinition}[Vertical Line Test:]~\\[0.5mm]
If a vertical line can be drawn that intersects the graph more than once, the graph is not the graph of a function with $x$ as the independent variable and $y$ as the dependent variable. \\[0.5em]
		\begin{minipage}{0.5\linewidth}
			\begin{center}
				\defexample
					\captionof{figure}{Does Not Pass}
					\begin{tikzpicture}
			 		\begin{axis}[
		 				framedaxes,
		 				width=5.5cm,height=5.5cm,
		 				xlabel={$x$},
		 				ylabel={$y$},
		 				xmin=-8,xmax=8,
		 				ymin=-8,ymax=8,
						xtick={-6,-4,...,6},
							minor xtick={-7,-5,...,7},
						ytick={-6,-4,...,6},
							minor ytick={-7,-5,...,7},
					         grid=both
		 				]
 				% use TeX as calculator:
%	 					\addplot[smooth,mark=*,first,line width=1.0pt,fill=white]coordinates{	(-4,2)	};
%	 					\addplot[smooth,mark=*,first,line width=1.0pt]coordinates{	(5,-4)	};
		 				\addplot[first,line width=1.5pt,samples=400,->]expression[domain=-2:8]{(x+2)^0.5+1};
		 				\addplot[first,line width=1.5pt,samples=400,->]expression[domain=-2:8]{-1*(x+2)^0.5+1};
						\addplot[first,line width=1.5pt,-,forget plot] coordinates {(-2,0.98) (-2,1.02)};
						\addplot[fifth,line width=0.75pt,<->] coordinates {(2,8) (2,-8)};
								\node at (axis cs:3.5,-7.5) {\color{fifth}\tiny{$x\,$=\,2}};
						\addplot[fifth,smooth,mark=*,line width=0.25pt]coordinates{	(2,3)	};
						\addplot[fifth,smooth,mark=*,line width=0.25pt]coordinates{	(2,-1)	};
			 		\end{axis}
		 		\end{tikzpicture}
				\label{fig:fan-def1}
			\end{center}
		\end{minipage}
		\begin{minipage}{0.5\linewidth}
			\begin{center}
				\defexample
					\captionof{figure}{Passes}
					\begin{tikzpicture}
			 		\begin{axis}[
		 				framedaxes,
		 				width=5.5cm,height=5.5cm,
		 				xlabel={$x$},
		 				ylabel={$y$},
		 				xmin=-8,xmax=8,
		 				ymin=-8,ymax=8,
						xtick={-6,-4,...,6},
							minor xtick={-7,-5,...,7},
						ytick={-6,-4,...,6},
							minor ytick={-7,-5,...,7},
					         grid=both
		 				]
 				% use TeX as calculator:
%	 					\addplot[smooth,mark=*,first,line width=1.0pt,fill=white]coordinates{	(-4,2)	};
%	 					\addplot[smooth,mark=*,first,line width=1.0pt]coordinates{	(5,-4)	};
		 				\addplot[first,line width=1.5pt,samples=400,->]expression[domain=-2:8]{(x+2)^0.5+1};
		 				\addplot[first,line width=1.5pt,samples=400,<-]expression[domain=-8:-2]{-1*(-x-2)^0.5+1};
						\addplot[first,line width=1.5pt,-,forget plot] coordinates {(-2,0.98) (-2,1.02)};
%						\addplot[myred,line width=1.0pt,dashed,<->] coordinates {(3,8) (3,-8)};
%								\node at (axis cs:3.25,-7.5) {\color{myred}\tiny{$x\,$=\,3}};
			 		\end{axis}
		 		\end{tikzpicture}
				\label{fig:fan-def2}
			\end{center}
		\end{minipage}
\end{myDefinition}






%======================================================
 \newpage
%======================================================

\subsection*{Exit Exercises} \label{exit-functions-and-notation}


\begin{myExit}
	\begin{enumerate}
		\item What is the definition of a function?  
		\vfill
		\item What are two examples of functions that you use in your daily life outside of school?  Explain how these are functions, referencing the definition of a function.
		\vfill
		\vfill
		\item What do you look for in the graph of a relation to determine if the graph is the graph of a function or not?  Fully explain your answer.
		\vfill
		\end{enumerate}
\end{myExit}

	\begin{multicols}{2}
\begin{myExit}
If $f(x) = 2x^2-7x$, evaluate $f(-3)$.
\end{myExit}
\begin{myExit}
If $g(x) = x^2+6x$, solve $g(x)=16$.
\end{myExit}
	\end{multicols}
		\vfill
		\vfill
		\vfill





\exitlikert{functions}





