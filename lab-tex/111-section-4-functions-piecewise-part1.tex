%============================================================
% MTH 111Z Project - Template File
% Part of Section 3.2 from OpenStax OER
%	Updated 202302
%============================================================


\section{Piecewise-Defined Functions} \label{functions-piecewise}

In this section, we'll explore piecewise-defined functions, which are functions constructed from pieces of several other functions.  We'll find the domain and range of these types of functions, as well as graph, evaluate, solve them.  And given the graph of a piecewise-defined function, we'll construct the formula for the graph's function. \\[0.5em]
\textbook{3.2}




\subsection*{Preparation Exercises} \label{prep-functions-piecewise}

\begin{myPrep}
A family event charges \$4/person, with a maximum of \$20 for any single family.  
	\begin{enumerate}
		\item How much will a family of three pay?
		\vfill
		\item How much will a family of seven pay?
		\vfill
		\item At what number of people does the calculation change from being per person to a single charge for the whole family?
		\vfill
		\vfill
	\end{enumerate}
\end{myPrep}

\begin{myPrep}
In November 2022, Portland General Electric set the rates for basic residential service as a function of the number of kilowatt-hours (kWh) of energy used.  The rates in November 2022 were \$0.0642/kWh when up to 1000 kWh (kilowatt-hours) are used and if greater than 1000 kWh are used, then the first 1000 kWh are billed at the \$0.0642/kWh rate and \$0.07002/kWh is charged for the energy usage greater than the initial 1000 kWh.  
	\begin{enumerate}
		\item What is the cost for using 740 kWh?
		\vfill
		\item What is the cost for using 1320 kWh?
		\vfill
		\item What is a formula to find the cost for using $x$ kWh if $x$ is greater than 1000 kWh?
		\vfill
		\vfill
	\end{enumerate}
\end{myPrep}


