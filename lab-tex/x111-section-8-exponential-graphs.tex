%============================================================
% MTH 111Z Project - Template File
% Section 6.2 from OpenStax OER
%	Updated 202302
%============================================================


%\fakesection{Graphs of Exponential Functions} \label{exponential-graph}
\section{Graphs of Exponential Functions} \label{exponential-graph}

%Introduction




\subsection*{Preparation Exercises} \label{prep-exponential-graph}

\begin{myPrep}
text

	\begin{enumerate}
		\item Question 1
		\vfill
		\item Question 2
		\vfill
		\item Question 3
		\vfill
		\item Question 4
		\vfill
		\item Question 5
		\vfill
	\end{enumerate}
\end{myPrep}


%======================================================
\newpage
%======================================================


\subsection*{Examples} \label{examples-exponential-graph}

This is where some introductory text would go.

\begin{myExample}
\begin{enumerate}

\item Here is the first example.
\vfill

\item Here is the second example.
\vfill

\end{enumerate}
\end{myExample}

%======================================================
\newpage
%======================================================

\subsection*{Activities} \label{activities-exponential-graph}

\begin{myActivity}
\begin{enumerate}
	\item Here is the first activity. Maybe it's a station maze. Maybe it's a puzzle.
	\vfill
\end{enumerate}
\end{myActivity}

\begin{myActivity}
\begin{enumerate}
	\item Here's the second activity. It's better than the first.
	\vfill
\end{enumerate}
\end{myActivity}



%======================================================
 \newpage
%======================================================

\subsection*{Exit Exercises} \label{exit-exponential-graph}

~


\begin{myExit}
	\begin{enumerate}
		\item Question 1
		\vfill
		\item Question 2
		\vfill
		\item Question 3
		\vfill
		\item Question 4
		\vfill
		\item Question 5
		\vfill
	\end{enumerate}
\end{myExit}
\vfill































% %======================================================
% \newpage
% %======================================================

% \begin{myExample}
% What are the domains of the following functions?
% 	\begin{enumerate}
% 		\begin{multicols}{2}
% 		\item $f(x) = 7x(x+8)(x-3)$
% 		\item $g(t) = \dfrac{7t(t+8)}{t-3}$
% 		\end{multicols}
% 		\vfill
% 		\begin{multicols}{2}
% 		\item $j(n) = \sqrt{12-3n}$
% 		\item $k(w) = w^2 -8w-9$
% 		\end{multicols}
% 		\vfill
% 		\begin{multicols}{2}
% 		\item $m(a) = \dfrac{5a-9}{2a+8}$
% 		\item $r(u) = \dfrac{\sqrt{u+3}}{u-7}$
% 		\end{multicols}
% 		\vfill
% 		\vfill

% 	\end{enumerate}
% \end{myExample}


% {\bf Textbook Reference:}\\
% If you need a few more examples of set-builder and interval notations, \\please see the \href{https://openstax.org/books/algebra-and-trigonometry-2e/pages/3-2-domain-and-range#fs-id1165137677916}{Using Notations to Specify Domain and Range} section of \S3.2 in our book.

% %======================================================
% \newpage
% %======================================================

% \begin{center}
%   {\Large {\bf  Part 2: Finding Domains and Ranges from Graphs}}
% \end{center}

% \begin{myExample}
% Let $y=f(x)$ be defined by the graph below.  (Are you sure this is a function?  Why?)\\

% \begin{minipage}{0.4\linewidth}
% \begin{center}
% \begin{tikzpicture}
% 		\begin{axis}[
% 				framed,
% 				width=7cm,height=7cm,
% 				axis x line=middle,
% 				axis y line=middle,
% 				xlabel={$x$},
% 				ylabel={$y$},
% 				xmin=-8,xmax=8,
% 				ymin=-8,ymax=8,
%         xtick={-6,-4,...,6},
%         	minor xtick={-7,-5,...,7},
%         ytick={-6,-4,...,6},
%         	minor ytick={-7,-5,...,7},
%         grid=both
% 				]
% 				% use TeX as calculator:
% 					\addplot[smooth,mark=*,first,line width=1.0pt,fill=white]coordinates{	(-4,2)	};
% 					\addplot[smooth,mark=*,first,line width=1.0pt]coordinates{	(5,-4)	};
% 				\addplot[first,line width=1.5pt,samples=200]expression[domain=-4:4.99999]{2*(-x+5)^(1/2)-4};
% 		\end{axis}
% 		\end{tikzpicture}
% ~\\
% \end{center}
% \end{minipage}
% \begin{minipage}{0.6\linewidth}
% \begin{enumerate}
% 	\item Find $f(1)$.\\
% 	\item Find $f(5)$.\\
% 	\item Find $f(-4)$.\\
% 	\item Find $f(6)$.\\
% \end{enumerate}
% \end{minipage}

% \begin{enumerate}
% \setcounter{enumi}{4}
% 	\item State the domain of this function in interval notation and set-builder notation.\\~\\
% 	\item State the range of this function in interval notation and set-builder notation.\\~\\
% \end{enumerate}
% \end{myExample}


% \begin{myExample}
% Use $y=h(t)$ and $y=k(x)$ to answer the following.
% \vspace{-1cm}
% \begin{multicols}{2}
% \begin{center}
% \captionof{figure}{$y=h(t)$}
% \begin{tikzpicture}
% 		\begin{axis}[
% 				framed,
% 				width=6cm,height=6cm,
% 				xlabel={$t$},
% 				ylabel={$y$},
% 				xmin=-8,xmax=8,
% 				ymin=-8,ymax=8,
%         xtick={-6,-4,...,6},
%         	minor xtick={-7,-5,...,7},
%         ytick={-6,-4,...,6},
%         	minor ytick={-7,-5,...,7},
%         grid=both
% 				]
% 				% use TeX as calculator:
% 				\addplot[blue,line width=1.5pt,samples=400,-]expression[domain=-7:-4]{-x-4};
% 					\addplot[smooth,mark=*,blue,line width=2.0pt]coordinates{	(-7,3)	};
% 				\addplot[blue,line width=1.5pt,samples=400,-]expression[domain=-4:-3]{x+4};
% 				\addplot[blue,line width=1.5pt,samples=400,-]expression[domain=-3:-2]{(x+3)^2+1};
% 				\addplot[blue,line width=1.5pt,samples=400,-]expression[domain=-2:0]{-(x+1)^2+3};
% 				\addplot[blue,line width=1.5pt,samples=400,-]expression[domain=0:2]{-x+2};
% 				\addplot[blue,line width=1.5pt,samples=400,-]expression[domain=2:4]{(x-3)^2-1};
% 				\addplot[blue,line width=1.5pt,samples=400,-]expression[domain=4:6]{x-4};
% 					\addplot[smooth,mark=*,blue,line width=1pt,fill=white]coordinates{	(6,2)	};
% 		\end{axis}
% 		\end{tikzpicture}
% 		\label{fig:dr1}
% \end{center}

% \begin{center}
% \captionof{figure}{$y=k(x)$}
% \begin{tikzpicture}
% 		\begin{axis}[
% 				framed,
% 				width=6cm,height=6cm,
% 				xlabel={$x$},
% 				ylabel={$y$},
% 				xmin=-8,xmax=8,
% 				ymin=-8,ymax=8,
%         xtick={-6,-4,...,6},
%         	minor xtick={-7,-5,...,7},
%         ytick={-6,-4,...,6},
%         	minor ytick={-7,-5,...,7},
%         grid=both
% 				]
% 				% use TeX as calculator:
% 				\addplot[blue,line width=1.5pt,samples=400,-]expression[domain=-7:-5.99]{x+1};
% 				\addplot[smooth,mark=*,blue,line width=1pt,fill=white]coordinates{	(-7,-6)	};
% 				\addplot[blue,line width=1.5pt,samples=400,-]expression[domain=-6.01:-3.99]{2*(x)+7};
% 				\addplot[blue,line width=1.5pt,samples=400,-]expression[domain=-4.01:-2.99]{(x)+3};
% 				\addplot[blue,line width=1.5pt,samples=400,-]expression[domain=-3.01:-0.99]{-1/2*(x)-1.5};
% 				\addplot[blue,line width=1.5pt,samples=400,-]expression[domain=-1.01:0]{2*(x)^2-3};
% 				\addplot[blue,line width=1.5pt,samples=400,-]expression[domain=0:1.01]{-(x)^2-3};
% 				\addplot[blue,line width=1.5pt,samples=400,-]expression[domain=0.99:2]{-(x-2)^2-3};
% 				\addplot[blue,line width=1.5pt,samples=400,-]expression[domain=2:3.01]{2*(x-2)^2-3};
% 				\addplot[blue,line width=1.5pt,samples=400,-]expression[domain=2.99:5.01]{(x-3)^2-1};
% 				\addplot[blue,line width=1.5pt,samples=400,-]expression[domain=4.99:6]{3};
% 				\addplot[smooth,mark=*,blue,line width=1.5pt]coordinates{	(6,3)	};
% 		\end{axis}
% 		\end{tikzpicture}
% 		\label{fig:dr2}
% \end{center}
% \end{multicols}

% 		\begin{enumerate}\setlength{\itemsep}{0.85in}
% 		\begin{multicols}{2}
% 				\item		State the domain of $h$.
% 				\item		State the domain of $k$.
% 		\end{multicols}
% \vfill
% 		\begin{multicols}{2}
% 				\item		State the range of $h$.
% 				\item		State the range of $k$.
% 		\end{multicols}
% \vfill
% 		\end{enumerate}	
% \end{myExample}

% %======================================================
% \newpage
% %======================================================

% \begin{center}
%   {\Large {\bf  Part 3: Piecewise-Defined Functions}}
% \end{center}


% \begin{myDefinition} 
% A function that is defined by different formulas on different parts of its domain is called a\\
% {\bf \underline{piecewise-defined function}}.  Piecewise-defined functions take on the form:\\  
% 				\[
% 				f(x)=
% 					\begin{cases}
% 							\text{formula \#1}				&	\textrm{if}\ 	\ \			\text{$x$ is in this part of the domain}		\\
% 							\text{formula \#2}				&	\textrm{if}\ 	\ \			\text{$x$ is in this part of the domain}		\\
% 							\text{etc.}				&	\textrm{if}\ 	\ \			\text{etc.}		\\
% 					\end{cases}
% 				\]
% \ 

% \end{myDefinition}

% \begin{myExample}
% In Table~\ref{tab:tax1}, the 2011 federal income tax rates\footnote{\url{http://www.irs.gov/newsroom/article/0,,id=233465,00.html}} for 2011 are shown.  

% \renewcommand{\arraystretch}{1.5}
% \begin{table}[h!]
% \centering
% \caption{Federal Income Tax Percentage Rates for 2011}
% \label{tab:tax1} \small{
% 	\begin{tabular}{| c | c | c |}
% 	\hline 
% 		Income Amount $(x)$										& Percentage of Income Taxed (in $\%$) & Tax on $x$ Dollars of Income ($T(x)$, in \$) \\ \hline
% 		\	$0 \le x < 8500$										&				10		&	$0.10x$								\\ \hline			
% 			$8500 \le x < 34500$								&				15			&	$0.15x$							\\ \hline			
% 			$34500 \le x < 83600$								&				25			&	$0.25x$							\\ \hline			
% 			$83600 \le x < 174400$							&				28				&	$0.28x$						\\ \hline										
% 			$174400 \le x < 379150 $						&				33					&	$0.33x$					\\ \hline				
% 			$x \ge 379150$											&				35	&	$0.35x$									\\ \hline					
% 	\end{tabular}}
% \end{table}

% Notice that for each interval, the percentage of income taxed as a function of income is \emph{constant}.  If we graph each \emph{piece} over its respective interval, we obtain the following: \\
% 		\begin{minipage}{0.5\linewidth}

% \begin{center}
% 	\captionof{figure}{Graph of $y=T(x)$}
% 	\label{fig:b}
% 	   	\begin{tikzpicture}
% 			\begin{axis}[
% 				framed,
% 				height=3in,
% 				width=3in,
% 				xmin=0,xmax=500,
% 				ymin=0,ymax=180,
% 					xlabel style={at={(ticklabel cs:0.5)},anchor=near ticklabel},
%       		ylabel style={at={(ticklabel cs:0.5)},rotate=90,anchor=near ticklabel},
% 				xlabel={\tiny{$x$, income amount in \$1000s}},
% 				ylabel={\tiny{ $y$, tax amount in \$1000s}},
% 				xtick={0,50,100,...,450},
% 					 minor xtick={25,75,...,475},
% 				ytick={10,20,...,180},
% 				%	  minor ytick={1,2,3,...,39},
% 				grid=both,
% 				]
% 				% use TeX as calculator:
% 				\addplot[blue, line width=1.25pt]expression[domain=0:8.5,samples=200]{0.1*x};
% 						\addplot[smooth,mark=*,blue,line width=1.25pt]coordinates{	(0,0)	};
% 						\addplot[smooth,mark=*,blue,line width=1.25pt,fill=white]coordinates{	(8.5,0.85)	};
% 				\addplot[blue, line width=1.25pt]expression[domain=8.5:34.5,samples=200]{0.15*x};
% 						\addplot[smooth,mark=*,blue,line width=1.25pt]coordinates{	(8.5,1.275)	};
% 						\addplot[smooth,mark=*,blue,line width=1.25pt,fill=white]coordinates{	(34.5,5.175)	};
% 				\addplot[blue, line width=1.25pt]expression[domain=34.5:83.6,samples=200]{0.25*x};
% 						\addplot[smooth,mark=*,blue,line width=1.25pt]coordinates{	(34.5,8.625)	};
% 						\addplot[smooth,mark=*,blue,line width=1.25pt,fill=white]coordinates{	(83.6,20.9)	};
% 				\addplot[blue, line width=1.25pt]expression[domain=83.6:174.4,samples=200]{0.28*x};
% 						\addplot[smooth,mark=*,blue,line width=1.25pt]coordinates{	(83.6,23.408)	};
% 						\addplot[smooth,mark=*,blue,line width=1.25pt,fill=white]coordinates{	(174.4,48.832)	};
% 				\addplot[blue, line width=1.25pt]expression[domain=174.4:379.15,samples=200]{0.33*x};
% 						\addplot[smooth,mark=*,blue,line width=1.25pt]coordinates{	(174.4,57.552)	};
% 						\addplot[smooth,mark=*,blue,line width=1.25pt,fill=white]coordinates{	(379.15,125.1195)	};
% 				\addplot[blue, line width=1.25pt]expression[domain=379.15:500,samples=200,->]{0.35*x};
% 						\addplot[smooth,mark=*,blue,line width=1.25pt]coordinates{	(379.15,132.7025)	};
% 	\end{axis}
% 		\end{tikzpicture}
% \end{center}



% 		\end{minipage}
% 		\begin{minipage}{0.5\linewidth}
% Write the formula for $T(x)$.\\~\\~\\~\\~\\~\\~\\~\\~\\~\\~\\

% 		\end{minipage}
% \end{myExample}




% %======================================================
% \newpage
% %======================================================

% \begin{myExample}
% Consider the function $g(x) = |x|$, the absolute value function.
% \begin{enumerate}
% 	\item When you first learned about the absolute value, how did you describe what is measured?\\~\\~\\
% 	\item In MTH 95 (Intermediate Algebra), what was another way you learned that you could write $g$?\\~\\~\\
% 	\item If you had to explain to someone in words how to computer an absolute value, what would you say?\\~\\~\\
% 	\item The graph of $y=g(x)$ (the absolute value function) is given in Figure~\ref{fig:absval}.  \\
% 	Write this formula for this function as a piecewise-defined function.\\
% \begin{minipage}{0.5\linewidth}
% Formula: \\[4cm]

% \end{minipage}
% \begin{minipage}{0.5\linewidth}
% \begin{center}
% \captionof{figure}{}
% \begin{tikzpicture}
% 		\begin{axis}[
% 				framed,
% 				height=8cm,
% 				width=8cm,
% 				xlabel={$x$},
% 				ylabel={$y$},
% 				xmin=-8,xmax=8,
% 				ymin=-8,ymax=8,
%         xtick={-6,-4,...,6},
%         	minor xtick={-7,-5,...,7},
%         ytick={-6,-4,...,6},
%         	minor ytick={-7,-5,...,7},
%         grid=both
% 				]
% 				% use TeX as calculator:
% 				\addplot[blue,line width=1.25pt,samples=200,<-]expression[domain=-8:0]{-1*x};
% 				\addplot[blue,line width=1.25pt,samples=200,->]expression[domain=0:8]{x};
% 		\end{axis}
% 		\end{tikzpicture}
% 		\label{fig:absval}
% \end{center}
% \end{minipage}
% \end{enumerate}

% \end{myExample}


% %======================================================
% \newpage
% %======================================================




% \begin{myExample}

% Let $k$ and $m$ be as defined below.
% %\begin{minipage}{0.6\linewidth}
% \vspace{-1cm}
% \begin{multicols}{2}
% \begin{center}
% 				\[
% 				k(t)=
% 					\begin{cases}
% 						-2t-4								& \textrm{if}\ 	\ \			t \le -1			\\
% 						 2t+3							& \textrm{if}\ 	\ \			t > -1 		
% 					\end{cases}
% 				\]  
% \end{center}
% \begin{center}
% 				\[
% 				m(x)=
% 					\begin{cases}
% 						 x-3								& \textrm{if}\ 	\ \			0 < x \le 4 		\\
% 						-x-4								& \textrm{if}\ 	\ \			x \le 0	
% 					\end{cases}
% 				\]
% \end{center}
% \end{multicols}
% %\end{minipage}

% \begin{enumerate}
% \begin{multicols}{2}
% \item What is the domain of $k$?\\~\\~\\
% \item What is the domain of $m$?\\~\\~\\
% \end{multicols}
% \begin{multicols}{2}
% \item Find $k(10)$.\\~\\~\\~\\
% \item Find $m(3)$.\\~\\~\\~\\
% \end{multicols}
% \item Find $k(-9)$.\\~\\~\\~\\
% \item Find $m(0)$.\\~\\~\\~\\
% \begin{multicols}{2}
% \item Graph $y=k(t)$ in Figure~\ref{fig:pwd12} below.
% \item Graph $y=m(x)$ in Figure~\ref{fig:pwd22} below.
% \end{multicols}

% \vspace{-1cm}
% \begin{multicols}{2}
% \begin{center}
% \captionof{figure}{$y=k(t)$}
% \begin{tikzpicture}
% 		\begin{axis}[
% 				framed,
% 				height=7cm,
% 				width=7cm,
% 				xmin=-8,xmax=8,
% 				ymin=-8,ymax=8,
%         xtick={0},
%         	minor xtick={-7,-6,...,7},
%         ytick={0},
%         	minor ytick={-7,-6,...,7},
%         grid=both
% 				]
% 				% use TeX as calculator:
% 		\end{axis}
% 		\end{tikzpicture}
% 		\label{fig:pwd12}
% \end{center}

% \begin{center}
% \captionof{figure}{$y=m(x)$}
% \begin{tikzpicture}
% 		\begin{axis}[
% 				framed,
% 				height=7cm,
% 				width=7cm,
% 				xmin=-8,xmax=8,
% 				ymin=-8,ymax=8,
%         xtick={0},
%         	minor xtick={-7,-6,...,7},
%         ytick={0},
%         	minor ytick={-7,-6,...,7},
%         grid=both
% 				]
% 				% use TeX as calculator:
% 		\end{axis}
% 		\end{tikzpicture}
% 		\label{fig:pwd22}
% \end{center}
% \end{multicols}

% \end{enumerate}

% \end{myExample}


% %======================================================
% \newpage
% %======================================================


% \begin{myExample}
% In Figure~\ref{fig:piecewise4} is the graph of $y=F(x)$.	\\
% \begin{enumerate}%\setlength{\itemsep}{1.25in}
% \begin{minipage}[!t]{0.4\linewidth}
% \item Write the formula for the piecewise function $F$: \\[6cm]
% \end{minipage}
% \begin{minipage}{0.6\linewidth}
% \begin{center}
% \captionof{figure}{$y=F(x)$}
% \begin{tikzpicture}
% 		\begin{axis}[
% 				framed,
% 				height=8cm,
% 				width=8cm,
% 				xlabel={$x$},
% 				ylabel={$y$},
% 				xmin=-8,xmax=8,
% 				ymin=-8,ymax=8,
%         xtick={-6,-4,...,6},
%         	minor xtick={-7,-5,...,7},
%         ytick={-6,-4,...,6},
%         	minor ytick={-7,-5,...,7},
%         grid=both
% 				]
% 				% use TeX as calculator:
% 				\addplot[blue,line width=1.25pt,samples=200,<-]expression[domain=-8:-2]{1/2*(x+2)-3};
% 						\addplot[mark=*,blue,line width=1.0pt]coordinates{	(-2,-3)	};
% 				\addplot[blue,line width=1.25pt,samples=200]expression[domain=-2:3]{5};
% 						\addplot[mark=*,blue,line width=1.0pt,fill=white]coordinates{	(-2,5)	};
% 						\addplot[mark=*,blue,line width=1.0pt]coordinates{	(3,5)	};
% 				\addplot[blue,line width=1.25pt,samples=200]expression[domain=3:7]{-2*(x-6)-5};
% 						\addplot[mark=*,blue,line width=1.0pt,fill=white]coordinates{	(3,1)	};
% 						\addplot[mark=*,blue,line width=1.0pt,fill=white]coordinates{	(7,-7)	};
% 		\end{axis}
% 		\end{tikzpicture}
% 		\label{fig:piecewise4}
% \end{center}
% \end{minipage}
% %\vspace{11pt}
% \begin{multicols}{2}
% 	\item	Evaluate $F(0)$.	

% 	\item	Solve $F(x)=0$.	
% \end{multicols}
% ~\\
% \begin{multicols}{2}
% 	\item	Evaluate $F(2)$.
% 	\item	Solve $F(x)=-5$.	
% \end{multicols}
% ~\\

% 	\item	Solve $F(x)=-5$ both algebraically (using your formula from Part a) and graphically.	
% \vfill
% \end{enumerate}

% \end{myExample}






