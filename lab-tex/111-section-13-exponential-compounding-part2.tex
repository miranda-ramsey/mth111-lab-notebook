

%======================================================
\newpage
%======================================================


\subsection*{Practice Exercises} \label{practice-exponential-compounding}

\begin{myPractice}
How much is owed at the end of a 6 years if \$12,000 is borrowed at 6.4\% annual interest, compounded quarterly?
\vfill
\end{myPractice}

\begin{myPractice}
Maya said the formula she set up for an exercise to calculate the amount (in dollars) owed on a loan after some number of years is $A = 10250\left(1+\frac{0.04}{12}\right)^{120}$.  What is the principal of the loan, the nominal interest rate, the number of compounds per year, and the number of years of the loan?
\vfill
\end{myPractice}


\begin{myPractice}
If Marshon invests \$5000 at 5.8\% annual interest compounded continuously, how much will he have in the account in 13 years?
\vfill
\end{myPractice}



%======================================================
\newpage
%======================================================

\subsection*{Definitions} \label{def-exponential-compounding}

\begin{myDefinition}[Simple Interest]~\\[0.5mm]
The formula to calculate {\bf simple interest} is $${I = Prt}$$
where $I$ is the amount of interest earned, \\
$P$ is the principal or initial amount of money, \\
$r$ is the annual interest rate, and \\
$t$ is the number of years the interest is earned 
\end{myDefinition}

\begin{myDefinition}[Compound Interest]~\\[0.5mm]
The formula to calculate {\bf compound interest} is $${A = P \left(1+\dfrac{r}{n}\right)^{nt}}$$
where $A$ is total amount of money owed or earned,\\
$P$ is the principal, \\
$r$ is the nominal annual interest rate,  \\
$n$ is the number of times the interest is compounded per year, and \\
$t$ is the number of years that the interest is earned. \\
Note: This formula is used where there is a finite number of compounding per year.
\end{myDefinition}

\begin{myDefinition}[The Number $\boldsymbol{e}$]~\\[0.5mm]
\begin{minipage}{0.9\linewidth}
$\boldsymbol{e}$, also known as {\bf Euler's number}, is the irrational number that results from $$\lim\limits_{n\to \infty}\left(1+\frac{1}{n}\right)^n$$
\defexample You can watch this \href{https://youtu.be/AuA2EAgAegE}{Numberphile YouTube video about $e$}. 
\end{minipage}
\begin{minipage}{0.1\linewidth}
\flushright \qrcode[height=1cm]{https://youtu.be/AuA2EAgAegE}
\end{minipage}
%$\boldsymbol{e}$, also known as {\bf Euler's number}, is the irrational number that results from $$\lim\limits_{n\to \infty}\left(1+\frac{1}{n}\right)^n$$	
%\defexample You can watch this \href{https://youtu.be/AuA2EAgAegE}{Numberphile YouTube video about $e$}. 
%(url: youtu.be/AuA2EAgAegE)
\end{myDefinition}



\begin{myDefinition}[Continuously Compounded Interest]~\\[0.5mm]
The formula to calculate {\bf continuously compounded interest} is $${A = P e^{rt}}$$
where $A$ is total amount of money owed or earned,\\
$P$ is the principal, \\
$r$ is the nominal annual interest rate, and \\
$t$ is the number of years that the interest is earned. \\
Note: This formula is used when the compounding happens continuously.
\end{myDefinition}



\begin{myDefinition}[Effective Interest Rate]~\\[0.5mm]
The \textbf{effective interest rate, ${r_e}$} is the equivalent interest rate that, if compounded annually, would yield the same result after 1 year as compounding the stated nominal rate $n$ times per year or compounding the nominal rate continuously.\\[0.5em]
When compounding $n$ times per year, $r_e = \left(1 + \frac{r}{n}\right)^n - 1$, where $r$ is the nominal annual interest rate.\\
When compounding continuously, $r_e = e^r - 1$, where $r$ is the nominal annual interest rate.\\


\end{myDefinition}




%======================================================
 \newpage
%======================================================

\subsection*{Exit Exercises} \label{exit-exponential-compounding}

\begin{myExit}
In the year 2001, Alyssa opened a retirement account that earns a nominal interest rate 7.25\% per year. Her initial deposit was \$13,500. How much will the account be worth in 2025 if interest compounds monthly? How much more would she make if interest compounded continuously?\end{myExit}
\vfill

\begin{myExit}
Kyoko received a \$10,000 scholarship that she gets to invest for college.  Her bank has several investment accounts to choose from, all compounding daily.  Her goal is to have \$15,000 by the time she finishes high school in 6 years.  To the nearest hundredth of a percent, what should her minimum nominal interest rate be in order to reach her goal? 
\end{myExit}
\vfill

\begin{myExit}
A small business is planning on building a new facility in 9 years. They can invest money at 7\% annual interest, compounded daily, right now. If they need \$600,000 to build the new facility in 9 years, what is the minimum they need to invest to ensure they have \$600,000 in 9 years?
\end{myExit}
\vfill





\exitlikert{exponential functions}








