




%======================================================
\newpage
%======================================================


\subsection*{Practice Exercises} \label{practice-functions-transformations}

\begin{myPractice}
Let $f$ be a function.
\begin{enumerate}
%\item If $k_1(x) = f(-x)-4$, state the transformations that take the graph of $y=f(x)$ to the graph of $y=k_1(x)$.
%\vfill

\item If $k_1(x) = - f(x+3)$, state the transformations that take the graph of $y=f(x)$ to the graph of $y=k_1(x)$.
\vfill

%\item If $k_3(x) = \frac{1}{3} f(x-2)$, state the transformations that take the graph of $y=f(x)$ to the graph of $y=k_3(x)$.
%\vfill

\item If $k_2(x) =  f(\frac{1}{5}x)+3$, state the transformations that take the graph of $y=f(x)$ to the graph of $y=k_2(x)$.
\vfill
\end{enumerate}
\end{myPractice}

\begin{myPractice}
Let $g$ be a function for which we know that $g(3)=5$.
\begin{enumerate}
\item If $m(x) = -2g(x-4)+7$, state a sequence of transformations that takes the graph of $y=g(x)$ to the graph of $y=m(x)$?
\vfill
\vfill

\begin{multicols}{2}
\item What point do you know is on \\
the graph of $y=g(x)$?
\item What point do you know is on \\
the graph of $y=m(x)$?
\end{multicols}
\vfill

\end{enumerate}
\end{myPractice}

\newpage


\begin{myPractice}
Given in Figure~\ref{fig:prac-transform-1} is the graph of $y=f(x)$.  \\
Based on the graph, would you say that $f$ is even, odd, or neither? Explain your answer.\\
\begin{minipage}{0.4\linewidth}
	\begin{center}
	\captionof{figure}{$y=f(x)$}~\\[-0.8em]
	\begin{tikzpicture}
 		\begin{axis}[
 			framedaxes,
 			height=6cm,
 			width=6cm,
 			xlabel={$x$},
 			ylabel={$y$},
 			xmin=-8,xmax=8,
 			ymin=-8,ymax=8,
		        xtick={-6,-4,...,6},
		       	minor xtick={-11,-9,-7,...,11},
		        ytick={-6,-4,...,6},
	         	minor ytick={-7,-5,...,7},
		         grid=both
 			]
 		% use TeX as calculator:
				\addplot[first,line width=1.5pt,samples=400,<->]expression[domain=-7.9:7.9]{1/2*abs(x)*sin(180*x/4)};
%					\addplot[first,smooth,mark=*,line width=1pt,fill=first]coordinates{	(-3,1)	};
%					\addplot[first,smooth,mark=*,line width=1pt,fill=white]coordinates{	(-3,-4)	};
%				\addplot[first,line width=1.5pt,samples=400,-]expression[domain=-3:1]{-4};
%					\addplot[first,smooth,mark=*,line width=1pt,fill=white]coordinates{	(1,-4)	};
%					\addplot[first,smooth,mark=*,line width=1pt,fill=first]coordinates{	(2,3)	};
%				\addplot[first,line width=1.5pt,samples=400,->]expression[domain=2:7.5]{-2*(x-2)+3};
 		\end{axis}
 		\end{tikzpicture}
		\label{fig:prac-transform-1}
	\end{center}
\end{minipage}
~\\
\end{myPractice}


\begin{myPractice}
Algebraically determine if the function $g(x) = -2x^2-3x$ is even, odd, or neither.
\vfill
\vfill
\end{myPractice}



\begin{myPractice}
\begin{enumerate}
\item If $k_3(x) = 4\cdot f(6x-2)+5$, state the transformations that take the graph of $y=f(x)$ to the graph of $y=k_3(x)$.
\vfill

\item If $k_4(x) =  \frac{7}{3}\cdot f(\frac{1}{5}x-1)$, state the transformations that take the graph of $y=f(x)$ to the graph of $y=k_4(x)$.
\vfill
\end{enumerate}
\end{myPractice}




%======================================================
\newpage
%======================================================

\subsection*{Definitions} \label{def-functions-transformations}

\begin{myDefinition}[Vertical Shift:]~\\[0.5mm]
\begin{minipage}{0.9\linewidth}
Given a function $f$, the graph of  $g(x) = f(x) +k$ for some real number $k$ is a {\bf vertical shift} of the graph of $y=f(x)$.\\[2mm]
If $k>0$, $g$ will be the graph of $f$ shifted up by $k$ units.\\
If $k<0$, $g$ will be the graph of $f$ shifted down by $k$ units.\\[0.4em]
\defexample \href{https://tiny.cc/111Z-VertShift}{View this Desmos graph} to see an interactive example of the definition. % (url: tiny.cc/111Z-VertShift)
\end{minipage}
\begin{minipage}{0.1\linewidth}
\flushright \qrcode[height=1cm]{https://tiny.cc/111Z-VertShift}
\end{minipage}
\end{myDefinition}


\begin{myDefinition}[Horizontal Shift:]~\\[0.5mm]
\begin{minipage}{0.9\linewidth}
Given a function $f$, the graph of $g(x) = f(x-h)$ for some real number $h$ is a {\bf horizontal shift} of the graph of $y=f(x)$.\\[2mm]
If $h>0$, $g$ will be the graph of $f$ shifted right by $h$ units.\\
If $h<0$, $g$ will be the graph of $f$ shifted left by $h$ units.\\[0.4em]
\defexample \href{https://tiny.cc/111Z-HorizShift}{View this Desmos graph} to see an interactive example of the definition.  %(url: tiny.cc/111Z-HorizShift)
\end{minipage}
\begin{minipage}{0.1\linewidth}
\flushright \qrcode[height=1cm]{https://tiny.cc/111Z-HorizShift}
\end{minipage}
\end{myDefinition}


\begin{myDefinition}[Vertical Stretch/Compression:]~\\[0.5mm]
\begin{minipage}{0.9\linewidth}
Given a function $f$, the graph of $g(x)= a\cdot f(x) $ for some real number $a$, where $a\neq0$, is a {\bf vertical stretch} or a {\bf vertical compression} of the graph of $y=f(x)$.\\[2mm] 
If $a>1$, $g$ will be the graph of $f$ vertically stretched by a factor of $a$.\\
If $0<a<1$, $g$ will be the graph of $f$ vertically compressed by a factor of $a$.\\[2mm] 
If $a<0$, $g$ will be a combination of a vertical reflection over the $x$-axis {\it and} a vertical stretch or compression of $f$.\\[0.4em]
\defexample \href{https://tiny.cc/111Z-VertStretch}{View this Desmos graph} to see an interactive example of the definition. %(url: tiny.cc/111Z-VertStretch)
\end{minipage}
\begin{minipage}{0.1\linewidth}
\flushright \qrcode[height=1cm]{https://tiny.cc/111Z-VertStretch}
\end{minipage}
\end{myDefinition}



\begin{myDefinition}[Horizontal Stretch/Compression:]~\\[0.5mm]
\begin{minipage}{0.9\linewidth}
Given a function $f$, the graph of $g(x) = f(b\cdot x) $ for some real number $b$, where $b\neq0$, is a {\bf horizontal stretch} or a {\bf horizontal compression} of the graph of $y=f(x)$.\\[2mm] 
If $b>1$, $g$ will be the graph of $f$ horizontally compressed by a factor of $\frac{1}{b}$.\\
If $0<b<1$, $g$ will be the graph of $f$ horizontally stretched by a factor of $\frac{1}{b}$.\\[2mm]
If $b<0$, $g$ will be a combination of a horizontal reflection over the $y$-axis {\it and} a horizontal stretch or compression of $f$.\\[0.4em]
\defexample \href{https://tiny.cc/111Z-HorizStretch}{View this Desmos graph} to see an interactive example of the definition.  %(url: tiny.cc/111Z-HorizStretch)
\end{minipage}
\begin{minipage}{0.1\linewidth}
\flushright \qrcode[height=1cm]{https://tiny.cc/111Z-HorizStretch}
\end{minipage}
\end{myDefinition}

\newpage

\begin{myDefinition}[Vertical Reflection:]~\\[0.5mm]
\begin{minipage}{0.9\linewidth}
Given a function $f$, the graph of $y = -f(x)$ is a {\bf vertical reflection} of the graph of $y=f(x)$ over the $x$-axis.\\[0.4em]
\defexample \href{https://tiny.cc/111Z-VertReflect}{View this Desmos graph} to see an interactive example of the definition.  %(url: tiny.cc/111Z-VertReflect)
\end{minipage}
\begin{minipage}{0.1\linewidth}
\flushright \qrcode[height=1cm]{https://tiny.cc/111Z-VertReflect}
\end{minipage}
\end{myDefinition}


\begin{myDefinition}[Horizontal Reflection:]~\\[0.5mm]
\begin{minipage}{0.9\linewidth}
Given a function $f$, the graph of $y = f(-x)$ is a {\bf horizontal reflection} of the graph of $y=f(x)$ over the $y$-axis.\\[0.4em]
\defexample \href{https://tiny.cc/111Z-HorizReflect}{View this Desmos graph} to see an interactive example of the definition.  %(url: tiny.cc/111Z-HorizReflect)
\end{minipage}
\begin{minipage}{0.1\linewidth}
\flushright \qrcode[height=1cm]{https://tiny.cc/111Z-HorizReflect}
\end{minipage}
\end{myDefinition}


\begin{myDefinition}[Combined Transformations:]~\\[0.5mm]
Given a function $f$, the combined vertical transformations written in the form $y=a\cdot f(x)+k$, $a\neq 0$, would be applied in the order:
\setlist{itemsep=1pt}
\begin{itemize}
	\item a vertical reflection over the $x$-axis, if $a<0$
	\item a vertical stretch or compression by a factor of $|a|$
	\item a vertical shift up or down by $k$ units
\end{itemize}

Given a function $f$, the combined horizontal transformations written in the form $ y=f\left(b\cdot (x-h)\right)$, $b\neq 0$, would be applied in the order:
\setlist{itemsep=0pt}
\begin{itemize}
	\item a horizontal reflection over the $y$-axis, if $b<0$
	\item a horizontal stretch or compression by a factor of $\left|\frac{1}{b}\right|$
	\item a horizontal shift left or right by $h$ units
\end{itemize}

Given a function $f$, the combined horizontal transformations written in the form $y= f(bx-h)$, $b\neq 0$, would be applied in the order:
\setlist{itemsep=0pt}
\begin{itemize}
	\item a horizontal shift left or right by $h$ units
	\item a horizontal reflection over the $y$-axis, if $b<0$
	\item a horizontal stretch or compression by a factor of $\left|\frac{1}{b}\right|$
\end{itemize}
\end{myDefinition}


\begin{myDefinition}[Even Function:]~\\[0.5mm]
\begin{minipage}{0.9\linewidth}
Given a function $f$, if $f(-x)=f(x)$ for every input $x$, then $f$ is an  {\bf  even function}.\\
We describe even functions as being symmetrical about the $y$-axis.\\[0.4em]
\defexample \href{https://tiny.cc/111Z-EvenFunction}{View this Desmos graph} to see an interactive example of the definition.%  (url: tiny.cc/111Z-EvenFunction)
\end{minipage}
\begin{minipage}{0.1\linewidth}
\flushright \qrcode[height=1cm]{https://tiny.cc/111Z-EvenFunction}
\end{minipage}
\end{myDefinition}


\begin{myDefinition}[Odd Function:]~\\[0.5mm]
\begin{minipage}{0.9\linewidth}
Given a function $f$, if $ -f(-x)=f(x)$ for every input $x$, then $f$ is an {\bf odd function}.\\
We describe odd functions as being symmetrical about the origin.\\
Note: $f(x) = -f(-x)$ is equivalent to the statement $f(-x) = -f(x)$.\\[0.4em]
\defexample \href{https://tiny.cc/111Z-OddFunction}{View this Desmos graph} to see an interactive example of the definition. % (url: tiny.cc/111Z-OddFunction)
\end{minipage}
\begin{minipage}{0.1\linewidth}
\flushright \qrcode[height=1cm]{https://tiny.cc/111Z-OddFunction}
\end{minipage}
\end{myDefinition}





%======================================================
 \newpage
%======================================================

\subsection*{Exit Exercises} \label{exit-functions-transformations}




\begin{myExit}
	\begin{enumerate}
		\item What is meant by an ``inside'' change?  How do inside changes impact the graph of a function?
		\vfill
		\item What is meant by an ``outside'' change?  How do outside changes impact the graph of a function?
		\vfill
		\item What is the relationship between even and odd functions and transformations?
		\vfill
	\end{enumerate}
\end{myExit}


\begin{myExit}
$f$ is a function and $f(-20) = 32$.
	\begin{enumerate}
		\item If $g(x) = -\frac{1}{8} f(-2x+10)+5$, list the sequence of transformations that take the graph of $y=f(x)$ to the graph of $y=g(x)$.
		\vfill
		\vfill
		\item What point is on the graph of $y=f(x)$ and what point will be on the graph of $y=g(x)$?
		\vfill
	\end{enumerate}
\end{myExit}




\exitlikert{function transformations}



