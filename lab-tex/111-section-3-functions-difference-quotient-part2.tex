

%======================================================
\newpage
%======================================================


\subsection*{Practice Exercises} \label{practice-difference-quotient}

\begin{myPractice}
\begin{enumerate}

\item The function $m$ in Table~\ref{table:aroc-ex1} below shows the cost of movie tickets\footnote{Costs obtained from \url{https://www.natoonline.org/data/ticket-price} } in the U.S. in the year $t$.


%https://www.natoonline.org/data/ticket-price
	\begin{minipage}{0.3\linewidth}
		\renewcommand\arraystretch{1.5}
		\captionof{table}{Price of Movie \\Tickets in the U.S.}
		\begin{tabular}{c|c}
			$t$ & $m(t)$\\
			(year) & (in dollars)\\
			\hline
			1995 & 4.35\\
			1999 & 5.06\\
			2003 & 6.03\\
			2009 & 7.50\\
			2013 & 8.13\\
			2017 & 8.97\\
			2021 & 10.17
		\end{tabular}
		\label{table:aroc-ex1}
	\end{minipage}
	\begin{minipage}{0.7\linewidth}
%		\begin{enumerate}[label=\roman*.]
%			\item What is the average rate of change in the price of a movie ticket from 1995 to 2003? \\[3cm]
%			\item 
\begin{enumerate}
\item	What is the unit of the average rate of change in the price of a movie over any time period? \\[2cm]
\item	What is the average rate of change in the price of a movie ticket from 2003 to 2021? \\[4cm]
\end{enumerate}
%		\end{enumerate}
		\end{minipage}
~\\~\\~\\
	
	\item Given $f(n) = \frac{1}{3}n^2-1$, find the average rate of change of $f$ on the interval $[3,9]$.
	\vfill
	
\end{enumerate}
\end{myPractice}




%======================================================
\newpage
%======================================================


\begin{myPractice}
Find and simplify the difference quotient for each of the following functions.
\begin{enumerate}

\item $f(x) = -6x + 8$
\vfill

\item $g(x) = -2x^2 - 5x$
\vfill


\end{enumerate}
\end{myPractice}

%======================================================
\newpage
%======================================================

\subsection*{Definitions} \label{def-difference-quotient}


\begin{myDefinition}[Rate of Change:]~\\[0.5mm]
A {\bf rate of change} describes how the output values change in relation to a change in the input values.  \\
The unit for the rate of change is ``output unit(s) per input unit.''
\end{myDefinition}

\begin{myDefinition}[Average Rate of Change:]~\\[0.5mm]
The {\bf average rate of change} for a function $f$ between two input values $x_1$ and $x_2$ is the difference in their output values divided by the difference in the two input values.  The average rate of change is calculated using the formula
\[\text{average rate of change} = \dfrac{f(x_2)-f(x_1)}{x_2-x_1}, ~ x_1 \neq x_2\]
The average rate of change is the slope of the line between the two points $(x_1, f(x_1))$ and $(x_2, f(x_2))$.\\[0.5em]
\defexample The function $E(x)$ gives the cost of a dozen eggs (in dollars) $x$ year after 2010.  If we know $E(19) = 1.362$ and $E(23)=2.666$, we can find average rate of change as 
$$\begin{aligned}
\dfrac{E(23)-E(19)}{23\text{ years}-19\text{ years}} &= \dfrac{ \$2.666 - \$1.362 }{4 \text{ years}}\\[0.5em]
&= \dfrac{ \$1.304 }{4\text{ years}}\\[0.5em]
&\approx \$0.33/\text{year}
\end{aligned}$$
This shows that between 2019 and 2023, the cost of a dozen eggs increased on average by about \$0.33/year.
%https://fred.stlouisfed.org/series/APU0000708111
\end{myDefinition}

\begin{myDefinition}[Difference Quotient:]~\\[0.5mm]
The {\bf difference quotient} for a function $f$ is given by the formula
\[ \dfrac{f(x+h)-f(x)}{h}, ~ h\neq0\]  
The difference quotient is the average rate of change between the two points $(x,f(x))$ and $(x+h, f(x+h))$.\\[0.5em]
\defexample Given the function $f(x) = 3x^2-4x$, the difference quotient would be evaluated as 
$$\begin{aligned}
\dfrac{\boldsymbol{f(x+h)}-f(x)}{h} &= \dfrac{\boldsymbol{3(x+h)^2-4(x+h)} - \left(3x^2-4x\right)}{h}\\[0.5em]
&= \dfrac{\boldsymbol{3x^2 + 6xh + 3h^2 -4x -4h}  - 3x^2+4x}{h}\\[0.5em]
&=  \dfrac{6xh + 3h^2 -4h }{h}\\[0.5em]
&=  \dfrac{h(6x + 3h -4) }{h}\\[0.5em]
&=  6x + 3h -4,  \text{ for $h\neq0$}
\end{aligned}$$
\end{myDefinition}


%======================================================
 \newpage
%======================================================

\subsection*{Exit Exercises} \label{exit-difference-quotient}


\begin{myExit}
\begin{enumerate}
		\item What are two situations in your daily life that involve an average rate of change?  \\
		What are the units for these rates of change?
		\vfill
		\vfill
		\item What is the formula for the difference quotient for the function $k$ that has an input variable $p$?
		\vfill
		\vfill
		\item If you have a function $m$ that gives the price of a gallon of milk in the year $t$, what would be the unit for the average rate of change for $m$?
		\vfill
		\vfill
		\vfill
		\vfill
		\end{enumerate}
\end{myExit}

\newpage

\begin{myExit}
	Find and simplify the difference quotient for the function $f(t) = \dfrac{3}{t-6}$.
	\vfill
\end{myExit}


\exitlikert{the difference quotient}

