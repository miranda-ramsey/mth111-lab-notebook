%============================================================
% MTH 111 Project - Hub File
% Updated 202302
%============================================================

\documentclass[11pt,twoside]{report}
\usepackage{etex}
\reserveinserts{28}
\usepackage[usenames,dvipsnames]{color}
%\usepackage{tabularx}								% Ross Different
%\usepackage{tkz-euclide}								% Ross Different
\usepackage{graphicx}
\usepackage{amssymb}
%\usepackage{tasks}								% Ross Different
\usepackage[standard,framed,thmmarks]{ntheorem}	% needed for theorems, examples
\usepackage{enumerate}
\usepackage{pstricks} % needed for many things, including fancy shading in theorems, examples
\usepackage{enumitem}
\usepackage{placeins} % enables \FloatBarrier, useful for positioning figures/tables more precisely.
\usepackage{calc} % calculations such as dividing textwidth/2
\usepackage{multicol}
\usepackage{framed}	% needed for putting boxes round theorems, examples	
\usepackage{pgfplots}
%\pgfplotsset{compat=1.13}
\usepackage{needspace} % needed to keep examples looking good (with \hrule above and below)
\usepackage{fancyhdr} % headers and footersd
\usepgflibrary{arrows}
%\usepackage{nonfloat} % needed for putting figures in boxes- they don't float!
%\usepackage{refcheck} % useful for checking references
\usepackage{array}
\usepackage{tocloft}
\renewcommand\cftsecafterpnum{\vskip3pt}
\renewcommand\cftchapafterpnum{\vskip5pt}
\setlength{\cftbeforetoctitleskip}{0mm}
\setlength\cftaftertoctitleskip{20pt}
\setlength{\cftbeforechapskip}{4pt}
\usepackage{latexsym}
\usepackage{graphicx}
\usepackage{amscd}
\usepackage{multirow}
\usepackage{amsmath}
\usepackage{textcomp} % for cents symbol
\usepackage{lastpage}
\usepackage{setspace}
\usepackage{caption}
\usepackage{wasysym}
% \theoremstyle{definition}
\usepackage{tikz}
\usetikzlibrary{shapes,arrows,calc,positioning}
% \usepackage[charter]{mathdesign} % this is causing warnings with other packages.
\usepackage{hyperref} 
\pagestyle{fancy}
\setlength{\textwidth}{7.25in}			%for =>PDF, use 6.5in
\setlength{\textheight}{9.10in}		%for =>PDF, use 9.5in
\setlength{\oddsidemargin}{-0.375in}		
\setlength{\evensidemargin}{-0.375in}		
\setlength{\voffset}{-.75in}			%for=>PDF, use -0.75in
\setlength{\footskip}{15pt}
\usepackage[per-mode=symbol]{siunitx}
\usepackage[super]{nth} %nth added when appropriate
\usepackage{minitoc}
\usepackage{qrcode}

%\usepackage{dirtytalk} %for quotations											% Ross Different
%\usetkzobj{all}															% Ross Different
%\settasks{after-item-skip=2cm,after-skip=2cm} %for tasks instead of items				% Ross Different
%\usetikzlibrary{decorations.pathreplacing,calligraphy} %for braces on graph				% Ross Different
%\setlength\columnsep{30pt} %space between columns								% Ross Different	
%\newcolumntype{P}[1]{>{\centering\arraybackslash}p{#1}} %center content in cells of table	% Ross Different

\usepackage[T1]{fontenc}
\usepackage{lmodern}
\usepackage{url}


%\usepackage{draftwatermark}
%\SetWatermarkText{{\color{red!10}{DRAFT}}}
%\SetWatermarkScale{1}

% \ifpdf
  \usepackage{pdfcolmk}
% \fi
% % check if using xelatex rather than pdflatex
% \ifxetex
%   \usepackage{fontspec}
% \fi

\usepackage{tikz}  
\usepackage{pifont} % for dingbats
\providecommand{\HUGE}{\Huge} % if not using memoir
\newlength{\drop} % for my convenience
% specify the Webomints family
\newcommand*{\wb}[2]{\fontsize{#1}{#2}\usefont{U}{webo}{xl}{n}}
% select a (FontSite) font by its font family ID
\newcommand*{\FSfont}[1]{\fontencoding{T1}\fontfamily{#1}\selectfont}
% if you don?t have the FontSite fonts either \renewcommand*{\FSfont}[1]{}
% or use your own choice of family.
% select a (TeX Font) font by its font family ID
\newcommand*{\TXfont}[1]{\fontencoding{T1}\fontfamily{#1}\selectfont}
% Generic publisher?s logo
\newcommand*{\plogo}{\fbox{$\mathcal{PL}$}}
% Some shades
\definecolor{Dark}{gray}{0.2}
\definecolor{MedDark}{gray}{0.4}
\definecolor{Medium}{gray}{0.6}
\definecolor{Light}{gray}{0.8}
% Additional font series macros
\makeatletter
% light series
% e.g., kernel doc, section s: line 12 or thereabouts
\DeclareRobustCommand\ltseries
  {\not@math@alphabet\ltseries\relax
   \fontseries\ltdefault\selectfont}
% e.g., kernel doc, section t: line 32 or thereabouts
\newcommand{\ltdefault}{l}
% e.g., kernel doc, section v: line 19 or thereabouts
\DeclareTextFontCommand{\textlt}{\ltseries}
% heavy(bold) series
\DeclareRobustCommand\hbseries
  {\not@math@alphabet\hbseries\relax
   \fontseries\hbdefault\selectfont}
\newcommand{\hbdefault}{hb}
\DeclareTextFontCommand{\texthb}{\hbseries}
\makeatother

%==================================================
%  TITLE PAGE STUFF
%==================================================

\usepgflibrary{decorations.text} % LATEX and plain TEX and pure pgf
\usepgflibrary{decorations.footprints} % LATEX and plain TEX and pure pgf
\usepgflibrary{fadings} % LATEX and plain TEX and pure pgf
\usetikzlibrary{fadings}
\usepgflibrary{patterns} % LATEX and plain TEX and pure pgf
\usetikzlibrary{patterns}
\usepgflibrary{decorations.footprints} % LATEX and plain TEX and pure pgf
\usetikzlibrary{decorations.footprints} % LATEX and plain TEX when using Tik Z
\usepgflibrary{decorations.fractals} % LATEX and plain TEX and pure pgf
\usetikzlibrary{decorations.fractals} % LATEX and plain TEX when using Tik Z
\usepgflibrary{shapes.symbols} % LATEX and plain TEX and pure pgf
\usetikzlibrary{shapes.symbols} % LATEX and plain TEX when using Tik Z
\usetikzlibrary{shadows} % LATEX and plain TEX when using Tik Z

% enumerate settings
\setlist{itemsep=6pt}	%topsep=0.1em
%\renewcommand{\labelenumi}{ (\alph{enumi}) }			% makes enumeration (a) instead of (a).

\usepackage{placeins} % enables \FloatBarrier, useful for positioning figures/tables more precisely.
%\usepackage[dvips]{graphics}
%\usetikzlibrary{arrows}

\usepackage{pgfplots}

\tikzset{>=stealth}
\usetikzlibrary{backgrounds,arrows}
\tikzset{tight background}
%\usetikzlibrary{arrows.meta} 										%Ross Different
\pgfplotsset{every axis/.append style={
					width=7cm,
					height=7cm,
					line width=0.75pt,
					tick label style={font=\tiny},
					label style={font=\small},
					legend style={font=\tiny}
}}

\pgfplotsset{framed/.style={axis background/.style ={draw=black!75}}}

\pgfplotsset{framedaxes/.style={axis background/.style ={draw=black!75},axis x line=middle,axis y line=middle}}



%\usepackage{url}
\hypersetup{colorlinks=true,urlcolor=blue,linkcolor=black,breaklinks=true}        %for screen PDF
\usepackage{minitoc} % mini table-of-contents
\usepackage{xcolor}

%%%%   My colors for screen/color printing
 \definecolor{first}{rgb}{0.08,0.08,0.8}     %blue
 \definecolor{second}{rgb}{0,.55,0.05}       %green
 \definecolor{third}{rgb}{0.5,0.08,.7}       %purple
 \definecolor{fourth}{rgb}{0.9,0.1,0.9}      %pink
 \definecolor{fifth}{rgb}{1, 0.5, 0.1}  %orange

%%%% My colors for BW printing
% \definecolor{first}{rgb}{0,0,0}
% \definecolor{second}{rgb}{0.125,0.125,0.125}
% \definecolor{third}{rgb}{0.25,0.25,0.25}
% \definecolor{fourth}{rgb}{0.375,0.375,0.375}
% \definecolor{fifth}{rgb}{0.5,0.5,0.5}
% \hypersetup{colorlinks=true,urlcolor=black,linkcolor=black,breaklinks=true}        %for printed PDF


\newcommand{\cred}[0]{\color{red}}
\newcommand{\corange}[0]{\color{orange}}
\newcommand{\cgreen}[0]{\color{green}}
\newcommand{\cblue}[0]{\color{blue}}
\newcommand{\cpurple}[0]{\color{purple}}
\newcommand{\cbrown}[0]{\color{brown}}
\newcommand{\cgray}[0]{\color{gray}}

\setlength{\parindent}{0pt}

\date{}








%%%% Scot's Theorems

% example
\theoremstyle{break}
\theorembodyfont{\color{black}}
\theoremsymbol{}
%\theoremprework{{\blue\bigskip\needspace{\baselineskip}\hrule}}
\theorempostwork{}
\theoremseparator{:}
\shadecolor{blue}
%\newshadedtheorem{myexample}{EXAMPLE}[section]
\newtheorem{myExample}{\black Example}



% definitions
\theoremstyle{nonumberbreak}
\theorembodyfont{\color{black}}
\theoremsymbol{}
%\theoremprework{{\blue\bigskip\needspace{\baselineskip}\hrule}}
\theorempostwork{}
\theoremseparator{:}
\shadecolor{blue}
%\newshadedtheorem{myexample}{EXAMPLE}[section]
\newtheorem{myDefinitions}{\black Definitions and Vocabulary}



\newcommand{\firstpage}{~\hfill\thepage\hfill~}



%%%%%%%%%%%%%%%%%%%%




% preparation
\theoremstyle{break}
\theorembodyfont{\color{black}}
\theoremsymbol{}
\theoremprework{\vspace{1mm}}					
%\theorempostwork{}
\theoremseparator{:}
\shadecolor{blue}
%\newshadedtheorem{myexample}{EXAMPLE}[section]
\newtheorem{myPrep}{\black Preparation}

% exit
\theoremstyle{break}
\theorembodyfont{\color{black}}
\theoremsymbol{}
\theoremprework{\vspace{1mm}}					
\theorempostwork{}
\theoremseparator{:}
\shadecolor{blue}
%\newshadedtheorem{myexample}{EXAMPLE}[section]
\newtheorem{myExit}{\black Exit}

% activity
\theoremstyle{break}
\theorembodyfont{\color{black}}
\theoremsymbol{}
\theoremprework{\vspace{1mm}}					
%\theorempostwork{}
\theoremseparator{:}
\shadecolor{blue}
%\newshadedtheorem{myexample}{EXAMPLE}[section]
\newtheorem{myActivity}{\black Activity}

% exercise
\theoremstyle{break}
\theorembodyfont{\color{black}}
\theoremsymbol{}
\theoremprework{\vspace{1mm}}					
%\theorempostwork{}
\theoremseparator{:}
\shadecolor{blue}
\newtheorem{myExercise}{\black Exercise}

% definition
\theoremstyle{empty}
\theorembodyfont{\color{black}}
\theoremsymbol{}
%\theoremprework{\vspace{1mm}}					 
%\theoremprework{{\blue\bigskip\needspace{\baselineskip}\hrule}}
%\theorempostwork{\vspace{2.5mm}}
\theoremseparator{:}
\shadecolor{blue}
%\newshadedtheorem{myexample}{EXAMPLE}[section]
\newtheorem{myDefinition}{\black Definition}

% practice
\theoremstyle{break}
\theorembodyfont{\color{black}}
\theoremsymbol{}
\theoremprework{\vspace{1mm}}
%\theorempostwork{}
\theoremseparator{:}
\shadecolor{blue}
%\newshadedtheorem{myexample}{EXAMPLE}[section]
\newtheorem{myPractice}{\black Practice}

% framed thing
\theoremstyle{nonumberbreak}
\theorembodyfont{\color{black}}
\theoremsymbol{}
%\theoremprework{{\blue\bigskip\needspace{\baselineskip}\hrule}}
\theorempostwork{}
\theoremseparator{}
\shadecolor{blue}
%\newshadedtheorem{myexample}{EXAMPLE}[section]
\newframedtheorem{myFramed}{\black }

% properties
\theoremstyle{nonumberbreak}
\theorembodyfont{\color{black}}
\theoremsymbol{}
\theoremprework{\vspace{-3mm}}
\theorempostwork{\vspace{-4mm}}
\theoremseparator{:}
\shadecolor{blue}
%\newshadedtheorem{myexample}{EXAMPLE}[section]
\newtheorem{myProperties}{\black Properties}

% note
\theoremstyle{nonumberbreak}
\theorembodyfont{}
\theoremsymbol{}
%\theorempostwork{\color{blue}\hrule\bigskip}
\theoremprework{\vspace{-0.1cm}}
%\theorempostwork{\blue\hrule\needspace{\baselineskip}}
\theoremseparator{:}
%\shadecolor{green}
%\newframedtheorem{myProof}{Solution}
\newtheorem{myNote}{\black Important Note}

\def\theoremframecommand{%
      \psshadowbox[fillstyle=solid,fillcolor=yellow!30,linecolor=black]}
\newshadedtheorem{myshadedDefinition}{}
%\newframedtheorem{myDefinition}{}

\newcommand{\ph}{\phantom}
\newcommand{\ds}{\displaystyle}
\onehalfspace

\newcommand{\resetCounters}{
\setcounter{myPrep}{0}
\setcounter{myPractice}{0}
\setcounter{myActivity}{0}
\setcounter{myExit}{0}
\setcounter{myExercise}{0}
\setcounter{myExample}{0}
}

\newcommand{\fakesubsection}[1]{%
  \par\refstepcounter{subsection} % Increase section counter
  \subsectionmark{#1} % Add section mark (header)
%  \addcontentsline{toc}{}{\protect\numberline{\thesection}#1} % Add section to ToC
%  \addcontentsline{toc}{subsection}{\protect\numberline{\thesubsection}#1} % Add subsection to ToC
  % Add more content here, if needed.
}

\newcommand{\fakesection}[1]{%
  \par\refstepcounter{section} % Increase section counter
  \sectionmark{#1} % Add section mark (header)
  \addcontentsline{toc}{section}{\protect\numberline{\thesection}#1} % Add section to ToC
  % Add more content here, if needed.
}
 
\newcommand{\fakechapter}[1]{%
  \par\refstepcounter{chapter} % Increase section counter
  \chaptermark{#1} % Add section mark (header)
  \addcontentsline{toc}{chapter}{\protect\numberline{}#1} % Add section to ToC
  % Add more content here, if needed.
}


\newcommand{\exitlikert}[1]{%
On a scale of 1 - 5, how are you feeling with the concepts related to #1?\\
Place a mark on the scale below to indicate your overall comfort with the topics we saw in this section.
\begin{center}
 \includegraphics[scale=0.05]{worried} \begin{tabular}{ccccc}
 &  &  &  & \\
\hline
\hspace{1mm}1\hspace{4mm}  & \hspace{4mm}2\hspace{4mm} & \hspace{4mm}3\hspace{4mm} & \hspace{4mm}4\hspace{4mm} & \hspace{4mm}5\hspace{1mm} 
\end{tabular} \includegraphics[scale=0.05]{grin}
\end{center}
}

\newcommand{\defexample}{
{\textit{Example}}:
}

\newcommand{\fixthis}{
 {\color{red}{\Large \bf Fix something here.}}}

\newcommand{\textbook}[1]{ %
{\bf Textbook Reference:} This relates to content in \S#1 of {\it Algebra and Trigonometry 2e}.\\[-4mm]}

\newcommand{\textbooks}[2]{ %
{\bf Textbook Reference:} This relates to content in \S#1 and \S#2 of {\it Algebra and Trigonometry 2e}.\\[-4mm]}


\newcommand{\ignore}{
 {\color{red}{\Large \bf -- Ignore this section for now.}}}



\newcommand*{\titleRFscot}{\begingroup% Robert Frost, T&H p 149
\fontfamily{ptm}\selectfont 
%\setmainfont{Asana-Math} % FontSite Bergamo (Bembo)
\drop = 0.2\textheight
\centering
~
\vfill
\rule{0.8\textwidth}{0.8pt}\\[\baselineskip]
{\HUGE \bf Lab Manual and Lecture Notes}\\[0.14\drop]
{\Huge \bf for}\\[0.14\drop]
{\HUGE \bf Scot Leavitt's MTH 111 Course}\\[0.04\drop]
\rule{0.8\textwidth}{0.8pt}\\[\baselineskip]
%{\Large \plogo}\\[0.5\baselineskip]
%{\large\scshape year}\par
\vfill
%  {\color{red}{\Large \bf Draft: \today }}  \vfill

\vfill
{\large Revised February 2024\\ }
\endgroup}






\newcommand*{\titleRF}{\begingroup% Robert Frost, T&H p 149
\fontfamily{ptm}\selectfont 
%\setmainfont{Asana-Math} % FontSite Bergamo (Bembo)
\drop = 0.2\textheight
\centering
~
\vfill
\rule{0.7\textwidth}{0.8pt}\\[\baselineskip]
{\HUGE \bf MTH 111 Lab Manual}\\[0.04\drop]
\rule{0.7\textwidth}{0.8pt}\\[\baselineskip]
%{\Large \plogo}\\[0.5\baselineskip]
%{\large\scshape year}\par
\vfill
%  {\color{red}{\Large \bf Draft: \today }}  \vfill

\vfill
{\large Prepared by \\
Portland Community College\\
Mathematics Department \\
Revised February 2024\\ }
\endgroup}

%==================================================
% 				BEGIN DOCUMENT
%==================================================

\begin{document}

%\setlist[enumerate]{noitemsep,topsep=-10pt,itemsep=-5pt}

\fancyhf{}
\fancyfoot[LE,RO]{ \scriptsize{ Page \thepage} }
\fancyfoot[RE,LO]{\scriptsize{ \leftmark }}
\fancyhead[LE,RO]{\scriptsize{ \rightmark }}
\fancyhead[RE,LO]{\scriptsize{ MTH 111 Lab Manual }}
\renewcommand{\headrulewidth}{0.4pt} % Removes header line if 0pt
\fancyfootoffset[LE,LO]{0in}         % Moves center ??
\renewcommand{\footrulewidth}{0.4pt} % Removes header line if 0ptes header line if 0pt
\fancyfootoffset[LE,LO]{0in}         %Moves center ??
\renewcommand{\footrulewidth}{0.4pt} % Removes header line if 0pt

%==================================================
%           Title/ Info
%==================================================

\thispagestyle{empty}
\titleRFscot

%==================================================
\newpage
%==================================================

\dominitoc

\tableofcontents
\thispagestyle{empty}
\singlespacing

%==================================================
\newpage
%==================================================

\setcounter{page}{1}

\renewcommand{\labelenumi}{\alph{enumi})}


%%%%%%%%%%%%%
%\iffalse
%%%%%%%%%%%%%

% Preface
 %============================================================
% MTH 111 Project - Template File
% Chapter 0 - Preface
%	Updated August 2023
%============================================================


%\fakesection{Functions and Function Notation} \label{functions-and-notation}

\setcounter{chapter}{-1} 
%\fakechapter{Preface}
\chapter{Preface}  \label{chapter-preface}


This lab manual was designed with the intent of being used in the following manner.  In each section:
\setlist{itemsep=6pt}
\begin{itemize}
	\item Students will complete the Preparation Exercises before receiving instruction on the content.  Some instructors will have students do this before coming to class, while others might do it as a warmup in class.
	\item There will be some instructor-led presentation of the content.  This may be a formal class lecture, a discovery activity, a video lecture, or something else.
	\item Students will then engage in Practice Exercises in a group setting to reinforce their initial understanding of the foundational concepts.
	\item An instructor will then assign one or more group activities.  Because many of these activities are web-based and instructors can choose to use different activities, we have not included any of the possible activities in this document. 
	\item The Definitions are meant to provide a single location for all definitions and some key concepts.  Students can use this as a resource after having covered the topics in class.
	\item Students will complete some or all of the Exit Exercises, as decided by their instructor, to summarize their understanding of key concepts.
	\item In general, not every section of this lab manual will take the same amount of class time.  Some sections might be half-day topics, while others could be two-day topics.	
	\item Each instructor will identify for their class which exercises will be submitted as part of the course grade.
\end{itemize}
~

Course Resource Links:\\
{\it Algebra and Trigonometry 2e}:  \href{http://tiny.cc/111Z-Textbook}{http://tiny.cc/111Z-Textbook}\\
MTH 111 Supplement: \href{http://tiny.cc/111Z-Supplement}{http://tiny.cc/111Z-Supplement}

~



The editors would like to thank the following people for their contributions to this document:\\
Amy Cakebread, Bethany Downs, Wendy Fresh, Dave Froemke, Peter Haberman, Shane Horner, Ross Kouzes, Scot Leavitt, Emily Nelson, Kim Neuburger, Emily O'Sullivan, Jeff Pettit, Dennis Reynolds, Rebecca Ross, Heiko Spoddeck, Greta Swanson, Stephanie Yurasits



\begin{quote}

% Scot Leavitt
% Ross Kouzes
% Shane Horner
% 

\end{quote}


\newpage
~

%%%%%%%%%%%%%%%%%
%\fi
%%%%%%%%%%%%%%%%%








%%%%%%%%%%%%%
%\iffalse
%%%%%%%%%%%%%
% Ch 3 of OER
 %============================================================
% MTH 111Z Project - Template File
% Chapter 3.1 from OpenStax OER
%	Updated 202302
%============================================================


%\fakesection{Functions and Function Notation} \label{functions-and-notation}

\setcounter{chapter}{0} 
\chapter{Functions} \label{chapter-functions}

\minitoc

\newpage

~ %intentional blank page
% Do we include Damian's suggestion on a check list of topics or reference to CCOG topics on the second/back page of the chapter page?

\iffalse

\setcounter{section}{-1}
\section{Learning Objectives}

By the end of Section 1.1, you should be able to:
\setlist{itemsep=0pt}
\begin{itemize}
	\item know the definition of a function
	\item identify a function from a graph, table, set of ordered pairs, and equation
	\item evaluate a function for a given input
	\item solve an equation for a given output
\end{itemize}

By the end of Section 1.2, you should be able to:
\setlist{itemsep=0pt}
\begin{itemize}
	\item calculate the average rate of change for a function on a closed interval
	\item compute the difference quotient for a linear, quadratic, square root, or rational function
\end{itemize}


By the end of Section 1.3, you should be able to:
\begin{itemize}
	\item identify the domain of a function given as a graph, table, set of ordered pair, and equation
	\item identify the range of a function given as a graph, table, and set of ordered pair
	\item identify when a function is increasing, decreasing, or is constant
	\item identify when a function is positive or negative
	\item identify local and absolute minimum and maximum points and values.
\end{itemize}





\setlist{itemsep=6pt}

\fi


\resetCounters
 %============================================================
% MTH 111Z Project - Template File
% Section 3.1, 3.2 from OpenStax OER
%	Updated 202302
%============================================================

\section{Functions and Function Notation} \label{functions-and-notation}

In this section, we'll develop our understanding of functions and function notation, whether the function is presented as a set of ordered pairs, a table of values, a graph, or an equation.  We'll also learn how to evaluate a function given an input value and to solve for an input given a function's output value.\\[0.5em]
\textbooks{3.1}{3.2}

\subsection*{Preparation Exercises} \label{prep-functions-and-notation}


\begin{myPrep}
Many of us have at least one restaurant we would love to go to for a meal.  \\
Go online and find a menu for a restaurant you'd choose to eat at.
	\begin{enumerate}
		\item What is the name of the restaurant and what type of food do they serve?
		\vfill
		\item What is a dish you would like to get and how much does it cost?
		\vfill
		\item Are there any other dishes on the menu that costs the same as your dish? \\
		If so, what are they?
		\vfill
		\item If I ask you the price of any specific dish from the menu, do you know how much it costs? 
		\vfill
		\item If you know how much I paid for a dish, do you know exactly what I ordered based just on the price?  Explain your answer.
		\vfill
		\vfill
	\end{enumerate}
\end{myPrep}






%======================================================
\newpage
%======================================================





\subsection*{Practice Exercises} \label{practice-functions-and-notation}

\begin{myPractice}
Determine if each of the following relations show $y$ as a function of $x$.  \\
Explain your reasoning by referencing the definition of a function.  If a relation is not a function, provide a specific example why the definition was not satisfied.\\
Assume all ordered pairs are of the form $(x,y)$.

	\begin{enumerate}
		\item  $\{ (\text{red},\text{pepper}), ~ (\text{green},\text{pear}), ~ (\text{purple},\text{grape}), ~ (\text{orange},\text{orange}), ~ (\text{yellow},\text{pepper}), ~ (\text{red},\text{onion})\}$
		\vfill
		\item	
		\begin{minipage}{0.1\linewidth}
		\captionof{table}{}
		\end{minipage}~\\[-0.8em]
			\renewcommand\arraystretch{1.5}
			\begin{tabular}{|c|cccccc|}
			\hline
			$x$ & ~$-5$~ & ~$3$~ & ~$-1$~ & ~$2$~ & ~$4$~ & ~$6$~\\
			\hline
			$y$ & $9$ & ~$-4$~ & $2$ & ~$-4$~ & ~$-5$~ & ~$-6$~\\
			\hline
			\end{tabular}		
			\label{tab:fan-prex1}
		\vfill
		\begin{multicols}{2}
			\item
				\begin{minipage}{0.15\linewidth}
				\captionof{figure}{}
				\end{minipage}~\\[-0.8em]
				\begin{tikzpicture}
			 		\begin{axis}[
		 				framedaxes,
		 				width=5.5cm,height=5.5cm,
		 				xlabel={$x$},
		 				ylabel={$y$},
		 				xmin=-6,xmax=6,
		 				ymin=-6,ymax=6,
						xtick={-4,-2,...,4},
							minor xtick={-7,-5,...,7},
						ytick={-4,-2,...,4},
							minor ytick={-7,-5,...,7},
					         grid=both
		 				]
 				% use TeX as calculator:
%	 					\addplot[smooth,mark=*,first,line width=1.0pt,fill=white]coordinates{	(-4,2)	};
%	 					\addplot[smooth,mark=*,first,line width=1.0pt]coordinates{	(5,-4)	};
		 				\addplot[first,line width=1.5pt,samples=200,<->]expression[domain=-4.826:4.59036]{0.05*(x+4)*(x+2)*(x-1)*(x-4)};
			 		\end{axis}
		 		\end{tikzpicture}
				\label{fig:fan-prex1}
			\item	
				\begin{minipage}{0.15\linewidth}
				\captionof{figure}{}
				\end{minipage}~\\[-0.8em]
				\begin{tikzpicture}
			 		\begin{axis}[
		 				framedaxes,
		 				width=5.5cm,height=5.5cm,
		 				xlabel={$x$},
		 				ylabel={$y$},
		 				xmin=-6,xmax=6,
		 				ymin=-6,ymax=6,
						xtick={-4,-2,...,4},
							minor xtick={-7,-5,...,7},
						ytick={-4,-2,...,4},
							minor ytick={-7,-5,...,7},
					         grid=both
		 				]
 				% use TeX as calculator:
%	 					\addplot[smooth,mark=*,first,line width=1.0pt,fill=white]coordinates{	(-4,2)	};
%	 					\addplot[smooth,mark=*,first,line width=1.0pt]coordinates{	(5,-4)	};
		 				\addplot[first,line width=1.5pt,samples=400,->]expression[domain=-1:6]{(x+1)^0.5+2};
		 				\addplot[first,line width=1.5pt,samples=400]expression[domain=-1:1]{-0.5*(x+1)^0.5+2};
		 				\addplot[first,line width=1.5pt,samples=400]expression[domain=1:3]{0.5*(-(x-3))^0.5+0.586};
		 				\addplot[first,line width=1.5pt,samples=400]expression[domain=0:3]{-0.5*(-(x-3))^0.5+0.586};
		 				\addplot[first,line width=1.5pt,samples=400]expression[domain=-3:0]{0.5*((x+3))^0.5-1.146};
		 				\addplot[first,line width=1.5pt,->,samples=400]expression[domain=-3:6.01]{-1*((x+3))^0.5-1.146};
						\addplot[first,line width=1.5pt,-,forget plot] coordinates {(-1,1.98) (-1,2.02)};
						\addplot[first,line width=1.5pt,-,forget plot] coordinates {(3,0.566) (3,0.606)};
						\addplot[first,line width=1.5pt,-,forget plot] coordinates {(-3,-1.166) (-3,-1.126)};
						
			 		\end{axis}
		 		\end{tikzpicture}
				\label{fig:fan-prex2}
		\end{multicols}
		\vfill
	\end{enumerate}
\end{myPractice}


%======================================================
\newpage
%======================================================


\begin{myPractice}
	\begin{enumerate}
		\item Let $y=g(x)$ be defined by the set of $(x,y)$ ordered pairs:\\
		 $\{ (-60,5), ~ (-23,4), ~ (-4,3), ~ (3,2), ~ (4,1), ~ (5,0), ~ (12,-1)\}$.
			\begin{enumerate}[label=\roman*.]
				\begin{multicols}{2}
				\item Find $g(5)$.
				\item Solve $g(x)=3$.
				\end{multicols}
			\end{enumerate}
		\vfill

		\item Let $y=h(x)$ be defined by the graph in Figure~\ref{fig:fan-ex1} below.
		
		\begin{minipage}{0.4\linewidth}
					\captionof{figure}{$y=h(x)$}~\\[-0.8em]
					\begin{tikzpicture}
			 		\begin{axis}[
		 				framedaxes,
		 				width=7cm,height=7cm,
		 				xlabel={$x$},
		 				ylabel={$y$},
		 				xmin=-8,xmax=8,
		 				ymin=-8,ymax=8,
						xtick={-6,-4,...,6},
							minor xtick={-7,-5,...,7},
						ytick={-6,-4,...,6},
							minor ytick={-7,-5,...,7},
					         grid=both
		 				]
 				% use TeX as calculator:
%	 					\addplot[smooth,mark=*,first,line width=1.0pt,fill=white]coordinates{	(-4,2)	};
%	 					\addplot[smooth,mark=*,first,line width=1.0pt]coordinates{	(5,-4)	};
						\addplot[first,line width=1.5pt,samples=400,-]expression[domain=-7:-5.99]{x+1};
						\addplot[first,smooth,mark=*,line width=1pt,fill=white]coordinates{	(-7,-6)	};
						\addplot[first,line width=1.5pt,samples=400,-]expression[domain=-6.01:-3.99]{2*(x)+7};
						\addplot[first,line width=1.5pt,samples=400,-]expression[domain=-4.01:-2.99]{(x)+3};
						\addplot[first,line width=1.5pt,samples=400,-]expression[domain=-3.01:-0.99]{-1/2*(x)-1.5};
						\addplot[first,line width=1.5pt,samples=400,-]expression[domain=-1.01:0]{2*(x)^2-3};
						\addplot[first,line width=1.5pt,samples=400,-]expression[domain=0:1.01]{-(x)^2-3};
						\addplot[first,line width=1.5pt,samples=400,-]expression[domain=0.99:2]{-(x-2)^2-3};
						\addplot[first,line width=1.5pt,samples=400,-]expression[domain=2:3.01]{2*(x-2)^2-3};
						\addplot[first,line width=1.5pt,samples=400,-]expression[domain=2.99:5.01]{(x-3)^2-1};
						\addplot[first,line width=1.5pt,samples=400,-]expression[domain=4.99:6]{3};
						\addplot[first,smooth,mark=*,line width=1.5pt]coordinates{	(6,3)	};
			 		\end{axis}
		 		\end{tikzpicture}
				\label{fig:fan-ex1}
		\end{minipage}
		\begin{minipage}{0.4\linewidth}
			\begin{enumerate}[label=\roman*.]
				\item Find $h(-4)$.\vspace{2cm}
				\item Solve $h(x)=-3$.\vspace{2cm}
			\end{enumerate}
		\end{minipage}
		\item Let $f(x) = x^2-3$.
		\begin{enumerate}[label=\roman*.]
			\begin{multicols}{2}
				\item Find $f(-4)$.
				\item Solve $f(x)=46$.
			\end{multicols}
		\end{enumerate}
		\vfill
		\vfill
	

	\end{enumerate}
\end{myPractice}



%======================================================
\newpage
%======================================================

\subsection*{Definitions} \label{def-functions-and-notation}

\begin{myDefinition}[{Relation:}]~\\[0.5mm]
A {\bf relation} is a set of $(x,y)$ ordered pairs.  \\[0.5em]
The variable $x$ is called the {\bf independent variable} or {\bf input variable}.  \\
Each individual $x$-value is referred to as an {\bf input} or {\bf input value}.\\[0.5em]
The variable $y$ is called the {\bf dependent variable} or {\bf output variable}.  \\
Each individual $y$-value is referred to as an {\bf output} or {\bf output value}. \\[0.5em]
\defexample The set $\{ (0,-2), ~ (1, -1), ~ (2, 0), ~ (1, -3),~ (3,-4) \}$ is a relation.

\end{myDefinition}

\begin{myDefinition}[Function:]~\\[0.5mm]
A {\bf function} is a relation where each possible input value is paired with {\it exactly one} output value.\\
We say, ``The output is a function of the input,'' and often write this algebraically as $y = f(x)$. \\[0.5em]
\defexample The set $\{ (0,-2), ~ (1, -1), ~ (4, 0), ~ (9,1),~ (16,2) \}$ is a function.\\
\defexample The set $\{ (0,-2), ~ (\boldsymbol{1}, -1), ~ (2, 0), ~ (\boldsymbol{1}, -3),~ (3,-4) \}$ is a relation, but {\it not} a function.

\end{myDefinition}

\begin{myDefinition}[Domain and Range:]~\\[0.5mm]
The {\bf domain} of a relation or function is the set of all possible input values.\\
The {\bf range} of a relation or function is the set of all possible output values.\\[0.5em]
\defexample Given the function $\{ (0,-2), ~ (4, 0), ~ (1, -1), ~ (9,1),~ (16,2) \}$,\\
the domain of the function is $\{ 0,\,4,\,1,\,9,\,16\}$ and the range of the function is $\{ -2,\,0,\,-1,\,1,\,2\}$.

\end{myDefinition}


\begin{myDefinition}[Vertical Line Test:]~\\[0.5mm]
If a vertical line can be drawn that intersects the graph more than once, the graph is not the graph of a function with $x$ as the independent variable and $y$ as the dependent variable. \\[0.5em]
		\begin{minipage}{0.5\linewidth}
			\begin{center}
				\defexample
					\captionof{figure}{Does Not Pass}
					\begin{tikzpicture}
			 		\begin{axis}[
		 				framedaxes,
		 				width=5.5cm,height=5.5cm,
		 				xlabel={$x$},
		 				ylabel={$y$},
		 				xmin=-8,xmax=8,
		 				ymin=-8,ymax=8,
						xtick={-6,-4,...,6},
							minor xtick={-7,-5,...,7},
						ytick={-6,-4,...,6},
							minor ytick={-7,-5,...,7},
					         grid=both
		 				]
 				% use TeX as calculator:
%	 					\addplot[smooth,mark=*,first,line width=1.0pt,fill=white]coordinates{	(-4,2)	};
%	 					\addplot[smooth,mark=*,first,line width=1.0pt]coordinates{	(5,-4)	};
		 				\addplot[first,line width=1.5pt,samples=400,->]expression[domain=-2:8]{(x+2)^0.5+1};
		 				\addplot[first,line width=1.5pt,samples=400,->]expression[domain=-2:8]{-1*(x+2)^0.5+1};
						\addplot[first,line width=1.5pt,-,forget plot] coordinates {(-2,0.98) (-2,1.02)};
						\addplot[fifth,line width=0.75pt,<->] coordinates {(2,8) (2,-8)};
								\node at (axis cs:3.5,-7.5) {\color{fifth}\tiny{$x\,$=\,2}};
						\addplot[fifth,smooth,mark=*,line width=0.25pt]coordinates{	(2,3)	};
						\addplot[fifth,smooth,mark=*,line width=0.25pt]coordinates{	(2,-1)	};
			 		\end{axis}
		 		\end{tikzpicture}
				\label{fig:fan-def1}
			\end{center}
		\end{minipage}
		\begin{minipage}{0.5\linewidth}
			\begin{center}
				\defexample
					\captionof{figure}{Passes}
					\begin{tikzpicture}
			 		\begin{axis}[
		 				framedaxes,
		 				width=5.5cm,height=5.5cm,
		 				xlabel={$x$},
		 				ylabel={$y$},
		 				xmin=-8,xmax=8,
		 				ymin=-8,ymax=8,
						xtick={-6,-4,...,6},
							minor xtick={-7,-5,...,7},
						ytick={-6,-4,...,6},
							minor ytick={-7,-5,...,7},
					         grid=both
		 				]
 				% use TeX as calculator:
%	 					\addplot[smooth,mark=*,first,line width=1.0pt,fill=white]coordinates{	(-4,2)	};
%	 					\addplot[smooth,mark=*,first,line width=1.0pt]coordinates{	(5,-4)	};
		 				\addplot[first,line width=1.5pt,samples=400,->]expression[domain=-2:8]{(x+2)^0.5+1};
		 				\addplot[first,line width=1.5pt,samples=400,<-]expression[domain=-8:-2]{-1*(-x-2)^0.5+1};
						\addplot[first,line width=1.5pt,-,forget plot] coordinates {(-2,0.98) (-2,1.02)};
%						\addplot[myred,line width=1.0pt,dashed,<->] coordinates {(3,8) (3,-8)};
%								\node at (axis cs:3.25,-7.5) {\color{myred}\tiny{$x\,$=\,3}};
			 		\end{axis}
		 		\end{tikzpicture}
				\label{fig:fan-def2}
			\end{center}
		\end{minipage}
\end{myDefinition}






%======================================================
 \newpage
%======================================================

\subsection*{Exit Exercises} \label{exit-functions-and-notation}


\begin{myExit}
	\begin{enumerate}
		\item What is the definition of a function?  
		\vfill
		\item What are two examples of functions that you use in your daily life outside of school?  Explain how these are functions, referencing the definition of a function.
		\vfill
		\vfill
		\item What do you look for in the graph of a relation to determine if the graph is the graph of a function or not?  Fully explain your answer.
		\vfill
		\end{enumerate}
\end{myExit}

	\begin{multicols}{2}
\begin{myExit}
If $f(x) = 2x^2-7x$, evaluate $f(-3)$.
\end{myExit}
\begin{myExit}
If $g(x) = x^2+6x$, solve $g(x)=16$.
\end{myExit}
	\end{multicols}
		\vfill
		\vfill
		\vfill





\exitlikert{functions}








\newpage
~

\resetCounters
%============================================================
% MTH 111Z Project - Template File
% Section 3.2, 3.3 from OpenStax OER
%	Updated 202302
%============================================================

\section{Domain, Range, and Behaviors of Functions} \label{functions-domain-range-behavior}

In this section, we'll learn to identify the domain and range of functions given in various forms, as well as determine when a function exhibits important behaviors.\\[0.5em]
\textbooks{3.2}{3.3}

\subsection*{Preparation Exercises} \label{prep-functions-domain-range-behavior}

\begin{myPrep}
	\begin{enumerate}
		\item For any real number $k$ other than 0, what is $\dfrac{0}{k}$ and why?
		\vfill
		\item For any real number $k$ other than 0, what is $\dfrac{k}{0}$ and why?
		\vfill
		\item Given $f(x) = \sqrt{x+2}$, evaluate $f(23)$ and $f(-18)$. 
		\vfill
		\vfill
	\end{enumerate}
\end{myPrep}

\begin{myPrep}
	\begin{enumerate}
		\item ~\\ \vspace{-7mm}~\\ \begin{minipage}{0.9\linewidth}
Draw $x > -2$ on a number line and write the set of values in both interval and \\
		 set-builder notations.\\
		Hint: \href{http://tiny.cc/111Z-IntSetNotation}{Here is a review video} of these two notations. 
		\end{minipage}
		\begin{minipage}{0.1\linewidth}
		\flushright \qrcode[height=1cm]{https://tiny.cc/111Z-DomRang}
		\end{minipage}

		\vfill
		\item Draw $x\leq 6$ on a number line and write the set of values in both interval  and set-builder notations.
		\vfill
	\end{enumerate}
		

\end{myPrep}




%======================================================
\newpage
%======================================================

\subsection*{Practice Exercises} \label{practice-functions-domain-range-behavior}


\begin{myPractice}
Algebraically find the domain of the following functions.\\
State the domains in both interval notation and set-builder notation.
\begin{enumerate}
	\begin{multicols}{3}
	\item $f(x) = \sqrt{-2x+18}$
	\item $g(t) = \sqrt[3]{3t-24}$
	\item $h(k) = \dfrac{k+3}{k-9}$
	\end{multicols}
\vfill
\vfill
\end{enumerate}
\end{myPractice}

\begin{myPractice}
Find the domain and range of the function $p$ graphed in Figure~\ref{fig:drb-ex1}.\\
State both in interval notation and set-builder notation.\\

\begin{minipage}{0.3\linewidth}
	\begin{center}
	\captionof{figure}{$y=p(x)$}~\\[-0.8em]
	\begin{tikzpicture}
 		\begin{axis}[
 			framedaxes,
 			height=6cm,
 			width=6cm,
 			xlabel={$x$},
 			ylabel={$y$},
 			xmin=-8,xmax=8,
 			ymin=-8,ymax=8,
		        xtick={-6,-4,...,6},
		       	minor xtick={-9,-7,-5,...,7,9},
		        ytick={-6,-4,...,6},
	         	minor ytick={-7,-5,...,7},
		         grid=both
 			]
 		% use TeX as calculator:

		 	\addplot[first,line width=1.5pt,samples=200,-]expression[domain=-6:5]
				{0.1*(x+5)*(x+1)*(x-4)};
			\addplot[first,smooth,mark=*,line width=1pt,fill=white]coordinates{	(-6,-5)	};
			\addplot[first,smooth,mark=*,line width=1pt,fill=first]coordinates{	(5,6)	};
 		\end{axis}
 		\end{tikzpicture}
		\label{fig:drb-ex1}
	\end{center}
	\end{minipage}
	\vfill
\end{myPractice}


%======================================================
\newpage
%======================================================

\begin{myPractice}
Below are the graphs of $y=r(t)$ in Figure~\ref{fig:drb-ex2} and $y=s(t)$ in Figure~\ref{fig:drb-ex3}.\\

\begin{minipage}{0.5\linewidth}
	\begin{center}
	\captionof{figure}{$y=r(t)$}~\\[-0.8em]
	\begin{tikzpicture}
 		\begin{axis}[
 			framedaxes,
 			height=7cm,
 			width=7cm,
 			xlabel={$t$},
 			ylabel={$y$},
 			xmin=-8,xmax=8,
 			ymin=-8,ymax=8,
		        xtick={-6,-4,...,6},
		       	minor xtick={-11,-9,-7,...,11},
		        ytick={-6,-4,...,6},
	         	minor ytick={-7,-5,...,7},
		         grid=both
 			]
 		% use TeX as calculator:
				\addplot[first,line width=1.5pt,samples=400,-]expression[domain=-7:-4]{2/3*(x+4)^2+1};
					\addplot[first,smooth,mark=*,line width=2.0pt]coordinates{	(-7,7)	};
				\addplot[first,line width=1.5pt,samples=400,-]expression[domain=-4:-3]{1};
				\addplot[first,line width=1.5pt,samples=400,-]expression[domain=-3:-2]{(x+3)^2+1};
				\addplot[first,line width=1.5pt,samples=400,-]expression[domain=-2:0]{-(x+1)^2+3};
				\addplot[first,line width=1.5pt,samples=400,-]expression[domain=0:2]{-2*x+2};
				\addplot[first,line width=1.5pt,samples=400,-]expression[domain=2:4]{(x-3)^2-3};
				\addplot[first,line width=1.5pt,samples=400,-]expression[domain=4:6]{2*(x-4)-2};
					\addplot[first,smooth,mark=*,line width=1pt,fill=white]coordinates{	(6,2)	};
 		\end{axis}
 		\end{tikzpicture}
		\label{fig:drb-ex2}
	\end{center}
\end{minipage}
\begin{minipage}{0.5\linewidth}
	\begin{center}
	\captionof{figure}{$y=s(t)$}~\\[-0.8em]
	\begin{tikzpicture}
 		\begin{axis}[
 			framedaxes,
 			height=7cm,
 			width=7cm,
 			xlabel={$t$},
 			ylabel={$y$},
 			xmin=-8,xmax=8,
 			ymin=-8,ymax=8,
		        xtick={-6,-4,...,6},
		       	minor xtick={-11,-9,-7,...,11},
		        ytick={-6,-4,...,6},
	         	minor ytick={-7,-5,...,7},
		         grid=both
 			]
 		% use TeX as calculator:
				\addplot[first,line width=1.5pt,samples=400,-]expression[domain=-7:-5.99]{x+2};
					\addplot[first,smooth,mark=*,line width=1pt,fill=first]coordinates{	(-7,-5)	};
				\addplot[first,line width=1.5pt,samples=400,-]expression[domain=-6.01:-3.99]{2*(x)+8};
				\addplot[first,line width=1.5pt,samples=400,-]expression[domain=-4.01:-2.97]{(x)+4};
				\addplot[first,line width=1.5pt,samples=400,-]expression[domain=-3.03:-0.99]{-1/2*(x)-0.5};
				\addplot[first,line width=1.5pt,samples=400,-]expression[domain=-1.01:0]{2*(x)^2-2};
				\addplot[first,line width=1.5pt,samples=400,-]expression[domain=0:1.025]{-(x)^2-2};
				\addplot[first,line width=1.5pt,samples=400,-]expression[domain=0.975:2]{-(x-2)^2-2};
				\addplot[first,line width=1.5pt,samples=400,-]expression[domain=2:3.01]{2*(x-2)^2-2};
				\addplot[first,line width=1.5pt,samples=400,-]expression[domain=2.965:5.02]{(x-3)^2};
				\addplot[first,line width=1.5pt,samples=400,-]expression[domain=4.97:6]{4};
					\addplot[first,smooth,mark=*,line width=1.5pt,fill=white]coordinates{	(6,4)	};
 		\end{axis}
 		\end{tikzpicture}
		\label{fig:drb-ex3}
	\end{center}
\end{minipage}
\begin{enumerate}
	\begin{multicols}{2}
		\item Over what intervals is $r$ increasing?
		\item Over what intervals is $s$ negative?
	\end{multicols}
	\vfill
	
	\begin{multicols}{2}
		\item What is the absolute maximum value of $r$?
		\item State any local minimum points of $s$.
	\end{multicols}
	\vfill
	
	\begin{multicols}{2}
		\item Over what intervals is $r$ constant?
		\item Over what intervals is $s$ decreasing?
	\end{multicols}
	\vfill
	
	\begin{multicols}{2}
		\item Over what interval is $r$ positive?
		\item What is the absolute minimum value of $s$?
	\end{multicols}
	\vfill
	
\end{enumerate}

\end{myPractice}




%======================================================
\newpage
%======================================================

\subsection*{Definitions} \label{def-functions-domain-range-behavior}

\begin{myDefinition}[Domain and Range:]~\\[0.5mm]
\begin{minipage}{0.9\linewidth}
The {\bf domain} of a function is the set of all possible input values for the function.\\
The {\bf range} of a function is the set of all possible output values for the function.\\
The domain and range are commonly stated using interval notation or set-builder notation.\\[0.4em]
\defexample \href{https://tiny.cc/111Z-DomRang}{View this Desmos graph} to see an interactive example of these definitions.  %(url: tiny.cc/111Z-DomRang)
\end{minipage}
\begin{minipage}{0.1\linewidth}
\flushright \qrcode[height=1cm]{https://tiny.cc/111Z-DomRang}
\end{minipage}
\end{myDefinition}


\begin{myDefinition}[Positive and Negative:]~\\[0.5mm]
\begin{minipage}{0.9\linewidth}
A function $f$ is {\bf positive} if the output values are greater than 0.  $f$ is positive when $f(x)>0$.\\
A function $f$ is {\bf negative} if the output values are less than 0. $f$ is negative  when $f(x)<0$.\\[0.4em]
\defexample \href{https://tiny.cc/111Z-PosNeg}{View this Desmos graph} to see an interactive example of these definitions.  %(url: tiny.cc/111Z-PosNeg)
\end{minipage}
\begin{minipage}{0.1\linewidth}
\flushright \qrcode[height=1cm]{https://tiny.cc/111Z-PosNeg}
\end{minipage}

\end{myDefinition}

\begin{myDefinition}[Increasing, Decreasing, and Constant:]~\\[0.5mm]
\begin{minipage}{0.9\linewidth}
Let $f$ be a function that is defined on an open interval $I$, with $a$ and $b$ in $I$ and $b>a$.  \\


$f$ is {\bf increasing} on $I$ if $f(b)>f(a)$ for all $a$ and $b$ in $I$.\\
In other words, as you move left-to-right on the interval $I$, your $y$-values increase.\\

$f$ is {\bf decreasing} on $I$ if $f(b)<f(a)$ for all $a$ and $b$ in $I$.\\
In other words, as you move left-to-right on the interval $I$, your $y$-values decrease.\\

$f$ is {\bf constant} on $I$ if $f(b)=f(a)$ for all $a$ and $b$ in $I$.\\
In other words, as you move left-to-right on the interval $I$, your $y$-values do not change.\\[0.4em]
\defexample \href{https://tiny.cc/111Z-IncDec}{View this Desmos graph} to see an interactive example of these definitions. %(url: tiny.cc/111Z-IncDec)
\end{minipage}
\begin{minipage}{0.1\linewidth}
\flushright \qrcode[height=1cm]{https://tiny.cc/111Z-IncDec}
\end{minipage}
\end{myDefinition}


\begin{myDefinition}[Local Minimum or Maximum:]~\\[0.5mm]
Given a function $f$ that is defined on an open interval $I$, with $c$ in $I$.  \\
\begin{minipage}{0.6\linewidth}
$f$ has a {\bf local maximum} at $x=c$ if $f(c) \geq f(x)$ for all $x$ in $I$.\\
The {\bf local maximum value} of $f$ is the output $f(c)$.\\

$f$ has a {\bf local minimum} at $x=c$ if $f(c) \leq f(x)$ for all $x$ in $I$.\\
The {\bf local minimum value} of $f$ is the output $f(c)$.\\[0.4em]

\defexample In Figure~\ref{fig:drb-def1}, the function has two local minimum points and one local maximum point.\\
The local minimum value of about $0.9$ occurs at $x=-3$.\\
The local minimum value of about $-4.3$ occurs at $x=2$.\\
The local maximum value of about $3.5$ occurs at $x=-1$.
\end{minipage}
\begin{minipage}{0.4\linewidth}
		\vspace{1cm}
		\begin{center}
			\defexample			
			\captionof{figure}{Local Extrema}
			\begin{tikzpicture}
		 		\begin{axis}[
		 			framedaxes,
					width=5.5cm,height=5.5cm,
		 			xlabel={$x$},
		 			ylabel={$y$},
		 			xmin=-8,xmax=8,
					ymin=-8,ymax=8,
					xtick={-6,-4,...,6},
						minor xtick={-7,-5,...,7},
					ytick={-6,-4,-2,0,2,6},
						minor ytick={-7,...,7},
				        grid=both
		 			]
 				% use TeX as calculator:
		 			\addplot[first,line width=1.5pt,samples=200,-]expression[domain=-4:3]{1/8*x^4+1/3*x^3-5/4*x^2-3*x+2};
					\addplot[first,smooth,mark=*,line width=0.05pt]coordinates{(-4,4.667)	};
					\addplot[first,smooth,mark=*,line width=0.25pt]coordinates{(-3,0.875)	};
						\node at (axis cs:-5.3,0.5) {\color{first}\tiny{local min}};
					\addplot[first,smooth,mark=*,line width=0.25pt]coordinates{(-1,3.542)	};
						\node at (axis cs:-1.1,4.892) {\color{first}\tiny{local}};
						\node at (axis cs:-1.1,4.242) {\color{first}\tiny{max}};
					\addplot[first,smooth,mark=*,line width=0.25pt]coordinates{(2,-4.333)	};
						\node at (axis cs:4.5,-4.5) {\color{first}\tiny{local min}};
					\addplot[first,smooth,mark=*,line width=0.05pt]coordinates{(3,0.8735)	};
		 		\end{axis}
	 		\end{tikzpicture}
			\label{fig:drb-def1}
		\end{center}
	\end{minipage}
\end{myDefinition}

\newpage

\begin{myDefinition}[Absolute Minimum or Maximum:]~\\[0.5mm]
\begin{minipage}{0.6\linewidth}
$f$ has an {\bf absolute maximum} at $x=c$ if $f(c) \geq f(x)$ for all $x$ in the domain of $f$.\\
The {\bf absolute maximum value} of $f$ is the output $f(c)$.\\

$f$ has an {\bf absolute minimum} at $x=c$ if $f(c) \leq f(x)$ for all $x$ in the domain of $f$.\\
The {\bf absolute minimum value} of $f$ is the output $f(c)$.\\[0.4em]

\defexample In Figure~\ref{fig:drb-def2}, the function has an absolute minimum point and an absolute maximum point.\\
The absolute minimum value of about $-4.3$ occurs at $x=2$.\\
The absolute maximum value of about $4.8$ occurs at $x=-4$.
\end{minipage}
\begin{minipage}{0.4\linewidth}
		\vspace{1cm}
		\begin{center}
			\defexample			
			\captionof{figure}{Absolute Extrema}
			\begin{tikzpicture}
		 		\begin{axis}[
		 			framedaxes,
					width=5.5cm,height=5.5cm,
		 			xlabel={$x$},
		 			ylabel={$y$},
		 			xmin=-8,xmax=8,
					ymin=-8,ymax=8,
					xtick={-6,-4,...,6},
						minor xtick={-7,-5,...,7},
					ytick={-6,-4,-2,0,2,6},
						minor ytick={-7,...,7},
				        grid=both
		 			]
 				% use TeX as calculator:
		 			\addplot[first,line width=1.5pt,samples=200,-]expression[domain=-4:3]{1/8*x^4+1/3*x^3-5/4*x^2-3*x+2};
					\addplot[first,smooth,mark=*,line width=0.25pt]coordinates{(-4,4.667)	};
						\node at (axis cs:-4,5.8) {\color{first}\tiny{absolute}};
						\node at (axis cs:-4,5.2) {\color{first}\tiny{max}};
					\addplot[first,smooth,mark=*,line width=0.25pt]coordinates{(3,0.8735)	};
					\addplot[first,smooth,mark=*,line width=0.25pt]coordinates{(2,-4.333)	};
						\node at (axis cs:5,-4.8) {\color{first}\tiny{absolute min}};
%					\addplot[first,smooth,mark=*,line width=0.25pt]coordinates{(-3,0.875)	};
%						\node at (axis cs:-5.3,0.5) {\color{first}\tiny{local min}};
%					\addplot[first,smooth,mark=*,line width=0.25pt]coordinates{(-1,3.542)	};
%						\node at (axis cs:-1.1,4.892) {\color{first}\tiny{local}};
%						\node at (axis cs:-1.1,4.242) {\color{first}\tiny{max}};
%						\node at (axis cs:4.5,-4.5) {\color{first}\tiny{local min}};

%		 			\addplot[first,line width=1.5pt,samples=400,->]expression[domain=-2:8]{(x+2)^0.5+1};
%		 			\addplot[first,line width=1.5pt,samples=400,->]expression[domain=-2:8]{-1*(x+2)^0.5+1};
%					\addplot[first,line width=1.5pt,-,forget plot] coordinates {(-2,0.98) (-2,1.02)};
%					\addplot[red,line width=0.75pt,<->] coordinates {(2,8) (2,-8)};
%						\node at (axis cs:3.5,-7.5) {\color{red!60}\tiny{$x\,$=\,2}};
%					\addplot[red,smooth,mark=*,line width=0.25pt]coordinates{	(2,3)	};
%					\addplot[red,smooth,mark=*,line width=0.25pt]coordinates{	(2,-1)	};
		 		\end{axis}
	 		\end{tikzpicture}
			\label{fig:drb-def2}
	\end{center}
\end{minipage}
\end{myDefinition}




%======================================================
 \newpage
%======================================================

\subsection*{Exit Exercises} \label{exit-functions-domain-range-behavior}

\begin{myExit}
What is the domain of $f(x) = \sqrt{x}$?  What is the domain of $g(x)= \sqrt[3]{x}$?  Why are these domains different?
\end{myExit}

\vfill
\vfill
\vfill

\begin{myExit}
Graphically speaking, what is the difference between a function being negative and a function decreasing?
\end{myExit}

\vfill
\vfill
\vfill

\begin{myExit}
For the function $F$ graphed in Figure~\ref{fig:exit-functions-domain-range-behavior-1}, answer the following.
\begin{center}
	\captionof{figure}{$y=F(x)$}
	\label{fig:exit-functions-domain-range-behavior-1}
	\begin{tikzpicture}
 		\begin{axis}[
 			framedaxes,
 			height=4.7cm,
 			width=8cm,
 			xlabel={$x$},
 			ylabel={$y$},
 			xmin=-10,xmax=10,
 			ymin=-5,ymax=5,
		        xtick={-8,-6,...,8},
		       	minor xtick={-9,-7,-5,...,7,9},
		        ytick={-4,-2,...,6},
	         	minor ytick={-7,-5,...,7},
		         grid=both
 			]
 		% use TeX as calculator:
 			\addplot[first,line width=1.25pt,samples=200,<-]expression[domain=-8.871:-7]{-2*(x+7)^2+2};
 			\addplot[first,line width=1.25pt,samples=200,-]expression[domain=-7:-5]{-0.5*(x+7)^2+2};
 			\addplot[first,line width=1.25pt,samples=200,-]expression[domain=-5:-4]{(x+4)^2-1};
 			\addplot[first,line width=1.25pt,samples=200,-]expression[domain=-4:-2]{-1};
 			\addplot[first,line width=1.25pt,samples=200,-]expression[domain=-2:-1]{-1*(x+2)^2-1};
 			\addplot[first,line width=1.25pt,samples=200,-]expression[domain=-1:1]{(x)^2-3};
 			\addplot[first,line width=1.25pt,samples=200,-]expression[domain=1:3]{2*(x-1)-2};
 			\addplot[first,line width=1.25pt,samples=200,-]expression[domain=3:5]{-0.5*(x-5)^2+4};
 			\addplot[first,line width=1.25pt,samples=200,-]expression[domain=5:7]{-1*(x-5)^2+4};
 			\addplot[first,line width=1.25pt,samples=200,->]expression[domain=7:10]{4/(x-6)-4};
 		\end{axis}
 		\end{tikzpicture}
 \end{center}


\begin{enumerate}
\begin{multicols}{2}
	\item Over what intervals is $F$ increasing?
	\item What is the range of $F$?
\end{multicols}
	\vfill
\begin{multicols}{2}
	\item Over what intervals is $F$ negative?
	\item What are any local minimum points on $F$?
\end{multicols}
	\vfill
\begin{multicols}{2}
	\item Over what intervals is $F$ constant?
	\item What is the absolute maximum value of $F$?
\end{multicols}
	\vfill
\end{enumerate}
\end{myExit}



\exitlikert{the graphical behaviors of functions}




\newpage
~



\resetCounters
 %============================================================
% MTH 111Z Project - Template File
% Section 3.3 from OpenStax OER
%	Updated 202302
%============================================================

\section{Average Rates of Change and The Difference Quotient} \label{functions-difference-quotient}

In this section, we'll learn how to calculate the average rate of change of a function's output between two specific inputs, as well as evaluate the difference quotient, which is a general form of the average rate of change for a function.\\[0.5em]
\textbook{3.3}




\subsection*{Preparation Exercises} \label{prep-difference-quotient}

\begin{myPrep}
Suppose you're driving south on I-5 in Oregon and you pass mile marker 294 in Portland at 1:35 PM.  Later, you pass mile marker 194 in Eugene at 3:05 PM. 
	\begin{enumerate}
		\item What was your average speed (in miles per hour) of the trip from Portland to Eugene?
		\vfill
		\item What was your speed at any particular moment, say as you drove past mile marker 256 in Salem?
		\vfill
	\end{enumerate}
\end{myPrep}

\begin{myPrep}
Let $f(x) = x^2 - 2x$.  Evaluate and simplify the following:
	\begin{enumerate}
		\begin{multicols}{2}
		\item $f(4)$
		\item $f(-6)$
		\end{multicols}
		\vfill
		\begin{multicols}{2}
		\item $f(a)$
		\item $f(a+h)$
		\end{multicols}
		\vfill
	\end{enumerate}
\end{myPrep}




%======================================================
\newpage
%======================================================


\subsection*{Practice Exercises} \label{practice-difference-quotient}

\begin{myPractice}
\begin{enumerate}

\item The function $m$ in Table~\ref{table:aroc-ex1} below shows the cost of movie tickets\footnote{Costs obtained from \url{https://www.natoonline.org/data/ticket-price} } in the U.S. in the year $t$.


%https://www.natoonline.org/data/ticket-price
	\begin{minipage}{0.3\linewidth}
		\renewcommand\arraystretch{1.5}
		\captionof{table}{Price of Movie \\Tickets in the U.S.}
		\begin{tabular}{c|c}
			$t$ & $m(t)$\\
			(year) & (in dollars)\\
			\hline
			1995 & 4.35\\
			1999 & 5.06\\
			2003 & 6.03\\
			2009 & 7.50\\
			2013 & 8.13\\
			2017 & 8.97\\
			2021 & 10.17
		\end{tabular}
		\label{table:aroc-ex1}
	\end{minipage}
	\begin{minipage}{0.7\linewidth}
%		\begin{enumerate}[label=\roman*.]
%			\item What is the average rate of change in the price of a movie ticket from 1995 to 2003? \\[3cm]
%			\item 
\begin{enumerate}
\item	What is the unit of the average rate of change in the price of a movie over any time period? \\[2cm]
\item	What is the average rate of change in the price of a movie ticket from 2003 to 2021? \\[4cm]
\end{enumerate}
%		\end{enumerate}
		\end{minipage}
~\\~\\~\\
	
	\item Given $f(n) = \frac{1}{3}n^2-1$, find the average rate of change of $f$ on the interval $[3,9]$.
	\vfill
	
\end{enumerate}
\end{myPractice}




%======================================================
\newpage
%======================================================


\begin{myPractice}
Find and simplify the difference quotient for each of the following functions.
\begin{enumerate}

\item $f(x) = -6x + 8$
\vfill

\item $g(x) = -2x^2 - 5x$
\vfill


\end{enumerate}
\end{myPractice}

%======================================================
\newpage
%======================================================

\subsection*{Definitions} \label{def-difference-quotient}


\begin{myDefinition}[Rate of Change:]~\\[0.5mm]
A {\bf rate of change} describes how the output values change in relation to a change in the input values.  \\
The unit for the rate of change is ``output unit(s) per input unit.''
\end{myDefinition}

\begin{myDefinition}[Average Rate of Change:]~\\[0.5mm]
The {\bf average rate of change} for a function $f$ between two input values $x_1$ and $x_2$ is the difference in their output values divided by the difference in the two input values.  The average rate of change is calculated using the formula
\[\text{average rate of change} = \dfrac{f(x_2)-f(x_1)}{x_2-x_1}, ~ x_1 \neq x_2\]
The average rate of change is the slope of the line between the two points $(x_1, f(x_1))$ and $(x_2, f(x_2))$.\\[0.5em]
\defexample The function $E(x)$ gives the cost of a dozen eggs (in dollars) $x$ year after 2010.  If we know $E(19) = 1.362$ and $E(23)=2.666$, we can find average rate of change as 
$$\begin{aligned}
\dfrac{E(23)-E(19)}{23\text{ years}-19\text{ years}} &= \dfrac{ \$2.666 - \$1.362 }{4 \text{ years}}\\[0.5em]
&= \dfrac{ \$1.304 }{4\text{ years}}\\[0.5em]
&\approx \$0.33/\text{year}
\end{aligned}$$
This shows that between 2019 and 2023, the cost of a dozen eggs increased on average by about \$0.33/year.
%https://fred.stlouisfed.org/series/APU0000708111
\end{myDefinition}

\begin{myDefinition}[Difference Quotient:]~\\[0.5mm]
The {\bf difference quotient} for a function $f$ is given by the formula
\[ \dfrac{f(x+h)-f(x)}{h}, ~ h\neq0\]  
The difference quotient is the average rate of change between the two points $(x,f(x))$ and $(x+h, f(x+h))$.\\[0.5em]
\defexample Given the function $f(x) = 3x^2-4x$, the difference quotient would be evaluated as 
$$\begin{aligned}
\dfrac{\boldsymbol{f(x+h)}-f(x)}{h} &= \dfrac{\boldsymbol{3(x+h)^2-4(x+h)} - \left(3x^2-4x\right)}{h}\\[0.5em]
&= \dfrac{\boldsymbol{3x^2 + 6xh + 3h^2 -4x -4h}  - 3x^2+4x}{h}\\[0.5em]
&=  \dfrac{6xh + 3h^2 -4h }{h}\\[0.5em]
&=  \dfrac{h(6x + 3h -4) }{h}\\[0.5em]
&=  6x + 3h -4,  \text{ for $h\neq0$}
\end{aligned}$$
\end{myDefinition}


%======================================================
 \newpage
%======================================================

\subsection*{Exit Exercises} \label{exit-difference-quotient}


\begin{myExit}
\begin{enumerate}
		\item What are two situations in your daily life that involve an average rate of change?  \\
		What are the units for these rates of change?
		\vfill
		\vfill
		\item What is the formula for the difference quotient for the function $k$ that has an input variable $p$?
		\vfill
		\vfill
		\item If you have a function $m$ that gives the price of a gallon of milk in the year $t$, what would be the unit for the average rate of change for $m$?
		\vfill
		\vfill
		\vfill
		\vfill
		\end{enumerate}
\end{myExit}

\newpage

\begin{myExit}
	Find and simplify the difference quotient for the function $f(t) = \dfrac{3}{t-6}$.
	\vfill
\end{myExit}


\exitlikert{the difference quotient}








\resetCounters
%============================================================
% MTH 111Z Project - Template File
% Part of Section 3.2 from OpenStax OER
%	Updated 202302
%============================================================


\section{Piecewise-Defined Functions} \label{functions-piecewise}

In this section, we'll explore piecewise-defined functions, which are functions constructed from pieces of several other functions.  We'll find the domain and range of these types of functions, as well as graph, evaluate, solve them.  And given the graph of a piecewise-defined function, we'll construct the formula for the graph's function. \\[0.5em]
\textbook{3.2}




\subsection*{Preparation Exercises} \label{prep-functions-piecewise}

\begin{myPrep}
A family event charges \$4/person, with a maximum of \$20 for any single family.  
	\begin{enumerate}
		\item How much will a family of three pay?
		\vfill
		\item How much will a family of seven pay?
		\vfill
		\item At what number of people does the calculation change from being per person to a single charge for the whole family?
		\vfill
		\vfill
	\end{enumerate}
\end{myPrep}

\begin{myPrep}
In November 2022, Portland General Electric set the rates for basic residential service as a function of the number of kilowatt-hours (kWh) of energy used.  The rates in November 2022 were \$0.0642/kWh when up to 1000 kWh (kilowatt-hours) are used and if greater than 1000 kWh are used, then the first 1000 kWh are billed at the \$0.0642/kWh rate and \$0.07002/kWh is charged for the energy usage greater than the initial 1000 kWh.  
	\begin{enumerate}
		\item What is the cost for using 740 kWh?
		\vfill
		\item What is the cost for using 1320 kWh?
		\vfill
		\item What is a formula to find the cost for using $x$ kWh if $x$ is greater than 1000 kWh?
		\vfill
		\vfill
	\end{enumerate}
\end{myPrep}




%======================================================
\newpage
%======================================================


\subsection*{Practice Exercises} \label{practice-functions-piecewise}

\begin{myPractice}
~\\[-5mm]
	\begin{minipage}{0.5\linewidth}
		Let $f(x)=
			\begin{cases}
				x^2-4	&	\textrm{if}\ 	\ \	-2\leq x<3		\\
				\frac{2}{3}x-1	&	\textrm{if}\ 	\ \	x\geq 3		\\
			\end{cases}
		$
		\begin{enumerate}
			\item Evaluate $f(5)$.\\[12mm]
			\item What is the domain of $f$?\\[12mm]
			\item Graph $y=f(x)$ in Figure~\ref{fig:pw-ex1}.
		\end{enumerate}
	\end{minipage}
	\begin{minipage}{0.5\linewidth}
	\begin{center}
	\captionof{figure}{$y=f(x)$}~\\[-0.8em]
	\begin{tikzpicture}
 		\begin{axis}[
 			framed,
 			height=7cm,
 			width=7cm,
 			xlabel={},
 			ylabel={},
 			xmin=-8,xmax=8,
 			ymin=-8,ymax=8,
		        xtick={-16,16},
		       	minor xtick={-17,...,17},
		        ytick={-16,16},
	         	minor ytick={-17,...,17},
		         grid=both
 			]
 		% use TeX as calculator:
 		\end{axis}
 		\end{tikzpicture}
		\label{fig:pw-ex1}
	\end{center}

	\end{minipage}
\end{myPractice}
~\\[10mm]

\begin{myPractice}
The graph of $y=g(x)$ is in Figure~\ref{fig:pw-ex2}.\\

	\begin{minipage}{0.4\linewidth}
	\begin{center}
	\captionof{figure}{$y=g(x)$}~\\[-0.8em]
	\begin{tikzpicture}
 		\begin{axis}[
 			framedaxes,
 			height=7cm,
 			width=7cm,
 			xlabel={$x$},
 			ylabel={$y$},
 			xmin=-8,xmax=8,
 			ymin=-8,ymax=8,
		        xtick={-6,-4,...,6},
		       	minor xtick={-11,-9,-7,...,11},
		        ytick={-6,-4,...,6},
	         	minor ytick={-7,-5,...,7},
		         grid=both
 			]
 		% use TeX as calculator:
				\addplot[first,line width=1.5pt,samples=400,<-]expression[domain=-8:-3]{2/3*(x+3)+1};
					\addplot[first,smooth,mark=*,line width=1pt,fill=first]coordinates{	(-3,1)	};
					\addplot[first,smooth,mark=*,line width=1pt,fill=white]coordinates{	(-3,-4)	};
				\addplot[first,line width=1.5pt,samples=400,-]expression[domain=-3:1]{-4};
					\addplot[first,smooth,mark=*,line width=1pt,fill=white]coordinates{	(1,-4)	};
					\addplot[first,smooth,mark=*,line width=1pt,fill=first]coordinates{	(2,3)	};
				\addplot[first,line width=1.5pt,samples=400,->]expression[domain=2:7.5]{-2*(x-2)+3};
 		\end{axis}
 		\end{tikzpicture}
		\label{fig:pw-ex2}
	\end{center}
	\end{minipage}
	\begin{minipage}{0.6\linewidth}
	\begin{enumerate}
		\begin{multicols}{2}
			\item Evaluate $g(-2)$.
			\item Solve $g(x)=-1$.
		\end{multicols}
		~\\
		\begin{multicols}{2}
		\item What is the domain of $g$? 
		\item What is the range of $g$? 
		\end{multicols}
		~\\
		\item State the formula for the function $g$.\\[10mm]
	\end{enumerate}
	\end{minipage}
\end{myPractice}
\vfill



%======================================================
\newpage
%======================================================

\subsection*{Definitions} \label{def-functions-piecewise}

\begin{myDefinition}[Piecewise-Defined Function:]~\\[0.5mm]
A {\bf piecewise-defined function} is a function which uses different formulas for calculating the output on different intervals of its domain.  Each formula is used on a distinct interval of the domain.  \\

The notation we use to write a piecewise-defined function is:
\[
	f(x)=
	\begin{cases}
		\text{formula \#1}	&	\textrm{if}\ 	\ \	\text{$x$ is in this part of the domain of $f$}		\\
		\text{formula \#2}	&	\textrm{if}\ 	\ \	\text{$x$ is in this other part of the domain of $f$}		\\
		\text{etc.}			&	\textrm{if}\ 	\ \	\text{etc.}		\\
	\end{cases}
\]

The domain of the function is the union of the intervals used by the separate formulas.\\[0.5em]
\defexample \\
\begin{minipage}{0.5\linewidth}
The function $$f(x)=
			\begin{cases}
				-\frac{1}{2}(x+3)^2+4	&	\textrm{if}\ 	\ \	-5< x\leq1		\\
				\sqrt{x-1}+3	&	\textrm{if}\ 	\ \	x> 1		\\
			\end{cases}
		$$
is a piece-wise defined function. \\[0.75em]
Its graph is in Figure~\ref{fig:pw-def1} to the right. \\[0.75em]
Its domain is $(-5, \infty)$ and its range is $[-4,\infty)$.		
\end{minipage}
\begin{minipage}{0.5\linewidth}
	\begin{center}
			\captionof{figure}{$y=f(x)$}~\\[-0.8em]
			\begin{tikzpicture}
 				\begin{axis}[
 					framedaxes,
 					height=5.5cm,
 					width=5.5cm,
		 			xlabel={$x$},
 					ylabel={$y$},
 					xmin=-8,xmax=8,
 					ymin=-8,ymax=8,
				        xtick={-6,-4,...,6},
				       	minor xtick={-11,-9,-7,...,11},
				        ytick={-6,-4,...,6},
	         			minor ytick={-7,-5,...,7},
		      		   grid=both
 					]
		 		% use TeX as calculator:
				\addplot[first,line width=1.5pt,samples=400,-]expression[domain=-5:1]{-0.5*(x+3)^2+4};
					\addplot[first,smooth,mark=*,line width=1pt,fill=white]coordinates{	(-5,2)	};
					\addplot[first,smooth,mark=*,line width=1pt,fill=first]coordinates{	(1,-4)	};
				\addplot[first,line width=1.5pt,samples=400,->]expression[domain=-1:8]{(x-1)^0.5+3};
					\addplot[first,smooth,mark=*,line width=1pt,fill=white]coordinates{	(1,3)	};
		 		\end{axis}
		 		\end{tikzpicture}
				\label{fig:pw-def1}
			\end{center}
\end{minipage}
\end{myDefinition}

%Note: Dennis prefers ``expression'' instead of ``formula''.








%======================================================
 \newpage
%======================================================

\subsection*{Exit Exercises } \label{exit-functions-piecewise}

%%%%%%%%
\iffalse 
\begin{myExit}
Use $F(x)=
			\begin{cases}
				-\frac{1}{2}x^2+1	&	\textrm{if}\ 	\ \	 x\leq2		\\
				-\frac{1}{2}x+5	&	\textrm{if}\ 	\ \	x> 4		\\
			\end{cases}$ to answer the following.
	\begin{enumerate}
	\begin{minipage}{0.5\linewidth}
		\item Graph $y=F(x)$ in Figure~\ref{fig:pw-exit1} below.
		\begin{center}
			\captionof{figure}{$y=F(x)$}
			\begin{tikzpicture}
 				\begin{axis}[
 					framed,
 					height=7cm,
 					width=7cm,
		 			xlabel={$x$},
 					ylabel={$y$},
 					xmin=-8,xmax=8,
 					ymin=-8,ymax=8,
				        xtick={-6,-4,...,6},
				       	minor xtick={-11,-9,-7,...,11},
				        ytick={-6,-4,...,6},
	         			minor ytick={-7,-5,...,7},
		      		   grid=both
 					]
		 		% use TeX as calculator:
		 		\end{axis}
		 		\end{tikzpicture}
				\label{fig:pw-exit1}
			\end{center}
	\end{minipage}
	\begin{minipage}{0.1\linewidth}
	\end{minipage}
	\begin{minipage}{0.4\linewidth}
		\item Evaluate $F(-3)$.\\[15mm]
		\item Solve $F(x) = 1$ from your graph.\\[15mm]
	\end{minipage}		
	
	\end{enumerate}
\end{myExit}

\fi
%%%%%%%%

\begin{myExit}
Use $F(x)=
			\begin{cases}
				-\frac{1}{2}x^2+1	&	\textrm{if}\ 	\ \	 x\leq2		\\
				-\frac{1}{2}x+5	&	\textrm{if}\ 	\ \	x> 4		\\
			\end{cases}$ to answer the following.
	\begin{enumerate}
	\begin{minipage}{0.5\linewidth}
		\item Evaluate $F(8)$.
	\end{minipage}
	\begin{minipage}{0.5\linewidth}
		\item Evaluate $F(-6)$.
	\end{minipage}			
	\end{enumerate}
\vfill
\vfill
\end{myExit}


\begin{myExit}
In November 2022, Portland General Electric set the rates for basic residential service to be \$0.0642/kWh for up to 1000 kWh used and then \$0.07002/kWh for any usage greater than 1000 kWh.  Find a piecewise-defined function that gives the cost of electricity used (in dollars) as a function of x, the amount of kWh used.
\vfill
\vfill
\vfill
\end{myExit}



\begin{myExit}~\\[-5mm]
	\begin{minipage}{0.5\linewidth}
		$y=G(x)$ is graphed in Figure~\ref{fig:pw-exit2} below.
		\begin{center}
			\captionof{figure}{$y=G(x)$}
			\begin{tikzpicture}
 				\begin{axis}[
 					framedaxes,
 					height=7cm,
 					width=7cm,
		 			xlabel={$x$},
 					ylabel={$y$},
 					xmin=-8,xmax=8,
 					ymin=-8,ymax=8,
				        xtick={-6,-4,...,6},
				       	minor xtick={-11,-9,-7,...,11},
				        ytick={-6,-4,...,6},
	         			minor ytick={-7,-5,...,7},
		      		   grid=both
 					]
		 		% use TeX as calculator:
				\addplot[first,line width=1.5pt,samples=400,<-]expression[domain=-7.666:-4]{-3*(x+4)-3};
					\addplot[first,smooth,mark=*,line width=1pt,fill=first]coordinates{	(-4,-3)	};
					\addplot[first,smooth,mark=*,line width=1pt,fill=white]coordinates{	(-4,1)	};
				\addplot[first,line width=1.5pt,samples=400,-]expression[domain=-4:2]{3/4*(x+4)+1};
					\addplot[first,smooth,mark=*,line width=1pt,fill=white]coordinates{	(2,5.5)	};
					\addplot[first,smooth,mark=*,line width=1pt,fill=first]coordinates{	(3,3)	};
				\addplot[first,line width=1.5pt,samples=400,->]expression[domain=3:8]{-1*(x-3)+3};
		 		\end{axis}
		 		\end{tikzpicture}
				\label{fig:pw-exit2}
			\end{center}
			\vspace{1cm}
	\end{minipage}
	\begin{minipage}{0.1\linewidth}
	\end{minipage}
	\begin{minipage}{0.4\linewidth}
		\begin{enumerate}
			\item State the formula for the function $G$.\\[30mm]
			\item Evaluate $G(-4)$ from the graph.\\[12.5mm]
			\item Solve $G(x) = 0$ from the graph.\\[12.5mm]
		\end{enumerate}
	\end{minipage}		
	
\end{myExit}
\vfill




\exitlikert{piecewise-defined functions}













\resetCounters
%============================================================
% MTH 111Z Project - Template File
% Section 3.4 from OpenStax OER
%	Updated 202302
%============================================================

\section{Algebraic Combinations of Functions and Function Composition} \label{functions-algebra-and-composition}

In this section, we'll learn about several ways we can combine functions algebraically, as well as use the output from one function as the input for another.  As we've done before, we'll investigate these ideas with functions presented as graphs, as tables, and as formulas.  \\[0.5em]
\textbook{3.4}



\subsection*{Preparation Exercises } \label{prep-functions-algebra-and-composition}

\begin{myPrep}
Suppose it costs a bakery \$3,000 for rent and utilities each month and each loaf of bread costs \$2.15 to produce.  The bakery sells each loaf of bread for \$6.49.
	\begin{enumerate}
		\item Find a function $C$ that calculates the total cost (in dollars) each month to produce $x$ loaves of bread.
		\vfill
		\item Find a function $R$ that calculates the total revenue (in dollars) each month from selling $x$ loaves of bread.
		\vfill
		\item Find a function $P$ that calculates the profit (in dollars) from producing and selling $x$ loaves of bread each month.  (Note: The profit is the difference between the revenue and costs.)
		\vfill
	\end{enumerate}
\end{myPrep}

\begin{myPrep}
The function $r=f(t)=0.35t$ gives the radius (in inches) of the circular pattern formed when a drop of water hits a pond $t$ seconds after the drop of water hits the pond's surface.  The function $a = g(r) = \pi r^2$ gives the area of a circle (in square inches) when the circle has a radius of $r$ inches.\\[0.5em]
How would you determine the area of the circle 9 seconds after a water drop lands on the pond?
\vfill
\end{myPrep}







%======================================================
\newpage
%======================================================


\subsection*{Practice Exercises  } \label{practice-functions-algebra-and-composition}



\begin{myPractice}
Let $f(x) = x^2-4x$, $g(x) = \sqrt{3x+1}$, $h$ be defined by Table~\ref{tab:algebra-1}, and $k$ defined by Figure~\ref{fig:algebra-2}.\\

\begin{minipage}{0.5\linewidth}
	\begin{center}	
	\captionof{table}{$h(x)$}
	\renewcommand\arraystretch{1.5}
	\begin{tabular}{c|c}
	~$x$~ & ~$h(x)$~\\
	\hline
	~$-4$~ & ~$8$~\\
	~$-2$~ & ~$-1$~\\
	~$1$~ & ~$3$~\\
	~$2$~ & ~$6$~\\
	~$4$~ & ~$-7$~\\
	~$8$~ & ~$-5$~\\
	\end{tabular}		
	\label{tab:algebra-1}
	\end{center}
\end{minipage}
\begin{minipage}{0.5\linewidth}
	\begin{center}
	\captionof{figure}{$y=k(x)$}~\\[-0.8em]
	\begin{tikzpicture}
 		\begin{axis}[
 			framedaxes,
 			height=7cm,
 			width=7cm,
 			xlabel={$x$},
 			ylabel={$y$},
 			xmin=-8,xmax=8,
 			ymin=-8,ymax=8,
		        xtick={-6,-4,...,6},
		       	minor xtick={-11,-9,-7,...,11},
		        ytick={-6,-4,...,6},
	         	minor ytick={-7,-5,...,7},
		         grid=both
 			]
 		% use TeX as calculator:
				\addplot[first,line width=1.5pt,samples=400,<-]expression[domain=-8:-3]{2/3*(x+3)+1};
					\addplot[first,smooth,mark=*,line width=1pt,fill=first]coordinates{	(-3,1)	};
					\addplot[first,smooth,mark=*,line width=1pt,fill=white]coordinates{	(-3,-4)	};
				\addplot[first,line width=1.5pt,samples=400,-]expression[domain=-3:1]{-4};
					\addplot[first,smooth,mark=*,line width=1pt,fill=white]coordinates{	(1,-4)	};
					\addplot[first,smooth,mark=*,line width=1pt,fill=first]coordinates{	(2,3)	};
				\addplot[first,line width=1.5pt,samples=400,->]expression[domain=2:7.5]{-2*(x-2)+3};
 		\end{axis}
 		\end{tikzpicture}
		\label{fig:algebra-2}
	\end{center}
\end{minipage}

\begin{enumerate}
\item Evaluate the following:
	\begin{enumerate}[label=\roman*)]
	\begin{multicols}{2}
		\item $(f-k)(-3)$
		\item $(g\cdot h)(1)$
	\end{multicols}
		\vfill
	\end{enumerate}
\item Evaluate the following:
	\begin{enumerate}[label=\roman*)]
	\begin{multicols}{2}
		\item $(k\circ g)(5)$
		\item $(f\circ h)(8)$
	\end{multicols}
		\vfill
	\end{enumerate}
\end{enumerate}
\end{myPractice}

%======================================================
\newpage
%======================================================



\begin{myPractice}
\item Let $F(x)=\dfrac{x+5}{x-3}$, $G(x)=x^2-4$, and $H(x)=\sqrt{2x+19}$\\

 State the domain of the following functions:
	\begin{enumerate}
		\item $\dfrac{H}{G}$
		\vfill
		\item $F \circ H$
		\vfill
	\end{enumerate}
\end{myPractice}

%======================================================
\newpage
%======================================================

\subsection*{Definitions} \label{def-functions-algebra-and-composition}

\begin{myDefinition}[Algebraic Combinations of Functions]~\\[0.5mm]
Two functions $f$ and $g$ can be combined using the operations of addition, subtraction, multiplication, or division as follows:\\

$\boldsymbol{f+g}$ is defined for all values of $x$ in the domain of both $f$ and $g$ as \[(f+g)(x) = f(x)+g(x) \]

$\boldsymbol{f-g}$ is defined for all values of $x$ in the domain of both $f$ and $g$ as \[(f-g)(x) = f(x)-g(x) \]

$\boldsymbol{f\cdot g}$ is defined for all values of $x$ in the domain of both $f$ and $g$ as \[(f\cdot g)(x) = f(x) \cdot g(x) \]

$\boldsymbol{\dfrac{f}{g}}$ is defined for all values of $x$ in the domain of both $f$ and $g$ and where $g(x)\neq0$  as \[\left(\dfrac{f}{g}\right)(x) = \dfrac{f(x)}{g(x)} \]


\end{myDefinition}

\begin{myDefinition}[Function Composition]~\\[0.5mm]
The {\bf composition of functions}, $\boldsymbol{f\circ g}$ occurs when the output of one function $g$ is used as the input of another function $f$ and is defined as
\[ (f \circ g) (x) = f(g(x))\]
The domain of $f\circ g$ is all values of $x$ in the domain of $g$ where the values of $g(x)$ are in the domain of $f$.\\

Note that $f\cdot g$ is used for the product of two functions, while $f \circ g$ is used for composition.
\end{myDefinition}




%======================================================
 \newpage
%======================================================

\subsection*{Exit Exercises     } \label{exit-functions-algebra-and-composition}



\begin{myExit}
Answer the following in general for two functions $f$ and $g$.
	\begin{enumerate}
		\item What is meant by $(f + g)(6)$?  Explain both algebraically, as well as in written words.
		\vfill
		\item For two functions $f$ and $g$, how do you find the domain of $f + g$, $f-g$, or $f\cdot g$? \\
			What else do you need to consider for $\dfrac{f}{g}$?
		\vfill
		\item What is meant by $(f \circ g)(-4)$?  Explain both algebraically, as well as in written words.
		\vfill
		\item In general, does the order of composition matter? Does $(f \circ g)(x)$ yield the same thing as $(g \circ f)(x)$?
		\vfill
	\end{enumerate}
\end{myExit}

\begin{myExit}
Let $f(x) = x^2 + 7x$, $g(x) = \sqrt{5x-1}$, and $h(x) = \dfrac{x+1}{x-2} $.
	\begin{enumerate}
		\begin{multicols}{2}
			\item Evaluate $(g-f)(10)$.
			\item Evaluate $(h\circ f )(x)$.
		\end{multicols}
		\vfill
	\end{enumerate}
\end{myExit}





\exitlikert{function algebra and function composition}


















\resetCounters
%============================================================
% MTH 111 Project - Template File
% Section 3.5 from OpenStax OER
%	Updated August 2023
%============================================================

\section{Graph Transformations and Symmetry} \label{functions-transformations}

In this section, we'll look at ways we can algebraically manipulate the inputs and outputs of a function and see the impact this has on the overall function and its graph.  We'll also look at the symmetry that some functions have related and how this symmetry relates to some of the transformations we'll see.    \\[0.5em]
\textbook{3.5}



\subsection*{Preparation Exercises} \label{prep-functions-transformations}

\begin{myPrep}
Suppose the following graph represents the function $A$ which gives the air temperature (in $^\circ$F above $65^\circ$F) in a classroom $t$ hours after midnight each weekday.
\begin{center}
	\captionof{figure}{$y=A(t)$}
	\begin{tikzpicture}
 		\begin{axis}[
 			framedaxes,
			height=6cm,
			width=12cm,
 			xlabel={$t$},
% 			ylabel={$y$},
 			xmin=0,xmax=24,
 			ymin=-6,ymax=6,
		        xtick={0,2,...,24},
		       	minor xtick={0,1,...,24},
		        ytick={-4,-2,0,2,4},
	         	minor ytick={-7,-5,...,7},
		xlabel style={at={(ticklabel cs:0.5)},anchor=center, below=18mm},
	      	ylabel style={at={(ticklabel cs:0.5,-1)},rotate=90,anchor=center, above=0.25mm},
		xlabel={\tiny $t$, hours after midnight},
		ylabel={\tiny $y$, degrees above $65^\circ$F},
		         grid=both
 			]
 		% use TeX as calculator:
			\addplot[first,line width=1.5pt,samples=200]expression[domain=0:5]{-3};
			\addplot[first,line width=1.5pt,samples=200]expression[domain=5:7]{2.5*(x-5)-3};
			\addplot[first,line width=1.5pt,samples=200]expression[domain=7:20]{2};
			\addplot[first,line width=1.5pt,samples=200]expression[domain=20:24]{-5/4*(x-20)+2};
			\addplot[first,smooth,mark=*,line width=1pt,fill=first,only marks]coordinates{(0,-3) (5,-3) (7,2) (20,2) (24,-3)	};
 		\end{axis}
		\end{tikzpicture}
		\label{fig:transform-1}
	\end{center}
	\begin{enumerate}
		\item Let $W$ be a new function that has the same schedule as the room used by $A$, but is for a room that is always two degrees warmer than the room used by $A$.	  In Figure~\ref{fig:transform-1}, sketch the graph of the function $W$.
		\vfill
		\item What expression could we use for $W$ to represent the function used to create the graph for the room that is always two degrees warmer than the room used for $A$?
		\vfill
		\item If we were to vertically shift the graph of $y=A(t)$ down by one unit in the $y$ direction to create a new function $C$, what would that mean in the context of the temperature of the room and what expression could we use to represent $C$?
		\vfill
%		\item Suppose someone changes the schedule so that each temperature change occurs two hours earlier.  If the function $E$ represents this change, what expression would represent this new the function $E$?\\
%		\vfill
	\end{enumerate}
\end{myPrep}







%======================================================
\newpage
%======================================================


\subsection*{Practice Exercises} \label{practice-functions-transformations}

\begin{myPractice}
Let $f$ be a function.
\begin{enumerate}
%\item If $k_1(x) = f(-x)-4$, state the transformations that take the graph of $y=f(x)$ to the graph of $y=k_1(x)$.
%\vfill

\item If $k_1(x) = - f(x+3)$, state the transformations that take the graph of $y=f(x)$ to the graph of $y=k_1(x)$.
\vfill

%\item If $k_3(x) = \frac{1}{3} f(x-2)$, state the transformations that take the graph of $y=f(x)$ to the graph of $y=k_3(x)$.
%\vfill

\item If $k_2(x) =  f(\frac{1}{5}x)+3$, state the transformations that take the graph of $y=f(x)$ to the graph of $y=k_2(x)$.
\vfill
\end{enumerate}
\end{myPractice}

\begin{myPractice}
Let $g$ be a function for which we know that $g(3)=5$.
\begin{enumerate}
\item If $m(x) = -2g(x-4)+7$, state a sequence of transformations that takes the graph of $y=g(x)$ to the graph of $y=m(x)$?
\vfill
\vfill

\begin{multicols}{2}
\item What point do you know is on \\
the graph of $y=g(x)$?
\item What point do you know is on \\
the graph of $y=m(x)$?
\end{multicols}
\vfill

\end{enumerate}
\end{myPractice}

\newpage


\begin{myPractice}
Given in Figure~\ref{fig:prac-transform-1} is the graph of $y=f(x)$.  \\
Based on the graph, would you say that $f$ is even, odd, or neither? Explain your answer.\\
\begin{minipage}{0.4\linewidth}
	\begin{center}
	\captionof{figure}{$y=f(x)$}~\\[-0.8em]
	\begin{tikzpicture}
 		\begin{axis}[
 			framedaxes,
 			height=6cm,
 			width=6cm,
 			xlabel={$x$},
 			ylabel={$y$},
 			xmin=-8,xmax=8,
 			ymin=-8,ymax=8,
		        xtick={-6,-4,...,6},
		       	minor xtick={-11,-9,-7,...,11},
		        ytick={-6,-4,...,6},
	         	minor ytick={-7,-5,...,7},
		         grid=both
 			]
 		% use TeX as calculator:
				\addplot[first,line width=1.5pt,samples=400,<->]expression[domain=-7.9:7.9]{1/2*abs(x)*sin(180*x/4)};
%					\addplot[first,smooth,mark=*,line width=1pt,fill=first]coordinates{	(-3,1)	};
%					\addplot[first,smooth,mark=*,line width=1pt,fill=white]coordinates{	(-3,-4)	};
%				\addplot[first,line width=1.5pt,samples=400,-]expression[domain=-3:1]{-4};
%					\addplot[first,smooth,mark=*,line width=1pt,fill=white]coordinates{	(1,-4)	};
%					\addplot[first,smooth,mark=*,line width=1pt,fill=first]coordinates{	(2,3)	};
%				\addplot[first,line width=1.5pt,samples=400,->]expression[domain=2:7.5]{-2*(x-2)+3};
 		\end{axis}
 		\end{tikzpicture}
		\label{fig:prac-transform-1}
	\end{center}
\end{minipage}
~\\
\end{myPractice}


\begin{myPractice}
Algebraically determine if the function $g(x) = -2x^2-3x$ is even, odd, or neither.
\vfill
\vfill
\end{myPractice}



\begin{myPractice}
\begin{enumerate}
\item If $k_3(x) = 4\cdot f(6x-2)+5$, state the transformations that take the graph of $y=f(x)$ to the graph of $y=k_3(x)$.
\vfill

\item If $k_4(x) =  \frac{7}{3}\cdot f(\frac{1}{5}x-1)$, state the transformations that take the graph of $y=f(x)$ to the graph of $y=k_4(x)$.
\vfill
\end{enumerate}
\end{myPractice}




%======================================================
\newpage
%======================================================

\subsection*{Definitions} \label{def-functions-transformations}

\begin{myDefinition}[Vertical Shift:]~\\[0.5mm]
\begin{minipage}{0.9\linewidth}
Given a function $f$, the graph of  $g(x) = f(x) +k$ for some real number $k$ is a {\bf vertical shift} of the graph of $y=f(x)$.\\[2mm]
If $k>0$, $g$ will be the graph of $f$ shifted up by $k$ units.\\
If $k<0$, $g$ will be the graph of $f$ shifted down by $k$ units.\\[0.4em]
\defexample \href{https://tiny.cc/111Z-VertShift}{View this Desmos graph} to see an interactive example of the definition. % (url: tiny.cc/111Z-VertShift)
\end{minipage}
\begin{minipage}{0.1\linewidth}
\flushright \qrcode[height=1cm]{https://tiny.cc/111Z-VertShift}
\end{minipage}
\end{myDefinition}


\begin{myDefinition}[Horizontal Shift:]~\\[0.5mm]
\begin{minipage}{0.9\linewidth}
Given a function $f$, the graph of $g(x) = f(x-h)$ for some real number $h$ is a {\bf horizontal shift} of the graph of $y=f(x)$.\\[2mm]
If $h>0$, $g$ will be the graph of $f$ shifted right by $h$ units.\\
If $h<0$, $g$ will be the graph of $f$ shifted left by $h$ units.\\[0.4em]
\defexample \href{https://tiny.cc/111Z-HorizShift}{View this Desmos graph} to see an interactive example of the definition.  %(url: tiny.cc/111Z-HorizShift)
\end{minipage}
\begin{minipage}{0.1\linewidth}
\flushright \qrcode[height=1cm]{https://tiny.cc/111Z-HorizShift}
\end{minipage}
\end{myDefinition}


\begin{myDefinition}[Vertical Stretch/Compression:]~\\[0.5mm]
\begin{minipage}{0.9\linewidth}
Given a function $f$, the graph of $g(x)= a\cdot f(x) $ for some real number $a$, where $a\neq0$, is a {\bf vertical stretch} or a {\bf vertical compression} of the graph of $y=f(x)$.\\[2mm] 
If $a>1$, $g$ will be the graph of $f$ vertically stretched by a factor of $a$.\\
If $0<a<1$, $g$ will be the graph of $f$ vertically compressed by a factor of $a$.\\[2mm] 
If $a<0$, $g$ will be a combination of a vertical reflection over the $x$-axis {\it and} a vertical stretch or compression of $f$.\\[0.4em]
\defexample \href{https://tiny.cc/111Z-VertStretch}{View this Desmos graph} to see an interactive example of the definition. %(url: tiny.cc/111Z-VertStretch)
\end{minipage}
\begin{minipage}{0.1\linewidth}
\flushright \qrcode[height=1cm]{https://tiny.cc/111Z-VertStretch}
\end{minipage}
\end{myDefinition}



\begin{myDefinition}[Horizontal Stretch/Compression:]~\\[0.5mm]
\begin{minipage}{0.9\linewidth}
Given a function $f$, the graph of $g(x) = f(b\cdot x) $ for some real number $b$, where $b\neq0$, is a {\bf horizontal stretch} or a {\bf horizontal compression} of the graph of $y=f(x)$.\\[2mm] 
If $b>1$, $g$ will be the graph of $f$ horizontally compressed by a factor of $\frac{1}{b}$.\\
If $0<b<1$, $g$ will be the graph of $f$ horizontally stretched by a factor of $\frac{1}{b}$.\\[2mm]
If $b<0$, $g$ will be a combination of a horizontal reflection over the $y$-axis {\it and} a horizontal stretch or compression of $f$.\\[0.4em]
\defexample \href{https://tiny.cc/111Z-HorizStretch}{View this Desmos graph} to see an interactive example of the definition.  %(url: tiny.cc/111Z-HorizStretch)
\end{minipage}
\begin{minipage}{0.1\linewidth}
\flushright \qrcode[height=1cm]{https://tiny.cc/111Z-HorizStretch}
\end{minipage}
\end{myDefinition}

\newpage

\begin{myDefinition}[Vertical Reflection:]~\\[0.5mm]
\begin{minipage}{0.9\linewidth}
Given a function $f$, the graph of $y = -f(x)$ is a {\bf vertical reflection} of the graph of $y=f(x)$ over the $x$-axis.\\[0.4em]
\defexample \href{https://tiny.cc/111Z-VertReflect}{View this Desmos graph} to see an interactive example of the definition.  %(url: tiny.cc/111Z-VertReflect)
\end{minipage}
\begin{minipage}{0.1\linewidth}
\flushright \qrcode[height=1cm]{https://tiny.cc/111Z-VertReflect}
\end{minipage}
\end{myDefinition}


\begin{myDefinition}[Horizontal Reflection:]~\\[0.5mm]
\begin{minipage}{0.9\linewidth}
Given a function $f$, the graph of $y = f(-x)$ is a {\bf horizontal reflection} of the graph of $y=f(x)$ over the $y$-axis.\\[0.4em]
\defexample \href{https://tiny.cc/111Z-HorizReflect}{View this Desmos graph} to see an interactive example of the definition.  %(url: tiny.cc/111Z-HorizReflect)
\end{minipage}
\begin{minipage}{0.1\linewidth}
\flushright \qrcode[height=1cm]{https://tiny.cc/111Z-HorizReflect}
\end{minipage}
\end{myDefinition}


\begin{myDefinition}[Combined Transformations:]~\\[0.5mm]
Given a function $f$, the combined vertical transformations written in the form $y=a\cdot f(x)+k$, $a\neq 0$, would be applied in the order:
\setlist{itemsep=1pt}
\begin{itemize}
	\item a vertical reflection over the $x$-axis, if $a<0$
	\item a vertical stretch or compression by a factor of $|a|$
	\item a vertical shift up or down by $k$ units
\end{itemize}

Given a function $f$, the combined horizontal transformations written in the form $ y=f\left(b\cdot (x-h)\right)$, $b\neq 0$, would be applied in the order:
\setlist{itemsep=0pt}
\begin{itemize}
	\item a horizontal reflection over the $y$-axis, if $b<0$
	\item a horizontal stretch or compression by a factor of $\left|\frac{1}{b}\right|$
	\item a horizontal shift left or right by $h$ units
\end{itemize}

Given a function $f$, the combined horizontal transformations written in the form $y= f(bx-h)$, $b\neq 0$, would be applied in the order:
\setlist{itemsep=0pt}
\begin{itemize}
	\item a horizontal shift left or right by $h$ units
	\item a horizontal reflection over the $y$-axis, if $b<0$
	\item a horizontal stretch or compression by a factor of $\left|\frac{1}{b}\right|$
\end{itemize}
\end{myDefinition}


\begin{myDefinition}[Even Function:]~\\[0.5mm]
\begin{minipage}{0.9\linewidth}
Given a function $f$, if $f(-x)=f(x)$ for every input $x$, then $f$ is an  {\bf  even function}.\\
We describe even functions as being symmetrical about the $y$-axis.\\[0.4em]
\defexample \href{https://tiny.cc/111Z-EvenFunction}{View this Desmos graph} to see an interactive example of the definition.%  (url: tiny.cc/111Z-EvenFunction)
\end{minipage}
\begin{minipage}{0.1\linewidth}
\flushright \qrcode[height=1cm]{https://tiny.cc/111Z-EvenFunction}
\end{minipage}
\end{myDefinition}


\begin{myDefinition}[Odd Function:]~\\[0.5mm]
\begin{minipage}{0.9\linewidth}
Given a function $f$, if $ -f(-x)=f(x)$ for every input $x$, then $f$ is an {\bf odd function}.\\
We describe odd functions as being symmetrical about the origin.\\
Note: $f(x) = -f(-x)$ is equivalent to the statement $f(-x) = -f(x)$.\\[0.4em]
\defexample \href{https://tiny.cc/111Z-OddFunction}{View this Desmos graph} to see an interactive example of the definition. % (url: tiny.cc/111Z-OddFunction)
\end{minipage}
\begin{minipage}{0.1\linewidth}
\flushright \qrcode[height=1cm]{https://tiny.cc/111Z-OddFunction}
\end{minipage}
\end{myDefinition}





%======================================================
 \newpage
%======================================================

\subsection*{Exit Exercises} \label{exit-functions-transformations}




\begin{myExit}
	\begin{enumerate}
		\item What is meant by an ``inside'' change?  How do inside changes impact the graph of a function?
		\vfill
		\item What is meant by an ``outside'' change?  How do outside changes impact the graph of a function?
		\vfill
		\item What is the relationship between even and odd functions and transformations?
		\vfill
	\end{enumerate}
\end{myExit}


\begin{myExit}
$f$ is a function and $f(-20) = 32$.
	\begin{enumerate}
		\item If $g(x) = -\frac{1}{8} f(-2x+10)+5$, list the sequence of transformations that take the graph of $y=f(x)$ to the graph of $y=g(x)$.
		\vfill
		\vfill
		\item What point is on the graph of $y=f(x)$ and what point will be on the graph of $y=g(x)$?
		\vfill
	\end{enumerate}
\end{myExit}




\exitlikert{function transformations}





\newpage
~





\resetCounters
%============================================================
% MTH 111Z Project - Template File
% Section 3.7 from OpenStax OER
%	Updated 202302
%============================================================

\section{Inverse Functions} \label{functions-inverse}

In this section, we'll see what happens if you turn a function inside-out and make the output become the input and the input become the output.  We'll also explore when doing this will result in a function and what it means if it does.   \\[0.5em]
\textbook{3.7}


\subsection*{Preparation Exercises} \label{prep-functions-inverse}

\begin{myPrep}
Consider the following set of ordered pairs:\\
$$ \{ (\text{Alaska}, 1), ~ (\text{Washington},10), ~ (\text{Oregon},6), ~ (\text{Idaho}, 2),~ (\text{Nevada}, 4),~ (\text{Hawaii}, 2),~ (\text{California}, 52)\} $$
\begin{enumerate}
	\item Does this set represent a function?  Answer by referencing the definition of a function.
	\vfill
	\item If we were to swap the $x$ and $y$-values, would this new set be a function? Explain your answer.\\ 
$ \{ (1, \text{Alaska}), ~ (10, \text{Washington}), ~ (6, \text{Oregon}), ~ (2, \text{Idaho}),~ (4, \text{Nevada}),~ (2, \text{Hawaii}),~ (52, \text{California})\} $
	\vfill
\end{enumerate}
\end{myPrep}

\begin{myPrep}
The formula to convert a temperature $F$ in degrees Fahrenheit to a temperature $C$ in degrees Celsius is given by the function
\[ C = g(F) = \frac{5}{9}(F-32)  \]

\begin{enumerate}
	\item For each temperature in degrees Fahrenheit, how many temperatures in degrees Celsius are produced by this formula?
	\vfill
	\item For each temperature in degrees Celsius, how many temperatures in degrees Fahrenheit are produced by this formula?
	\vfill
	\item Would it be true to state that $C$ is a function of $F$ and also that $F$ is a function of $C$?  Why or why not?
	\vfill

\end{enumerate}
\end{myPrep}






%======================================================
\newpage
%======================================================


\subsection*{Practice Exercises} \label{practice-functions-inverse}

\begin{myPractice}
$f$ is defined in Table~\ref{tab:invprac1}.
	\begin{center}
			\renewcommand{\arraystretch}{1.75}
		\captionof{table}{}
		\label{tab:invprac1}
	%	\begin{tabular}{| m{1cm} | c | c | c | c | c |}
\begin{tabular}{|>{\centering\arraybackslash}m{1cm}| >{\centering\arraybackslash}m{1cm} >{\centering\arraybackslash}m{1cm} >{\centering\arraybackslash}m{1cm} >{\centering\arraybackslash}m{1cm} >{\centering\arraybackslash}m{1cm}|>{\centering\arraybackslash}m{1cm}|}		\hline
				$x$ 		&	-4	& 	-2	&	 0	&	2	&	 4	\ \\ \hline
				$f(x)$	&	4	&	2	&	-2	&	-4	&	0			\\ \hline
		\end{tabular} 
		\end{center}
\vspace{11pt}

Evaluate the following.
\begin{multicols}{3}
		\begin{enumerate}\setlength{\itemsep}{0.75in}
			\item	$f(0)$
		
			\item $f^{-1}(0)$
		
			\item	$f^{-1}(-4)$ 
		
		\end{enumerate}
		
\end{multicols}
\end{myPractice}
~\\

\begin{myPractice}
$y=g(x)$ is defined in Figure~\ref{fig:invprac1}.\\
\begin{minipage}{0.4\linewidth}
	\begin{center}
	\captionof{figure}{$y=f(x)$}~\\[-0.8em]
	\begin{tikzpicture}
 		\begin{axis}[
 			framedaxes,
 			height=7cm,
 			width=7cm,
 			xlabel={$x$},
 			ylabel={$y$},
 			xmin=-8,xmax=8,
 			ymin=-8,ymax=8,
		        xtick={-6,-4,...,6},
		       	minor xtick={-11,-9,-7,...,11},
		        ytick={-6,-4,...,6},
	         	minor ytick={-7,-5,...,7},
		         grid=both
 			]
 		% use TeX as calculator:
%	\addplot[first,line width=1.25pt,samples=200,<-]expression[domain=-8:-3]{1/2*(x+4)^2+5};
	\addplot[first,line width=1.25pt,samples=200,<-]expression[domain=-2:-1]{-1*(x+3)^2+9};
	\addplot[first,line width=1.25pt,samples=200,-]expression[domain=-1:1]{1*(x-1)^2+1};
	\addplot[first,line width=1.25pt,samples=200,-]expression[domain=1:3]{-1/2*(x-1)^2+1};
	\addplot[first,line width=1.25pt,samples=200,-]expression[domain=3:5]{1/2*(x-5)^2-3};
	\addplot[first,line width=1.25pt,samples=200,->]expression[domain=5:8]{-1/2*(x-5)^2-3};
 		\end{axis}
 		\end{tikzpicture}
		\label{fig:invprac1}
	\end{center}
\end{minipage}
\begin{minipage}{0.6\linewidth}
Evaluate the following.
	\begin{enumerate}\setlength{\itemsep}{0.5in}
		\item	$g(5)$
		\item $g^{-1}(5)$	
		\item	$g^{-1}(-1)$ 
		\end{enumerate}

\end{minipage}


\end{myPractice}

\begin{myPractice}
Graph the function $f(x) = 4+\sqrt[3]{x-1}$ in a graphing utility (such as Desmos) to confirm that it is a one-to-one function and then find the formula for $f^{-1}$.
\vfill
\vfill
\vfill
\end{myPractice}


\newpage



\begin{myPractice}
Graph the function $g(x) = 2-\sqrt{x+3}$ in a graphing utility.
\begin{enumerate}
	\item Is $g$ a one-to-one function? Why or why not? 
	\vfill
	\vfill

	\item State the domain and range of $g$.
	\vfill

	\item Algebraically find the formula for $g^{-1}$.
	\vfill
	\vfill
	\vfill
	
	\item State the domain and range of $g^{-1}$.
	\vfill

\end{enumerate}
\end{myPractice}



%======================================================
\newpage
%======================================================

\subsection*{Definitions} \label{def-functions-inverse}

\begin{myDefinition}[One-to-One Function:]~\\[0.5mm]
A function $f$ is said to be {\bf one-to-one} if for every possible output ($y$-value) in the range of $f$, there is exactly one input ($x$-value) in the domain of $f$. \\

In other words, in a one-to-one function, each possible input is paired with exactly one output {\bf AND} each possible output is paired with exactly one input.\\[0.5em]
\defexample The set $\{ (0,-2), ~ (1, \boldsymbol{-1}), ~ (4, 0), ~ (9,\boldsymbol{-1}),~ (16,-3) \}$ is a function, but it is \textit{not} one-to-one.\\
\defexample The set $\{ (0,-2), ~ (1, -1), ~ (2, 0), ~ (3, 1),~ (4,2) \}$ is a function and \textit{is} one-to-one.
\end{myDefinition}

\begin{myDefinition}[Horizontal Line Test:]~\\[0.5mm]
If a horizontal line can be drawn that intersects the graph of a function more than once, the graph is not the graph of a one-to-one function.\\[0.5em]
		\begin{minipage}{0.5\linewidth}
			\begin{center}
				\defexample
					\captionof{figure}{Does Not Pass}
					\begin{tikzpicture}
			 		\begin{axis}[
		 				framedaxes,
		 				width=5.5cm,height=5.5cm,
		 				xlabel={$x$},
		 				ylabel={$y$},
		 				xmin=-8,xmax=8,
		 				ymin=-8,ymax=8,
						xtick={-6,-4,...,6},
							minor xtick={-7,-5,...,7},
						ytick={-6,-4,...,6},
							minor ytick={-7,-5,...,7},
					         grid=both
		 				]
 				% use TeX as calculator:
%	 					\addplot[smooth,mark=*,first,line width=1.0pt,fill=white]coordinates{	(-4,2)	};
%	 					\addplot[smooth,mark=*,first,line width=1.0pt]coordinates{	(5,-4)	};
		 				\addplot[first,line width=1.5pt,samples=400,<->]expression[domain=-5.899:3.899]{0.5*(x+1)^2-4};
%						\addplot[first,line width=1.5pt,-,forget plot] coordinates {(-2,0.98) (-2,1.02)};
						\addplot[fifth,line width=0.75pt,<->] coordinates {(-8,-2) (8,-2)};
								\node at (axis cs:5,-2.5) {\color{fifth}\tiny{$y$\,=\,$-2$}};
						\addplot[fifth,smooth,mark=*,line width=0.25pt]coordinates{	(-3,-2)	};
						\addplot[fifth,smooth,mark=*,line width=0.25pt]coordinates{	(1,-2)	};
			 		\end{axis}
		 		\end{tikzpicture}
				\label{fig:inv-def1}
			\end{center}
		\end{minipage}
		\begin{minipage}{0.5\linewidth}
			\begin{center}
				\defexample
					\captionof{figure}{Passes}
					\begin{tikzpicture}
			 		\begin{axis}[
		 				framedaxes,
		 				width=5.5cm,height=5.5cm,
		 				xlabel={$x$},
		 				ylabel={$y$},
		 				xmin=-8,xmax=8,
		 				ymin=-8,ymax=8,
						xtick={-6,-4,...,6},
							minor xtick={-7,-5,...,7},
						ytick={-6,-4,...,6},
							minor ytick={-7,-5,...,7},
					         grid=both
		 				]
 				% use TeX as calculator:
%	 					\addplot[smooth,mark=*,first,line width=1.0pt,fill=white]coordinates{	(-4,2)	};
%	 					\addplot[smooth,mark=*,first,line width=1.0pt]coordinates{	(5,-4)	};
		 				\addplot[first,line width=1.5pt,samples=400,->]expression[domain=-2:8]{(x+2)^0.5+1};
		 				\addplot[first,line width=1.5pt,samples=400,<-]expression[domain=-8:-2]{-1*(-x-2)^0.5+1};
						\addplot[first,line width=1.5pt,-,forget plot] coordinates {(-2,0.98) (-2,1.02)};
%						\addplot[myred,line width=1.0pt,dashed,<->] coordinates {(3,8) (3,-8)};
%								\node at (axis cs:3.25,-7.5) {\color{myred}\tiny{$x\,$=\,3}};
			 		\end{axis}
		 		\end{tikzpicture}
				\label{fig:inv-def2}
			\end{center}
		\end{minipage}
\end{myDefinition}

\begin{myDefinition}[Inverse Function:]~\\[0.5mm]
If a function $f$ is one-to-one, then the function has an {\bf inverse}, $f^{-1}$.\\

Two functions, $f$ and $f^{-1}$ are {\bf inverse functions} if and only if both of the following are true:
\begin{itemize}
\item $f\left(f^{-1}(x)\right) = x$ for all $x$ in the domain of $f^{-1}$.
\item $f^{-1}\left(f(x)\right) = x$ for all $x$ in the domain of $f$.
\end{itemize}
The domain of a function $f$ is the range of the inverse function $f^{-1}$.\\
The range of a function $f$ is the domain of the inverse function $f^{-1}$.\\[0.5em]
\defexample \\
If $f$ is defined by the set $\{ (0,-2), ~ (1, -1), ~ (2, 0), ~ (3, 1),~ (4,2) \}$, \\
then $f^{-1}$ is the set $\{ (-2,0), ~ (-1, 1), ~ (0, 2), ~ (1, 3),~ (2,4) \}$.\\[0.3em]
The domain of $f$ is $\{ 0, ~ 1, ~ 2, ~ 3, ~ 4 \}$ and the range is $\{ -2, \, -1, ~ 0, ~ 1,~ 2 \}$.\\[0.3em]
The domain of $f^{-1}$ is $\{ -2, \, -1, ~ 0, ~ 1,~ 2 \}$ and the range is $\{ 0, ~ 1, ~ 2, ~ 3, ~ 4 \}$.\\

\end{myDefinition}





%======================================================
 \newpage
%======================================================

\subsection*{Exit Exercises} \label{exit-functions-inverse}



\begin{myExit}
	\begin{enumerate}
		\item Explain what is meant by the phrase ``one-to-one'' and how you can tell from the graph of the function if the function is one-to-one?
		\vfill
		\vfill
		\item What is the relationship of the domain and range of a function $f$ and its inverse function $f^{-1}$?
		\vfill
		\item What happens when you compose two functions that are inverses of each other?
		\vfill
		\item Why is $g(x)=(4x-1)^3$ invertible and $h(x)=(4x-1)^2$ is not?
		\vfill
		\item Given $m(x) = 19+(-3x+1)^5$, find $m^{-1}$.
		\vfill
		\vfill
	\end{enumerate}
\end{myExit}
\vfill




\exitlikert{inverse functions}















% Ch 6 of OER
\include{111-chapter-2-exponential-and-logarithmic-functions}

\resetCounters
%============================================================
% MTH 111 Project - Template File
% Section 6.1 from OpenStax OER
%	Updated August 2023
%============================================================

\section{Exponential Functions} \label{exponential-intro}

In this section, we'll investigate a new type of function that has a constant {\it percent} rate of change, whether it's a constant percent of increase or a constant percent of decrease.  \\[0.5em]
\textbooks{6.1}{6.2}



\subsection*{Preparation Exercises} \label{prep-exponential-intro}

\begin{myPrep}
Suppose you have a water filter that can remove 60\% of the contaminants in the water each time the water passes through the filter.
	\begin{enumerate}
		\item If 60\% of the contaminants are removed with each pass, what percent still remains after a single pass?
		\vfill
		\item If you start with 156.25 mcg of contaminants in some water, how much will be left after you pass the water through the filter once?
		\vfill
		\item How much will be left after a second pass through the filter?
		\vfill
		\item How much will be left after a third pass through the filter?
		\vfill
		\item Will the filter ever remove all of the contaminants from the water?  Why or why not?
		\vfill
	\end{enumerate}
\end{myPrep}



%======================================================
\newpage
%======================================================



%======================================================
\newpage
%======================================================


\subsection*{Practice Exercises} \label{practice-exponential-intro}

\begin{myPractice}
Which of the following functions are exponential functions? \\
For the exponential functions, identify if they represent exponential growth or decay.
\begin{enumerate}
\begin{multicols}{4}
\item $f(x) = 2 \left(\frac{3}{4}\right)^x$
\item $g(x) = 3 \left(-5\right)^x$
\item $h(x) = 4 \left(x\right)^7$
\item $j(x) = 5 9^{x+2}$
\end{multicols}
\vfill
\end{enumerate}
\end{myPractice}


\begin{myPractice}
A baseball card was worth \$50 in 1995 and its value has increased by 7\% per year every year since then.  Find a formula for a function $V$ that models the value of the baseball card $t$ years after 1995.  
\vfill
Evaluate $V(21)$ and explain its meaning in context.
\vfill
\vfill

\end{myPractice}




\begin{myPractice}
Match each function with one of the graphs in Figure~\ref{fig:exp-practice1}.\\
\begin{minipage}{0.5\linewidth}
	   \begin{center}
 		  \captionof{figure}{}~\\[-0.8em]
   		\label{fig:e1}
					\begin{tikzpicture}
					\begin{axis}[
					framedaxes,
					height=2.5in,
					width=3.5in,
					xlabel={$x$},
					ylabel={$y$},
					xmin=-6,xmax=6,
					ymin=-1,ymax=6,
					grid=both,
					xtick={-15,15},
					ytick={-15,15},
					]
					% use TeX as calculator:
%								\addplot[myred,dashed,line width=0.75pt,<->]expression[domain=-4:4,samples=200]{-.05} ;
					\addplot[first,<->,line width=1.0pt]expression[domain=-6:1,samples=400]{2* 3^x};
					\addplot[first,<->,line width=1.0pt]expression[domain=-6:2.71,samples=400]{2* 1.5^x};
					\addplot[first,<->,line width=1.0pt]expression[domain=-1.58:6,samples=400]{2*0.5^x};
					\addplot[first,<->,line width=1.0pt]expression[domain=-6:3.8,samples=400]{3* 1.2^x};
					\node at (axis cs:-1.8,5.5) {$\boldsymbol{A}$};	
					\node at (axis cs:1.3,5.5) {$\boldsymbol{B}$};	
					\node at (axis cs:2.2,5.5) {$\boldsymbol{C}$};	
					\node at (axis cs:4,5.5) {$\boldsymbol{D}$};	
					\end{axis}
					\end{tikzpicture}
					\label{fig:exp-practice1}
			\end{center}
\end{minipage}
\begin{minipage}{0.5\linewidth}
	\setlist{itemsep=16pt}
	\begin{itemize}
		\item $F(x) = 2(3)^x$ is graph \underline{~~~~~~~}
		\item $G(x) = 2 (0.5)^x$ is graph \underline{~~~~~~~}
		\item $H(x) = 3(1.2)^x$ is graph \underline{~~~~~~~}
		\item $J(x) = 2(1.5)^x$ is graph \underline{~~~~~~~}
	\end{itemize}
\end{minipage}
\end{myPractice}
\vfill





%======================================================
\newpage
%======================================================

\subsection*{Definitions} \label{def-exponential-intro}

\begin{myDefinition}[Exponential Function:]~\\[0.5mm]
For any real number $x$, an {\bf exponential function} is a function of the form $\boldsymbol{f(x) = a\cdot b^x}$ where
	\begin{itemize}
	\setlength{\itemsep}{0in}
		\item $a$ is a non-zero real number
		\item $b$ is any positive real number, where $b\neq 1$
	\end{itemize}
Exponential functions grow or decay with a constant {\it percent} rate of change.  
\end{myDefinition}

\begin{myDefinition}[Key Characteristics of Exponential Functions:]~\\[0.5mm]
For an exponential function $f(x) = a\cdot b^x$, with $a> 0$ and $b>0,b\neq1$, we have the following:\\[3mm]
\begin{minipage}{0.6\linewidth}
	\begin{itemize}
	\setlength{\itemsep}{1mm}
		\item $(0,a)$ is the vertical intercept.
		\item There is no horizontal intercept.
		\item The domain of $f$ is $(-\infty, \infty)$.
		\item The range of $f$ is $(0,\infty)$.
		\item $f$ is a one-to-one function.
		\item The horizontal asymptote is $y=0$.
		\item If $b>1$, then $f$ is an increasing function and
			\begin{itemize}
			\setlength{\itemsep}{0in}
				\item[\small{•}] as $x\rightarrow \infty$, $f(x)\rightarrow \infty$, and
				\item[\small{•}] as $x\rightarrow -\infty$, $f(x)\rightarrow 0$.
			\end{itemize}	
		\item If $0<b<1$, then $f$ is a decreasing function and
			\begin{itemize}
			\setlength{\itemsep}{0in}
				\item[\small{•}] as $x\rightarrow \infty$, $f(x)\rightarrow 0$, and
				\item[\small{•}] as $x\rightarrow -\infty$, $f(x)\rightarrow \infty$.
			\end{itemize}	
	\end{itemize}
	\end{minipage}
	\begin{minipage}{0.4\linewidth}
			\begin{center}
			\captionof{figure}{$y=a\cdot b^x, b>1$}~\\[-0.8em]
			\begin{tikzpicture}
 				\begin{axis}[
 					framedaxes,
		 			height=5cm,
 					width=5cm,
 					xlabel={$x$},
 					ylabel={$y$},
		 			xmin=-6,xmax=6,
 					ymin=-6,ymax=6,
				        xtick={-16,16},
				       	minor xtick={-16,16},
				        ytick={-16,16},
	        		 	minor ytick={-16,16},
				         grid=both
 					]
		 		% use TeX as calculator:
				 	\addplot[first,line width=1.5pt,samples=200,<->]expression[domain=-6:2.584] {2^x};
					\addplot[first,line width=1.0pt,mark=*] coordinates {(0,1)};
						\node at (axis cs:-1.5,1.5) {{\tiny $(0,a)$}};									
		 		\end{axis}
 				\end{tikzpicture}
				\label{fig:expintro-def1}
%				\end{center}

%			\begin{center}
			\captionof{figure}{$y=a\cdot b^x,0<b<1$}
			\begin{tikzpicture}
 				\begin{axis}[
 					framedaxes,
		 			height=5cm,
 					width=5cm,
 					xlabel={$x$},
 					ylabel={$y$},
		 			xmin=-6,xmax=6,
 					ymin=-6,ymax=6,
				        xtick={-16,16},
				       	minor xtick={-16,16},
				        ytick={-16,16},
	        		 	minor ytick={-16,16},
				         grid=both
 					]
		 		% use TeX as calculator:
				 	\addplot[first,line width=1.5pt,samples=200,<->]expression[domain=-2.584:6] {0.5^x};
					\addplot[first,line width=1.0pt,mark=*] coordinates {(0,1)};
						\node at (axis cs:1.5,1.5) {{\tiny $(0,a)$}};									
		 		\end{axis}
 				\end{tikzpicture}
				\label{fig:expintro-def2}
				\end{center}

	\end{minipage}
~\\[0.5em]	
\begin{minipage}{0.9\linewidth}
\defexample \href{https://tiny.cc/111Z-ExpFunction}{View this Desmos graph} to see an interactive example of a exponential function.  %(url: tiny.cc/111Z-
\end{minipage}
\begin{minipage}{0.1\linewidth}
\flushright \qrcode[height=1cm]{https://tiny.cc/111Z-ExpFunction}
\end{minipage}
ExpFunction)
\end{myDefinition}

\begin{myDefinition}[The Number $e$:]~\\[0.5mm]
\begin{minipage}{0.65\linewidth}
The number $\boldsymbol{e}$ was discovered in the late 1600's by Jacob Bernoulli. Later in the 1700's, Leonard Euler discovered many of its interesting properties.\\

$\boldsymbol{e}$ can be approximated by 2.718281828459, though its decimal form does not end and does not repeat.  It is an irrational number.\\

The graph of the function given by $f(x) = e^x$ looks a lot like the graphs of the functions given by $g(x) = 2^x$ and $h(x) = 3^x$, as shown in Figure~\ref{fig:edef1}. 
 \vspace{8mm}

\end{minipage}
\begin{minipage}{0.35\linewidth}
	   \begin{center}
 		  \captionof{figure}{}
   		\label{fig:edef1}
					\begin{tikzpicture}
					\begin{axis}[
					framedaxes,
					legend pos=north west,
					legend cell align=left,
					height=6.5cm,
					width=7cm,
					xlabel={$x$},
					ylabel={$y$},
					xmin=-4,xmax=4,
					ymin=-1,ymax=7,
					grid=both,
					xtick={-3,-2,...,3},
					ytick={0,1,...,6},
					]
					% use TeX as calculator:
%								\addplot[myred,dashed,line width=0.75pt,<->]expression[domain=-4:4,samples=200]{-.05} ;
					\addplot[third!35,dashed,<->,line width=1.0pt]expression[domain=-4:2.807354,samples=200]{2^x};
					\addplot[first,<->,line width=1.5pt]expression[domain=-4:1.945910,samples=200]{2.718281828^x};
					\addplot[second,dashed,<->,line width=1.0pt]expression[domain=-4:1.771243,samples=200]{3^x};
								%\node at (axis cs:-3.,-0.25) {\tiny{$y\,$=\,0}};								    
					\legend{$y=2^x$, $y=e^x$, $y=3^x$}
					\end{axis}
					\end{tikzpicture}
			\end{center}
\end{minipage}
\end{myDefinition}




%======================================================
 \newpage
%======================================================

\subsection*{Exit Exercises} \label{exit-exponential-intro}




\begin{myExit}
Given the formula for an exponential function, you should be able to look at the formula and identify if the function will represent exponential growth or exponential decay.  How can you do this?  \\[0.5em]
Give an example of a symbolic function for each, one exponential growth and one exponential decay, as part of your explanation.
\vfill
\vfill
\vfill
\end{myExit}


\begin{myExit}
At the start of an experiment, a population of bacteria has 5 million bacteria and the population is decreasing by 13\% every 6 hours.  Find a formula for the function $B$ that gives the number of bacteria (in millions) remaining after $n$ 6-hour time intervals. \\[0.4in]

Evaluate $B(8)$ and explain its meaning in context.
\vfill
\vfill
\vfill
\end{myExit}


\begin{myExit}
%Given $f(x) = 5^x$ and $g(x)=-7\cdot5^{x-2}+1$, state the sequence of transformations that takes the graph of $y=f(x)$ to the graph of $y=g(x)$ and also state the range of $g$.
%REPLACED WITH:
Find an exponential function $f$ that satisfies $f(2)=\dfrac{3}{8}$ and $f(-1)=\dfrac{8}{9}$.

\vfill
\vfill
\vfill
\vfill
\vfill
\end{myExit}






\exitlikert{exponential functions}


















\resetCounters
%============================================================
% MTH 111Z Project - Template File
% Section 6.3 from OpenStax OER
%	Updated 202302
%============================================================

\section{Logarithmic Functions } \label{logarithmic-intro}

In this section, we'll look at the inverse of exponential functions in general, as well as focus on the inverses of a few exponential functions with commonly used bases.\\[0.5em]
\textbooks{6.3}{6.4}


\subsection*{Preparation Exercises } \label{prep-logarithmic-intro}

\begin{myPrep}
~\\ \vspace{-12mm}
\begin{enumerate}
	\begin{multicols}{2}
		\item Rewrite $\sqrt[7]{2187}=3$ \\ as an exponential statement.
		\item Rewrite $2^{10} = 1024$ \\as a radical statement.
	\end{multicols}
	\vspace{0.7 in}
\end{enumerate}
\end{myPrep}

\begin{myPrep}
Let $f(x) = 2^x$.  The graph of $y=f(x)$ is in Figure~\ref{fig:log-practice1}.\\
\begin{enumerate}
	\begin{minipage}{0.6\linewidth}
		\item Evaluate $f(5)$.
		\vspace{0.35 in}\\

		\item Evaluate $f(-2)$.
		\vspace{0.35 in}\\
		~

		\item Solve $f(x)=8$.
	
	\end{minipage}
	\begin{minipage}{0.4\linewidth}
		\begin{center}
		\captionof{figure}{$f(x)=2^x$}~\\[-0.8em]
		\begin{tikzpicture}
			\begin{axis}[
				framedaxes,
				width=2.25in,
				height=2.25in,
				xlabel={$x$},
				ylabel={$y$},
				xmin=-6,xmax=6,
				ymin=-6,ymax=12,
			        xtick={-5,...,5},
				        	minor xtick={-3,-1,...,9},
			        ytick={-4,-2,...,10},
      	  				minor ytick={-5,3,...,11},
			        grid=none
				]
				% use TeX as calculator:
				\addplot[first,line width=1.5pt,<->]expression[domain=-6:3.58,samples=200]{{1 * (2)^x}};
		\end{axis}
		\end{tikzpicture}
		\label{fig:log-practice1}
		\end{center}
	\end{minipage}
	\vspace{0.6 in}
	
	\item $f$ is a one-to-one function. How do you know and what does this mean?
	\vfill
	\vfill
	\begin{multicols}{3}
	\item What is $f^{-1}(32)$?
	\item What is $f^{-1}(\frac{1}{4})$?
	\item What is $f^{-1}(1)$?
	\end{multicols}
	\vfill
	\end{enumerate}
\end{myPrep}



%======================================================
\newpage
%======================================================


\subsection*{Practice Exercises } \label{practice-logarithmic-intro}

\begin{myPractice}
Convert each exponential statement into a logarithmic statement.
	\begin{enumerate}
		\begin{multicols}{3}
			\item $3^4=81$
			\item $5^{-2}=\dfrac{1}{25}$
			\item $7^{0}=1$
		\end{multicols}
	\end{enumerate}
	\vfill
\end{myPractice}


\begin{myPractice}
Evaluate each logarithm without a calculator.
	\begin{enumerate}
		\begin{multicols}{3}
			\item $\log \left(10000\right)$
			\item $\log_{25}\left(5\right)$
			\item $\log_4\left(\frac{1}{64}\right)$
		\end{multicols}
	\end{enumerate}
	\vfill
\end{myPractice}


\begin{myPractice}
Let $f(x)=7^x$ and $g(x)= \log_7(x)$.
	\begin{enumerate}
		\begin{multicols}{2}
			\item What is the domain of $f$?
			\item What is the range of $f$?
		\end{multicols}
		\vfill
		\begin{multicols}{2}
			\item What is the domain of $g$?
			\item What is the range of $g$?
		\end{multicols}
		\vfill
	\end{enumerate}
\end{myPractice}

\begin{myPractice}
Match each function with one of the graphs in Figure~\ref{fig:log-practice2}.\\
\begin{minipage}{0.5\linewidth}
	   \begin{center}
 		  \captionof{figure}{}
   		\label{fig:e1}
					\begin{tikzpicture}
					\begin{axis}[
					framedaxes,
					height=2.5in,
					width=3.5in,
					xlabel={$x$},
					ylabel={$y$},
					xmin=-1,xmax=6,
					ymin=-6,ymax=6,
					grid=both,
					xtick={-15,15},
					ytick={-15,15},
					]
					% use TeX as calculator:
%								\addplot[myred,dashed,line width=0.75pt,<->]expression[domain=-4:4,samples=200]{-.05} ;
					\addplot[first,<->,line width=1.0pt]expression[domain=0.0014:6,samples=400]{ln(x)/ln(3)};
					\addplot[first,<->,line width=1.0pt]expression[domain=0.042:6,samples=400]{ln(x)/ln(1.7)};
					\addplot[first,<->,line width=1.0pt]expression[domain=0.016:6,samples=400]{ln(x)/ln(1/2)};
					\addplot[first,<->,line width=1.0pt]expression[domain=0.21:4.78,samples=400]{ln(x)/ln(1.3)};
					\node at (axis cs:5,5.5) {$\boldsymbol{A}$};	
					\node at (axis cs:5.5,3.75) {$\boldsymbol{B}$};	
					\node at (axis cs:5.5,1) {$\boldsymbol{C}$};	
					\node at (axis cs:5.5,-3) {$\boldsymbol{D}$};	
					\end{axis}
					\end{tikzpicture}
					\label{fig:log-practice2}
			\end{center}
\end{minipage}
\begin{minipage}{0.5\linewidth}
	\setlist{itemsep=16pt}
	\begin{itemize}
		\item $f(x) = \log_{1.7}(x)$ is graph \underline{~~~~~~~}
		\item $g(x) = \log_{1/2}(x)$ is graph \underline{~~~~~~~}
		\item $h(x) =  \log_{1.3}(x)$ is graph \underline{~~~~~~~}
		\item $j(x) =  \log_{3}(x)$ is graph \underline{~~~~~~~}
	\end{itemize}
\end{minipage}
\end{myPractice}









%======================================================
\newpage
%======================================================

\subsection*{Definitions } \label{def-logarithmic-intro}

\begin{myDefinition}[Logarithm:]~\\[0.5mm]
For any real number $x>0$, the \textbf{logarithm with base $\boldsymbol{b}$ of $\boldsymbol{x}$}, where $b>0$ and $b\neq 1$, is denoted by $\boldsymbol{\log_b (x)}$ and is defined by
			$$
					y = \log_b (x) \ \  \textrm{if and only if} \ \  x = b^y
			$$
\end{myDefinition}

\begin{myDefinition}[Common Logarithm:]~\\[0.5mm]
The {\bf common logarithm}, $\boldsymbol{\log(x)}$, is a logarithm with base 10 and satisfies
$$ \ y=\log(x) \ \ \text{is equivalent to} \ \ 10^y=x, \text{for } x>0 $$
 \end{myDefinition}

\begin{myDefinition}[Natural Logarithm:]~\\[0.5mm]
The {\bf natural logarithm}, $\boldsymbol{\ln(x)}$, is a logarithm with base $e$ and satisfies the following:
$$ \ y=\ln(x) \ \ \text{is equivalent to} \ \ e^y=x, \text{for } x>0 $$
%For $x>0$, $y=\ln(x)$ is equivalent to $e^y=x$. 
\end{myDefinition}


\begin{myDefinition}[Key Characteristics of Logarithmic Functions:]~\\[0.5mm]
For a logarthmic function $f(x) = \log_b(x)$, with $b> 0$, $b\neq1$, and $x>0$, we have the following:\\[3mm]
\begin{minipage}{0.6\linewidth}
	\begin{itemize}
	\setlength{\itemsep}{1mm}
		\item $(1,0)$ is the horizontal intercept.
		\item There is no vertical intercept.
		\item The domain of $f$ is $(0, \infty)$.
		\item The range of $f$ is $(-\infty,\infty)$.
		\item $f$ is a one-to-one function.
		\item The vertical asymptote is $x=0$.
		\item If $b>1$, then $f$ is an increasing function and
			\begin{itemize}
			\setlength{\itemsep}{0in}
				\item[\small{•}] as $x\rightarrow \infty$, $f(x)\rightarrow \infty$, and
				\item[\small{•}] as $x\rightarrow 0^+$, $f(x)\rightarrow -\infty$.
			\end{itemize}	
		\item If $0<b<1$, then $f$ is a decreasing function and
			\begin{itemize}
			\setlength{\itemsep}{0in}
				\item[\small{•}] as $x\rightarrow \infty$, $f(x)\rightarrow -\infty$, and
				\item[\small{•}] as $x\rightarrow 0^+$, $f(x)\rightarrow \infty$.
			\end{itemize}	
	\end{itemize}
	\end{minipage}
	\begin{minipage}{0.4\linewidth}
			\begin{center}
			\captionof{figure}{$y=\log_b(x), b>1$}~\\[-0.8em]
			\begin{tikzpicture}
 				\begin{axis}[
 					framedaxes,
		 			height=5cm,
 					width=5cm,
 					xlabel={$x$},
 					ylabel={$y$},
		 			xmin=-6,xmax=6,
 					ymin=-6,ymax=6,
				        xtick={-16,16},
				       	minor xtick={-16,16},
				        ytick={-16,16},
	        		 	minor ytick={-16,16},
				         grid=both
 					]
		 		% use TeX as calculator:
					\addplot[first,line width=1.5pt,<->]expression[domain=0.016:6,samples=200]{{ln(x)/ln(2)}};
					\addplot[first,line width=1.0pt,mark=*] coordinates {(1,0)};
					\node at (axis cs:1.75,-0.75) {{\tiny $(1,0)$}};	
					\addplot[smooth,mark=*,first,line width=1.5pt]coordinates{ (2,1)	};
					\node at (axis cs:1.5,1.75) {{\tiny $(b,1)$}};		
		 		\end{axis}
 				\end{tikzpicture}
				\label{fig:log-def1}
%				\end{center}

%			\begin{center}
			\captionof{figure}{$y=\log_b(x), 0<b<1$}
			\begin{tikzpicture}
 				\begin{axis}[
 					framedaxes,
		 			height=5cm,
 					width=5cm,
 					xlabel={$x$},
 					ylabel={$y$},
		 			xmin=-6,xmax=6,
 					ymin=-6,ymax=6,
				        xtick={-16,16},
				       	minor xtick={-16,16},
				        ytick={-16,16},
	        		 	minor ytick={-16,16},
				         grid=both
 					]
		 		% use TeX as calculator:
					\addplot[first,line width=1.5pt,<->]expression[domain=0.088:6,samples=200]{{ln(x)/ln(2/3)}};
					\node at (axis cs:2.5,-0.5) {{\tiny $(1,0)$}};					
					\addplot[smooth,mark=*,first,line width=1.5pt]coordinates{(1,0)	};
					\addplot[smooth,mark=*,first,line width=1.5pt]coordinates{(0.444,2)	};
					\node at (axis cs:1.5,2.75) {{\tiny $(b,1)$}};
		 		\end{axis}
 				\end{tikzpicture}
				\label{fig:log-def2}
				\end{center}

	\end{minipage}
	~\\[0.5em]	
\begin{minipage}{0.9\linewidth}
\defexample \href{https://tiny.cc/111Z-LogFunction}{View this Desmos graph} to see an interactive example of a logarithmic function.  %(url: tiny.cc/111Z-LogFunction)
\end{minipage}
\begin{minipage}{0.1\linewidth}
\flushright \qrcode[height=1cm]{https://tiny.cc/111Z-LogFunction}
\end{minipage}
\end{myDefinition}


%======================================================
 \newpage
%======================================================

\subsection*{Exit Exercises} \label{exit-logarithmic-intro}

\begin{myExit}
	\begin{enumerate}
		\item Why do we call $b$ the base of the logarithm $\log_b(x)$?  Evaluate $\log_2(16)$ and use this in your answer.
		\vfill
		\vfill
		\item Evaluate $\log_9(3)$ and then restate your logarithmic statement as an exponential statement.
		\vfill
		\item For any $b>0$, where $b\neq1$, what is the domain of $f(x)= \log_b(x)$ and why is this the domain of $f$?
		\vfill
		\vfill
		\item Does a logarithmic function have a horizontal or vertical asymptote and why?
		\vfill
		\vfill
		\item Given $f(x) = \log_8(x)$ and $g(x)=-3\cdot \log_8(x+3)-2$, state the sequence of transformations that takes the graph of $y=f(x)$ to the graph of $y=g(x)$ and also state the domain of $g$.
		\vfill
		\vfill
	\end{enumerate}
\end{myExit}
\vfill


\exitlikert{logarithmic functions}




\newpage
~










\resetCounters
%============================================================
% MTH 111Z Project - Template File
% Section 6.5 from OpenStax OER
%	Updated 202302
%============================================================

\section{Properties of Logarithms} \label{logarithms-properties}

In this section, we'll investigate some important properties of logarithms, which will help us to be able to solve exponential and logarithmic equations. \\[0.5em]
\textbook{6.5}

\subsection*{Preparation Exercises} \label{prep-logarithms-properties}

\begin{myPrep}
Simplify each expression in the left column without a calculator.  \\
State the corresponding exponent rule in the right column.
	\begin{enumerate}
		\begin{multicols}{2}
			\item $8^1$\\
			$b^1$
		\end{multicols}
		\vfill

		\begin{multicols}{2}
			\item $7^0$\\
			$b^0$
		\end{multicols}
		\vfill

		\begin{multicols}{2}
			\item $6^{-2}$\\
			$b^{-n}$
		\end{multicols}
		\vfill

		\begin{multicols}{2}
			\item $5^2 \cdot 5^9$\\
			$b^n \cdot b^m$
		\end{multicols}
		\vfill

		\begin{multicols}{2}
			\item $\dfrac{4^7}{4^3}$\\
			 $\dfrac{b^n}{b^m}$
		\end{multicols}
		\vfill
		
		\begin{multicols}{2}
			\item  $\left(3^2 \right)^5$\\
			$\left(b^n \right)^m$
		\end{multicols}
		\vfill
	\end{enumerate}
\end{myPrep}

\begin{myPrep}
Evaluate each logarithm without a calculator.
	\begin{enumerate}
		\begin{multicols}{2}
			\item $\log_5 (1)$
			\item $\log_2 (2)$
		\end{multicols}
		\vfill

		\begin{multicols}{2}
			\item$\log (1)$	
			\item $\ln (e)$
		\end{multicols}
		\vfill
	\end{enumerate}
\end{myPrep}





%======================================================
\newpage
%======================================================


\subsection*{Practice Exercises} \label{practice-logarithms-properties}
\begin{myPractice}
Expand $\log_6(5x^3y)$ as much as possible by rewriting the expression as a sum, difference, or product of logs or constant factors.
\end{myPractice}
\vfill

\begin{myPractice}
Rewrite $\frac{1}{2}\ln(x+5)-6\ln(x)$ as a single logarithm.
\end{myPractice}
\vfill

\begin{myPractice}
Find the exact value of $\log_{5}(75)-\log_{5}(3)+\log_2(16)$ without using a calculator.
\end{myPractice}
\vfill

\begin{myPractice}
Saskia and José disagree.  Saskia says that $\log_3(x+2)-\log_3(x+1) = \dfrac{\log_3(x+2)}{\log_3(x+1)}$, but José says that's wrong.  \\
Who is correct and why?
\end{myPractice}
\vfill






%======================================================
\newpage
%======================================================

\subsection*{Properties} \label{def-logarithms-properties}

\begin{myDefinition}[Four Properties of Logarithms]~\\[0.5mm]
Given any real number $x$ and any positive real number $b$, with $b\neq 1$,\\[0.5em]
	\begin{minipage}{0.4\linewidth}
			$$\begin{aligned} 
			\log_b(1)&=0\\[0.5em]
			\log_b(b)&=1
			\end{aligned}$$
	\end{minipage}
	\begin{minipage}{0.4\linewidth}
			$$\begin{aligned} 
			\log_b(b^x)&=x\\[0.6em]
			b^{\log_b(x)}&=x
			\end{aligned}$$

%			$$\log_b(b^x)=x$$
%			$$b^{\log_b(x)}=x, \ x>0$$
	\end{minipage}
\end{myDefinition}


\begin{myDefinition}[Product Rule for Logarithms]~\\[0.5mm]
Given any positive real numbers $M$, $N$, and $b$, with $b\neq 1$,
\[\log_b\left(M\cdot N\right) = \log_b(M)+\log_b(N)\]
\end{myDefinition}

\begin{myDefinition}[Quotient Rule for Logarithms]~\\[0.5mm]
Given any positive real numbers $M$, $N$, and $b$, with $b\neq 1$,
\[\log_b\left(\dfrac{M}{N}\right) = \log_b(M)-\log_b(N)\]
\end{myDefinition}

\begin{myDefinition}[Power Rule for Logarithms]~\\[0.5mm]
Given any real number $n$, positive real numbers $M$ and $b$, with $b\neq 1$,
\[\log_b\left(M^n\right) = n\cdot \log_b(M)\]
\end{myDefinition}

\begin{myDefinition}[Change of Base Formula]~\\[0.5mm]
Given positive real numbers $M$, $n$, and $b$, with $b\neq 1$ and $n\neq1$,
\[\log_b(M) = \dfrac{ \log_n(M)}{\log_n(b)}\]
\end{myDefinition}





%======================================================
 \newpage
%======================================================

\subsection*{Exit Exercises} \label{exit-logarithms-properties}



\begin{myExit}
	\begin{enumerate}
		\item Only one of these is true.  Which one and why? 
		\begin{enumerate}[label=\roman*)]
			\begin{minipage}{0.55\linewidth}
				 \item $ \log_2(8) +\log_2(16) = \log_2(8\cdot16) $
			\end{minipage}
			\begin{minipage}{0.45\linewidth}
			 	\item $ \log_2(8) \cdot \log_2(16)  =\log_2(8+16) $
			\end{minipage}
			\end{enumerate}
		\vfill
		
		\item Only one of these is true.  Which one and why? 
		\begin{enumerate}[label=\roman*)]
			\begin{minipage}{0.55\linewidth}
			 	\item $ \dfrac{\log_3(27)}{ \log_3(9)}  =\log_3(27-9) $
			\end{minipage}
			\begin{minipage}{0.45\linewidth}
				\item $ \log_3(27) -\log_3(9) = \log_3\left(\frac{27}{9}\right) $
			\end{minipage}
			\end{enumerate}
		\vfill
		
		\item Rewrite each of these as a single logarithm.
		\begin{enumerate}[label=\roman*)]
			\begin{minipage}{0.55\linewidth}
				 \item $ \log(10) - \log(2) + \log(3)$
			\end{minipage}
			\begin{minipage}{0.45\linewidth}
				 \item $ \ln(a) - \ln(b) - \ln(c)$
			\end{minipage}
			\end{enumerate}
		\vfill
		
		\item Fully expand each of the following.
		\begin{enumerate}[label=\roman*)]
			\begin{minipage}{0.55\linewidth}
				 \item $ \log_2\left( \dfrac{x^2}{y^3z} \right)$
			\end{minipage}
			\begin{minipage}{0.45\linewidth}
				 \item $ \ln(e^5n^3\sqrt[4]{k})$
			\end{minipage}
			\end{enumerate}
		\vfill
	\end{enumerate}
\end{myExit}


\exitlikert{properties of logarithms}



























\resetCounters
%============================================================
% MTH 111Z Project - Template File
% Section 6.6 from OpenStax OER
%	Updated 202302
%============================================================

\section{Exponential and Logarithmic Equations} \label{exponential-and-logarithmic-equations}

In this section, we'll now combine what we know about exponential and logarithmic functions, as well as the properties of logarithms, in order to solve exponential and logarithmic equations.      \\[0.5em]
\textbook{6.6}


\subsection*{Preparation Exercises} \label{prep-exponential-and-logarithmic-equations}

\begin{myPrep}
	\begin{enumerate}
		\item Rewrite each logarithmic statement as an exponential statement.
			\begin{multicols}{2}
			\begin{enumerate}
				\item $\log_3(243) = 5$
				\item $\log_{16}(4) = \frac{1}{2}$
			\end{enumerate}
			\end{multicols}
			\vfill
		\item Solve $\log_5(x)=3$ by first rewriting it as an exponential statement.
		\vfill
		\item If $f(x) = x+8$, what is the inverse function of $f$?
		\vfill
		\item If $f(x) = 5x$, what is the inverse function of $f$?
		\vfill
		\item How do inverse operations or inverse functions help us solve equations?
		\vfill
	\end{enumerate}
\end{myPrep}



%======================================================
\newpage
%======================================================


\subsection*{Practice Exercises} \label{practice-exponential-and-logarithmic-equations}


\begin{myPractice}
Solve each of the following algebraically.  Use a calculator to approximate any irrational solutions.
	\begin{enumerate}
		\begin{multicols}{2}
			\item	$ 3^{5x-6} = 81$
			\item $9^{x-11} = 7 $
		\end{multicols}
		\vfill
		\vfill
		\begin{multicols}{2}
			\item	$ e^{x+3} +4= 19$
			\item	$ 5^{x-6}= 3^{2x+7}$
		\end{multicols}
		\vfill
		\vfill
		\vfill
	\end{enumerate}
\end{myPractice}

\newpage
\begin{myPractice}
Solve each of the following algebraically.  Be sure to confirm any solutions are not extraneous.
	\begin{enumerate}
		\begin{multicols}{2}
			\item $\log_6(3x+1) =\log_6(x-9) $
			\item	$\log_7(x-4)+3=5$
		\end{multicols}
		\vfill
		\vfill
		\item $\log_3(x-2)= 1-\log_3(x-4)$
		\vfill
		\vfill
		\vfill
	\end{enumerate}
\end{myPractice}


%======================================================
\newpage
%======================================================

\subsection*{Definitions and Properties} \label{def-exponential-and-logarithmic-equations}

\begin{myDefinition}[Logarithm:]~\\[0.5mm]
For any real number $x>0$, the \textbf{logarithm with base $\boldsymbol{b}$ of $\boldsymbol{x}$}, where $b>0$ and $b\neq 1$, is denoted by $\boldsymbol{\log_b (x)}$ and is defined by
			$$
					y = \log_b (x) \ \  \textrm{if and only if} \ \  x = b^y
			$$
\end{myDefinition}


\begin{myDefinition}[One-to-One Property of Exponential Functions]~\\[0.5mm]
For any algebraic expressions $S$ and $T$, and any positive real number $b$, with $b\neq1$,
$$b^S=b^T \text{ if and only if } S=T$$
\end{myDefinition}


\begin{myDefinition}[One-to-One Property of Logarithmic Functions]~\\[0.5mm]
For any algebraic expressions $S>0$, $T>0$, and any positive real number $b$, where $b\neq1$,
$$\log_b(S)=\log_b(T) \text{ if and only if } S=T$$ 
Note: Because $\log_b(x)$ has the domain $(0,\infty)$ for all $b>0, b\neq1$, when we solve an equation involving logarithms, {\bf we must always check} to see if the solution we've found is valid or if it is an extraneous solution. 
\end{myDefinition}








%======================================================
 \newpage
%======================================================

\subsection*{Exit Exercises} \label{exit-exponential-and-logarithmic-equations}



\begin{myExit}
	\begin{enumerate}
		\item What's the general process for solving exponential equations that have one exponential expression in them?
		\vfill
		\item Why can logarithmic equations have extraneous solutions and how can an extraneous solution be recognized?
		\vfill
		\begin{multicols}{2}
			\item Solve $4e^{2k+1}+3=27$.
			\item Solve $\log_8(5x+12)-\log_8(x)=\log_8(2)$.
		\end{multicols}
		\vfill
		\vfill
		\vfill
	\end{enumerate}
\end{myExit}
\vfill


\exitlikert{exponential and logarithmic equations}














\newpage
~

\resetCounters
%============================================================
% MTH 111Z Project - Template File
% Part of Section 6.1 from OpenStax OER
%	Updated 202302
%============================================================

\section{Exponential Functions and Compound Interest} \label{exponential-compounding}

In this section, we'll build upon our understanding of the general exponent function $f(x) = a\cdot b^x$ and see how variations of this formula are used in financial situations. \\[0.5em]
\textbook{6.1}


\subsection*{Preparation Exercises} \label{prep-exponential-compounding}

\begin{myPrep}
The function $V(t) = 52.8(1.093)^t$ represents the value (in thousands of dollars) of a collectable car $t$ years after June 1st, 2015.
	\begin{enumerate}
		\item What was the value of the car on June 1st, 2015?
		\vfill
		\item Is the value of the car increasing or decreasing and at what rate is the value of the car changing?
		\vfill
	\end{enumerate}
\end{myPrep}

\begin{myPrep}
Simple Interest is calculated using the formula $I = P r t$, where $I$ is the interest earned, $P$ is the principal or initial amount of money, $r$ is the interest rate, and $t$ is the amount of time that the interest is earned.
	\begin{enumerate}
		\item How much simple interest is owed after 7 years on a \$4,000 loan that earns 6\% annual simple interest?  How much will need to be paid back in total at the end of 7 years?
				\vfill
		\vfill
		\item How much simple interest is earned after 6 years on a \$3,000 investment that earns 4\% annual simple interest?  
		\vfill
		\vfill
	\end{enumerate}
\end{myPrep}





%======================================================
\newpage
%======================================================


\subsection*{Practice Exercises} \label{practice-exponential-compounding}

\begin{myPractice}
How much is owed at the end of a 6 years if \$12,000 is borrowed at 6.4\% annual interest, compounded quarterly?
\vfill
\end{myPractice}

\begin{myPractice}
Maya said the formula she set up for an exercise to calculate the amount (in dollars) owed on a loan after some number of years is $A = 10250\left(1+\frac{0.04}{12}\right)^{120}$.  What is the principal of the loan, the nominal interest rate, the number of compounds per year, and the number of years of the loan?
\vfill
\end{myPractice}


\begin{myPractice}
If Marshon invests \$5000 at 5.8\% annual interest compounded continuously, how much will he have in the account in 13 years?
\vfill
\end{myPractice}



%======================================================
\newpage
%======================================================

\subsection*{Definitions} \label{def-exponential-compounding}

\begin{myDefinition}[Simple Interest]~\\[0.5mm]
The formula to calculate {\bf simple interest} is $${I = Prt}$$
where $I$ is the amount of interest earned, \\
$P$ is the principal or initial amount of money, \\
$r$ is the annual interest rate, and \\
$t$ is the number of years the interest is earned 
\end{myDefinition}

\begin{myDefinition}[Compound Interest]~\\[0.5mm]
The formula to calculate {\bf compound interest} is $${A = P \left(1+\dfrac{r}{n}\right)^{nt}}$$
where $A$ is total amount of money owed or earned,\\
$P$ is the principal, \\
$r$ is the nominal annual interest rate,  \\
$n$ is the number of times the interest is compounded per year, and \\
$t$ is the number of years that the interest is earned. \\
Note: This formula is used where there is a finite number of compounding per year.
\end{myDefinition}

\begin{myDefinition}[The Number $\boldsymbol{e}$]~\\[0.5mm]
\begin{minipage}{0.9\linewidth}
$\boldsymbol{e}$, also known as {\bf Euler's number}, is the irrational number that results from $$\lim\limits_{n\to \infty}\left(1+\frac{1}{n}\right)^n$$
\defexample You can watch this \href{https://youtu.be/AuA2EAgAegE}{Numberphile YouTube video about $e$}. 
\end{minipage}
\begin{minipage}{0.1\linewidth}
\flushright \qrcode[height=1cm]{https://youtu.be/AuA2EAgAegE}
\end{minipage}
%$\boldsymbol{e}$, also known as {\bf Euler's number}, is the irrational number that results from $$\lim\limits_{n\to \infty}\left(1+\frac{1}{n}\right)^n$$	
%\defexample You can watch this \href{https://youtu.be/AuA2EAgAegE}{Numberphile YouTube video about $e$}. 
%(url: youtu.be/AuA2EAgAegE)
\end{myDefinition}



\begin{myDefinition}[Continuously Compounded Interest]~\\[0.5mm]
The formula to calculate {\bf continuously compounded interest} is $${A = P e^{rt}}$$
where $A$ is total amount of money owed or earned,\\
$P$ is the principal, \\
$r$ is the nominal annual interest rate, and \\
$t$ is the number of years that the interest is earned. \\
Note: This formula is used when the compounding happens continuously.
\end{myDefinition}



\begin{myDefinition}[Effective Interest Rate]~\\[0.5mm]
The \textbf{effective interest rate, ${r_e}$} is the equivalent interest rate that, if compounded annually, would yield the same result after 1 year as compounding the stated nominal rate $n$ times per year or compounding the nominal rate continuously.\\[0.5em]
When compounding $n$ times per year, $r_e = \left(1 + \frac{r}{n}\right)^n - 1$, where $r$ is the nominal annual interest rate.\\
When compounding continuously, $r_e = e^r - 1$, where $r$ is the nominal annual interest rate.\\


\end{myDefinition}




%======================================================
 \newpage
%======================================================

\subsection*{Exit Exercises} \label{exit-exponential-compounding}

\begin{myExit}
In the year 2001, Alyssa opened a retirement account that earns a nominal interest rate 7.25\% per year. Her initial deposit was \$13,500. How much will the account be worth in 2025 if interest compounds monthly? How much more would she make if interest compounded continuously?\end{myExit}
\vfill

\begin{myExit}
Kyoko received a \$10,000 scholarship that she gets to invest for college.  Her bank has several investment accounts to choose from, all compounding daily.  Her goal is to have \$15,000 by the time she finishes high school in 6 years.  To the nearest hundredth of a percent, what should her minimum nominal interest rate be in order to reach her goal? 
\end{myExit}
\vfill

\begin{myExit}
A small business is planning on building a new facility in 9 years. They can invest money at 7\% annual interest, compounded daily, right now. If they need \$600,000 to build the new facility in 9 years, what is the minimum they need to invest to ensure they have \$600,000 in 9 years?
\end{myExit}
\vfill





\exitlikert{exponential functions}











\resetCounters
%============================================================
% MTH 111Z Project - Template File
% Part of Section 6.1 from OpenStax OER
%	Updated 202302
%============================================================

\section{Exponential and Logarithmic Models} \label{exponential-and-logarithmic-models}

In this section, we'll investigate a few ways that exponential and logarithmic functions are used to model the world around us, as well as how logarithms help us to quantify sound levels, acidity, and earthquakes.\\[0.5em]
\textbook{6.7}


\subsection*{Preparation Exercises} \label{prep-exponential-and-logarithmic-models}


\begin{myPrep}
Find the formula for an exponential function $f$ that satisfies $f(-2) = \dfrac{16}{27}$ and $f(3)=\dfrac{9}{2}$.	
\vfill
\end{myPrep}

\begin{myPrep}
The substance Einsteinium-253 decays at a continuous rate of about 3.3862\% per day.  If a scientist starts with a 60 mg sample of Einsteinium-253, how long until there are only 30 mg remaining?
\vfill
\end{myPrep}

\begin{myPrep}
The function $V(t) = 32.8(1.093)^t$ represents the value (in thousands of dollars) of a collectable car $t$ years after June 1st, 2015.  How long will it take for the value of the car to reach \$65,600?
\vfill
\end{myPrep}






%======================================================
\newpage
%======================================================


\subsection*{Practice Exercises} \label{practice-exponential-and-logarithmic-models}

\begin{myPractice}
A doctor prescribes 175 milligrams of a drug that decays by 20\% each hour.
\begin{enumerate}
\item Write an exponential model representing $D$, the amount of the drug (in mg) remaining in the patient's system, $t$ hours after having the drug administered.  (Note: 20\% per hour is not a continuous rate.)\\[0.35in]
	\item To the nearest hour, what is the half-life of the drug?
\vfill
	\item When will there be 25 mg of the drug remaining in the patient's system?
\vfill
\end{enumerate}\end{myPractice}


\begin{myPractice}
%https://www.statcan.gc.ca/en/subjects-start/population_and_demography/40-million
In 2022, Canada's population\footnote{Growth rate obtained from \url{https://www.statcan.gc.ca/en/subjects-start/population_and_demography/40-million}} grew by about 2.7\%.  If Canada's population were to maintain that same growth rate, how long would it take for Canada's population to double?
\vfill
\end{myPractice}

\begin{myPractice}
Cobalt-60\footnote{Half-life value obtained from \url{https://www.britannica.com/science/cobalt-60}} is used for radiotherapy and has a half-life of 5.26 years.  Find the continuous annual rate of decay.
\vfill
\end{myPractice}




%======================================================
\newpage
%======================================================

\subsection*{Definitions} \label{def-exponential-and-logarithmic-models}

\begin{myDefinition}[Doubling Time]~\\[0.5mm]
The amount of time it takes an exponential growth model to increase to double the starting value.
\end{myDefinition}

\begin{myDefinition}[Half-Life]~\\[0.5mm]
The amount of time it takes an exponential decay model to decay to half of the starting value.
\end{myDefinition}


\begin{myDefinition}[Exponential Function of the Form $\boldsymbol{f(t)=ae^{kt}}$:]~\\[0.5mm]
For any real number $t$, an {\bf exponential function with the form} $\boldsymbol{f(t) = a e^{kt}}$ where
	\begin{itemize}
	\setlength{\itemsep}{0in}
		\item $a$ is a non-zero real number
		\item $k$ is any non-zero real number and is the continuous growth rate.
	\end{itemize}
Exponential functions of the form $f(t)=ae^{kt}$ are often referred to as a {\bf continuous growth model}.  
\end{myDefinition}


\begin{myDefinition}[Key Characteristics of $\boldsymbol{f(t) = ae^{kt}}$:]~\\[0.5mm]
For an exponential function $f(t) = ae^{kt}$, with $a> 0$, we have the following:\\[3mm]
\begin{minipage}{0.6\linewidth}
	\begin{itemize}
	\setlength{\itemsep}{1mm}
		\item $(0,a)$ is the vertical intercept.
		\item There is no horizontal intercept.
		\item The domain of $f$ is $(-\infty, \infty)$.
		\item The range of $f$ is $(0,\infty)$.
		\item $f$ is a one-to-one function.
		\item The horizontal asymptote is $y=0$.
		\item If $k>0$, then $f$ is an increasing function and
			\begin{itemize}
			\setlength{\itemsep}{0in}
				\item[\small{•}] as $t\rightarrow \infty$, $f(t)\rightarrow \infty$, and
				\item[\small{•}] as $t\rightarrow -\infty$, $f(t)\rightarrow 0$.
			\end{itemize}	
		\item If $k<0$, then $f$ is a decreasing function and
			\begin{itemize}
			\setlength{\itemsep}{0in}
				\item[\small{•}] as $t\rightarrow \infty$, $f(t)\rightarrow 0$, and
				\item[\small{•}] as $t\rightarrow -\infty$, $f(t)\rightarrow \infty$.
			\end{itemize}	
	\end{itemize}
	\end{minipage}
	\begin{minipage}{0.4\linewidth}
			\begin{center}
			\captionof{figure}{$y=ae^{kt}, k>0$}~\\[-0.8em]
			\begin{tikzpicture}
 				\begin{axis}[
 					framedaxes,
		 			height=5cm,
 					width=5cm,
 					xlabel={$t$},
 					ylabel={$y$},
		 			xmin=-6,xmax=6,
 					ymin=-6,ymax=6,
				        xtick={-16,16},
				       	minor xtick={-16,16},
				        ytick={-16,16},
	        		 	minor ytick={-16,16},
				         grid=both
 					]
		 		% use TeX as calculator:
				 	\addplot[first,line width=1.5pt,samples=200,<->]expression[domain=-6:2.584] {2^x};
					\addplot[first,line width=1.0pt,mark=*] coordinates {(0,1)};
%						\node at (axis cs:-1.5,1.5) {{\tiny $(0,a)$}};									
		 		\end{axis}
 				\end{tikzpicture}
				\label{fig:explog-def1}
%				\end{center}

%			\begin{center}
			\captionof{figure}{$y=ae^{kt}, k<0$}
			\begin{tikzpicture}
 				\begin{axis}[
 					framedaxes,
		 			height=5cm,
 					width=5cm,
 					xlabel={$t$},
 					ylabel={$y$},
		 			xmin=-6,xmax=6,
 					ymin=-6,ymax=6,
				        xtick={-16,16},
				       	minor xtick={-16,16},
				        ytick={-16,16},
	        		 	minor ytick={-16,16},
				         grid=both
 					]
		 		% use TeX as calculator:
				 	\addplot[first,line width=1.5pt,samples=200,<->]expression[domain=-2.584:6] {0.5^x};
					\addplot[first,line width=1.0pt,mark=*] coordinates {(0,1)};
%						\node at (axis cs:1.5,1.5) {{\tiny $(0,a)$}};									
		 		\end{axis}
 				\end{tikzpicture}
				\label{fig:explog-def2}
				\end{center}

	\end{minipage}
\end{myDefinition}



\newpage

\begin{myDefinition}[Decibels]~\\[0.5mm]
The loudness $L(x)$, measured in {\bf decibels (dB)}, of a sound of intensity $x$, measured in watts/m$^2$, is defined by $$L(x) = 10 \log \left( \frac{x}{I_0} \right),$$ where $I_0 = 10^{-12}$ watts/m$^2$ and approximately represents the least intense sound that a human ear can detect.
\end{myDefinition}

\begin{myDefinition}[pH]~\\[0.5mm]
The {\bf pH of a chemical solution} is used to measure the acidity or alkalinity of the solution.  

The formula used to calculate the pH of a solution is $$\text{pH}=-\log\left[\text{H}^+\right],$$ where $\left[\text{H}^+\right]$ is the concentration of hydrogen ions in moles per liter. 

pH values range from 0 (acidic) to 14 (basic), with 7 being neutral.
\end{myDefinition}


\begin{myDefinition}[Richter scale]~\\[0.5mm]
The Richter scale is one way of converting seismographic readings into numbers that provide a reference for measuring the magnitude $M$ of an earthquake.  All earthquakes are compared to a zero-level earthquake whose seismographic reading measures 0.001 millimeter at a distance of 100 kilometers from the epicenter.  

An earthquake whose seismographic reading measures $x$ millimeters has a magnitude $M(x)$, give by $$M(x)=\log\left(\dfrac{x}{x_0}\right),$$ where $x_0=10^{-3}$ is the reading of a zero-level earthquake the same distance from the epicenter.\\
\end{myDefinition}



%======================================================
 \newpage
%======================================================

\subsection*{Exit Exercises} \label{exit-exponential-and-logarithmic-models}

\begin{myExit}
	\begin{enumerate}
		\item For what type of exponential models would we discuss the half-life?  What does the half-life measure?
		\vfill
		\item For what type of exponential models would we discuss the doubling time?  What does the doubling time measure?
		\vfill
	\end{enumerate}
\end{myExit}


\begin{myExit}
A patient receives an injection of 20 mg of a medicine that decays exponentially.  45 minutes after the injection, there are 8 mg of medicine left in her body.   What is the half-life of this medication?
\vfill
\vfill
\end{myExit}

\begin{myExit}
A research student is working with a culture of bacteria that doubles in size every 13 minutes.  The initial population count was 1350 bacteria.  Find an exponential function that expresses the bacteria population, $P$, as a function of $t$, the number of {\bf hours} after the experiment began.
\vfill
\vfill
\end{myExit}


\exitlikert{exponential functions}















% Ch 5 of OER
 %============================================================
% MTH 111Z Project - Template File
% Chapter 3.1 from OpenStax OER
%	Updated 202302
%============================================================


%\fakesection{Functions and Function Notation} \label{functions-and-notation}

\chapter{Polynomial and Rational Functions} \label{chapter-polynomial-and-rational-functions}

\minitoc

\newpage

~ %intentional blank page
% Do we include Damian's suggestion on a check list of topics or reference to CCOG topics on the second/back page of the chapter page?

\iffalse

\setcounter{section}{-1}
\section{Learning Objectives}

By the end of Section 3.1, you should be able to:
\setlist{itemsep=0pt}
\begin{itemize}
	\item objective 1
	\item objective 2
	\item objective 3
	\item objective 4
\end{itemize}

By the end of Section 3.2, you should be able to:
\begin{itemize}
	\item objective 1
	\item objective 2
	\item objective 3
	\item objective 4
\end{itemize}





\setlist{itemsep=6pt}

\fi

\resetCounters
%============================================================
% MTH 111Z Project - Template File
% Section 5.2 from OpenStax OER
%	Updated 202302
%============================================================

\section{Polynomial Functions and Some Key Characteristics} \label{functions-polynomial-long-term}

In this section, we'll learn about the long-term behaviors of polynomial functions and how to identify this, as well as other key attributes of polynomial functions, based on their algebraic formula.     \\[0.5em]
\textbook{5.2}


\subsection*{Preparation Exercises} \label{prep-functions-polynomial-long-term}

\begin{myPrep}
~\\ \vspace{-12mm}
	\begin{enumerate}
		\begin{multicols}{2}
		\item Completely factor $2x^2+2x-12$.
		\item Completely factor $ 2x^2+3x-2$.
		\end{multicols}
		\vfill
		\vfill
		\item Algebraically find and state the $x$-intercepts of the graph of $f(x) = 2x^2+2x-12$.
		\vfill
		\item Algebraically find and state  the $x$-intercepts of the graph of $f(x) = 2x^2+3x-2$.
		\vfill
		\item Algebraically find and state  the $y$-intercept of the graph of $f(x) = 2x^2+3x-2$.
		\vfill
	\end{enumerate}
\end{myPrep}







%======================================================
\newpage
%======================================================


\subsection*{Practice Exercises} \label{practice-functions-polynomial-long-term}


\begin{myPractice}
Let $f(x) = \frac{1}{2} x^4 +3x^3 -16x$, which in factored form is $f(x) = \frac{1}{2}x(x-2)(x+4)^2$.
	\begin{enumerate}
		\item Describe the end behavior of the graph of $y=f(x)$.
		\vfill

		\item At most how many turning points does the graph of $y=f(x)$ have?
		\vfill

		\item What is the $y$-intercept of $y=f(x)$?
		\vfill

		\item What are the $x$-intercept(s) of $y=f(x)$?
		\vfill
	\end{enumerate}
\end{myPractice}



\begin{myPractice}
Let $g(x) = 2x^3-5x^2-4x+12$, which in factored form is $g(x) = (2x+3)(x-2)^2$.
	\begin{enumerate}
		\item Describe the end behavior of the graph of $y=g(x)$.
		\vfill

		\item At most how many turning points does the graph of $y=g(x)$ have?
		\vfill

		\item What is the $y$-intercept of $y=g(x)$?
		\vfill

		\item What are the $x$-intercept(s) of $y=g(x)$?
		\vfill
	\end{enumerate}
\end{myPractice}



%======================================================
\newpage
%======================================================

\subsection*{Definitions} \label{def-functions-polynomial-long-term}


\begin{myDefinition}[Power Function]~\\[0.5mm]
A {\bf power function} is a function that can be represented in the form $${f(x) = kx^p},$$
where $k$ and $p$ are real numbers.  $k$ is called the coefficient.
\end{myDefinition}

\begin{myDefinition}[Polynomial Function]~\\[0.5mm]
A \textbf{polynomial function} is of the form $${p(x) = a_n x^n + a_{n-1} x^{n-1} + \cdots + a_1 x + a_0},$$
where $a_n, ~a_{n-1}, ~\dots,~ a_1, ~a_0$ are real numbers, $a_n\neq 0$, and $n$ is a non-negative integer.
			
The \textbf{leading term} is the highest degree term, $a_n x^n$. 

The \textbf{degree} of the polynomial is $n$.
			
The \textbf{leading coefficient} is the coefficient of the leading term, $a_n$. \\[0.5em]
\defexample The polynomial $5x^2-8x+4$ has a leading term of $5x^2$, is second degree polynomial,  and has a leading coefficient of $5$
\end{myDefinition}


\begin{myDefinition}[$\boldsymbol{x}$-intercept or Horizontal Intercept]~\\[0.5mm]
A {\bf horizontal intercept} or {\bf $\boldsymbol{x}$-intercept} of a graph is a point where the graph intersects the horizontal or $x$-axis.  This occurs when the function has an output value of 0.\\[0.5em]
\defexample We can find the horizontal intercepts of the graph of $f(x) = 5x^2-8x+4$ by solving $f(x)=0$.
\end{myDefinition}

\begin{myDefinition}[$\boldsymbol{y}$-intercept or Vertical Intercept]~\\[0.5mm]
A {\bf vertical intercept} or {\bf $\boldsymbol{y}$-intercept} of a graph is a point where the graph intersects the vertical or $y$-axis.  This occurs when the function has an input value of 0.\\[0.5em]
\defexample We can find the vertical intercept of the graph of $f(x) = 5x^2-8x+4$ by evaluating $f(0)$.
\end{myDefinition}

\begin{myDefinition}[End Behavior or Long-Term Behavior]~\\[0.5mm]
The {\bf end behavior} or {\bf long-term behavior} of a polynomial function is determined by its leading term.  The long-term behavior of the polynomial function will be consistent with the power function that is the leading term of the polynomial.\\[0.5em]
\defexample The end behavior of $f(x) = \boldsymbol{5x^4}{}-7x^2+8x-9$ will be the same as the end behavior of $y=5x^4$.
\end{myDefinition}


\begin{myDefinition}[Turning Point]~\\[0.5mm]
A {\bf turning point} of a polynomial function is a point at which the graph changes from increasing to decreasing or from decreasing to increasing.\\[0.5em]
\defexample The graph of  $f(x) = 5x^4-7x^2+8x-9$ has at most 3 turning points.
\end{myDefinition}


\begin{myDefinition}[Root or Zero]~\\[0.5mm]
A {\bf root} or {\bf zero} of a polynomial function is a value $r$ for which $f(r)=0$.\\
$r$ is a zero of a polynomial function $f$ if and only if $(x-r)$ is a factor of $f$.\\[0.5em]
\defexample If $7$ is a zero of a polynomial, then $(x-7)$ is a factor of the polynomial.\\
\defexample  If $(x+3)$ is a factor of a polynomial, then $-3$ is a zero of the polynomial.
\end{myDefinition}




%======================================================
 \newpage
%======================================================

\subsection*{Exit Exercises} \label{exit-functions-polynomial-long-term}

\begin{myExit}
	\begin{enumerate}
		\item What does the degree of a polynomial function tell you about the graph of the function?
		\vfill
		\item What does the leading coefficient of a polynomial function tell you about the graph of the function?
		\vfill
		\item If you know the graph of a polynomial function has 7 turning points, what can you say about the degree of the function?
		\vfill
		\item What do we know about a polynomial function's formula if we know the function has the following behavior?\\
		As $x\rightarrow -\infty$, $f(x)\rightarrow \infty$ and as $x\rightarrow \infty$, $f(x)\rightarrow -\infty$.
		\vfill
	\end{enumerate}
\end{myExit}


\begin{myExit}
State the degree, leading coefficient, long-term behavior, $y$-intercept, and the $x$-intercepts for function \\$f(x) = \frac{1}{3}(x-4)(x+5)^2(x+1)$.
		\vfill
\end{myExit}
\vfill

\exitlikert{polynomial functions}











\newpage
~

\resetCounters
%============================================================
% MTH 111Z Project - Template File
% Section 5.3 from OpenStax OER
%	Updated 202302
%============================================================
\section{Graphs of Polynomial Functions} \label{functions-polynomial-graphs}

In this section, we'll focus on the short-term behaviors of polynomial functions and the location and behavior of the $x$-intercepts can be identified from their factored algebraic form.  We'll then use this knowledge to both graph polynomial functions, as well as construct polynomial functions based on a graph. \\[0.5em]
\textbook{5.3}




\subsection*{Preparation Exercises} \label{prep-functions-polynomial-graphs}


\begin{myPrep}
Let $f(x) =-2(x+3)(x-2)^2(x+1)$.
	\begin{enumerate}
		\item Describe the end behavior of the graph of $y=f(x)$.
		\vfill

		\item What are the $x$-intercept(s) of $y=f(x)$?
		\vfill

		\item What is the $y$-intercept of $y=f(x)$?
		\vfill
	
		\item At most how many turning points does the graph of $y=f(x)$ have?
		\vfill

	\end{enumerate}
\end{myPrep}

\begin{myPrep}

Let $g(x) =-2(x+3)^2(x-2)(x+1)^3$.
	\begin{enumerate}
		\item Describe the end behavior of the graph of $y=g(x)$.
		\vfill

		\item What are the $x$-intercept(s) of $y=g(x)$?
		\vfill

		\item What is the $y$-intercept of $y=g(x)$?
		\vfill
	
		\item At most how many turning points does the graph of $y=g(x)$ have?
		\vfill

	\end{enumerate}
\end{myPrep}





%======================================================
\newpage
%======================================================


\subsection*{Practice Exercises} \label{practice-functions-polynomial-graphs}

\begin{myPractice}
	Let $f(x) =\frac{1}{2}(x-2)(x+1)^2(x-4)$. \\
\begin{minipage}{0.7\linewidth}
	\begin{enumerate}
		\item What is the end behavior of the graph of $y=f(x)$?  Why?\\[0.25in]
		\item What are the $x$-intercepts and their behaviors? Why?\\[0.5in]
		\item What is the $y$-intercept?\\[0.25in]
		\item Sketch a graph of $y=f(x)$ in Figure~\ref{fig:poly-pr1} to the right.
	\end{enumerate}
\end{minipage}
\begin{minipage}{0.3\linewidth}
	\begin{center}
	\captionof{figure}{$y=f(x)$}~\\[-0.8em]
	\begin{tikzpicture}
 		\begin{axis}[
				framed,
				width=7cm, 
				height=7cm,				
				xlabel={},
				ylabel={},
				xmin=-8,xmax=8,
				ymin=-8,ymax=8,
				xtick={-16,16},
        	minor xtick={-17,-16,...,17},
				ytick={-16,16},
        	minor ytick={-17,-16,...,17},
        grid=both
 			]
 		% use TeX as calculator:
 		\end{axis}
 		\end{tikzpicture}
		\label{fig:poly-pr1}
	\end{center}
\end{minipage}


\vfill
\end{myPractice}



\begin{myPractice}
	Given in Figure~\ref{fig:poly-pr2} is the graph of a polynomial function $g$. \\
\begin{minipage}{0.35\linewidth}
	\begin{center}
	\captionof{figure}{$y=g(x)$}~\\[-0.8em]
	\begin{tikzpicture}
 		\begin{axis}[
 			framedaxes,
 			height=7cm,
 			width=7cm,
 			xlabel={$x$},
 			ylabel={$y$},
 			xmin=-8,xmax=8,
 			ymin=-8,ymax=8,
		        xtick={-6,-4,...,6},
		       	minor xtick={-11,-9,-7,...,11},
		        ytick={-6,-4,...,6},
	         	minor ytick={-7,-5,...,7},
		         grid=both
 			]
 		% use TeX as calculator:
			\addplot[first,line width=1.25pt,<->,samples=400]expression[domain=-4.153:4.44]{1/36*(x+3)^2*(x-1)^2*(x-4)};

 		\end{axis}
 		\end{tikzpicture}
		\label{fig:poly-pr2}
	\end{center}
\end{minipage}
\begin{minipage}{0.65\linewidth}
	\begin{enumerate}
		\item Based on the end behavior and other characteristics of the graph of $y=g(x)$, what are the possibilities for the degree of $g$?\\[0.25in]
		\item Based on the end behavior of the graph of $y=g(x)$, is the leading coefficient of $g$ positive or negative?\\[0.25in]
		\item Based on the $x$-intercepts, what are the linear factors of $g$ and their powers?\\[0.5in]
		\item What is a possible formula for the function $g$?\\[0.25in]
	\end{enumerate}
\end{minipage}


\vfill
\end{myPractice}



%======================================================
\newpage
%======================================================

\subsection*{Definitions} \label{def-functions-polynomial-graphs}

\begin{myDefinition}[Multiplicity]~\\[0.5mm]
The {\bf multiplicity} of a factor is the number of times that factor occurs in the factored version of the polynomial.  We will also refer to the {\bf multiplicity of the zero} for the zero associated with this factor.\\[0.5em]
\defexample For $f(x) = 3(x-5)(x+4)^2$, the multiplicity of $5$ is one and the multiplicity of $-4$ is two.\\[0.5em]
The sum of the multiplicities of the real roots for a polynomial function is less than or equal to the degree of the polynomial.
\end{myDefinition}

\begin{myDefinition}[Even Multiplicity]~\\[0.5mm]
When a root or zero has even multiplicity, then the graphical behavior at the associated $x$-intercept is that the graph will touch, but not cross, the $x$-axis at that $x$-intercept.
\end{myDefinition}

\begin{myDefinition}[Odd Multiplicity]~\\[0.5mm]
When a root or zero has odd multiplicity, then the graphical behavior at the associated $x$-intercept is that the graph will cross over the $x$-axis at that $x$-intercept.\\

When the multiplicity of a root is one, the graph will cross through the $x$-intercept in a somewhat linear manner.\\

When the multiplicity of a root is a larger odd number, the graph will cross through the $x$-intercept by flattening out as it crosses.\\

\begin{minipage}{0.9\linewidth}
\defexample \href{https://tiny.cc/111Z-Multiplicity}{View this Desmos graph} to see an interactive example how the multiplicity of a root of a polynomial function impacts the behavior of the related $x$-intercept.  %(url: tiny.cc/111Z-Multiplicity)
\end{minipage}
\begin{minipage}{0.1\linewidth}
\flushright \qrcode[height=1cm]{https://tiny.cc/111Z-Multiplicity}
\end{minipage}
\end{myDefinition}



%======================================================
 \newpage
%======================================================

\subsection*{Exit Exercises} \label{exit-functions-polynomial-graphs}

\begin{myExit}
	\begin{enumerate}
		\item What is the difference between a zero and an $x$-intercept of a polynomial function?
		\vfill
		\item How is the factored version of a polynomial function useful when graphing the function?  What does the factored version help us to be able to quickly identify?
		\vfill
		\item How is the expanded (non-factored) version of a polynomial function useful when graphing the function?  What does the non-factored version help us to be able to quickly identify?
		\vfill
		\item If the graph of a polynomial function touches, but doesn't cross, the $x$-axis at a point $(k,0)$, what do we know about the factored form of that function?
		\vfill
	\end{enumerate}
\end{myExit}

\begin{myExit}
Identify a possible formula for the polynomial function $F$ whose graph is in Figure~\ref{fig:poly-ex1}.\\
There is a point on the graph at $(-3,21)$.\\
\begin{minipage}{0.6\linewidth}
~
\end{minipage}
\begin{minipage}{0.4\linewidth}
	\begin{center}
	\captionof{figure}{$y=F(x)$}~\\[-0.8em]
	\begin{tikzpicture}
 		\begin{axis}[
 			framedaxes,
 			height=7cm,
 			width=7cm,
 			xlabel={$x$},
 			ylabel={$y$},
 			xmin=-8,xmax=8,
 			ymin=-32,ymax=32,
		        xtick={-6,-4,...,6},
		       	minor xtick={-11,-9,-7,...,11},
		        ytick={-28,-24,...,28},
%	         	minor ytick={-30,-28,...,30},
		         grid=both
 			]
 		% use TeX as calculator:
			\addplot[first,line width=1.25pt,<->,samples=400]expression[domain=-5.494:5.88]{1/42*x^2*(x+5)*(x-4)^2};
			\addplot[mark=*,first] plot coordinates {(-3,21)	};
 		\end{axis}
 		\end{tikzpicture}
		\label{fig:poly-ex1}
	\end{center}
\end{minipage}

\end{myExit}


\vfill

\exitlikert{polynomial functions}






























\resetCounters
%============================================================
% MTH 111Z Project - Template File
% Section 5.6 from OpenStax OER
%	Updated 202302
%============================================================


\section{Rational Functions and Some Key Characteristics} \label{functions-rational-short-term}

In this section, we'll look at rational functions and see how the concepts we explored with polynomial functions carry over to a function that is a fraction of two polynomials.   \\[0.5em]
\textbook{5.6}


\subsection*{Preparation Exercises} \label{prep-functions-rational-short-term}

\begin{myPrep}
	\begin{enumerate}
		\item For any real number $k$ other than 0, what is $\dfrac{0}{k}$ and why?
		\vfill
		\item For any real number $k$ other than 0, what is $\dfrac{k}{0}$ and why?
		\vfill
		\item What is the domain of $f(x) = \dfrac{3x^2-3x-36}{2x^2+2x-60}$?
		\vfill
	\end{enumerate}
\end{myPrep}

\begin{myPrep}
	Let $g(x) =-7(x+2)^2(x-1)^3(x+5)$. 
	\begin{enumerate}
		\item What are the $x$-intercepts and their behaviors?
		\vfill
		\vfill
		\item What is the $y$-intercept?
		\vfill
	\end{enumerate}
\end{myPrep}






%======================================================
\newpage
%======================================================


\subsection*{Practice Exercises} \label{practice-functions-rational-short-term}


\begin{myPractice}
Let $r(x) = \dfrac{p(x)}{q(x)}$.
\begin{enumerate}
	\item What usually happens on the graph of $r(x)$ when $p(x)=0$?
	\vfill
	\item What usually happens on the graph of $r(x)$ when $q(x)=0$?
	\vfill
	\end{enumerate}
\end{myPractice}

\begin{myPractice}
Let $R(x) = \dfrac{5(x+3)(x-4)^2}{(x-2)^2(x+1)}$.  
\begin{enumerate}
\begin{multicols}{2}
\item What is the domain of $R$?
\item What is the $y$-intercept of $R$?
\end{multicols}
\vfill
\begin{multicols}{2}
\item What are the zeros of $R$?
\item What are the $x$-intercepts of $R$?
\end{multicols}
\vfill
\begin{multicols}{2}
\item What happens when $x=-1$?
\item What happens when $x=2$?
\end{multicols}
\vfill

\end{enumerate}
\end{myPractice}

%%%%%%%%%%
\newpage

\begin{myPractice}
When working with a rational function, 
\begin{enumerate}
	\item What is the behavior of a vertical asymptote that comes from a root of the denominator with odd multiplicity?  Include at least two sketches of possible examples in your answer.
	\vfill
	\item What is the behavior of a vertical asymptote that comes from a root of the denominator with even multiplicity?  Include at least two sketches of possible examples in your answer.
	\vfill
	\end{enumerate}
\end{myPractice}



\begin{myPractice}~\\[-0.5em]
\begin{minipage}{0.55\linewidth}
Sketch a graph of a function $y=F(x)$ in Figure~\ref{fig:rat-pr1} that has:
	\begin{itemize}
		\item a vertical asymptote at $x=-2$ with even multiplicity, 
		\item a $y$-intercept at $(0,-2)$, 
		\item an $x$-intercept at $(-4,0)$ with even multiplicity, 
		\item an $x$-intercept at $(3,0)$ with odd multiplicity, and
		\item there are no other vertical asymptotes or $x$-intercepts.
	\end{itemize}
~\\~\\~\\~\\
\end{minipage}
\begin{minipage}{0.05\linewidth}
~
\end{minipage}
\begin{minipage}{0.4\linewidth}
\begin{center}
	\captionof{figure}{$y=F(x)$}~\\[-0.8em]
	\begin{tikzpicture}
	\begin{axis}[
				framed,
				legend cell align=left, legend pos=north west,
				width=7cm, 
				height=7cm,				
				xlabel={},
				ylabel={},
				xmin=-6,xmax=6,
				ymin=-6,ymax=6,
				xtick={-16,16},
        	minor xtick={-7,-6,...,7},
				ytick={-16,16},
        	minor ytick={-7,-6,...,7},
        grid=both
				]
		\end{axis}
		\end{tikzpicture}
		\label{fig:rat-pr1}
\end{center}
\end{minipage}
\end{myPractice}
\vfill


%======================================================
\newpage
%======================================================

\subsection*{Definitions} \label{def-functions-rational-short-term}


\begin{myDefinition}[Rational Function]~\\[0.5mm]
A {\bf rational function} is a function that can be written as the quotient of two polynomial functions, where the denominator is not 0.
$$r(x)=\dfrac{p(x)}{q(x)}, \text{where $p$ and $q$ are polynomial functions and $q(x)\neq0$}$$
\end{myDefinition}

\begin{myDefinition}[Vertical Asymptote]~\\[0.5mm]
Given a function $f$ and a real number $a$, a {\bf vertical asymptote} of the graph of $y=f(x)$ is a vertical line $x=a$ where $f(x)$ tends toward positive or negative infinity as $x$ approaches $a$ from either the left or the right. We write this as:
$$\text{As $x\rightarrow a^{-}$, $f(x) \rightarrow \pm \infty$ or as $x\rightarrow a^{+}$, $f(x) \rightarrow \pm \infty$.}$$
\defexample $x=1$ is a vertical asymptote of $f(x)=\dfrac{x-5}{x-1}$ due to the following:
\begin{itemize}
\item As $x$ approaches $x=1$ from the left, the $y$-values increase towards $\infty$.\\
Mathematically, we write this: As $x\rightarrow 1^{-}$, $f(x)\rightarrow\infty$.
\item As $x$ approaches $x=1$ from the right, the $y$-values decrease towards $-\infty$.\\
Mathematically, we write this: As $x\rightarrow 1^{+}$, $f(x)\rightarrow -\infty$.
\end{itemize}
\begin{minipage}{0.9\linewidth}
\href{https://tiny.cc/111Z-VertAsymp}{View this Desmos graph} to see an interactive version of this example.  %(url: tiny.cc/111Z-VertAsymp)
\end{minipage}
\begin{minipage}{0.1\linewidth}
\flushright \qrcode[height=1cm]{https://tiny.cc/111Z-VertAsymp}
\end{minipage}
\end{myDefinition}


\begin{myDefinition}[Multiplicity and Vertical Asymptotes]~\\[0.5mm]
The multiplicity of a root of the denominator of a rational function impacts the behavior of the vertical asymptote.\\

If the rational function is factored and reduced to its simplest terms, {\bf a root of the denominator with even multiplicity} will produce a vertical asymptote that approaches $\infty$ on both sides or that approaches $-\infty$ on both sides.\\

If the rational function is factored and reduced to its simplest terms, {\bf a root of the denominator with odd multiplicity} will produce a vertical asymptote that approaches $\infty$ on one side and $-\infty$ on the other.  \\

\begin{minipage}{0.9\linewidth}
\defexample \href{http://tiny.cc/111Z-MultiplicityAndVA}{View this Desmos graph} to see an interactive example of this concept.  %(url: tiny.cc/111Z-MultiplicityAndVA)
\end{minipage}
\begin{minipage}{0.1\linewidth}
\flushright \qrcode[height=1cm]{http://tiny.cc/111Z-MultiplicityAndVA}
\end{minipage}
\end{myDefinition}



\begin{myDefinition}[Horizontal Asymptote]~\\[0.5mm]
Given a function $f$ and a real number $L$, a {\bf horizontal asymptote} of a graph of $y=f(x)$ is a horizontal line $y=L$ where $f(x)$ tends toward $L$  as $x$ approaches $\infty$ or as $x$ approaches $-\infty$. We write this as:
$$\text{As $x\rightarrow \infty$ or as $x\rightarrow -\infty$, $f(x) \rightarrow L$.}$$
\begin{minipage}{0.9\linewidth}
\defexample \href{http://tiny.cc/111Z-HorizAsymp}{View this Desmos graph} to see an interactive example of the definition.  %(url: tiny.cc/111Z-HorizAsymp)
\end{minipage}
\begin{minipage}{0.1\linewidth}
\flushright \qrcode[height=1cm]{http://tiny.cc/111Z-HorizAsymp}
\end{minipage}
\end{myDefinition}





%======================================================
 \newpage
%======================================================

\subsection*{Exit Exercises} \label{exit-functions-rational-short-term}


\begin{myExit}
	\begin{enumerate}
		\item Explain when and why rational functions have vertical asymptotes?
		\vfill
		\item If $(-5,0)$ is an $x$-intercept of a rational function $R$, what do you know about the formula for $R(x)$?
		\vfill
		\item If $x=4$ is a vertical asymptote of a rational function $R$, what do you know about the formula for $R(x)$?
		\vfill
	\end{enumerate}
\end{myExit}

\begin{myExit}
Let $R(x)=\dfrac{4x(x-7)^2(x+4)}{5(x+7)(x+1)}$.
\begin{enumerate}
	\item What is the domain of $R$?  Answer using both interval and set-builder notations.
\vfill
\begin{multicols}{2}
	\item What are the $x$-intercepts of the graph of $R$?
	\item What are the vertical asymptotes of the graph of $R$?
\end{multicols}
\vfill
	\end{enumerate}
\end{myExit}




\exitlikert{rational functions}




























% %======================================================
% \newpage
% %======================================================

% \begin{myExample}
% What are the domains of the following functions?
% 	\begin{enumerate}
% 		\begin{multicols}{2}
% 		\item $f(x) = 7x(x+8)(x-3)$
% 		\item $g(t) = \dfrac{7t(t+8)}{t-3}$
% 		\end{multicols}
% 		\vfill
% 		\begin{multicols}{2}
% 		\item $j(n) = \sqrt{12-3n}$
% 		\item $k(w) = w^2 -8w-9$
% 		\end{multicols}
% 		\vfill
% 		\begin{multicols}{2}
% 		\item $m(a) = \dfrac{5a-9}{2a+8}$
% 		\item $r(u) = \dfrac{\sqrt{u+3}}{u-7}$
% 		\end{multicols}
% 		\vfill
% 		\vfill

% 	\end{enumerate}
% \end{myExample}


% {\bf Textbook Reference:}\\
% If you need a few more examples of set-builder and interval notations, \\please see the \href{https://openstax.org/books/algebra-and-trigonometry-2e/pages/3-2-domain-and-range#fs-id1165137677916}{Using Notations to Specify Domain and Range} section of \S3.2 in our book.

% %======================================================
% \newpage
% %======================================================

% \begin{center}
%   {\Large {\bf  Part 2: Finding Domains and Ranges from Graphs}}
% \end{center}

% \begin{myExample}
% Let $y=f(x)$ be defined by the graph below.  (Are you sure this is a function?  Why?)\\

% \begin{minipage}{0.4\linewidth}
% \begin{center}
% \begin{tikzpicture}
% 		\begin{axis}[
% 				framed,
% 				width=7cm,height=7cm,
% 				axis x line=middle,
% 				axis y line=middle,
% 				xlabel={$x$},
% 				ylabel={$y$},
% 				xmin=-8,xmax=8,
% 				ymin=-8,ymax=8,
%         xtick={-6,-4,...,6},
%         	minor xtick={-7,-5,...,7},
%         ytick={-6,-4,...,6},
%         	minor ytick={-7,-5,...,7},
%         grid=both
% 				]
% 				% use TeX as calculator:
% 					\addplot[smooth,mark=*,first,line width=1.0pt,fill=white]coordinates{	(-4,2)	};
% 					\addplot[smooth,mark=*,first,line width=1.0pt]coordinates{	(5,-4)	};
% 				\addplot[first,line width=1.5pt,samples=200]expression[domain=-4:4.99999]{2*(-x+5)^(1/2)-4};
% 		\end{axis}
% 		\end{tikzpicture}
% ~\\
% \end{center}
% \end{minipage}
% \begin{minipage}{0.6\linewidth}
% \begin{enumerate}
% 	\item Find $f(1)$.\\
% 	\item Find $f(5)$.\\
% 	\item Find $f(-4)$.\\
% 	\item Find $f(6)$.\\
% \end{enumerate}
% \end{minipage}

% \begin{enumerate}
% \setcounter{enumi}{4}
% 	\item State the domain of this function in interval notation and set-builder notation.\\~\\
% 	\item State the range of this function in interval notation and set-builder notation.\\~\\
% \end{enumerate}
% \end{myExample}


% \begin{myExample}
% Use $y=h(t)$ and $y=k(x)$ to answer the following.
% \vspace{-1cm}
% \begin{multicols}{2}
% \begin{center}
% \captionof{figure}{$y=h(t)$}
% \begin{tikzpicture}
% 		\begin{axis}[
% 				framed,
% 				width=6cm,height=6cm,
% 				xlabel={$t$},
% 				ylabel={$y$},
% 				xmin=-8,xmax=8,
% 				ymin=-8,ymax=8,
%         xtick={-6,-4,...,6},
%         	minor xtick={-7,-5,...,7},
%         ytick={-6,-4,...,6},
%         	minor ytick={-7,-5,...,7},
%         grid=both
% 				]
% 				% use TeX as calculator:
% 				\addplot[blue,line width=1.5pt,samples=400,-]expression[domain=-7:-4]{-x-4};
% 					\addplot[smooth,mark=*,blue,line width=2.0pt]coordinates{	(-7,3)	};
% 				\addplot[blue,line width=1.5pt,samples=400,-]expression[domain=-4:-3]{x+4};
% 				\addplot[blue,line width=1.5pt,samples=400,-]expression[domain=-3:-2]{(x+3)^2+1};
% 				\addplot[blue,line width=1.5pt,samples=400,-]expression[domain=-2:0]{-(x+1)^2+3};
% 				\addplot[blue,line width=1.5pt,samples=400,-]expression[domain=0:2]{-x+2};
% 				\addplot[blue,line width=1.5pt,samples=400,-]expression[domain=2:4]{(x-3)^2-1};
% 				\addplot[blue,line width=1.5pt,samples=400,-]expression[domain=4:6]{x-4};
% 					\addplot[smooth,mark=*,blue,line width=1pt,fill=white]coordinates{	(6,2)	};
% 		\end{axis}
% 		\end{tikzpicture}
% 		\label{fig:dr1}
% \end{center}

% \begin{center}
% \captionof{figure}{$y=k(x)$}
% \begin{tikzpicture}
% 		\begin{axis}[
% 				framed,
% 				width=6cm,height=6cm,
% 				xlabel={$x$},
% 				ylabel={$y$},
% 				xmin=-8,xmax=8,
% 				ymin=-8,ymax=8,
%         xtick={-6,-4,...,6},
%         	minor xtick={-7,-5,...,7},
%         ytick={-6,-4,...,6},
%         	minor ytick={-7,-5,...,7},
%         grid=both
% 				]
% 				% use TeX as calculator:
% 				\addplot[blue,line width=1.5pt,samples=400,-]expression[domain=-7:-5.99]{x+1};
% 				\addplot[smooth,mark=*,blue,line width=1pt,fill=white]coordinates{	(-7,-6)	};
% 				\addplot[blue,line width=1.5pt,samples=400,-]expression[domain=-6.01:-3.99]{2*(x)+7};
% 				\addplot[blue,line width=1.5pt,samples=400,-]expression[domain=-4.01:-2.99]{(x)+3};
% 				\addplot[blue,line width=1.5pt,samples=400,-]expression[domain=-3.01:-0.99]{-1/2*(x)-1.5};
% 				\addplot[blue,line width=1.5pt,samples=400,-]expression[domain=-1.01:0]{2*(x)^2-3};
% 				\addplot[blue,line width=1.5pt,samples=400,-]expression[domain=0:1.01]{-(x)^2-3};
% 				\addplot[blue,line width=1.5pt,samples=400,-]expression[domain=0.99:2]{-(x-2)^2-3};
% 				\addplot[blue,line width=1.5pt,samples=400,-]expression[domain=2:3.01]{2*(x-2)^2-3};
% 				\addplot[blue,line width=1.5pt,samples=400,-]expression[domain=2.99:5.01]{(x-3)^2-1};
% 				\addplot[blue,line width=1.5pt,samples=400,-]expression[domain=4.99:6]{3};
% 				\addplot[smooth,mark=*,blue,line width=1.5pt]coordinates{	(6,3)	};
% 		\end{axis}
% 		\end{tikzpicture}
% 		\label{fig:dr2}
% \end{center}
% \end{multicols}

% 		\begin{enumerate}\setlength{\itemsep}{0.85in}
% 		\begin{multicols}{2}
% 				\item		State the domain of $h$.
% 				\item		State the domain of $k$.
% 		\end{multicols}
% \vfill
% 		\begin{multicols}{2}
% 				\item		State the range of $h$.
% 				\item		State the range of $k$.
% 		\end{multicols}
% \vfill
% 		\end{enumerate}	
% \end{myExample}

% %======================================================
% \newpage
% %======================================================

% \begin{center}
%   {\Large {\bf  Part 3: Piecewise-Defined Functions}}
% \end{center}


% \begin{myDefinition} 
% A function that is defined by different formulas on different parts of its domain is called a\\
% {\bf \underline{piecewise-defined function}}.  Piecewise-defined functions take on the form:\\  
% 				\[
% 				f(x)=
% 					\begin{cases}
% 							\text{formula \#1}				&	\textrm{if}\ 	\ \			\text{$x$ is in this part of the domain}		\\
% 							\text{formula \#2}				&	\textrm{if}\ 	\ \			\text{$x$ is in this part of the domain}		\\
% 							\text{etc.}				&	\textrm{if}\ 	\ \			\text{etc.}		\\
% 					\end{cases}
% 				\]
% \ 

% \end{myDefinition}

% \begin{myExample}
% In Table~\ref{tab:tax1}, the 2011 federal income tax rates\footnote{\url{http://www.irs.gov/newsroom/article/0,,id=233465,00.html}} for 2011 are shown.  

% \renewcommand{\arraystretch}{1.5}
% \begin{table}[h!]
% \centering
% \caption{Federal Income Tax Percentage Rates for 2011}
% \label{tab:tax1} \small{
% 	\begin{tabular}{| c | c | c |}
% 	\hline 
% 		Income Amount $(x)$										& Percentage of Income Taxed (in $\%$) & Tax on $x$ Dollars of Income ($T(x)$, in \$) \\ \hline
% 		\	$0 \le x < 8500$										&				10		&	$0.10x$								\\ \hline			
% 			$8500 \le x < 34500$								&				15			&	$0.15x$							\\ \hline			
% 			$34500 \le x < 83600$								&				25			&	$0.25x$							\\ \hline			
% 			$83600 \le x < 174400$							&				28				&	$0.28x$						\\ \hline										
% 			$174400 \le x < 379150 $						&				33					&	$0.33x$					\\ \hline				
% 			$x \ge 379150$											&				35	&	$0.35x$									\\ \hline					
% 	\end{tabular}}
% \end{table}

% Notice that for each interval, the percentage of income taxed as a function of income is \emph{constant}.  If we graph each \emph{piece} over its respective interval, we obtain the following: \\
% 		\begin{minipage}{0.5\linewidth}

% \begin{center}
% 	\captionof{figure}{Graph of $y=T(x)$}
% 	\label{fig:b}
% 	   	\begin{tikzpicture}
% 			\begin{axis}[
% 				framed,
% 				height=3in,
% 				width=3in,
% 				xmin=0,xmax=500,
% 				ymin=0,ymax=180,
% 					xlabel style={at={(ticklabel cs:0.5)},anchor=near ticklabel},
%       		ylabel style={at={(ticklabel cs:0.5)},rotate=90,anchor=near ticklabel},
% 				xlabel={\tiny{$x$, income amount in \$1000s}},
% 				ylabel={\tiny{ $y$, tax amount in \$1000s}},
% 				xtick={0,50,100,...,450},
% 					 minor xtick={25,75,...,475},
% 				ytick={10,20,...,180},
% 				%	  minor ytick={1,2,3,...,39},
% 				grid=both,
% 				]
% 				% use TeX as calculator:
% 				\addplot[blue, line width=1.25pt]expression[domain=0:8.5,samples=200]{0.1*x};
% 						\addplot[smooth,mark=*,blue,line width=1.25pt]coordinates{	(0,0)	};
% 						\addplot[smooth,mark=*,blue,line width=1.25pt,fill=white]coordinates{	(8.5,0.85)	};
% 				\addplot[blue, line width=1.25pt]expression[domain=8.5:34.5,samples=200]{0.15*x};
% 						\addplot[smooth,mark=*,blue,line width=1.25pt]coordinates{	(8.5,1.275)	};
% 						\addplot[smooth,mark=*,blue,line width=1.25pt,fill=white]coordinates{	(34.5,5.175)	};
% 				\addplot[blue, line width=1.25pt]expression[domain=34.5:83.6,samples=200]{0.25*x};
% 						\addplot[smooth,mark=*,blue,line width=1.25pt]coordinates{	(34.5,8.625)	};
% 						\addplot[smooth,mark=*,blue,line width=1.25pt,fill=white]coordinates{	(83.6,20.9)	};
% 				\addplot[blue, line width=1.25pt]expression[domain=83.6:174.4,samples=200]{0.28*x};
% 						\addplot[smooth,mark=*,blue,line width=1.25pt]coordinates{	(83.6,23.408)	};
% 						\addplot[smooth,mark=*,blue,line width=1.25pt,fill=white]coordinates{	(174.4,48.832)	};
% 				\addplot[blue, line width=1.25pt]expression[domain=174.4:379.15,samples=200]{0.33*x};
% 						\addplot[smooth,mark=*,blue,line width=1.25pt]coordinates{	(174.4,57.552)	};
% 						\addplot[smooth,mark=*,blue,line width=1.25pt,fill=white]coordinates{	(379.15,125.1195)	};
% 				\addplot[blue, line width=1.25pt]expression[domain=379.15:500,samples=200,->]{0.35*x};
% 						\addplot[smooth,mark=*,blue,line width=1.25pt]coordinates{	(379.15,132.7025)	};
% 	\end{axis}
% 		\end{tikzpicture}
% \end{center}



% 		\end{minipage}
% 		\begin{minipage}{0.5\linewidth}
% Write the formula for $T(x)$.\\~\\~\\~\\~\\~\\~\\~\\~\\~\\~\\

% 		\end{minipage}
% \end{myExample}




% %======================================================
% \newpage
% %======================================================

% \begin{myExample}
% Consider the function $g(x) = |x|$, the absolute value function.
% \begin{enumerate}
% 	\item When you first learned about the absolute value, how did you describe what is measured?\\~\\~\\
% 	\item In MTH 95 (Intermediate Algebra), what was another way you learned that you could write $g$?\\~\\~\\
% 	\item If you had to explain to someone in words how to computer an absolute value, what would you say?\\~\\~\\
% 	\item The graph of $y=g(x)$ (the absolute value function) is given in Figure~\ref{fig:absval}.  \\
% 	Write this formula for this function as a piecewise-defined function.\\
% \begin{minipage}{0.5\linewidth}
% Formula: \\[4cm]

% \end{minipage}
% \begin{minipage}{0.5\linewidth}
% \begin{center}
% \captionof{figure}{}
% \begin{tikzpicture}
% 		\begin{axis}[
% 				framed,
% 				height=8cm,
% 				width=8cm,
% 				xlabel={$x$},
% 				ylabel={$y$},
% 				xmin=-8,xmax=8,
% 				ymin=-8,ymax=8,
%         xtick={-6,-4,...,6},
%         	minor xtick={-7,-5,...,7},
%         ytick={-6,-4,...,6},
%         	minor ytick={-7,-5,...,7},
%         grid=both
% 				]
% 				% use TeX as calculator:
% 				\addplot[blue,line width=1.25pt,samples=200,<-]expression[domain=-8:0]{-1*x};
% 				\addplot[blue,line width=1.25pt,samples=200,->]expression[domain=0:8]{x};
% 		\end{axis}
% 		\end{tikzpicture}
% 		\label{fig:absval}
% \end{center}
% \end{minipage}
% \end{enumerate}

% \end{myExample}


% %======================================================
% \newpage
% %======================================================




% \begin{myExample}

% Let $k$ and $m$ be as defined below.
% %\begin{minipage}{0.6\linewidth}
% \vspace{-1cm}
% \begin{multicols}{2}
% \begin{center}
% 				\[
% 				k(t)=
% 					\begin{cases}
% 						-2t-4								& \textrm{if}\ 	\ \			t \le -1			\\
% 						 2t+3							& \textrm{if}\ 	\ \			t > -1 		
% 					\end{cases}
% 				\]  
% \end{center}
% \begin{center}
% 				\[
% 				m(x)=
% 					\begin{cases}
% 						 x-3								& \textrm{if}\ 	\ \			0 < x \le 4 		\\
% 						-x-4								& \textrm{if}\ 	\ \			x \le 0	
% 					\end{cases}
% 				\]
% \end{center}
% \end{multicols}
% %\end{minipage}

% \begin{enumerate}
% \begin{multicols}{2}
% \item What is the domain of $k$?\\~\\~\\
% \item What is the domain of $m$?\\~\\~\\
% \end{multicols}
% \begin{multicols}{2}
% \item Find $k(10)$.\\~\\~\\~\\
% \item Find $m(3)$.\\~\\~\\~\\
% \end{multicols}
% \item Find $k(-9)$.\\~\\~\\~\\
% \item Find $m(0)$.\\~\\~\\~\\
% \begin{multicols}{2}
% \item Graph $y=k(t)$ in Figure~\ref{fig:pwd12} below.
% \item Graph $y=m(x)$ in Figure~\ref{fig:pwd22} below.
% \end{multicols}

% \vspace{-1cm}
% \begin{multicols}{2}
% \begin{center}
% \captionof{figure}{$y=k(t)$}
% \begin{tikzpicture}
% 		\begin{axis}[
% 				framed,
% 				height=7cm,
% 				width=7cm,
% 				xmin=-8,xmax=8,
% 				ymin=-8,ymax=8,
%         xtick={0},
%         	minor xtick={-7,-6,...,7},
%         ytick={0},
%         	minor ytick={-7,-6,...,7},
%         grid=both
% 				]
% 				% use TeX as calculator:
% 		\end{axis}
% 		\end{tikzpicture}
% 		\label{fig:pwd12}
% \end{center}

% \begin{center}
% \captionof{figure}{$y=m(x)$}
% \begin{tikzpicture}
% 		\begin{axis}[
% 				framed,
% 				height=7cm,
% 				width=7cm,
% 				xmin=-8,xmax=8,
% 				ymin=-8,ymax=8,
%         xtick={0},
%         	minor xtick={-7,-6,...,7},
%         ytick={0},
%         	minor ytick={-7,-6,...,7},
%         grid=both
% 				]
% 				% use TeX as calculator:
% 		\end{axis}
% 		\end{tikzpicture}
% 		\label{fig:pwd22}
% \end{center}
% \end{multicols}

% \end{enumerate}

% \end{myExample}


% %======================================================
% \newpage
% %======================================================


% \begin{myExample}
% In Figure~\ref{fig:piecewise4} is the graph of $y=F(x)$.	\\
% \begin{enumerate}%\setlength{\itemsep}{1.25in}
% \begin{minipage}[!t]{0.4\linewidth}
% \item Write the formula for the piecewise function $F$: \\[6cm]
% \end{minipage}
% \begin{minipage}{0.6\linewidth}
% \begin{center}
% \captionof{figure}{$y=F(x)$}
% \begin{tikzpicture}
% 		\begin{axis}[
% 				framed,
% 				height=8cm,
% 				width=8cm,
% 				xlabel={$x$},
% 				ylabel={$y$},
% 				xmin=-8,xmax=8,
% 				ymin=-8,ymax=8,
%         xtick={-6,-4,...,6},
%         	minor xtick={-7,-5,...,7},
%         ytick={-6,-4,...,6},
%         	minor ytick={-7,-5,...,7},
%         grid=both
% 				]
% 				% use TeX as calculator:
% 				\addplot[blue,line width=1.25pt,samples=200,<-]expression[domain=-8:-2]{1/2*(x+2)-3};
% 						\addplot[mark=*,blue,line width=1.0pt]coordinates{	(-2,-3)	};
% 				\addplot[blue,line width=1.25pt,samples=200]expression[domain=-2:3]{5};
% 						\addplot[mark=*,blue,line width=1.0pt,fill=white]coordinates{	(-2,5)	};
% 						\addplot[mark=*,blue,line width=1.0pt]coordinates{	(3,5)	};
% 				\addplot[blue,line width=1.25pt,samples=200]expression[domain=3:7]{-2*(x-6)-5};
% 						\addplot[mark=*,blue,line width=1.0pt,fill=white]coordinates{	(3,1)	};
% 						\addplot[mark=*,blue,line width=1.0pt,fill=white]coordinates{	(7,-7)	};
% 		\end{axis}
% 		\end{tikzpicture}
% 		\label{fig:piecewise4}
% \end{center}
% \end{minipage}
% %\vspace{11pt}
% \begin{multicols}{2}
% 	\item	Evaluate $F(0)$.	

% 	\item	Solve $F(x)=0$.	
% \end{multicols}
% ~\\
% \begin{multicols}{2}
% 	\item	Evaluate $F(2)$.
% 	\item	Solve $F(x)=-5$.	
% \end{multicols}
% ~\\

% 	\item	Solve $F(x)=-5$ both algebraically (using your formula from Part a) and graphically.	
% \vfill
% \end{enumerate}

% \end{myExample}












\resetCounters
%============================================================
% MTH 111Z Project - Template File
% Section 5.6 from OpenStax OER
%	Updated 202302
%============================================================

\section{Graphs of Rational Functions} \label{functions-rational-graphs}

In this section, we'll continue working with rational functions.  We'll examine the long-term or end behavior and see what happens when there is a common root of the numerator and denominator.  We'll then use everything we've learned about rational functions to graph them, as well as construct rational functions based on a given graph.
      \\[0.5em]
\textbook{5.6}



\subsection*{Preparation Exercises} \label{prep-functions-rational-graphs}


\begin{myPrep}
Suppose a bakery has daily fixed costs of \$150.  To produce a single loaf of bread costs an additional \$2.75  for labor and materials.
\begin{enumerate}
\item Find a function that calculates the daily cost per loaf, $C$ in dollars, to produce $x$ loaves of bread.  \\
Hint: To find the cost per loaf, you need to divide the total costs by the number of loaves of bread.
\vfill
\item If the bakery produce 5 loaves of bread, what is the cost per loaf?  
\vfill
\item If the bakery produce 50 loaves of bread, what is the cost per loaf?
\vfill
\item If the bakery produce 500 loaves of bread, what is the cost per loaf?
\vfill
\item What is happening to the per loaf cost as the number of loaves made increases?
\vfill
\item Is there a limit to the how low the cost per loaf can go?
\vfill
\end{enumerate}
\end{myPrep}





%======================================================
\newpage
%======================================================


\subsection*{Practice Exercises} \label{practice-functions-rational-graphs}

\begin{myPractice}
The function $r(x)=\dfrac{ -6(x+5)^2(x-4) }{ (x-7) }$ does not have a horizontal asymptote.  How could you change the formula for this function so that it does have a horizontal asymptote?
\vfill
\end{myPractice}


\begin{myPractice}
Let $R(x)=\dfrac{3(x+2)(x-4)}{(x+2)(x+3)}$.  Answer the following without using a calculator.
\begin{enumerate}
\begin{multicols}{2}
	\item What is the domain of $R$?
	\item What is the $y$-intercept of graph of $R$?
\end{multicols}
\vfill
\begin{multicols}{2}
	\item What are the $x$-intercepts of the graph of $R$?
	\item What are the vertical asymptotes of the graph of $R$?
\end{multicols}
\vfill
\begin{multicols}{2}
	\item Does the graph of $R$ have any holes? \\ Why or why not?
	\item If the graph of $R$ has any holes,\\ find the coordinates of the hole(s).
\end{multicols}
\vfill
\begin{multicols}{2}
	\item Does the graph of $R$ have a horizontal  \\ asymptote?  Why or why not?
	\item If the graph of $R$ has a horizontal \\ asymptote, state its equation.
	\end{multicols}
\vfill
	\end{enumerate}
\end{myPractice}




\newpage 

\begin{myPractice}
Graph $R(x) =\dfrac{3(x+2)(x-4)}{(x+2)(x+3)}$ in Figure~\ref{fig:func-rat-prac-1}.  Clearly identify all vertical asymptotes, $x$-intercepts, the $y$-intercept, any holes, and any horizontal asymptote.
		
		\begin{minipage}{0.5\linewidth}
			\
		\end{minipage}
		\begin{minipage}{0.5\linewidth}
\begin{center}
\captionof{figure}{$y=r(x)$}
\label{fig:func-rat-prac-1}
\begin{tikzpicture}
\begin{axis}[
				framed,
				width=9cm, 
				height=9cm,				
				xlabel={},
				ylabel={},
				xmin=-8,xmax=8,
				ymin=-8,ymax=8,
				xtick={-16,16},
        	minor xtick={-7,...,7},
				ytick={-16,16},
        	minor ytick={-7,...,7},
        grid=both
				]
				% use TeX as calculator:
%	\addplot[first,line width=1.0pt,dashed,<->] coordinates {(-7.9,1.5) (7.9,1.5)};
%			\node at (axis cs:-7.,1.5) {\footnotesize{$y\,$=\,2}};								    
%	\addplot[red,line width=1.0pt,dashed,<->] coordinates {(1,8) (1,-8)};
%			\node at (axis cs:-0.25,-7.5) {\footnotesize{$x\,$=\,1}};								    
%	\addplot[red,line width=1.0pt,dashed,<->] coordinates {(3,8) (3,-8)};
%			\node at (axis cs:4.25,-7.5) {\footnotesize{$x\,$=\,3}};								    
%	\addplot[first,line width=1.0pt,<->,samples=400]expression[domain=-8:.891]{3*(x-2)^2/(2*(x-3)*(x-1))};
%	\addplot[first,line width=1.0pt,<->,samples=400]expression[domain=1.082:2.918]{3*(x-2)^2/(2*(x-3)*(x-1))};
%	\addplot[first,line width=1.0pt,<->,samples=400]expression[domain=3.109:8]{3*(x-2)^2/(2*(x-3)*(x-1))};
%	\addplot[first,line width=1.0pt,mark=*] coordinates {(0,2)};
%			\node at (axis cs:-1,2.5) {\footnotesize{$\left(0,2\right)$}};
%	\addplot[first,line width=1.0pt,fill=white,mark=*] coordinates {(5,2.27)};
		\end{axis}
		\end{tikzpicture}
\end{center}		
\end{minipage}
\vfill

\end{myPractice}

\begin{myPractice}
Find a possible formula for the rational function graphed in Figure~\ref{fig:func-rat-prac-2}.  Take into account the vertical asymptote(s), the $x$-intercept(s), the $y$-intercept, and whether or not there is a horizontal asymptote.
		
		\begin{minipage}{0.5\linewidth}
			\
		\end{minipage}
		\begin{minipage}{0.5\linewidth}
\begin{center}
\captionof{figure}{$y=r(x)$}
\label{fig:func-rat-prac-2}
\begin{tikzpicture}
\begin{axis}[
				framedaxes,
				width=8cm, 
				height=8cm,				
				xlabel={$x$},
				ylabel={$y$},
				xmin=-8,xmax=8,
				ymin=-8,ymax=8,
				xtick={-6,-4,...,6},
        	minor xtick={-7,-5,...,7},
				ytick={-6,-4,...,6},
        	minor ytick={-7,-5,...,7},
        grid=both
				]
				% use TeX as calculator:
	\addplot[third,line width=1.0pt,dashed,<->] coordinates {(-7.9,-2) (7.9,-2)};
			\node at (axis cs:6.5,-2.5) {\color{fifth}\footnotesize{$y\,$=\,$-2$}};
	\addplot[fifth,line width=1.0pt,dashed,<->] coordinates {(-2,8) (-2,-8)};
			\node at (axis cs:-3.5,-7.5) {\color{fifth}\footnotesize{$x\,$=\,$-2$}};								    
%	\addplot[red,line width=1.0pt,dashed,<->] coordinates {(3,8) (3,-8)};
%			\node at (axis cs:4.25,-7.5) {\footnotesize{$x\,$=\,3}};								    
	\addplot[first,line width=1.0pt,<->,samples=400]expression[domain=-8:-2.704]{ (-2*(x-4)*(x+3))/((x+2)^2)   };
	\addplot[first,line width=1.0pt,<->,samples=400]expression[domain=-0.296:8]{ (-2*(x-4)*(x+3))/((x+2)^2) };
%	\addplot[first,line width=1.0pt,<->,samples=400]expression[domain=3.109:8]{ (-2*(x-4)*(x+3))/((x+2)^2) };
	\addplot[first,line width=1.0pt,mark=*] coordinates {(0,6)};
			\node at (axis cs:1,6.5) {\footnotesize{$\left(0,6\right)$}};
%	\addplot[first,line width=1.0pt,fill=white,mark=*] coordinates {(5,2.27)};
		\end{axis}
		\end{tikzpicture}
\end{center}		
\end{minipage}
\vfill
\end{myPractice}



%======================================================
\newpage
%======================================================

\subsection*{Definitions} \label{def-functions-rational-graphs}


\begin{myDefinition}[Holes or Removable Discontinuities ]~\\[0.5mm]
A {\bf hole} or {\bf removable discontinuity} for a rational function $r$ occurs at an $x$-value that is a root of both the numerator and denominator and whose multiplicity in the numerator is greater than or equal to its multiplicity in the denominator.  \\
Note: If the multiplicity of the root is greater in the denominator, then the root will create a vertical asymptote. \\[0.5em]
\defexample There is a removable discontinuity at $x=1$ for $f(x)=\dfrac{(x-1)(x-5)}{(x-1)(x+3)}$ and $g(x)=\dfrac{(x-1)^2(x-5)}{(x-1)(x+3)}$, \\
but $x=1$ is a vertical asymptote for $h(x)=\dfrac{(x-1)(x-5)}{(x-1)^2(x+3)}$.
\end{myDefinition}


\begin{myDefinition}[Horizontal Asymptotes of Rational Functions]~\\[0.5mm]
Whether a rational function has is a horizontal asymptote can be determined by comparing the degrees of the numerator and denominator.
\begin{itemize}
\item If the degree of the denominator is greater than that of the numerator, the function will have a horizontal asymptote of $y=0$.
\item If the degree of the denominator is equal to that of the numerator, the function will have a horizontal asymptote.  The equation of the horizontal asymptote will be based on the ratio of the leading coefficients.
\item If the degree of the denominator is less than that of the numerator, the function will not have a horizontal asymptote.
\end{itemize}
\defexample $f(x)=\dfrac{6x^5-8x^3+2}{3x^6-9x+5}$ has a horizontal asymptote of $y=0$.\\[0.5em]
\indent \qquad$g(x)=\dfrac{6x^6-8x^3+2}{3x^6-9x+5}$ has a horizontal asymptote of $y=2$.\\[0.5em]
\indent \qquad$h(x)=\dfrac{6x^7-8x^3+2}{3x^6-9x+5}$ has no horizontal asymptote.
\end{myDefinition}





%======================================================
 \newpage
%======================================================

\subsection*{Exit Exercises} \label{exit-functions-rational-graphs}

\begin{myExit}
Let $R(x)=\dfrac{5(x+4)^2(x-2)}{2(x+4)(x+1)^2}$.
\begin{enumerate}
	\item What is the domain of $R$?  Answer using both interval and set-builder notations.
\vfill
\begin{multicols}{2}
	\item What are the $x$-intercepts of the graph of $R$?
	\item What are the vertical asymptotes of the graph of $R$?
\end{multicols}
\vfill
\begin{multicols}{2}
	\item Does the graph of $R$ have any holes? \\ Why or why not?
	\item Does the graph of $R$ have a horizontal asymptote? \\ If it does, what is the horizontal asymptote?
\end{multicols}
\vfill
	\end{enumerate}
\end{myExit}


\begin{myExit}
Find a possible formula for the rational function graphed in Figure~\ref{fig:func-rat-exit-2}.  Take into account the vertical asymptote(s), the $x$-intercept(s), the $y$-intercept, and whether or not there is a horizontal asymptote.
		
		\begin{minipage}{0.5\linewidth}
			\
		\end{minipage}
		\begin{minipage}{0.5\linewidth}
\begin{center}
\captionof{figure}{$y=r(x)$}
\label{fig:func-rat-exit-2}
\begin{tikzpicture}
\begin{axis}[
				framedaxes,
				width=8cm, 
				height=8cm,				
				xlabel={$x$},
				ylabel={$y$},
				xmin=-8,xmax=8,
				ymin=-8,ymax=8,
				xtick={-6,-4,...,6},
        	minor xtick={-7,-5,...,7},
				ytick={-6,-4,...,6},
        	minor ytick={-7,-5,...,7},
        grid=both
				]
				% use TeX as calculator:
	\addplot[first,line width=1.0pt,dashed,<->] coordinates {(-7.9,1.5) (7.9,1.5)};
%			\node at (axis cs:-7.,1.5) {\footnotesize{$y\,$=\,2}};								    
	\addplot[fifth,line width=1.0pt,dashed,<->] coordinates {(1,8) (1,-8)};
			\node at (axis cs:-0.25,-7.5) {\footnotesize{$x\,$=\,$1$}};								    
	\addplot[fifth,line width=1.0pt,dashed,<->] coordinates {(3,8) (3,-8)};
			\node at (axis cs:4.25,-7.5) {\footnotesize{$x\,$=\,$3$}};								    
	\addplot[first,line width=1.0pt,<->,samples=400]expression[domain=-8:.891]{3*(x-2)^2/(2*(x-3)*(x-1))};
	\addplot[first,line width=1.0pt,<->,samples=400]expression[domain=1.082:2.918]{3*(x-2)^2/(2*(x-3)*(x-1))};
	\addplot[first,line width=1.0pt,<->,samples=400]expression[domain=3.109:8]{3*(x-2)^2/(2*(x-3)*(x-1))};
	\addplot[first,line width=1.0pt,mark=*] coordinates {(0,2)};
			\node at (axis cs:-1,2.5) {\footnotesize{$\left(0,2\right)$}};
%	\addplot[first,line width=1.0pt,fill=white,mark=*] coordinates {(5,2.27)};
		\end{axis}
		\end{tikzpicture}
\end{center}		
\end{minipage}
\end{myExit}


\exitlikert{rational functions}






























% %======================================================
% \newpage
% %======================================================

% \begin{myExample}
% What are the domains of the following functions?
% 	\begin{enumerate}
% 		\begin{multicols}{2}
% 		\item $f(x) = 7x(x+8)(x-3)$
% 		\item $g(t) = \dfrac{7t(t+8)}{t-3}$
% 		\end{multicols}
% 		\vfill
% 		\begin{multicols}{2}
% 		\item $j(n) = \sqrt{12-3n}$
% 		\item $k(w) = w^2 -8w-9$
% 		\end{multicols}
% 		\vfill
% 		\begin{multicols}{2}
% 		\item $m(a) = \dfrac{5a-9}{2a+8}$
% 		\item $r(u) = \dfrac{\sqrt{u+3}}{u-7}$
% 		\end{multicols}
% 		\vfill
% 		\vfill

% 	\end{enumerate}
% \end{myExample}


% {\bf Textbook Reference:}\\
% If you need a few more examples of set-builder and interval notations, \\please see the \href{https://openstax.org/books/algebra-and-trigonometry-2e/pages/3-2-domain-and-range#fs-id1165137677916}{Using Notations to Specify Domain and Range} section of \S3.2 in our book.

% %======================================================
% \newpage
% %======================================================

% \begin{center}
%   {\Large {\bf  Part 2: Finding Domains and Ranges from Graphs}}
% \end{center}

% \begin{myExample}
% Let $y=f(x)$ be defined by the graph below.  (Are you sure this is a function?  Why?)\\

% \begin{minipage}{0.4\linewidth}
% \begin{center}
% \begin{tikzpicture}
% 		\begin{axis}[
% 				framed,
% 				width=7cm,height=7cm,
% 				axis x line=middle,
% 				axis y line=middle,
% 				xlabel={$x$},
% 				ylabel={$y$},
% 				xmin=-8,xmax=8,
% 				ymin=-8,ymax=8,
%         xtick={-6,-4,...,6},
%         	minor xtick={-7,-5,...,7},
%         ytick={-6,-4,...,6},
%         	minor ytick={-7,-5,...,7},
%         grid=both
% 				]
% 				% use TeX as calculator:
% 					\addplot[smooth,mark=*,first,line width=1.0pt,fill=white]coordinates{	(-4,2)	};
% 					\addplot[smooth,mark=*,first,line width=1.0pt]coordinates{	(5,-4)	};
% 				\addplot[first,line width=1.5pt,samples=200]expression[domain=-4:4.99999]{2*(-x+5)^(1/2)-4};
% 		\end{axis}
% 		\end{tikzpicture}
% ~\\
% \end{center}
% \end{minipage}
% \begin{minipage}{0.6\linewidth}
% \begin{enumerate}
% 	\item Find $f(1)$.\\
% 	\item Find $f(5)$.\\
% 	\item Find $f(-4)$.\\
% 	\item Find $f(6)$.\\
% \end{enumerate}
% \end{minipage}

% \begin{enumerate}
% \setcounter{enumi}{4}
% 	\item State the domain of this function in interval notation and set-builder notation.\\~\\
% 	\item State the range of this function in interval notation and set-builder notation.\\~\\
% \end{enumerate}
% \end{myExample}


% \begin{myExample}
% Use $y=h(t)$ and $y=k(x)$ to answer the following.
% \vspace{-1cm}
% \begin{multicols}{2}
% \begin{center}
% \captionof{figure}{$y=h(t)$}
% \begin{tikzpicture}
% 		\begin{axis}[
% 				framed,
% 				width=6cm,height=6cm,
% 				xlabel={$t$},
% 				ylabel={$y$},
% 				xmin=-8,xmax=8,
% 				ymin=-8,ymax=8,
%         xtick={-6,-4,...,6},
%         	minor xtick={-7,-5,...,7},
%         ytick={-6,-4,...,6},
%         	minor ytick={-7,-5,...,7},
%         grid=both
% 				]
% 				% use TeX as calculator:
% 				\addplot[blue,line width=1.5pt,samples=400,-]expression[domain=-7:-4]{-x-4};
% 					\addplot[smooth,mark=*,blue,line width=2.0pt]coordinates{	(-7,3)	};
% 				\addplot[blue,line width=1.5pt,samples=400,-]expression[domain=-4:-3]{x+4};
% 				\addplot[blue,line width=1.5pt,samples=400,-]expression[domain=-3:-2]{(x+3)^2+1};
% 				\addplot[blue,line width=1.5pt,samples=400,-]expression[domain=-2:0]{-(x+1)^2+3};
% 				\addplot[blue,line width=1.5pt,samples=400,-]expression[domain=0:2]{-x+2};
% 				\addplot[blue,line width=1.5pt,samples=400,-]expression[domain=2:4]{(x-3)^2-1};
% 				\addplot[blue,line width=1.5pt,samples=400,-]expression[domain=4:6]{x-4};
% 					\addplot[smooth,mark=*,blue,line width=1pt,fill=white]coordinates{	(6,2)	};
% 		\end{axis}
% 		\end{tikzpicture}
% 		\label{fig:dr1}
% \end{center}

% \begin{center}
% \captionof{figure}{$y=k(x)$}
% \begin{tikzpicture}
% 		\begin{axis}[
% 				framed,
% 				width=6cm,height=6cm,
% 				xlabel={$x$},
% 				ylabel={$y$},
% 				xmin=-8,xmax=8,
% 				ymin=-8,ymax=8,
%         xtick={-6,-4,...,6},
%         	minor xtick={-7,-5,...,7},
%         ytick={-6,-4,...,6},
%         	minor ytick={-7,-5,...,7},
%         grid=both
% 				]
% 				% use TeX as calculator:
% 				\addplot[blue,line width=1.5pt,samples=400,-]expression[domain=-7:-5.99]{x+1};
% 				\addplot[smooth,mark=*,blue,line width=1pt,fill=white]coordinates{	(-7,-6)	};
% 				\addplot[blue,line width=1.5pt,samples=400,-]expression[domain=-6.01:-3.99]{2*(x)+7};
% 				\addplot[blue,line width=1.5pt,samples=400,-]expression[domain=-4.01:-2.99]{(x)+3};
% 				\addplot[blue,line width=1.5pt,samples=400,-]expression[domain=-3.01:-0.99]{-1/2*(x)-1.5};
% 				\addplot[blue,line width=1.5pt,samples=400,-]expression[domain=-1.01:0]{2*(x)^2-3};
% 				\addplot[blue,line width=1.5pt,samples=400,-]expression[domain=0:1.01]{-(x)^2-3};
% 				\addplot[blue,line width=1.5pt,samples=400,-]expression[domain=0.99:2]{-(x-2)^2-3};
% 				\addplot[blue,line width=1.5pt,samples=400,-]expression[domain=2:3.01]{2*(x-2)^2-3};
% 				\addplot[blue,line width=1.5pt,samples=400,-]expression[domain=2.99:5.01]{(x-3)^2-1};
% 				\addplot[blue,line width=1.5pt,samples=400,-]expression[domain=4.99:6]{3};
% 				\addplot[smooth,mark=*,blue,line width=1.5pt]coordinates{	(6,3)	};
% 		\end{axis}
% 		\end{tikzpicture}
% 		\label{fig:dr2}
% \end{center}
% \end{multicols}

% 		\begin{enumerate}\setlength{\itemsep}{0.85in}
% 		\begin{multicols}{2}
% 				\item		State the domain of $h$.
% 				\item		State the domain of $k$.
% 		\end{multicols}
% \vfill
% 		\begin{multicols}{2}
% 				\item		State the range of $h$.
% 				\item		State the range of $k$.
% 		\end{multicols}
% \vfill
% 		\end{enumerate}	
% \end{myExample}

% %======================================================
% \newpage
% %======================================================

% \begin{center}
%   {\Large {\bf  Part 3: Piecewise-Defined Functions}}
% \end{center}


% \begin{myDefinition} 
% A function that is defined by different formulas on different parts of its domain is called a\\
% {\bf \underline{piecewise-defined function}}.  Piecewise-defined functions take on the form:\\  
% 				\[
% 				f(x)=
% 					\begin{cases}
% 							\text{formula \#1}				&	\textrm{if}\ 	\ \			\text{$x$ is in this part of the domain}		\\
% 							\text{formula \#2}				&	\textrm{if}\ 	\ \			\text{$x$ is in this part of the domain}		\\
% 							\text{etc.}				&	\textrm{if}\ 	\ \			\text{etc.}		\\
% 					\end{cases}
% 				\]
% \ 

% \end{myDefinition}

% \begin{myExample}
% In Table~\ref{tab:tax1}, the 2011 federal income tax rates\footnote{\url{http://www.irs.gov/newsroom/article/0,,id=233465,00.html}} for 2011 are shown.  

% \renewcommand{\arraystretch}{1.5}
% \begin{table}[h!]
% \centering
% \caption{Federal Income Tax Percentage Rates for 2011}
% \label{tab:tax1} \small{
% 	\begin{tabular}{| c | c | c |}
% 	\hline 
% 		Income Amount $(x)$										& Percentage of Income Taxed (in $\%$) & Tax on $x$ Dollars of Income ($T(x)$, in \$) \\ \hline
% 		\	$0 \le x < 8500$										&				10		&	$0.10x$								\\ \hline			
% 			$8500 \le x < 34500$								&				15			&	$0.15x$							\\ \hline			
% 			$34500 \le x < 83600$								&				25			&	$0.25x$							\\ \hline			
% 			$83600 \le x < 174400$							&				28				&	$0.28x$						\\ \hline										
% 			$174400 \le x < 379150 $						&				33					&	$0.33x$					\\ \hline				
% 			$x \ge 379150$											&				35	&	$0.35x$									\\ \hline					
% 	\end{tabular}}
% \end{table}

% Notice that for each interval, the percentage of income taxed as a function of income is \emph{constant}.  If we graph each \emph{piece} over its respective interval, we obtain the following: \\
% 		\begin{minipage}{0.5\linewidth}

% \begin{center}
% 	\captionof{figure}{Graph of $y=T(x)$}
% 	\label{fig:b}
% 	   	\begin{tikzpicture}
% 			\begin{axis}[
% 				framed,
% 				height=3in,
% 				width=3in,
% 				xmin=0,xmax=500,
% 				ymin=0,ymax=180,
% 					xlabel style={at={(ticklabel cs:0.5)},anchor=near ticklabel},
%       		ylabel style={at={(ticklabel cs:0.5)},rotate=90,anchor=near ticklabel},
% 				xlabel={\tiny{$x$, income amount in \$1000s}},
% 				ylabel={\tiny{ $y$, tax amount in \$1000s}},
% 				xtick={0,50,100,...,450},
% 					 minor xtick={25,75,...,475},
% 				ytick={10,20,...,180},
% 				%	  minor ytick={1,2,3,...,39},
% 				grid=both,
% 				]
% 				% use TeX as calculator:
% 				\addplot[blue, line width=1.25pt]expression[domain=0:8.5,samples=200]{0.1*x};
% 						\addplot[smooth,mark=*,blue,line width=1.25pt]coordinates{	(0,0)	};
% 						\addplot[smooth,mark=*,blue,line width=1.25pt,fill=white]coordinates{	(8.5,0.85)	};
% 				\addplot[blue, line width=1.25pt]expression[domain=8.5:34.5,samples=200]{0.15*x};
% 						\addplot[smooth,mark=*,blue,line width=1.25pt]coordinates{	(8.5,1.275)	};
% 						\addplot[smooth,mark=*,blue,line width=1.25pt,fill=white]coordinates{	(34.5,5.175)	};
% 				\addplot[blue, line width=1.25pt]expression[domain=34.5:83.6,samples=200]{0.25*x};
% 						\addplot[smooth,mark=*,blue,line width=1.25pt]coordinates{	(34.5,8.625)	};
% 						\addplot[smooth,mark=*,blue,line width=1.25pt,fill=white]coordinates{	(83.6,20.9)	};
% 				\addplot[blue, line width=1.25pt]expression[domain=83.6:174.4,samples=200]{0.28*x};
% 						\addplot[smooth,mark=*,blue,line width=1.25pt]coordinates{	(83.6,23.408)	};
% 						\addplot[smooth,mark=*,blue,line width=1.25pt,fill=white]coordinates{	(174.4,48.832)	};
% 				\addplot[blue, line width=1.25pt]expression[domain=174.4:379.15,samples=200]{0.33*x};
% 						\addplot[smooth,mark=*,blue,line width=1.25pt]coordinates{	(174.4,57.552)	};
% 						\addplot[smooth,mark=*,blue,line width=1.25pt,fill=white]coordinates{	(379.15,125.1195)	};
% 				\addplot[blue, line width=1.25pt]expression[domain=379.15:500,samples=200,->]{0.35*x};
% 						\addplot[smooth,mark=*,blue,line width=1.25pt]coordinates{	(379.15,132.7025)	};
% 	\end{axis}
% 		\end{tikzpicture}
% \end{center}



% 		\end{minipage}
% 		\begin{minipage}{0.5\linewidth}
% Write the formula for $T(x)$.\\~\\~\\~\\~\\~\\~\\~\\~\\~\\~\\

% 		\end{minipage}
% \end{myExample}




% %======================================================
% \newpage
% %======================================================

% \begin{myExample}
% Consider the function $g(x) = |x|$, the absolute value function.
% \begin{enumerate}
% 	\item When you first learned about the absolute value, how did you describe what is measured?\\~\\~\\
% 	\item In MTH 95 (Intermediate Algebra), what was another way you learned that you could write $g$?\\~\\~\\
% 	\item If you had to explain to someone in words how to computer an absolute value, what would you say?\\~\\~\\
% 	\item The graph of $y=g(x)$ (the absolute value function) is given in Figure~\ref{fig:absval}.  \\
% 	Write this formula for this function as a piecewise-defined function.\\
% \begin{minipage}{0.5\linewidth}
% Formula: \\[4cm]

% \end{minipage}
% \begin{minipage}{0.5\linewidth}
% \begin{center}
% \captionof{figure}{}
% \begin{tikzpicture}
% 		\begin{axis}[
% 				framed,
% 				height=8cm,
% 				width=8cm,
% 				xlabel={$x$},
% 				ylabel={$y$},
% 				xmin=-8,xmax=8,
% 				ymin=-8,ymax=8,
%         xtick={-6,-4,...,6},
%         	minor xtick={-7,-5,...,7},
%         ytick={-6,-4,...,6},
%         	minor ytick={-7,-5,...,7},
%         grid=both
% 				]
% 				% use TeX as calculator:
% 				\addplot[blue,line width=1.25pt,samples=200,<-]expression[domain=-8:0]{-1*x};
% 				\addplot[blue,line width=1.25pt,samples=200,->]expression[domain=0:8]{x};
% 		\end{axis}
% 		\end{tikzpicture}
% 		\label{fig:absval}
% \end{center}
% \end{minipage}
% \end{enumerate}

% \end{myExample}


% %======================================================
% \newpage
% %======================================================




% \begin{myExample}

% Let $k$ and $m$ be as defined below.
% %\begin{minipage}{0.6\linewidth}
% \vspace{-1cm}
% \begin{multicols}{2}
% \begin{center}
% 				\[
% 				k(t)=
% 					\begin{cases}
% 						-2t-4								& \textrm{if}\ 	\ \			t \le -1			\\
% 						 2t+3							& \textrm{if}\ 	\ \			t > -1 		
% 					\end{cases}
% 				\]  
% \end{center}
% \begin{center}
% 				\[
% 				m(x)=
% 					\begin{cases}
% 						 x-3								& \textrm{if}\ 	\ \			0 < x \le 4 		\\
% 						-x-4								& \textrm{if}\ 	\ \			x \le 0	
% 					\end{cases}
% 				\]
% \end{center}
% \end{multicols}
% %\end{minipage}

% \begin{enumerate}
% \begin{multicols}{2}
% \item What is the domain of $k$?\\~\\~\\
% \item What is the domain of $m$?\\~\\~\\
% \end{multicols}
% \begin{multicols}{2}
% \item Find $k(10)$.\\~\\~\\~\\
% \item Find $m(3)$.\\~\\~\\~\\
% \end{multicols}
% \item Find $k(-9)$.\\~\\~\\~\\
% \item Find $m(0)$.\\~\\~\\~\\
% \begin{multicols}{2}
% \item Graph $y=k(t)$ in Figure~\ref{fig:pwd12} below.
% \item Graph $y=m(x)$ in Figure~\ref{fig:pwd22} below.
% \end{multicols}

% \vspace{-1cm}
% \begin{multicols}{2}
% \begin{center}
% \captionof{figure}{$y=k(t)$}
% \begin{tikzpicture}
% 		\begin{axis}[
% 				framed,
% 				height=7cm,
% 				width=7cm,
% 				xmin=-8,xmax=8,
% 				ymin=-8,ymax=8,
%         xtick={0},
%         	minor xtick={-7,-6,...,7},
%         ytick={0},
%         	minor ytick={-7,-6,...,7},
%         grid=both
% 				]
% 				% use TeX as calculator:
% 		\end{axis}
% 		\end{tikzpicture}
% 		\label{fig:pwd12}
% \end{center}

% \begin{center}
% \captionof{figure}{$y=m(x)$}
% \begin{tikzpicture}
% 		\begin{axis}[
% 				framed,
% 				height=7cm,
% 				width=7cm,
% 				xmin=-8,xmax=8,
% 				ymin=-8,ymax=8,
%         xtick={0},
%         	minor xtick={-7,-6,...,7},
%         ytick={0},
%         	minor ytick={-7,-6,...,7},
%         grid=both
% 				]
% 				% use TeX as calculator:
% 		\end{axis}
% 		\end{tikzpicture}
% 		\label{fig:pwd22}
% \end{center}
% \end{multicols}

% \end{enumerate}

% \end{myExample}


% %======================================================
% \newpage
% %======================================================


% \begin{myExample}
% In Figure~\ref{fig:piecewise4} is the graph of $y=F(x)$.	\\
% \begin{enumerate}%\setlength{\itemsep}{1.25in}
% \begin{minipage}[!t]{0.4\linewidth}
% \item Write the formula for the piecewise function $F$: \\[6cm]
% \end{minipage}
% \begin{minipage}{0.6\linewidth}
% \begin{center}
% \captionof{figure}{$y=F(x)$}
% \begin{tikzpicture}
% 		\begin{axis}[
% 				framed,
% 				height=8cm,
% 				width=8cm,
% 				xlabel={$x$},
% 				ylabel={$y$},
% 				xmin=-8,xmax=8,
% 				ymin=-8,ymax=8,
%         xtick={-6,-4,...,6},
%         	minor xtick={-7,-5,...,7},
%         ytick={-6,-4,...,6},
%         	minor ytick={-7,-5,...,7},
%         grid=both
% 				]
% 				% use TeX as calculator:
% 				\addplot[blue,line width=1.25pt,samples=200,<-]expression[domain=-8:-2]{1/2*(x+2)-3};
% 						\addplot[mark=*,blue,line width=1.0pt]coordinates{	(-2,-3)	};
% 				\addplot[blue,line width=1.25pt,samples=200]expression[domain=-2:3]{5};
% 						\addplot[mark=*,blue,line width=1.0pt,fill=white]coordinates{	(-2,5)	};
% 						\addplot[mark=*,blue,line width=1.0pt]coordinates{	(3,5)	};
% 				\addplot[blue,line width=1.25pt,samples=200]expression[domain=3:7]{-2*(x-6)-5};
% 						\addplot[mark=*,blue,line width=1.0pt,fill=white]coordinates{	(3,1)	};
% 						\addplot[mark=*,blue,line width=1.0pt,fill=white]coordinates{	(7,-7)	};
% 		\end{axis}
% 		\end{tikzpicture}
% 		\label{fig:piecewise4}
% \end{center}
% \end{minipage}
% %\vspace{11pt}
% \begin{multicols}{2}
% 	\item	Evaluate $F(0)$.	

% 	\item	Solve $F(x)=0$.	
% \end{multicols}
% ~\\
% \begin{multicols}{2}
% 	\item	Evaluate $F(2)$.
% 	\item	Solve $F(x)=-5$.	
% \end{multicols}
% ~\\

% 	\item	Solve $F(x)=-5$ both algebraically (using your formula from Part a) and graphically.	
% \vfill
% \end{enumerate}

% \end{myExample}












\newpage
~


\include{111-chapter-4-appendices}







\resetCounters
%============================================================
% MTH 111Z Project - Template File
% Section 3.5 from OpenStax OER
%	Updated 202302
%============================================================


\section{Direct Links} \label{section-supplement-tinycc}

In the event that some TinyCC links from the MTH 111 Lab Manual break, here are the direct links used at the time of publication.\\

Please note that while the direct links to Desmos graphs might no longer be the most current version of the Desmos graphs, any older graphs should still be reasonably useful.

\subsection*{Preface Links} \label{linksfor-chapter1}

\setlist{itemsep=2.5pt}
\begin{itemize}
	\item {\it Algebra and Trigonometry} 2e: https://openstax.org/details/books/algebra-and-trigonometry-2e
	\item MTH 111Z Supplement: https://spot.pcc.edu/math/mth111-112-supplement-landing.html
\end{itemize}


\subsection*{Chapter 1 Links} \label{linksfor-chapter1}

\setlist{itemsep=2.5pt}
\begin{itemize}
	\item \S1.2 Interval and Set-Builder Notations: https://youtu.be/aLvRu8Int4M
	\item \S1.2 Domain and Range: https://www.desmos.com/calculator/dgrur8be3f
	\item \S1.2 Positive and Negative: https://www.desmos.com/calculator/v8qsikcufe
	\item \S1.2 Increasing, Decreasing, and Constant: https://www.desmos.com/calculator/ondcewmuy0
	\item \S1.6 Vertical Shift: https://www.desmos.com/calculator/sv2boowrlu
	\item \S1.6 Horizontal Shift: https://www.desmos.com/calculator/eloexg8kaz
	\item \S1.6 Vertical Stretch/Compression: https://www.desmos.com/calculator/m6uervj5h6
	\item \S1.6 Horizontal Stretch/Compression: https://www.desmos.com/calculator/0he8y9sftj
	\item \S1.6 Vertical Reflection: https://www.desmos.com/calculator/ow1t0ajggh
	\item \S1.6 Horizontal Reflection: https://www.desmos.com/calculator/0pgds68h5c
	\item \S1.6 Even Function: https://www.desmos.com/calculator/aqintrlh6h
	\item \S1.6 Odd Function: https://www.desmos.com/calculator/qtpfhmydes
\end{itemize}


\subsection*{Chapter 2 Links} \label{linksfor-chapter2}
\setlist{itemsep=2.5pt}
\begin{itemize}
	\item \S2.1 Exponential Function: https://www.desmos.com/calculator/zcp3gaj5ws
	\item \S2.2 Logarithmic Function: https://www.desmos.com/calculator/9bqm1kbl4i
	\item \S2.5 The Number $e$: https://youtu.be/AuA2EAgAegE
\end{itemize}

\subsection*{Chapter 3 Links} \label{linksfor-chapter3}
\begin{itemize}
	\item \S3.2 Multiplicity and Polynomials: https://www.desmos.com/calculator/b3pmc7es7z
	\item \S3.3 Vertical Asymptote: https://www.desmos.com/calculator/t10efozl3i
	\item \S3.3 Multiplicity and Vertical Asymptotes:  https://www.desmos.com/calculator/u1ed06auqc
	\item \S3.3 Horizontal Asymptote: https://www.desmos.com/calculator/9bqtdmnl4w
\end{itemize}






%%%%%%%%%%%%%%%%%%%%%%
\iffalse
%%%%%%%%%%%%%%%%%%%%%%






\include{111-chapter-4-appendices}

\resetCounters
%============================================================
% MTH 111Z Project - Template File
% Section 3.5 from OpenStax OER
%	Updated 202302
%============================================================


%\fakesection{Function Transformations} \label{functions-transformations}
\section{Function Transformations Supplement} \label{supplement-transformations}

%Introduction





\subsection*{Examples} \label{examples-functions-transformations}

\begin{myExample}~\\ \vspace{-18mm}~

\begin{minipage}{0.5\linewidth}
Table~\ref{tab:transform-example1} defines the functions $f$, $g$, and $h$. \\
Express $g(x)$ and $h(x)$ in terms of $f$.\\
\end{minipage}
\begin{minipage}{0.5\linewidth}
\begin{center}
\captionof{table}{}
\renewcommand\arraystretch{1.5}
\begin{tabular}{c|ccccccc}
	\hline
		$x$ & $-3$ & $-2$ & $-1$ & $0$ & $1$ & $2$ & $3$ \\
	\hline
		$f(x)$ & $8$ & $6$ & $4$ & $2$ & $0$ & $-1$ & $-2$ \\
	\hline
		$g(x)$ & $-8$ & $-6$ & $-4$ & $-2$ & $0$ & $1$ & $2$ \\
	\hline
		$h(x)$ & $5$ & $3$ & $1$ & $1$ & $-3$ & $-4$ & $-5$ \\
	\hline
\end{tabular}
\label{tab:transform-example1}\\[0.5em]
\end{center}
\end{minipage}

{\bf Answer:}\\
Since every value of $g(x)$ is the opposite of the value $f(x)$ for the same value of $x$, we can say that $g(x) = -1\cdot f(x)$ or more simply that $g(x) = -f(x)$.

Since every value of $h(x)$ is three less than the value $f(x)$ for the same value of $x$, we can say that $h(x) =  f(x)-3$.


\end{myExample}

\begin{myExample}
\begin{enumerate}
\item If $f(x)=x^2$ and $g(x)=2x^2+5$, express $g(x)$ in terms of $f$.\\[0.5em]
{\bf Answer:}\\
Since $f(x) = x^2$, we can substitute $f(x)$ in place of $x^2$ in $g(x)=2x^2+5$.  This gives us $g(x) = 2f(x)+5$.


\item If $f(x)=x^2$ and $h(x)=(x+5)^2-3$, express $h(x)$ in terms of $f$.\\[0.5em]
{\bf Answer:}\\
In $(x+5)^2$, the $5$ is being added inside of what is being squared.  To put this in terms of $f(x)= x^2$, we need to add $5$ to $x$ inside the function notation to have $f(x+5)= (x+5)^2$. 

We can now substitute $f(x+5)$ in place of $(x+5)^2$ in $h(x)= (x+5)^2-3$, giving us $h(x) = f(x+5)-3$.
\end{enumerate}
\end{myExample}


\begin{myExample}
MTH 111 Supplement Example
\end{myExample}


\begin{myExample}
MTH 111 Supplement Example
\end{myExample}

\begin{myExample}
MTH 111 Supplement Example
\end{myExample}




























% %======================================================
% \newpage
% %======================================================

% \begin{myExample}
% What are the domains of the following functions?
% 	\begin{enumerate}
% 		\begin{multicols}{2}
% 		\item $f(x) = 7x(x+8)(x-3)$
% 		\item $g(t) = \dfrac{7t(t+8)}{t-3}$
% 		\end{multicols}
% 		\vfill
% 		\begin{multicols}{2}
% 		\item $j(n) = \sqrt{12-3n}$
% 		\item $k(w) = w^2 -8w-9$
% 		\end{multicols}
% 		\vfill
% 		\begin{multicols}{2}
% 		\item $m(a) = \dfrac{5a-9}{2a+8}$
% 		\item $r(u) = \dfrac{\sqrt{u+3}}{u-7}$
% 		\end{multicols}
% 		\vfill
% 		\vfill

% 	\end{enumerate}
% \end{myExample}


% {\bf Textbook Reference:}\\
% If you need a few more examples of set-builder and interval notations, \\please see the \href{https://openstax.org/books/algebra-and-trigonometry-2e/pages/3-2-domain-and-range#fs-id1165137677916}{Using Notations to Specify Domain and Range} section of \S3.2 in our book.

% %======================================================
% \newpage
% %======================================================

% \begin{center}
%   {\Large {\bf  Part 2: Finding Domains and Ranges from Graphs}}
% \end{center}

% \begin{myExample}
% Let $y=f(x)$ be defined by the graph below.  (Are you sure this is a function?  Why?)\\

% \begin{minipage}{0.4\linewidth}
% \begin{center}
% \begin{tikzpicture}
% 		\begin{axis}[
% 				framed,
% 				width=7cm,height=7cm,
% 				axis x line=middle,
% 				axis y line=middle,
% 				xlabel={$x$},
% 				ylabel={$y$},
% 				xmin=-8,xmax=8,
% 				ymin=-8,ymax=8,
%         xtick={-6,-4,...,6},
%         	minor xtick={-7,-5,...,7},
%         ytick={-6,-4,...,6},
%         	minor ytick={-7,-5,...,7},
%         grid=both
% 				]
% 				% use TeX as calculator:
% 					\addplot[smooth,mark=*,first,line width=1.0pt,fill=white]coordinates{	(-4,2)	};
% 					\addplot[smooth,mark=*,first,line width=1.0pt]coordinates{	(5,-4)	};
% 				\addplot[first,line width=1.5pt,samples=200]expression[domain=-4:4.99999]{2*(-x+5)^(1/2)-4};
% 		\end{axis}
% 		\end{tikzpicture}
% ~\\
% \end{center}
% \end{minipage}
% \begin{minipage}{0.6\linewidth}
% \begin{enumerate}
% 	\item Find $f(1)$.\\
% 	\item Find $f(5)$.\\
% 	\item Find $f(-4)$.\\
% 	\item Find $f(6)$.\\
% \end{enumerate}
% \end{minipage}

% \begin{enumerate}
% \setcounter{enumi}{4}
% 	\item State the domain of this function in interval notation and set-builder notation.\\~\\
% 	\item State the range of this function in interval notation and set-builder notation.\\~\\
% \end{enumerate}
% \end{myExample}


% \begin{myExample}
% Use $y=h(t)$ and $y=k(x)$ to answer the following.
% \vspace{-1cm}
% \begin{multicols}{2}
% \begin{center}
% \captionof{figure}{$y=h(t)$}
% \begin{tikzpicture}
% 		\begin{axis}[
% 				framed,
% 				width=6cm,height=6cm,
% 				xlabel={$t$},
% 				ylabel={$y$},
% 				xmin=-8,xmax=8,
% 				ymin=-8,ymax=8,
%         xtick={-6,-4,...,6},
%         	minor xtick={-7,-5,...,7},
%         ytick={-6,-4,...,6},
%         	minor ytick={-7,-5,...,7},
%         grid=both
% 				]
% 				% use TeX as calculator:
% 				\addplot[blue,line width=1.5pt,samples=400,-]expression[domain=-7:-4]{-x-4};
% 					\addplot[smooth,mark=*,blue,line width=2.0pt]coordinates{	(-7,3)	};
% 				\addplot[blue,line width=1.5pt,samples=400,-]expression[domain=-4:-3]{x+4};
% 				\addplot[blue,line width=1.5pt,samples=400,-]expression[domain=-3:-2]{(x+3)^2+1};
% 				\addplot[blue,line width=1.5pt,samples=400,-]expression[domain=-2:0]{-(x+1)^2+3};
% 				\addplot[blue,line width=1.5pt,samples=400,-]expression[domain=0:2]{-x+2};
% 				\addplot[blue,line width=1.5pt,samples=400,-]expression[domain=2:4]{(x-3)^2-1};
% 				\addplot[blue,line width=1.5pt,samples=400,-]expression[domain=4:6]{x-4};
% 					\addplot[smooth,mark=*,blue,line width=1pt,fill=white]coordinates{	(6,2)	};
% 		\end{axis}
% 		\end{tikzpicture}
% 		\label{fig:dr1}
% \end{center}

% \begin{center}
% \captionof{figure}{$y=k(x)$}
% \begin{tikzpicture}
% 		\begin{axis}[
% 				framed,
% 				width=6cm,height=6cm,
% 				xlabel={$x$},
% 				ylabel={$y$},
% 				xmin=-8,xmax=8,
% 				ymin=-8,ymax=8,
%         xtick={-6,-4,...,6},
%         	minor xtick={-7,-5,...,7},
%         ytick={-6,-4,...,6},
%         	minor ytick={-7,-5,...,7},
%         grid=both
% 				]
% 				% use TeX as calculator:
% 				\addplot[blue,line width=1.5pt,samples=400,-]expression[domain=-7:-5.99]{x+1};
% 				\addplot[smooth,mark=*,blue,line width=1pt,fill=white]coordinates{	(-7,-6)	};
% 				\addplot[blue,line width=1.5pt,samples=400,-]expression[domain=-6.01:-3.99]{2*(x)+7};
% 				\addplot[blue,line width=1.5pt,samples=400,-]expression[domain=-4.01:-2.99]{(x)+3};
% 				\addplot[blue,line width=1.5pt,samples=400,-]expression[domain=-3.01:-0.99]{-1/2*(x)-1.5};
% 				\addplot[blue,line width=1.5pt,samples=400,-]expression[domain=-1.01:0]{2*(x)^2-3};
% 				\addplot[blue,line width=1.5pt,samples=400,-]expression[domain=0:1.01]{-(x)^2-3};
% 				\addplot[blue,line width=1.5pt,samples=400,-]expression[domain=0.99:2]{-(x-2)^2-3};
% 				\addplot[blue,line width=1.5pt,samples=400,-]expression[domain=2:3.01]{2*(x-2)^2-3};
% 				\addplot[blue,line width=1.5pt,samples=400,-]expression[domain=2.99:5.01]{(x-3)^2-1};
% 				\addplot[blue,line width=1.5pt,samples=400,-]expression[domain=4.99:6]{3};
% 				\addplot[smooth,mark=*,blue,line width=1.5pt]coordinates{	(6,3)	};
% 		\end{axis}
% 		\end{tikzpicture}
% 		\label{fig:dr2}
% \end{center}
% \end{multicols}

% 		\begin{enumerate}\setlength{\itemsep}{0.85in}
% 		\begin{multicols}{2}
% 				\item		State the domain of $h$.
% 				\item		State the domain of $k$.
% 		\end{multicols}
% \vfill
% 		\begin{multicols}{2}
% 				\item		State the range of $h$.
% 				\item		State the range of $k$.
% 		\end{multicols}
% \vfill
% 		\end{enumerate}	
% \end{myExample}

% %======================================================
% \newpage
% %======================================================

% \begin{center}
%   {\Large {\bf  Part 3: Piecewise-Defined Functions}}
% \end{center}


% \begin{myDefinition} 
% A function that is defined by different formulas on different parts of its domain is called a\\
% {\bf \underline{piecewise-defined function}}.  Piecewise-defined functions take on the form:\\  
% 				\[
% 				f(x)=
% 					\begin{cases}
% 							\text{formula \#1}				&	\textrm{if}\ 	\ \			\text{$x$ is in this part of the domain}		\\
% 							\text{formula \#2}				&	\textrm{if}\ 	\ \			\text{$x$ is in this part of the domain}		\\
% 							\text{etc.}				&	\textrm{if}\ 	\ \			\text{etc.}		\\
% 					\end{cases}
% 				\]
% \ 

% \end{myDefinition}

% \begin{myExample}
% In Table~\ref{tab:tax1}, the 2011 federal income tax rates\footnote{\url{http://www.irs.gov/newsroom/article/0,,id=233465,00.html}} for 2011 are shown.  

% \renewcommand{\arraystretch}{1.5}
% \begin{table}[h!]
% \centering
% \caption{Federal Income Tax Percentage Rates for 2011}
% \label{tab:tax1} \small{
% 	\begin{tabular}{| c | c | c |}
% 	\hline 
% 		Income Amount $(x)$										& Percentage of Income Taxed (in $\%$) & Tax on $x$ Dollars of Income ($T(x)$, in \$) \\ \hline
% 		\	$0 \le x < 8500$										&				10		&	$0.10x$								\\ \hline			
% 			$8500 \le x < 34500$								&				15			&	$0.15x$							\\ \hline			
% 			$34500 \le x < 83600$								&				25			&	$0.25x$							\\ \hline			
% 			$83600 \le x < 174400$							&				28				&	$0.28x$						\\ \hline										
% 			$174400 \le x < 379150 $						&				33					&	$0.33x$					\\ \hline				
% 			$x \ge 379150$											&				35	&	$0.35x$									\\ \hline					
% 	\end{tabular}}
% \end{table}

% Notice that for each interval, the percentage of income taxed as a function of income is \emph{constant}.  If we graph each \emph{piece} over its respective interval, we obtain the following: \\
% 		\begin{minipage}{0.5\linewidth}

% \begin{center}
% 	\captionof{figure}{Graph of $y=T(x)$}
% 	\label{fig:b}
% 	   	\begin{tikzpicture}
% 			\begin{axis}[
% 				framed,
% 				height=3in,
% 				width=3in,
% 				xmin=0,xmax=500,
% 				ymin=0,ymax=180,
% 					xlabel style={at={(ticklabel cs:0.5)},anchor=near ticklabel},
%       		ylabel style={at={(ticklabel cs:0.5)},rotate=90,anchor=near ticklabel},
% 				xlabel={\tiny{$x$, income amount in \$1000s}},
% 				ylabel={\tiny{ $y$, tax amount in \$1000s}},
% 				xtick={0,50,100,...,450},
% 					 minor xtick={25,75,...,475},
% 				ytick={10,20,...,180},
% 				%	  minor ytick={1,2,3,...,39},
% 				grid=both,
% 				]
% 				% use TeX as calculator:
% 				\addplot[blue, line width=1.25pt]expression[domain=0:8.5,samples=200]{0.1*x};
% 						\addplot[smooth,mark=*,blue,line width=1.25pt]coordinates{	(0,0)	};
% 						\addplot[smooth,mark=*,blue,line width=1.25pt,fill=white]coordinates{	(8.5,0.85)	};
% 				\addplot[blue, line width=1.25pt]expression[domain=8.5:34.5,samples=200]{0.15*x};
% 						\addplot[smooth,mark=*,blue,line width=1.25pt]coordinates{	(8.5,1.275)	};
% 						\addplot[smooth,mark=*,blue,line width=1.25pt,fill=white]coordinates{	(34.5,5.175)	};
% 				\addplot[blue, line width=1.25pt]expression[domain=34.5:83.6,samples=200]{0.25*x};
% 						\addplot[smooth,mark=*,blue,line width=1.25pt]coordinates{	(34.5,8.625)	};
% 						\addplot[smooth,mark=*,blue,line width=1.25pt,fill=white]coordinates{	(83.6,20.9)	};
% 				\addplot[blue, line width=1.25pt]expression[domain=83.6:174.4,samples=200]{0.28*x};
% 						\addplot[smooth,mark=*,blue,line width=1.25pt]coordinates{	(83.6,23.408)	};
% 						\addplot[smooth,mark=*,blue,line width=1.25pt,fill=white]coordinates{	(174.4,48.832)	};
% 				\addplot[blue, line width=1.25pt]expression[domain=174.4:379.15,samples=200]{0.33*x};
% 						\addplot[smooth,mark=*,blue,line width=1.25pt]coordinates{	(174.4,57.552)	};
% 						\addplot[smooth,mark=*,blue,line width=1.25pt,fill=white]coordinates{	(379.15,125.1195)	};
% 				\addplot[blue, line width=1.25pt]expression[domain=379.15:500,samples=200,->]{0.35*x};
% 						\addplot[smooth,mark=*,blue,line width=1.25pt]coordinates{	(379.15,132.7025)	};
% 	\end{axis}
% 		\end{tikzpicture}
% \end{center}



% 		\end{minipage}
% 		\begin{minipage}{0.5\linewidth}
% Write the formula for $T(x)$.\\~\\~\\~\\~\\~\\~\\~\\~\\~\\~\\

% 		\end{minipage}
% \end{myExample}




% %======================================================
% \newpage
% %======================================================

% \begin{myExample}
% Consider the function $g(x) = |x|$, the absolute value function.
% \begin{enumerate}
% 	\item When you first learned about the absolute value, how did you describe what is measured?\\~\\~\\
% 	\item In MTH 95 (Intermediate Algebra), what was another way you learned that you could write $g$?\\~\\~\\
% 	\item If you had to explain to someone in words how to computer an absolute value, what would you say?\\~\\~\\
% 	\item The graph of $y=g(x)$ (the absolute value function) is given in Figure~\ref{fig:absval}.  \\
% 	Write this formula for this function as a piecewise-defined function.\\
% \begin{minipage}{0.5\linewidth}
% Formula: \\[4cm]

% \end{minipage}
% \begin{minipage}{0.5\linewidth}
% \begin{center}
% \captionof{figure}{}
% \begin{tikzpicture}
% 		\begin{axis}[
% 				framed,
% 				height=8cm,
% 				width=8cm,
% 				xlabel={$x$},
% 				ylabel={$y$},
% 				xmin=-8,xmax=8,
% 				ymin=-8,ymax=8,
%         xtick={-6,-4,...,6},
%         	minor xtick={-7,-5,...,7},
%         ytick={-6,-4,...,6},
%         	minor ytick={-7,-5,...,7},
%         grid=both
% 				]
% 				% use TeX as calculator:
% 				\addplot[blue,line width=1.25pt,samples=200,<-]expression[domain=-8:0]{-1*x};
% 				\addplot[blue,line width=1.25pt,samples=200,->]expression[domain=0:8]{x};
% 		\end{axis}
% 		\end{tikzpicture}
% 		\label{fig:absval}
% \end{center}
% \end{minipage}
% \end{enumerate}

% \end{myExample}


% %======================================================
% \newpage
% %======================================================




% \begin{myExample}

% Let $k$ and $m$ be as defined below.
% %\begin{minipage}{0.6\linewidth}
% \vspace{-1cm}
% \begin{multicols}{2}
% \begin{center}
% 				\[
% 				k(t)=
% 					\begin{cases}
% 						-2t-4								& \textrm{if}\ 	\ \			t \le -1			\\
% 						 2t+3							& \textrm{if}\ 	\ \			t > -1 		
% 					\end{cases}
% 				\]  
% \end{center}
% \begin{center}
% 				\[
% 				m(x)=
% 					\begin{cases}
% 						 x-3								& \textrm{if}\ 	\ \			0 < x \le 4 		\\
% 						-x-4								& \textrm{if}\ 	\ \			x \le 0	
% 					\end{cases}
% 				\]
% \end{center}
% \end{multicols}
% %\end{minipage}

% \begin{enumerate}
% \begin{multicols}{2}
% \item What is the domain of $k$?\\~\\~\\
% \item What is the domain of $m$?\\~\\~\\
% \end{multicols}
% \begin{multicols}{2}
% \item Find $k(10)$.\\~\\~\\~\\
% \item Find $m(3)$.\\~\\~\\~\\
% \end{multicols}
% \item Find $k(-9)$.\\~\\~\\~\\
% \item Find $m(0)$.\\~\\~\\~\\
% \begin{multicols}{2}
% \item Graph $y=k(t)$ in Figure~\ref{fig:pwd12} below.
% \item Graph $y=m(x)$ in Figure~\ref{fig:pwd22} below.
% \end{multicols}

% \vspace{-1cm}
% \begin{multicols}{2}
% \begin{center}
% \captionof{figure}{$y=k(t)$}
% \begin{tikzpicture}
% 		\begin{axis}[
% 				framed,
% 				height=7cm,
% 				width=7cm,
% 				xmin=-8,xmax=8,
% 				ymin=-8,ymax=8,
%         xtick={0},
%         	minor xtick={-7,-6,...,7},
%         ytick={0},
%         	minor ytick={-7,-6,...,7},
%         grid=both
% 				]
% 				% use TeX as calculator:
% 		\end{axis}
% 		\end{tikzpicture}
% 		\label{fig:pwd12}
% \end{center}

% \begin{center}
% \captionof{figure}{$y=m(x)$}
% \begin{tikzpicture}
% 		\begin{axis}[
% 				framed,
% 				height=7cm,
% 				width=7cm,
% 				xmin=-8,xmax=8,
% 				ymin=-8,ymax=8,
%         xtick={0},
%         	minor xtick={-7,-6,...,7},
%         ytick={0},
%         	minor ytick={-7,-6,...,7},
%         grid=both
% 				]
% 				% use TeX as calculator:
% 		\end{axis}
% 		\end{tikzpicture}
% 		\label{fig:pwd22}
% \end{center}
% \end{multicols}

% \end{enumerate}

% \end{myExample}


% %======================================================
% \newpage
% %======================================================


% \begin{myExample}
% In Figure~\ref{fig:piecewise4} is the graph of $y=F(x)$.	\\
% \begin{enumerate}%\setlength{\itemsep}{1.25in}
% \begin{minipage}[!t]{0.4\linewidth}
% \item Write the formula for the piecewise function $F$: \\[6cm]
% \end{minipage}
% \begin{minipage}{0.6\linewidth}
% \begin{center}
% \captionof{figure}{$y=F(x)$}
% \begin{tikzpicture}
% 		\begin{axis}[
% 				framed,
% 				height=8cm,
% 				width=8cm,
% 				xlabel={$x$},
% 				ylabel={$y$},
% 				xmin=-8,xmax=8,
% 				ymin=-8,ymax=8,
%         xtick={-6,-4,...,6},
%         	minor xtick={-7,-5,...,7},
%         ytick={-6,-4,...,6},
%         	minor ytick={-7,-5,...,7},
%         grid=both
% 				]
% 				% use TeX as calculator:
% 				\addplot[blue,line width=1.25pt,samples=200,<-]expression[domain=-8:-2]{1/2*(x+2)-3};
% 						\addplot[mark=*,blue,line width=1.0pt]coordinates{	(-2,-3)	};
% 				\addplot[blue,line width=1.25pt,samples=200]expression[domain=-2:3]{5};
% 						\addplot[mark=*,blue,line width=1.0pt,fill=white]coordinates{	(-2,5)	};
% 						\addplot[mark=*,blue,line width=1.0pt]coordinates{	(3,5)	};
% 				\addplot[blue,line width=1.25pt,samples=200]expression[domain=3:7]{-2*(x-6)-5};
% 						\addplot[mark=*,blue,line width=1.0pt,fill=white]coordinates{	(3,1)	};
% 						\addplot[mark=*,blue,line width=1.0pt,fill=white]coordinates{	(7,-7)	};
% 		\end{axis}
% 		\end{tikzpicture}
% 		\label{fig:piecewise4}
% \end{center}
% \end{minipage}
% %\vspace{11pt}
% \begin{multicols}{2}
% 	\item	Evaluate $F(0)$.	

% 	\item	Solve $F(x)=0$.	
% \end{multicols}
% ~\\
% \begin{multicols}{2}
% 	\item	Evaluate $F(2)$.
% 	\item	Solve $F(x)=-5$.	
% \end{multicols}
% ~\\

% 	\item	Solve $F(x)=-5$ both algebraically (using your formula from Part a) and graphically.	
% \vfill
% \end{enumerate}

% \end{myExample}








\resetCounters
\section{Inverse Functions Supplement} \label{appendix-inverse}

\resetCounters
%============================================================
% MTH 111Z Project - Template File
% Section 6.1 from OpenStax OER
%	Updated 202302
%============================================================


%\fakesection{Exponential Functions} \label{exponential-intro}
\section{Exponential Functions Supplement} \label{appendix-exponential}

%Introduction


\subsection*{Examples} \label{examples-appendix-exponential}


\begin{myExample}
Find a formula for an exponential function $f$ that passes through the two points $(0,7)$ and $(5,7168)$. 
\end{myExample}

\begin{mySolution}
		Since the graph of $y=f(x)$ passes though $(0,7)$, we know that $f(0)=7$.  Combining that with $f(x) = a\cdot b^x$, we have
		\begin{align*}
			f(0)=7 \implies a\cdot b^0&=7 &&\text{(since $f(0) = a\cdot b^0$)}\\
			 a\cdot 1&=7 &&\text{(because $b^0=1$ if $b\neq0$)}\\
			 a&=7 &&
		\end{align*}
		Having determined $a=7$, we now have $f(x) = 7\cdot b^x$.  Since that graph also passes through $(5,7168)$, we also know $f(5)=7168$.  Putting these together, we have
		\begin{align*}
			f(5)=7168 \implies 7\cdot b^5&=7168 &&\text{(since $f(5) = 7\cdot b^5$)}\\
			 b^5&=1024 &&\text{(by dividing both sides of the equation by 7)}\\
			 b&=4 &&\text{(by taking the fifth root of both sides of the equation)}
		\end{align*}
		Finally, putting the $a$ and $b$ values that we calculated together, we have $f(x) = 7\cdot4^x$.
\end{mySolution}






\begin{myExample}
Find a formula for an exponential function $f$ that passes through the two points $(-2,\frac{3}{2})$ and $(3,48)$. 
\end{myExample}

\begin{mySolution}
	Since the graph of $y=f(x)$ passes though $(-2,\frac{3}{2})$, we know that $f(-2)=\frac{3}{2}$.  Combining that with $f(x) = a\cdot b^x$, we have 
		\begin{align*}
			f(-2)=\frac{3}{2} \implies a\cdot b^{-2}&=\frac{3}{2} &&\text{(since $f(-2) = a\cdot b^{-2}$)}\\
			 a&=\frac{3}{2}b^2 &&\text{(by multiplying both sides of the equation by $b^2$ to solve for $a$)}
		\end{align*}
		We will pause with that for a moment and consider that the graph also passes through $(3,48)$, which tells us that $f(3)=48$ or that $a \cdot b^3 = 48$.  If we substitute $a=\frac{3}{2}b^2$ into $a\cdot b^3=48$, we have:
		\begin{align*}
			a\cdot b^3=48 \implies  \frac{3}{2}b^2\cdot b^3 &=  48 &&\text{(since $a = \frac{3}{2}b^2$)}\\
			 \frac{3}{2} b^5&=48 &&\text{(since $b^2\cdot b^3 = b^5$)}\\
			 b^5&=32 &&\text{(by multiplying both sides of the equation by $\frac{2}{3}$)}\\
			 b&=2 &&\text{(by taking the third root of both sides of the equation)}
		\end{align*}
		Now with $b=2$, we can go back to $a = \frac{3}{2}b^2$ to find that $a= \frac{3}{2}\cdot 2^2$ or that $a=6$.

		Putting the $a$ and $b$ values that we calculated together, we have $f(x) = 6\cdot2^x$.
\end{mySolution}

















%%%%%
\newpage
%%%%%

\subsection*{Exercises} \label{exercises-appendix-exponential}


\begin{myExercise}
Find a formula for an exponential function $f$ that passes through the two points given. 
	\begin{enumerate}
		\begin{multicols}{2}
			\item  $(0,50)$ and $(3,400)$
			\item $(0,4)$ and $(4,\frac{1}{4})$
		\end{multicols}	
		\begin{multicols}{2}
			\item $(-1,\frac{2}{3})$ and $(2,18)$
			\item $(-2,\frac{125}{8})$ and $(1,8)$
		\end{multicols}	
	\end{enumerate}
\end{myExercise}

\begin{myExercise}
Find a formula for an exponential function $f$ that satisfies the given input-output pairs. 
	\begin{enumerate}
		\begin{multicols}{2}
			\item $f(-2)= 125$ and $f(3)=\frac{1}{25}$
			\item $f(-3)= \frac{27}{16}$ and $f(3)=\frac{4}{27}$
		\end{multicols}	
	\end{enumerate}
\end{myExercise}

\vfill


\begin{myAnswer}
~\\[-12mm]
	\begin{enumerate}
		\begin{multicols}{2}
			\item $f(x) = 4\cdot \left(\frac{1}{2}\right)^x$
			\item $f(x) = 10\cdot \left(\frac{4}{5}\right)^x$
		\end{multicols}	
	\end{enumerate}
\end{myAnswer}

\begin{myAnswer}
~\\[-12mm]
	\begin{enumerate}
		\begin{multicols}{2}
			\item $f(x)= 5\cdot \left(\frac{1}{5}\right)^x$
			\item $f(x)= \frac{1}{2} \cdot \left(\frac{2}{3}\right)^x$
		\end{multicols}	
	\end{enumerate}
\end{myAnswer}

\vfill




























% %======================================================
% \newpage
% %======================================================

% \begin{myExample}
% What are the domains of the following functions?
% 	\begin{enumerate}
% 		\begin{multicols}{2}
% 		\item $f(x) = 7x(x+8)(x-3)$
% 		\item $g(t) = \dfrac{7t(t+8)}{t-3}$
% 		\end{multicols}
% 		\vfill
% 		\begin{multicols}{2}
% 		\item $j(n) = \sqrt{12-3n}$
% 		\item $k(w) = w^2 -8w-9$
% 		\end{multicols}
% 		\vfill
% 		\begin{multicols}{2}
% 		\item $m(a) = \dfrac{5a-9}{2a+8}$
% 		\item $r(u) = \dfrac{\sqrt{u+3}}{u-7}$
% 		\end{multicols}
% 		\vfill
% 		\vfill

% 	\end{enumerate}
% \end{myExample}


% {\bf Textbook Reference:}\\
% If you need a few more examples of set-builder and interval notations, \\please see the \href{https://openstax.org/books/algebra-and-trigonometry-2e/pages/3-2-domain-and-range#fs-id1165137677916}{Using Notations to Specify Domain and Range} section of \S3.2 in our book.

% %======================================================
% \newpage
% %======================================================

% \begin{center}
%   {\Large {\bf  Part 2: Finding Domains and Ranges from Graphs}}
% \end{center}

% \begin{myExample}
% Let $y=f(x)$ be defined by the graph below.  (Are you sure this is a function?  Why?)\\

% \begin{minipage}{0.4\linewidth}
% \begin{center}
% \begin{tikzpicture}
% 		\begin{axis}[
% 				framed,
% 				width=7cm,height=7cm,
% 				axis x line=middle,
% 				axis y line=middle,
% 				xlabel={$x$},
% 				ylabel={$y$},
% 				xmin=-8,xmax=8,
% 				ymin=-8,ymax=8,
%         xtick={-6,-4,...,6},
%         	minor xtick={-7,-5,...,7},
%         ytick={-6,-4,...,6},
%         	minor ytick={-7,-5,...,7},
%         grid=both
% 				]
% 				% use TeX as calculator:
% 					\addplot[smooth,mark=*,first,line width=1.0pt,fill=white]coordinates{	(-4,2)	};
% 					\addplot[smooth,mark=*,first,line width=1.0pt]coordinates{	(5,-4)	};
% 				\addplot[first,line width=1.5pt,samples=200]expression[domain=-4:4.99999]{2*(-x+5)^(1/2)-4};
% 		\end{axis}
% 		\end{tikzpicture}
% ~\\
% \end{center}
% \end{minipage}
% \begin{minipage}{0.6\linewidth}
% \begin{enumerate}
% 	\item Find $f(1)$.\\
% 	\item Find $f(5)$.\\
% 	\item Find $f(-4)$.\\
% 	\item Find $f(6)$.\\
% \end{enumerate}
% \end{minipage}

% \begin{enumerate}
% \setcounter{enumi}{4}
% 	\item State the domain of this function in interval notation and set-builder notation.\\~\\
% 	\item State the range of this function in interval notation and set-builder notation.\\~\\
% \end{enumerate}
% \end{myExample}


% \begin{myExample}
% Use $y=h(t)$ and $y=k(x)$ to answer the following.
% \vspace{-1cm}
% \begin{multicols}{2}
% \begin{center}
% \captionof{figure}{$y=h(t)$}
% \begin{tikzpicture}
% 		\begin{axis}[
% 				framed,
% 				width=6cm,height=6cm,
% 				xlabel={$t$},
% 				ylabel={$y$},
% 				xmin=-8,xmax=8,
% 				ymin=-8,ymax=8,
%         xtick={-6,-4,...,6},
%         	minor xtick={-7,-5,...,7},
%         ytick={-6,-4,...,6},
%         	minor ytick={-7,-5,...,7},
%         grid=both
% 				]
% 				% use TeX as calculator:
% 				\addplot[blue,line width=1.5pt,samples=400,-]expression[domain=-7:-4]{-x-4};
% 					\addplot[smooth,mark=*,blue,line width=2.0pt]coordinates{	(-7,3)	};
% 				\addplot[blue,line width=1.5pt,samples=400,-]expression[domain=-4:-3]{x+4};
% 				\addplot[blue,line width=1.5pt,samples=400,-]expression[domain=-3:-2]{(x+3)^2+1};
% 				\addplot[blue,line width=1.5pt,samples=400,-]expression[domain=-2:0]{-(x+1)^2+3};
% 				\addplot[blue,line width=1.5pt,samples=400,-]expression[domain=0:2]{-x+2};
% 				\addplot[blue,line width=1.5pt,samples=400,-]expression[domain=2:4]{(x-3)^2-1};
% 				\addplot[blue,line width=1.5pt,samples=400,-]expression[domain=4:6]{x-4};
% 					\addplot[smooth,mark=*,blue,line width=1pt,fill=white]coordinates{	(6,2)	};
% 		\end{axis}
% 		\end{tikzpicture}
% 		\label{fig:dr1}
% \end{center}

% \begin{center}
% \captionof{figure}{$y=k(x)$}
% \begin{tikzpicture}
% 		\begin{axis}[
% 				framed,
% 				width=6cm,height=6cm,
% 				xlabel={$x$},
% 				ylabel={$y$},
% 				xmin=-8,xmax=8,
% 				ymin=-8,ymax=8,
%         xtick={-6,-4,...,6},
%         	minor xtick={-7,-5,...,7},
%         ytick={-6,-4,...,6},
%         	minor ytick={-7,-5,...,7},
%         grid=both
% 				]
% 				% use TeX as calculator:
% 				\addplot[blue,line width=1.5pt,samples=400,-]expression[domain=-7:-5.99]{x+1};
% 				\addplot[smooth,mark=*,blue,line width=1pt,fill=white]coordinates{	(-7,-6)	};
% 				\addplot[blue,line width=1.5pt,samples=400,-]expression[domain=-6.01:-3.99]{2*(x)+7};
% 				\addplot[blue,line width=1.5pt,samples=400,-]expression[domain=-4.01:-2.99]{(x)+3};
% 				\addplot[blue,line width=1.5pt,samples=400,-]expression[domain=-3.01:-0.99]{-1/2*(x)-1.5};
% 				\addplot[blue,line width=1.5pt,samples=400,-]expression[domain=-1.01:0]{2*(x)^2-3};
% 				\addplot[blue,line width=1.5pt,samples=400,-]expression[domain=0:1.01]{-(x)^2-3};
% 				\addplot[blue,line width=1.5pt,samples=400,-]expression[domain=0.99:2]{-(x-2)^2-3};
% 				\addplot[blue,line width=1.5pt,samples=400,-]expression[domain=2:3.01]{2*(x-2)^2-3};
% 				\addplot[blue,line width=1.5pt,samples=400,-]expression[domain=2.99:5.01]{(x-3)^2-1};
% 				\addplot[blue,line width=1.5pt,samples=400,-]expression[domain=4.99:6]{3};
% 				\addplot[smooth,mark=*,blue,line width=1.5pt]coordinates{	(6,3)	};
% 		\end{axis}
% 		\end{tikzpicture}
% 		\label{fig:dr2}
% \end{center}
% \end{multicols}

% 		\begin{enumerate}\setlength{\itemsep}{0.85in}
% 		\begin{multicols}{2}
% 				\item		State the domain of $h$.
% 				\item		State the domain of $k$.
% 		\end{multicols}
% \vfill
% 		\begin{multicols}{2}
% 				\item		State the range of $h$.
% 				\item		State the range of $k$.
% 		\end{multicols}
% \vfill
% 		\end{enumerate}	
% \end{myExample}

% %======================================================
% \newpage
% %======================================================

% \begin{center}
%   {\Large {\bf  Part 3: Piecewise-Defined Functions}}
% \end{center}


% \begin{myDefinition} 
% A function that is defined by different formulas on different parts of its domain is called a\\
% {\bf \underline{piecewise-defined function}}.  Piecewise-defined functions take on the form:\\  
% 				\[
% 				f(x)=
% 					\begin{cases}
% 							\text{formula \#1}				&	\textrm{if}\ 	\ \			\text{$x$ is in this part of the domain}		\\
% 							\text{formula \#2}				&	\textrm{if}\ 	\ \			\text{$x$ is in this part of the domain}		\\
% 							\text{etc.}				&	\textrm{if}\ 	\ \			\text{etc.}		\\
% 					\end{cases}
% 				\]
% \ 

% \end{myDefinition}

% \begin{myExample}
% In Table~\ref{tab:tax1}, the 2011 federal income tax rates\footnote{\url{http://www.irs.gov/newsroom/article/0,,id=233465,00.html}} for 2011 are shown.  

% \renewcommand{\arraystretch}{1.5}
% \begin{table}[h!]
% \centering
% \caption{Federal Income Tax Percentage Rates for 2011}
% \label{tab:tax1} \small{
% 	\begin{tabular}{| c | c | c |}
% 	\hline 
% 		Income Amount $(x)$										& Percentage of Income Taxed (in $\%$) & Tax on $x$ Dollars of Income ($T(x)$, in \$) \\ \hline
% 		\	$0 \le x < 8500$										&				10		&	$0.10x$								\\ \hline			
% 			$8500 \le x < 34500$								&				15			&	$0.15x$							\\ \hline			
% 			$34500 \le x < 83600$								&				25			&	$0.25x$							\\ \hline			
% 			$83600 \le x < 174400$							&				28				&	$0.28x$						\\ \hline										
% 			$174400 \le x < 379150 $						&				33					&	$0.33x$					\\ \hline				
% 			$x \ge 379150$											&				35	&	$0.35x$									\\ \hline					
% 	\end{tabular}}
% \end{table}

% Notice that for each interval, the percentage of income taxed as a function of income is \emph{constant}.  If we graph each \emph{piece} over its respective interval, we obtain the following: \\
% 		\begin{minipage}{0.5\linewidth}

% \begin{center}
% 	\captionof{figure}{Graph of $y=T(x)$}
% 	\label{fig:b}
% 	   	\begin{tikzpicture}
% 			\begin{axis}[
% 				framed,
% 				height=3in,
% 				width=3in,
% 				xmin=0,xmax=500,
% 				ymin=0,ymax=180,
% 					xlabel style={at={(ticklabel cs:0.5)},anchor=near ticklabel},
%       		ylabel style={at={(ticklabel cs:0.5)},rotate=90,anchor=near ticklabel},
% 				xlabel={\tiny{$x$, income amount in \$1000s}},
% 				ylabel={\tiny{ $y$, tax amount in \$1000s}},
% 				xtick={0,50,100,...,450},
% 					 minor xtick={25,75,...,475},
% 				ytick={10,20,...,180},
% 				%	  minor ytick={1,2,3,...,39},
% 				grid=both,
% 				]
% 				% use TeX as calculator:
% 				\addplot[blue, line width=1.25pt]expression[domain=0:8.5,samples=200]{0.1*x};
% 						\addplot[smooth,mark=*,blue,line width=1.25pt]coordinates{	(0,0)	};
% 						\addplot[smooth,mark=*,blue,line width=1.25pt,fill=white]coordinates{	(8.5,0.85)	};
% 				\addplot[blue, line width=1.25pt]expression[domain=8.5:34.5,samples=200]{0.15*x};
% 						\addplot[smooth,mark=*,blue,line width=1.25pt]coordinates{	(8.5,1.275)	};
% 						\addplot[smooth,mark=*,blue,line width=1.25pt,fill=white]coordinates{	(34.5,5.175)	};
% 				\addplot[blue, line width=1.25pt]expression[domain=34.5:83.6,samples=200]{0.25*x};
% 						\addplot[smooth,mark=*,blue,line width=1.25pt]coordinates{	(34.5,8.625)	};
% 						\addplot[smooth,mark=*,blue,line width=1.25pt,fill=white]coordinates{	(83.6,20.9)	};
% 				\addplot[blue, line width=1.25pt]expression[domain=83.6:174.4,samples=200]{0.28*x};
% 						\addplot[smooth,mark=*,blue,line width=1.25pt]coordinates{	(83.6,23.408)	};
% 						\addplot[smooth,mark=*,blue,line width=1.25pt,fill=white]coordinates{	(174.4,48.832)	};
% 				\addplot[blue, line width=1.25pt]expression[domain=174.4:379.15,samples=200]{0.33*x};
% 						\addplot[smooth,mark=*,blue,line width=1.25pt]coordinates{	(174.4,57.552)	};
% 						\addplot[smooth,mark=*,blue,line width=1.25pt,fill=white]coordinates{	(379.15,125.1195)	};
% 				\addplot[blue, line width=1.25pt]expression[domain=379.15:500,samples=200,->]{0.35*x};
% 						\addplot[smooth,mark=*,blue,line width=1.25pt]coordinates{	(379.15,132.7025)	};
% 	\end{axis}
% 		\end{tikzpicture}
% \end{center}



% 		\end{minipage}
% 		\begin{minipage}{0.5\linewidth}
% Write the formula for $T(x)$.\\~\\~\\~\\~\\~\\~\\~\\~\\~\\~\\

% 		\end{minipage}
% \end{myExample}




% %======================================================
% \newpage
% %======================================================

% \begin{myExample}
% Consider the function $g(x) = |x|$, the absolute value function.
% \begin{enumerate}
% 	\item When you first learned about the absolute value, how did you describe what is measured?\\~\\~\\
% 	\item In MTH 95 (Intermediate Algebra), what was another way you learned that you could write $g$?\\~\\~\\
% 	\item If you had to explain to someone in words how to computer an absolute value, what would you say?\\~\\~\\
% 	\item The graph of $y=g(x)$ (the absolute value function) is given in Figure~\ref{fig:absval}.  \\
% 	Write this formula for this function as a piecewise-defined function.\\
% \begin{minipage}{0.5\linewidth}
% Formula: \\[4cm]

% \end{minipage}
% \begin{minipage}{0.5\linewidth}
% \begin{center}
% \captionof{figure}{}
% \begin{tikzpicture}
% 		\begin{axis}[
% 				framed,
% 				height=8cm,
% 				width=8cm,
% 				xlabel={$x$},
% 				ylabel={$y$},
% 				xmin=-8,xmax=8,
% 				ymin=-8,ymax=8,
%         xtick={-6,-4,...,6},
%         	minor xtick={-7,-5,...,7},
%         ytick={-6,-4,...,6},
%         	minor ytick={-7,-5,...,7},
%         grid=both
% 				]
% 				% use TeX as calculator:
% 				\addplot[blue,line width=1.25pt,samples=200,<-]expression[domain=-8:0]{-1*x};
% 				\addplot[blue,line width=1.25pt,samples=200,->]expression[domain=0:8]{x};
% 		\end{axis}
% 		\end{tikzpicture}
% 		\label{fig:absval}
% \end{center}
% \end{minipage}
% \end{enumerate}

% \end{myExample}


% %======================================================
% \newpage
% %======================================================




% \begin{myExample}

% Let $k$ and $m$ be as defined below.
% %\begin{minipage}{0.6\linewidth}
% \vspace{-1cm}
% \begin{multicols}{2}
% \begin{center}
% 				\[
% 				k(t)=
% 					\begin{cases}
% 						-2t-4								& \textrm{if}\ 	\ \			t \le -1			\\
% 						 2t+3							& \textrm{if}\ 	\ \			t > -1 		
% 					\end{cases}
% 				\]  
% \end{center}
% \begin{center}
% 				\[
% 				m(x)=
% 					\begin{cases}
% 						 x-3								& \textrm{if}\ 	\ \			0 < x \le 4 		\\
% 						-x-4								& \textrm{if}\ 	\ \			x \le 0	
% 					\end{cases}
% 				\]
% \end{center}
% \end{multicols}
% %\end{minipage}

% \begin{enumerate}
% \begin{multicols}{2}
% \item What is the domain of $k$?\\~\\~\\
% \item What is the domain of $m$?\\~\\~\\
% \end{multicols}
% \begin{multicols}{2}
% \item Find $k(10)$.\\~\\~\\~\\
% \item Find $m(3)$.\\~\\~\\~\\
% \end{multicols}
% \item Find $k(-9)$.\\~\\~\\~\\
% \item Find $m(0)$.\\~\\~\\~\\
% \begin{multicols}{2}
% \item Graph $y=k(t)$ in Figure~\ref{fig:pwd12} below.
% \item Graph $y=m(x)$ in Figure~\ref{fig:pwd22} below.
% \end{multicols}

% \vspace{-1cm}
% \begin{multicols}{2}
% \begin{center}
% \captionof{figure}{$y=k(t)$}
% \begin{tikzpicture}
% 		\begin{axis}[
% 				framed,
% 				height=7cm,
% 				width=7cm,
% 				xmin=-8,xmax=8,
% 				ymin=-8,ymax=8,
%         xtick={0},
%         	minor xtick={-7,-6,...,7},
%         ytick={0},
%         	minor ytick={-7,-6,...,7},
%         grid=both
% 				]
% 				% use TeX as calculator:
% 		\end{axis}
% 		\end{tikzpicture}
% 		\label{fig:pwd12}
% \end{center}

% \begin{center}
% \captionof{figure}{$y=m(x)$}
% \begin{tikzpicture}
% 		\begin{axis}[
% 				framed,
% 				height=7cm,
% 				width=7cm,
% 				xmin=-8,xmax=8,
% 				ymin=-8,ymax=8,
%         xtick={0},
%         	minor xtick={-7,-6,...,7},
%         ytick={0},
%         	minor ytick={-7,-6,...,7},
%         grid=both
% 				]
% 				% use TeX as calculator:
% 		\end{axis}
% 		\end{tikzpicture}
% 		\label{fig:pwd22}
% \end{center}
% \end{multicols}

% \end{enumerate}

% \end{myExample}


% %======================================================
% \newpage
% %======================================================


% \begin{myExample}
% In Figure~\ref{fig:piecewise4} is the graph of $y=F(x)$.	\\
% \begin{enumerate}%\setlength{\itemsep}{1.25in}
% \begin{minipage}[!t]{0.4\linewidth}
% \item Write the formula for the piecewise function $F$: \\[6cm]
% \end{minipage}
% \begin{minipage}{0.6\linewidth}
% \begin{center}
% \captionof{figure}{$y=F(x)$}
% \begin{tikzpicture}
% 		\begin{axis}[
% 				framed,
% 				height=8cm,
% 				width=8cm,
% 				xlabel={$x$},
% 				ylabel={$y$},
% 				xmin=-8,xmax=8,
% 				ymin=-8,ymax=8,
%         xtick={-6,-4,...,6},
%         	minor xtick={-7,-5,...,7},
%         ytick={-6,-4,...,6},
%         	minor ytick={-7,-5,...,7},
%         grid=both
% 				]
% 				% use TeX as calculator:
% 				\addplot[blue,line width=1.25pt,samples=200,<-]expression[domain=-8:-2]{1/2*(x+2)-3};
% 						\addplot[mark=*,blue,line width=1.0pt]coordinates{	(-2,-3)	};
% 				\addplot[blue,line width=1.25pt,samples=200]expression[domain=-2:3]{5};
% 						\addplot[mark=*,blue,line width=1.0pt,fill=white]coordinates{	(-2,5)	};
% 						\addplot[mark=*,blue,line width=1.0pt]coordinates{	(3,5)	};
% 				\addplot[blue,line width=1.25pt,samples=200]expression[domain=3:7]{-2*(x-6)-5};
% 						\addplot[mark=*,blue,line width=1.0pt,fill=white]coordinates{	(3,1)	};
% 						\addplot[mark=*,blue,line width=1.0pt,fill=white]coordinates{	(7,-7)	};
% 		\end{axis}
% 		\end{tikzpicture}
% 		\label{fig:piecewise4}
% \end{center}
% \end{minipage}
% %\vspace{11pt}
% \begin{multicols}{2}
% 	\item	Evaluate $F(0)$.	

% 	\item	Solve $F(x)=0$.	
% \end{multicols}
% ~\\
% \begin{multicols}{2}
% 	\item	Evaluate $F(2)$.
% 	\item	Solve $F(x)=-5$.	
% \end{multicols}
% ~\\

% 	\item	Solve $F(x)=-5$ both algebraically (using your formula from Part a) and graphically.	
% \vfill
% \end{enumerate}

% \end{myExample}








\resetCounters
\include{111-section-24-appendix-logarithms}
 










%\resetCounters
%\include{111-section-8-exponential-graphs}


%\resetCounters
%\include{111-section-10-logarithms-graphs}




%%%%%%%%%%%%%%%%%%%%%%
\fi
%%%%%%%%%%%%%%%%%%%%%%



\end{document}