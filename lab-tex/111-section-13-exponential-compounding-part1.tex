%============================================================
% MTH 111Z Project - Template File
% Part of Section 6.1 from OpenStax OER
%	Updated 202302
%============================================================

\section{Exponential Functions and Compound Interest} \label{exponential-compounding}

In this section, we'll build upon our understanding of the general exponent function $f(x) = a\cdot b^x$ and see how variations of this formula are used in financial situations. \\[0.5em]
\textbook{6.1}


\subsection*{Preparation Exercises} \label{prep-exponential-compounding}

\begin{myPrep}
The function $V(t) = 52.8(1.093)^t$ represents the value (in thousands of dollars) of a collectable car $t$ years after June 1st, 2015.
	\begin{enumerate}
		\item What was the value of the car on June 1st, 2015?
		\vfill
		\item Is the value of the car increasing or decreasing and at what rate is the value of the car changing?
		\vfill
	\end{enumerate}
\end{myPrep}

\begin{myPrep}
Simple Interest is calculated using the formula $I = P r t$, where $I$ is the interest earned, $P$ is the principal or initial amount of money, $r$ is the interest rate, and $t$ is the amount of time that the interest is earned.
	\begin{enumerate}
		\item How much simple interest is owed after 7 years on a \$4,000 loan that earns 6\% annual simple interest?  How much will need to be paid back in total at the end of 7 years?
				\vfill
		\vfill
		\item How much simple interest is earned after 6 years on a \$3,000 investment that earns 4\% annual simple interest?  
		\vfill
		\vfill
	\end{enumerate}
\end{myPrep}


