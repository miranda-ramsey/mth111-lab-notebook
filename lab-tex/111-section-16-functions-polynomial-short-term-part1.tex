%============================================================
% MTH 111Z Project - Template File
% Section 5.3 from OpenStax OER
%	Updated 202302
%============================================================
\section{Graphs of Polynomial Functions} \label{functions-polynomial-graphs}

In this section, we'll focus on the short-term behaviors of polynomial functions and the location and behavior of the $x$-intercepts can be identified from their factored algebraic form.  We'll then use this knowledge to both graph polynomial functions, as well as construct polynomial functions based on a graph. \\[0.5em]
\textbook{5.3}




\subsection*{Preparation Exercises} \label{prep-functions-polynomial-graphs}


\begin{myPrep}
Let $f(x) =-2(x+3)(x-2)^2(x+1)$.
	\begin{enumerate}
		\item Describe the end behavior of the graph of $y=f(x)$.
		\vfill

		\item What are the $x$-intercept(s) of $y=f(x)$?
		\vfill

		\item What is the $y$-intercept of $y=f(x)$?
		\vfill
	
		\item At most how many turning points does the graph of $y=f(x)$ have?
		\vfill

	\end{enumerate}
\end{myPrep}

\begin{myPrep}

Let $g(x) =-2(x+3)^2(x-2)(x+1)^3$.
	\begin{enumerate}
		\item Describe the end behavior of the graph of $y=g(x)$.
		\vfill

		\item What are the $x$-intercept(s) of $y=g(x)$?
		\vfill

		\item What is the $y$-intercept of $y=g(x)$?
		\vfill
	
		\item At most how many turning points does the graph of $y=g(x)$ have?
		\vfill

	\end{enumerate}
\end{myPrep}

