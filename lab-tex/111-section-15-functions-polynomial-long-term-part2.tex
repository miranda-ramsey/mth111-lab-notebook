


%======================================================
\newpage
%======================================================


\subsection*{Practice Exercises} \label{practice-functions-polynomial-long-term}


\begin{myPractice}
Let $f(x) = \frac{1}{2} x^4 +3x^3 -16x$, which in factored form is $f(x) = \frac{1}{2}x(x-2)(x+4)^2$.
	\begin{enumerate}
		\item Describe the end behavior of the graph of $y=f(x)$.
		\vfill

		\item At most how many turning points does the graph of $y=f(x)$ have?
		\vfill

		\item What is the $y$-intercept of $y=f(x)$?
		\vfill

		\item What are the $x$-intercept(s) of $y=f(x)$?
		\vfill
	\end{enumerate}
\end{myPractice}



\begin{myPractice}
Let $g(x) = 2x^3-5x^2-4x+12$, which in factored form is $g(x) = (2x+3)(x-2)^2$.
	\begin{enumerate}
		\item Describe the end behavior of the graph of $y=g(x)$.
		\vfill

		\item At most how many turning points does the graph of $y=g(x)$ have?
		\vfill

		\item What is the $y$-intercept of $y=g(x)$?
		\vfill

		\item What are the $x$-intercept(s) of $y=g(x)$?
		\vfill
	\end{enumerate}
\end{myPractice}



%======================================================
\newpage
%======================================================

\subsection*{Definitions} \label{def-functions-polynomial-long-term}


\begin{myDefinition}[Power Function]~\\[0.5mm]
A {\bf power function} is a function that can be represented in the form $${f(x) = kx^p},$$
where $k$ and $p$ are real numbers.  $k$ is called the coefficient.
\end{myDefinition}

\begin{myDefinition}[Polynomial Function]~\\[0.5mm]
A \textbf{polynomial function} is of the form $${p(x) = a_n x^n + a_{n-1} x^{n-1} + \cdots + a_1 x + a_0},$$
where $a_n, ~a_{n-1}, ~\dots,~ a_1, ~a_0$ are real numbers, $a_n\neq 0$, and $n$ is a non-negative integer.
			
The \textbf{leading term} is the highest degree term, $a_n x^n$. 

The \textbf{degree} of the polynomial is $n$.
			
The \textbf{leading coefficient} is the coefficient of the leading term, $a_n$. \\[0.5em]
\defexample The polynomial $5x^2-8x+4$ has a leading term of $5x^2$, is second degree polynomial,  and has a leading coefficient of $5$
\end{myDefinition}


\begin{myDefinition}[$\boldsymbol{x}$-intercept or Horizontal Intercept]~\\[0.5mm]
A {\bf horizontal intercept} or {\bf $\boldsymbol{x}$-intercept} of a graph is a point where the graph intersects the horizontal or $x$-axis.  This occurs when the function has an output value of 0.\\[0.5em]
\defexample We can find the horizontal intercepts of the graph of $f(x) = 5x^2-8x+4$ by solving $f(x)=0$.
\end{myDefinition}

\begin{myDefinition}[$\boldsymbol{y}$-intercept or Vertical Intercept]~\\[0.5mm]
A {\bf vertical intercept} or {\bf $\boldsymbol{y}$-intercept} of a graph is a point where the graph intersects the vertical or $y$-axis.  This occurs when the function has an input value of 0.\\[0.5em]
\defexample We can find the vertical intercept of the graph of $f(x) = 5x^2-8x+4$ by evaluating $f(0)$.
\end{myDefinition}

\begin{myDefinition}[End Behavior or Long-Term Behavior]~\\[0.5mm]
The {\bf end behavior} or {\bf long-term behavior} of a polynomial function is determined by its leading term.  The long-term behavior of the polynomial function will be consistent with the power function that is the leading term of the polynomial.\\[0.5em]
\defexample The end behavior of $f(x) = \boldsymbol{5x^4}{}-7x^2+8x-9$ will be the same as the end behavior of $y=5x^4$.
\end{myDefinition}


\begin{myDefinition}[Turning Point]~\\[0.5mm]
A {\bf turning point} of a polynomial function is a point at which the graph changes from increasing to decreasing or from decreasing to increasing.\\[0.5em]
\defexample The graph of  $f(x) = 5x^4-7x^2+8x-9$ has at most 3 turning points.
\end{myDefinition}


\begin{myDefinition}[Root or Zero]~\\[0.5mm]
A {\bf root} or {\bf zero} of a polynomial function is a value $r$ for which $f(r)=0$.\\
$r$ is a zero of a polynomial function $f$ if and only if $(x-r)$ is a factor of $f$.\\[0.5em]
\defexample If $7$ is a zero of a polynomial, then $(x-7)$ is a factor of the polynomial.\\
\defexample  If $(x+3)$ is a factor of a polynomial, then $-3$ is a zero of the polynomial.
\end{myDefinition}




%======================================================
 \newpage
%======================================================

\subsection*{Exit Exercises} \label{exit-functions-polynomial-long-term}

\begin{myExit}
	\begin{enumerate}
		\item What does the degree of a polynomial function tell you about the graph of the function?
		\vfill
		\item What does the leading coefficient of a polynomial function tell you about the graph of the function?
		\vfill
		\item If you know the graph of a polynomial function has 7 turning points, what can you say about the degree of the function?
		\vfill
		\item What do we know about a polynomial function's formula if we know the function has the following behavior?\\
		As $x\rightarrow -\infty$, $f(x)\rightarrow \infty$ and as $x\rightarrow \infty$, $f(x)\rightarrow -\infty$.
		\vfill
	\end{enumerate}
\end{myExit}


\begin{myExit}
State the degree, leading coefficient, long-term behavior, $y$-intercept, and the $x$-intercepts for function \\$f(x) = \frac{1}{3}(x-4)(x+5)^2(x+1)$.
		\vfill
\end{myExit}
\vfill

\exitlikert{polynomial functions}








