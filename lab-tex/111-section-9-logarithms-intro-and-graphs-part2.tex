
%======================================================
\newpage
%======================================================


\subsection*{Practice Exercises } \label{practice-logarithmic-intro}

\begin{myPractice}
Convert each exponential statement into a logarithmic statement.
	\begin{enumerate}
		\begin{multicols}{3}
			\item $3^4=81$
			\item $5^{-2}=\dfrac{1}{25}$
			\item $7^{0}=1$
		\end{multicols}
	\end{enumerate}
	\vfill
\end{myPractice}


\begin{myPractice}
Evaluate each logarithm without a calculator.
	\begin{enumerate}
		\begin{multicols}{3}
			\item $\log \left(10000\right)$
			\item $\log_{25}\left(5\right)$
			\item $\log_4\left(\frac{1}{64}\right)$
		\end{multicols}
	\end{enumerate}
	\vfill
\end{myPractice}


\begin{myPractice}
Let $f(x)=7^x$ and $g(x)= \log_7(x)$.
	\begin{enumerate}
		\begin{multicols}{2}
			\item What is the domain of $f$?
			\item What is the range of $f$?
		\end{multicols}
		\vfill
		\begin{multicols}{2}
			\item What is the domain of $g$?
			\item What is the range of $g$?
		\end{multicols}
		\vfill
	\end{enumerate}
\end{myPractice}

\begin{myPractice}
Match each function with one of the graphs in Figure~\ref{fig:log-practice2}.\\
\begin{minipage}{0.5\linewidth}
	   \begin{center}
 		  \captionof{figure}{}
   		\label{fig:e1}
					\begin{tikzpicture}
					\begin{axis}[
					framedaxes,
					height=2.5in,
					width=3.5in,
					xlabel={$x$},
					ylabel={$y$},
					xmin=-1,xmax=6,
					ymin=-6,ymax=6,
					grid=both,
					xtick={-15,15},
					ytick={-15,15},
					]
					% use TeX as calculator:
%								\addplot[myred,dashed,line width=0.75pt,<->]expression[domain=-4:4,samples=200]{-.05} ;
					\addplot[first,<->,line width=1.0pt]expression[domain=0.0014:6,samples=400]{ln(x)/ln(3)};
					\addplot[first,<->,line width=1.0pt]expression[domain=0.042:6,samples=400]{ln(x)/ln(1.7)};
					\addplot[first,<->,line width=1.0pt]expression[domain=0.016:6,samples=400]{ln(x)/ln(1/2)};
					\addplot[first,<->,line width=1.0pt]expression[domain=0.21:4.78,samples=400]{ln(x)/ln(1.3)};
					\node at (axis cs:5,5.5) {$\boldsymbol{A}$};	
					\node at (axis cs:5.5,3.75) {$\boldsymbol{B}$};	
					\node at (axis cs:5.5,1) {$\boldsymbol{C}$};	
					\node at (axis cs:5.5,-3) {$\boldsymbol{D}$};	
					\end{axis}
					\end{tikzpicture}
					\label{fig:log-practice2}
			\end{center}
\end{minipage}
\begin{minipage}{0.5\linewidth}
	\setlist{itemsep=16pt}
	\begin{itemize}
		\item $f(x) = \log_{1.7}(x)$ is graph \underline{~~~~~~~}
		\item $g(x) = \log_{1/2}(x)$ is graph \underline{~~~~~~~}
		\item $h(x) =  \log_{1.3}(x)$ is graph \underline{~~~~~~~}
		\item $j(x) =  \log_{3}(x)$ is graph \underline{~~~~~~~}
	\end{itemize}
\end{minipage}
\end{myPractice}









%======================================================
\newpage
%======================================================

\subsection*{Definitions } \label{def-logarithmic-intro}

\begin{myDefinition}[Logarithm:]~\\[0.5mm]
For any real number $x>0$, the \textbf{logarithm with base $\boldsymbol{b}$ of $\boldsymbol{x}$}, where $b>0$ and $b\neq 1$, is denoted by $\boldsymbol{\log_b (x)}$ and is defined by
			$$
					y = \log_b (x) \ \  \textrm{if and only if} \ \  x = b^y
			$$
\end{myDefinition}

\begin{myDefinition}[Common Logarithm:]~\\[0.5mm]
The {\bf common logarithm}, $\boldsymbol{\log(x)}$, is a logarithm with base 10 and satisfies
$$ \ y=\log(x) \ \ \text{is equivalent to} \ \ 10^y=x, \text{for } x>0 $$
 \end{myDefinition}

\begin{myDefinition}[Natural Logarithm:]~\\[0.5mm]
The {\bf natural logarithm}, $\boldsymbol{\ln(x)}$, is a logarithm with base $e$ and satisfies the following:
$$ \ y=\ln(x) \ \ \text{is equivalent to} \ \ e^y=x, \text{for } x>0 $$
%For $x>0$, $y=\ln(x)$ is equivalent to $e^y=x$. 
\end{myDefinition}


\begin{myDefinition}[Key Characteristics of Logarithmic Functions:]~\\[0.5mm]
For a logarthmic function $f(x) = \log_b(x)$, with $b> 0$, $b\neq1$, and $x>0$, we have the following:\\[3mm]
\begin{minipage}{0.6\linewidth}
	\begin{itemize}
	\setlength{\itemsep}{1mm}
		\item $(1,0)$ is the horizontal intercept.
		\item There is no vertical intercept.
		\item The domain of $f$ is $(0, \infty)$.
		\item The range of $f$ is $(-\infty,\infty)$.
		\item $f$ is a one-to-one function.
		\item The vertical asymptote is $x=0$.
		\item If $b>1$, then $f$ is an increasing function and
			\begin{itemize}
			\setlength{\itemsep}{0in}
				\item[\small{•}] as $x\rightarrow \infty$, $f(x)\rightarrow \infty$, and
				\item[\small{•}] as $x\rightarrow 0^+$, $f(x)\rightarrow -\infty$.
			\end{itemize}	
		\item If $0<b<1$, then $f$ is a decreasing function and
			\begin{itemize}
			\setlength{\itemsep}{0in}
				\item[\small{•}] as $x\rightarrow \infty$, $f(x)\rightarrow -\infty$, and
				\item[\small{•}] as $x\rightarrow 0^+$, $f(x)\rightarrow \infty$.
			\end{itemize}	
	\end{itemize}
	\end{minipage}
	\begin{minipage}{0.4\linewidth}
			\begin{center}
			\captionof{figure}{$y=\log_b(x), b>1$}~\\[-0.8em]
			\begin{tikzpicture}
 				\begin{axis}[
 					framedaxes,
		 			height=5cm,
 					width=5cm,
 					xlabel={$x$},
 					ylabel={$y$},
		 			xmin=-6,xmax=6,
 					ymin=-6,ymax=6,
				        xtick={-16,16},
				       	minor xtick={-16,16},
				        ytick={-16,16},
	        		 	minor ytick={-16,16},
				         grid=both
 					]
		 		% use TeX as calculator:
					\addplot[first,line width=1.5pt,<->]expression[domain=0.016:6,samples=200]{{ln(x)/ln(2)}};
					\addplot[first,line width=1.0pt,mark=*] coordinates {(1,0)};
					\node at (axis cs:1.75,-0.75) {{\tiny $(1,0)$}};	
					\addplot[smooth,mark=*,first,line width=1.5pt]coordinates{ (2,1)	};
					\node at (axis cs:1.5,1.75) {{\tiny $(b,1)$}};		
		 		\end{axis}
 				\end{tikzpicture}
				\label{fig:log-def1}
%				\end{center}

%			\begin{center}
			\captionof{figure}{$y=\log_b(x), 0<b<1$}
			\begin{tikzpicture}
 				\begin{axis}[
 					framedaxes,
		 			height=5cm,
 					width=5cm,
 					xlabel={$x$},
 					ylabel={$y$},
		 			xmin=-6,xmax=6,
 					ymin=-6,ymax=6,
				        xtick={-16,16},
				       	minor xtick={-16,16},
				        ytick={-16,16},
	        		 	minor ytick={-16,16},
				         grid=both
 					]
		 		% use TeX as calculator:
					\addplot[first,line width=1.5pt,<->]expression[domain=0.088:6,samples=200]{{ln(x)/ln(2/3)}};
					\node at (axis cs:2.5,-0.5) {{\tiny $(1,0)$}};					
					\addplot[smooth,mark=*,first,line width=1.5pt]coordinates{(1,0)	};
					\addplot[smooth,mark=*,first,line width=1.5pt]coordinates{(0.444,2)	};
					\node at (axis cs:1.5,2.75) {{\tiny $(b,1)$}};
		 		\end{axis}
 				\end{tikzpicture}
				\label{fig:log-def2}
				\end{center}

	\end{minipage}
	~\\[0.5em]	
\begin{minipage}{0.9\linewidth}
\defexample \href{https://tiny.cc/111Z-LogFunction}{View this Desmos graph} to see an interactive example of a logarithmic function.  %(url: tiny.cc/111Z-LogFunction)
\end{minipage}
\begin{minipage}{0.1\linewidth}
\flushright \qrcode[height=1cm]{https://tiny.cc/111Z-LogFunction}
\end{minipage}
\end{myDefinition}


%======================================================
 \newpage
%======================================================

\subsection*{Exit Exercises} \label{exit-logarithmic-intro}

\begin{myExit}
	\begin{enumerate}
		\item Why do we call $b$ the base of the logarithm $\log_b(x)$?  Evaluate $\log_2(16)$ and use this in your answer.
		\vfill
		\vfill
		\item Evaluate $\log_9(3)$ and then restate your logarithmic statement as an exponential statement.
		\vfill
		\item For any $b>0$, where $b\neq1$, what is the domain of $f(x)= \log_b(x)$ and why is this the domain of $f$?
		\vfill
		\vfill
		\item Does a logarithmic function have a horizontal or vertical asymptote and why?
		\vfill
		\vfill
		\item Given $f(x) = \log_8(x)$ and $g(x)=-3\cdot \log_8(x+3)-2$, state the sequence of transformations that takes the graph of $y=f(x)$ to the graph of $y=g(x)$ and also state the domain of $g$.
		\vfill
		\vfill
	\end{enumerate}
\end{myExit}
\vfill


\exitlikert{logarithmic functions}




\newpage
~







