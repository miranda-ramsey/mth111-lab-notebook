
%======================================================
\newpage
%======================================================


\subsection*{Practice Exercises} \label{practice-exponential-intro}

\begin{myPractice}
Which of the following functions are exponential functions? \\
For the exponential functions, identify if they represent exponential growth or decay.
\begin{enumerate}
\begin{multicols}{4}
\item $f(x) = 2 \left(\frac{3}{4}\right)^x$
\item $g(x) = 3 \left(-5\right)^x$
\item $h(x) = 4 \left(x\right)^7$
\item $j(x) = 5 9^{x+2}$
\end{multicols}
\vfill
\end{enumerate}
\end{myPractice}


\begin{myPractice}
A baseball card was worth \$50 in 1995 and its value has increased by 7\% per year every year since then.  Find a formula for a function $V$ that models the value of the baseball card $t$ years after 1995.  
\vfill
Evaluate $V(21)$ and explain its meaning in context.
\vfill
\vfill

\end{myPractice}




\begin{myPractice}
Match each function with one of the graphs in Figure~\ref{fig:exp-practice1}.\\
\begin{minipage}{0.5\linewidth}
	   \begin{center}
 		  \captionof{figure}{}~\\[-0.8em]
   		\label{fig:e1}
					\begin{tikzpicture}
					\begin{axis}[
					framedaxes,
					height=2.5in,
					width=3.5in,
					xlabel={$x$},
					ylabel={$y$},
					xmin=-6,xmax=6,
					ymin=-1,ymax=6,
					grid=both,
					xtick={-15,15},
					ytick={-15,15},
					]
					% use TeX as calculator:
%								\addplot[myred,dashed,line width=0.75pt,<->]expression[domain=-4:4,samples=200]{-.05} ;
					\addplot[first,<->,line width=1.0pt]expression[domain=-6:1,samples=400]{2* 3^x};
					\addplot[first,<->,line width=1.0pt]expression[domain=-6:2.71,samples=400]{2* 1.5^x};
					\addplot[first,<->,line width=1.0pt]expression[domain=-1.58:6,samples=400]{2*0.5^x};
					\addplot[first,<->,line width=1.0pt]expression[domain=-6:3.8,samples=400]{3* 1.2^x};
					\node at (axis cs:-1.8,5.5) {$\boldsymbol{A}$};	
					\node at (axis cs:1.3,5.5) {$\boldsymbol{B}$};	
					\node at (axis cs:2.2,5.5) {$\boldsymbol{C}$};	
					\node at (axis cs:4,5.5) {$\boldsymbol{D}$};	
					\end{axis}
					\end{tikzpicture}
					\label{fig:exp-practice1}
			\end{center}
\end{minipage}
\begin{minipage}{0.5\linewidth}
	\setlist{itemsep=16pt}
	\begin{itemize}
		\item $F(x) = 2(3)^x$ is graph \underline{~~~~~~~}
		\item $G(x) = 2 (0.5)^x$ is graph \underline{~~~~~~~}
		\item $H(x) = 3(1.2)^x$ is graph \underline{~~~~~~~}
		\item $J(x) = 2(1.5)^x$ is graph \underline{~~~~~~~}
	\end{itemize}
\end{minipage}
\end{myPractice}
\vfill





%======================================================
\newpage
%======================================================

\subsection*{Definitions} \label{def-exponential-intro}

\begin{myDefinition}[Exponential Function:]~\\[0.5mm]
For any real number $x$, an {\bf exponential function} is a function of the form $\boldsymbol{f(x) = a\cdot b^x}$ where
	\begin{itemize}
	\setlength{\itemsep}{0in}
		\item $a$ is a non-zero real number
		\item $b$ is any positive real number, where $b\neq 1$
	\end{itemize}
Exponential functions grow or decay with a constant {\it percent} rate of change.  
\end{myDefinition}

\begin{myDefinition}[Key Characteristics of Exponential Functions:]~\\[0.5mm]
For an exponential function $f(x) = a\cdot b^x$, with $a> 0$ and $b>0,b\neq1$, we have the following:\\[3mm]
\begin{minipage}{0.6\linewidth}
	\begin{itemize}
	\setlength{\itemsep}{1mm}
		\item $(0,a)$ is the vertical intercept.
		\item There is no horizontal intercept.
		\item The domain of $f$ is $(-\infty, \infty)$.
		\item The range of $f$ is $(0,\infty)$.
		\item $f$ is a one-to-one function.
		\item The horizontal asymptote is $y=0$.
		\item If $b>1$, then $f$ is an increasing function and
			\begin{itemize}
			\setlength{\itemsep}{0in}
				\item[\small{•}] as $x\rightarrow \infty$, $f(x)\rightarrow \infty$, and
				\item[\small{•}] as $x\rightarrow -\infty$, $f(x)\rightarrow 0$.
			\end{itemize}	
		\item If $0<b<1$, then $f$ is a decreasing function and
			\begin{itemize}
			\setlength{\itemsep}{0in}
				\item[\small{•}] as $x\rightarrow \infty$, $f(x)\rightarrow 0$, and
				\item[\small{•}] as $x\rightarrow -\infty$, $f(x)\rightarrow \infty$.
			\end{itemize}	
	\end{itemize}
	\end{minipage}
	\begin{minipage}{0.4\linewidth}
			\begin{center}
			\captionof{figure}{$y=a\cdot b^x, b>1$}~\\[-0.8em]
			\begin{tikzpicture}
 				\begin{axis}[
 					framedaxes,
		 			height=5cm,
 					width=5cm,
 					xlabel={$x$},
 					ylabel={$y$},
		 			xmin=-6,xmax=6,
 					ymin=-6,ymax=6,
				        xtick={-16,16},
				       	minor xtick={-16,16},
				        ytick={-16,16},
	        		 	minor ytick={-16,16},
				         grid=both
 					]
		 		% use TeX as calculator:
				 	\addplot[first,line width=1.5pt,samples=200,<->]expression[domain=-6:2.584] {2^x};
					\addplot[first,line width=1.0pt,mark=*] coordinates {(0,1)};
						\node at (axis cs:-1.5,1.5) {{\tiny $(0,a)$}};									
		 		\end{axis}
 				\end{tikzpicture}
				\label{fig:expintro-def1}
%				\end{center}

%			\begin{center}
			\captionof{figure}{$y=a\cdot b^x,0<b<1$}
			\begin{tikzpicture}
 				\begin{axis}[
 					framedaxes,
		 			height=5cm,
 					width=5cm,
 					xlabel={$x$},
 					ylabel={$y$},
		 			xmin=-6,xmax=6,
 					ymin=-6,ymax=6,
				        xtick={-16,16},
				       	minor xtick={-16,16},
				        ytick={-16,16},
	        		 	minor ytick={-16,16},
				         grid=both
 					]
		 		% use TeX as calculator:
				 	\addplot[first,line width=1.5pt,samples=200,<->]expression[domain=-2.584:6] {0.5^x};
					\addplot[first,line width=1.0pt,mark=*] coordinates {(0,1)};
						\node at (axis cs:1.5,1.5) {{\tiny $(0,a)$}};									
		 		\end{axis}
 				\end{tikzpicture}
				\label{fig:expintro-def2}
				\end{center}

	\end{minipage}
~\\[0.5em]	
\begin{minipage}{0.9\linewidth}
\defexample \href{https://tiny.cc/111Z-ExpFunction}{View this Desmos graph} to see an interactive example of a exponential function.  %(url: tiny.cc/111Z-
\end{minipage}
\begin{minipage}{0.1\linewidth}
\flushright \qrcode[height=1cm]{https://tiny.cc/111Z-ExpFunction}
\end{minipage}
ExpFunction)
\end{myDefinition}

\begin{myDefinition}[The Number $e$:]~\\[0.5mm]
\begin{minipage}{0.65\linewidth}
The number $\boldsymbol{e}$ was discovered in the late 1600's by Jacob Bernoulli. Later in the 1700's, Leonard Euler discovered many of its interesting properties.\\

$\boldsymbol{e}$ can be approximated by 2.718281828459, though its decimal form does not end and does not repeat.  It is an irrational number.\\

The graph of the function given by $f(x) = e^x$ looks a lot like the graphs of the functions given by $g(x) = 2^x$ and $h(x) = 3^x$, as shown in Figure~\ref{fig:edef1}. 
 \vspace{8mm}

\end{minipage}
\begin{minipage}{0.35\linewidth}
	   \begin{center}
 		  \captionof{figure}{}
   		\label{fig:edef1}
					\begin{tikzpicture}
					\begin{axis}[
					framedaxes,
					legend pos=north west,
					legend cell align=left,
					height=6.5cm,
					width=7cm,
					xlabel={$x$},
					ylabel={$y$},
					xmin=-4,xmax=4,
					ymin=-1,ymax=7,
					grid=both,
					xtick={-3,-2,...,3},
					ytick={0,1,...,6},
					]
					% use TeX as calculator:
%								\addplot[myred,dashed,line width=0.75pt,<->]expression[domain=-4:4,samples=200]{-.05} ;
					\addplot[third!35,dashed,<->,line width=1.0pt]expression[domain=-4:2.807354,samples=200]{2^x};
					\addplot[first,<->,line width=1.5pt]expression[domain=-4:1.945910,samples=200]{2.718281828^x};
					\addplot[second,dashed,<->,line width=1.0pt]expression[domain=-4:1.771243,samples=200]{3^x};
								%\node at (axis cs:-3.,-0.25) {\tiny{$y\,$=\,0}};								    
					\legend{$y=2^x$, $y=e^x$, $y=3^x$}
					\end{axis}
					\end{tikzpicture}
			\end{center}
\end{minipage}
\end{myDefinition}




%======================================================
 \newpage
%======================================================

\subsection*{Exit Exercises} \label{exit-exponential-intro}




\begin{myExit}
Given the formula for an exponential function, you should be able to look at the formula and identify if the function will represent exponential growth or exponential decay.  How can you do this?  \\[0.5em]
Give an example of a symbolic function for each, one exponential growth and one exponential decay, as part of your explanation.
\vfill
\vfill
\vfill
\end{myExit}


\begin{myExit}
At the start of an experiment, a population of bacteria has 5 million bacteria and the population is decreasing by 13\% every 6 hours.  Find a formula for the function $B$ that gives the number of bacteria (in millions) remaining after $n$ 6-hour time intervals. \\[0.4in]

Evaluate $B(8)$ and explain its meaning in context.
\vfill
\vfill
\vfill
\end{myExit}


\begin{myExit}
%Given $f(x) = 5^x$ and $g(x)=-7\cdot5^{x-2}+1$, state the sequence of transformations that takes the graph of $y=f(x)$ to the graph of $y=g(x)$ and also state the range of $g$.
%REPLACED WITH:
Find an exponential function $f$ that satisfies $f(2)=\dfrac{3}{8}$ and $f(-1)=\dfrac{8}{9}$.

\vfill
\vfill
\vfill
\vfill
\vfill
\end{myExit}






\exitlikert{exponential functions}













