

%======================================================
\newpage
%======================================================


\subsection*{Practice Exercises} \label{practice-functions-inverse}

\begin{myPractice}
$f$ is defined in Table~\ref{tab:invprac1}.
	\begin{center}
			\renewcommand{\arraystretch}{1.75}
		\captionof{table}{}
		\label{tab:invprac1}
	%	\begin{tabular}{| m{1cm} | c | c | c | c | c |}
\begin{tabular}{|>{\centering\arraybackslash}m{1cm}| >{\centering\arraybackslash}m{1cm} >{\centering\arraybackslash}m{1cm} >{\centering\arraybackslash}m{1cm} >{\centering\arraybackslash}m{1cm} >{\centering\arraybackslash}m{1cm}|>{\centering\arraybackslash}m{1cm}|}		\hline
				$x$ 		&	-4	& 	-2	&	 0	&	2	&	 4	\ \\ \hline
				$f(x)$	&	4	&	2	&	-2	&	-4	&	0			\\ \hline
		\end{tabular} 
		\end{center}
\vspace{11pt}

Evaluate the following.
\begin{multicols}{3}
		\begin{enumerate}\setlength{\itemsep}{0.75in}
			\item	$f(0)$
		
			\item $f^{-1}(0)$
		
			\item	$f^{-1}(-4)$ 
		
		\end{enumerate}
		
\end{multicols}
\end{myPractice}
~\\

\begin{myPractice}
$y=g(x)$ is defined in Figure~\ref{fig:invprac1}.\\
\begin{minipage}{0.4\linewidth}
	\begin{center}
	\captionof{figure}{$y=f(x)$}~\\[-0.8em]
	\begin{tikzpicture}
 		\begin{axis}[
 			framedaxes,
 			height=7cm,
 			width=7cm,
 			xlabel={$x$},
 			ylabel={$y$},
 			xmin=-8,xmax=8,
 			ymin=-8,ymax=8,
		        xtick={-6,-4,...,6},
		       	minor xtick={-11,-9,-7,...,11},
		        ytick={-6,-4,...,6},
	         	minor ytick={-7,-5,...,7},
		         grid=both
 			]
 		% use TeX as calculator:
%	\addplot[first,line width=1.25pt,samples=200,<-]expression[domain=-8:-3]{1/2*(x+4)^2+5};
	\addplot[first,line width=1.25pt,samples=200,<-]expression[domain=-2:-1]{-1*(x+3)^2+9};
	\addplot[first,line width=1.25pt,samples=200,-]expression[domain=-1:1]{1*(x-1)^2+1};
	\addplot[first,line width=1.25pt,samples=200,-]expression[domain=1:3]{-1/2*(x-1)^2+1};
	\addplot[first,line width=1.25pt,samples=200,-]expression[domain=3:5]{1/2*(x-5)^2-3};
	\addplot[first,line width=1.25pt,samples=200,->]expression[domain=5:8]{-1/2*(x-5)^2-3};
 		\end{axis}
 		\end{tikzpicture}
		\label{fig:invprac1}
	\end{center}
\end{minipage}
\begin{minipage}{0.6\linewidth}
Evaluate the following.
	\begin{enumerate}\setlength{\itemsep}{0.5in}
		\item	$g(5)$
		\item $g^{-1}(5)$	
		\item	$g^{-1}(-1)$ 
		\end{enumerate}

\end{minipage}


\end{myPractice}

\begin{myPractice}
Graph the function $f(x) = 4+\sqrt[3]{x-1}$ in a graphing utility (such as Desmos) to confirm that it is a one-to-one function and then find the formula for $f^{-1}$.
\vfill
\vfill
\vfill
\end{myPractice}


\newpage



\begin{myPractice}
Graph the function $g(x) = 2-\sqrt{x+3}$ in a graphing utility.
\begin{enumerate}
	\item Is $g$ a one-to-one function? Why or why not? 
	\vfill
	\vfill

	\item State the domain and range of $g$.
	\vfill

	\item Algebraically find the formula for $g^{-1}$.
	\vfill
	\vfill
	\vfill
	
	\item State the domain and range of $g^{-1}$.
	\vfill

\end{enumerate}
\end{myPractice}



%======================================================
\newpage
%======================================================

\subsection*{Definitions} \label{def-functions-inverse}

\begin{myDefinition}[One-to-One Function:]~\\[0.5mm]
A function $f$ is said to be {\bf one-to-one} if for every possible output ($y$-value) in the range of $f$, there is exactly one input ($x$-value) in the domain of $f$. \\

In other words, in a one-to-one function, each possible input is paired with exactly one output {\bf AND} each possible output is paired with exactly one input.\\[0.5em]
\defexample The set $\{ (0,-2), ~ (1, \boldsymbol{-1}), ~ (4, 0), ~ (9,\boldsymbol{-1}),~ (16,-3) \}$ is a function, but it is \textit{not} one-to-one.\\
\defexample The set $\{ (0,-2), ~ (1, -1), ~ (2, 0), ~ (3, 1),~ (4,2) \}$ is a function and \textit{is} one-to-one.
\end{myDefinition}

\begin{myDefinition}[Horizontal Line Test:]~\\[0.5mm]
If a horizontal line can be drawn that intersects the graph of a function more than once, the graph is not the graph of a one-to-one function.\\[0.5em]
		\begin{minipage}{0.5\linewidth}
			\begin{center}
				\defexample
					\captionof{figure}{Does Not Pass}
					\begin{tikzpicture}
			 		\begin{axis}[
		 				framedaxes,
		 				width=5.5cm,height=5.5cm,
		 				xlabel={$x$},
		 				ylabel={$y$},
		 				xmin=-8,xmax=8,
		 				ymin=-8,ymax=8,
						xtick={-6,-4,...,6},
							minor xtick={-7,-5,...,7},
						ytick={-6,-4,...,6},
							minor ytick={-7,-5,...,7},
					         grid=both
		 				]
 				% use TeX as calculator:
%	 					\addplot[smooth,mark=*,first,line width=1.0pt,fill=white]coordinates{	(-4,2)	};
%	 					\addplot[smooth,mark=*,first,line width=1.0pt]coordinates{	(5,-4)	};
		 				\addplot[first,line width=1.5pt,samples=400,<->]expression[domain=-5.899:3.899]{0.5*(x+1)^2-4};
%						\addplot[first,line width=1.5pt,-,forget plot] coordinates {(-2,0.98) (-2,1.02)};
						\addplot[fifth,line width=0.75pt,<->] coordinates {(-8,-2) (8,-2)};
								\node at (axis cs:5,-2.5) {\color{fifth}\tiny{$y$\,=\,$-2$}};
						\addplot[fifth,smooth,mark=*,line width=0.25pt]coordinates{	(-3,-2)	};
						\addplot[fifth,smooth,mark=*,line width=0.25pt]coordinates{	(1,-2)	};
			 		\end{axis}
		 		\end{tikzpicture}
				\label{fig:inv-def1}
			\end{center}
		\end{minipage}
		\begin{minipage}{0.5\linewidth}
			\begin{center}
				\defexample
					\captionof{figure}{Passes}
					\begin{tikzpicture}
			 		\begin{axis}[
		 				framedaxes,
		 				width=5.5cm,height=5.5cm,
		 				xlabel={$x$},
		 				ylabel={$y$},
		 				xmin=-8,xmax=8,
		 				ymin=-8,ymax=8,
						xtick={-6,-4,...,6},
							minor xtick={-7,-5,...,7},
						ytick={-6,-4,...,6},
							minor ytick={-7,-5,...,7},
					         grid=both
		 				]
 				% use TeX as calculator:
%	 					\addplot[smooth,mark=*,first,line width=1.0pt,fill=white]coordinates{	(-4,2)	};
%	 					\addplot[smooth,mark=*,first,line width=1.0pt]coordinates{	(5,-4)	};
		 				\addplot[first,line width=1.5pt,samples=400,->]expression[domain=-2:8]{(x+2)^0.5+1};
		 				\addplot[first,line width=1.5pt,samples=400,<-]expression[domain=-8:-2]{-1*(-x-2)^0.5+1};
						\addplot[first,line width=1.5pt,-,forget plot] coordinates {(-2,0.98) (-2,1.02)};
%						\addplot[myred,line width=1.0pt,dashed,<->] coordinates {(3,8) (3,-8)};
%								\node at (axis cs:3.25,-7.5) {\color{myred}\tiny{$x\,$=\,3}};
			 		\end{axis}
		 		\end{tikzpicture}
				\label{fig:inv-def2}
			\end{center}
		\end{minipage}
\end{myDefinition}

\begin{myDefinition}[Inverse Function:]~\\[0.5mm]
If a function $f$ is one-to-one, then the function has an {\bf inverse}, $f^{-1}$.\\

Two functions, $f$ and $f^{-1}$ are {\bf inverse functions} if and only if both of the following are true:
\begin{itemize}
\item $f\left(f^{-1}(x)\right) = x$ for all $x$ in the domain of $f^{-1}$.
\item $f^{-1}\left(f(x)\right) = x$ for all $x$ in the domain of $f$.
\end{itemize}
The domain of a function $f$ is the range of the inverse function $f^{-1}$.\\
The range of a function $f$ is the domain of the inverse function $f^{-1}$.\\[0.5em]
\defexample \\
If $f$ is defined by the set $\{ (0,-2), ~ (1, -1), ~ (2, 0), ~ (3, 1),~ (4,2) \}$, \\
then $f^{-1}$ is the set $\{ (-2,0), ~ (-1, 1), ~ (0, 2), ~ (1, 3),~ (2,4) \}$.\\[0.3em]
The domain of $f$ is $\{ 0, ~ 1, ~ 2, ~ 3, ~ 4 \}$ and the range is $\{ -2, \, -1, ~ 0, ~ 1,~ 2 \}$.\\[0.3em]
The domain of $f^{-1}$ is $\{ -2, \, -1, ~ 0, ~ 1,~ 2 \}$ and the range is $\{ 0, ~ 1, ~ 2, ~ 3, ~ 4 \}$.\\

\end{myDefinition}





%======================================================
 \newpage
%======================================================

\subsection*{Exit Exercises} \label{exit-functions-inverse}



\begin{myExit}
	\begin{enumerate}
		\item Explain what is meant by the phrase ``one-to-one'' and how you can tell from the graph of the function if the function is one-to-one?
		\vfill
		\vfill
		\item What is the relationship of the domain and range of a function $f$ and its inverse function $f^{-1}$?
		\vfill
		\item What happens when you compose two functions that are inverses of each other?
		\vfill
		\item Why is $g(x)=(4x-1)^3$ invertible and $h(x)=(4x-1)^2$ is not?
		\vfill
		\item Given $m(x) = 19+(-3x+1)^5$, find $m^{-1}$.
		\vfill
		\vfill
	\end{enumerate}
\end{myExit}
\vfill




\exitlikert{inverse functions}




