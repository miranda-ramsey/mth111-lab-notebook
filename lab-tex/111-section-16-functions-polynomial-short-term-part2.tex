


%======================================================
\newpage
%======================================================


\subsection*{Practice Exercises} \label{practice-functions-polynomial-graphs}

\begin{myPractice}
	Let $f(x) =\frac{1}{2}(x-2)(x+1)^2(x-4)$. \\
\begin{minipage}{0.7\linewidth}
	\begin{enumerate}
		\item What is the end behavior of the graph of $y=f(x)$?  Why?\\[0.25in]
		\item What are the $x$-intercepts and their behaviors? Why?\\[0.5in]
		\item What is the $y$-intercept?\\[0.25in]
		\item Sketch a graph of $y=f(x)$ in Figure~\ref{fig:poly-pr1} to the right.
	\end{enumerate}
\end{minipage}
\begin{minipage}{0.3\linewidth}
	\begin{center}
	\captionof{figure}{$y=f(x)$}~\\[-0.8em]
	\begin{tikzpicture}
 		\begin{axis}[
				framed,
				width=7cm, 
				height=7cm,				
				xlabel={},
				ylabel={},
				xmin=-8,xmax=8,
				ymin=-8,ymax=8,
				xtick={-16,16},
        	minor xtick={-17,-16,...,17},
				ytick={-16,16},
        	minor ytick={-17,-16,...,17},
        grid=both
 			]
 		% use TeX as calculator:
 		\end{axis}
 		\end{tikzpicture}
		\label{fig:poly-pr1}
	\end{center}
\end{minipage}


\vfill
\end{myPractice}



\begin{myPractice}
	Given in Figure~\ref{fig:poly-pr2} is the graph of a polynomial function $g$. \\
\begin{minipage}{0.35\linewidth}
	\begin{center}
	\captionof{figure}{$y=g(x)$}~\\[-0.8em]
	\begin{tikzpicture}
 		\begin{axis}[
 			framedaxes,
 			height=7cm,
 			width=7cm,
 			xlabel={$x$},
 			ylabel={$y$},
 			xmin=-8,xmax=8,
 			ymin=-8,ymax=8,
		        xtick={-6,-4,...,6},
		       	minor xtick={-11,-9,-7,...,11},
		        ytick={-6,-4,...,6},
	         	minor ytick={-7,-5,...,7},
		         grid=both
 			]
 		% use TeX as calculator:
			\addplot[first,line width=1.25pt,<->,samples=400]expression[domain=-4.153:4.44]{1/36*(x+3)^2*(x-1)^2*(x-4)};

 		\end{axis}
 		\end{tikzpicture}
		\label{fig:poly-pr2}
	\end{center}
\end{minipage}
\begin{minipage}{0.65\linewidth}
	\begin{enumerate}
		\item Based on the end behavior and other characteristics of the graph of $y=g(x)$, what are the possibilities for the degree of $g$?\\[0.25in]
		\item Based on the end behavior of the graph of $y=g(x)$, is the leading coefficient of $g$ positive or negative?\\[0.25in]
		\item Based on the $x$-intercepts, what are the linear factors of $g$ and their powers?\\[0.5in]
		\item What is a possible formula for the function $g$?\\[0.25in]
	\end{enumerate}
\end{minipage}


\vfill
\end{myPractice}



%======================================================
\newpage
%======================================================

\subsection*{Definitions} \label{def-functions-polynomial-graphs}

\begin{myDefinition}[Multiplicity]~\\[0.5mm]
The {\bf multiplicity} of a factor is the number of times that factor occurs in the factored version of the polynomial.  We will also refer to the {\bf multiplicity of the zero} for the zero associated with this factor.\\[0.5em]
\defexample For $f(x) = 3(x-5)(x+4)^2$, the multiplicity of $5$ is one and the multiplicity of $-4$ is two.\\[0.5em]
The sum of the multiplicities of the real roots for a polynomial function is less than or equal to the degree of the polynomial.
\end{myDefinition}

\begin{myDefinition}[Even Multiplicity]~\\[0.5mm]
When a root or zero has even multiplicity, then the graphical behavior at the associated $x$-intercept is that the graph will touch, but not cross, the $x$-axis at that $x$-intercept.
\end{myDefinition}

\begin{myDefinition}[Odd Multiplicity]~\\[0.5mm]
When a root or zero has odd multiplicity, then the graphical behavior at the associated $x$-intercept is that the graph will cross over the $x$-axis at that $x$-intercept.\\

When the multiplicity of a root is one, the graph will cross through the $x$-intercept in a somewhat linear manner.\\

When the multiplicity of a root is a larger odd number, the graph will cross through the $x$-intercept by flattening out as it crosses.\\

\begin{minipage}{0.9\linewidth}
\defexample \href{https://tiny.cc/111Z-Multiplicity}{View this Desmos graph} to see an interactive example how the multiplicity of a root of a polynomial function impacts the behavior of the related $x$-intercept.  %(url: tiny.cc/111Z-Multiplicity)
\end{minipage}
\begin{minipage}{0.1\linewidth}
\flushright \qrcode[height=1cm]{https://tiny.cc/111Z-Multiplicity}
\end{minipage}
\end{myDefinition}



%======================================================
 \newpage
%======================================================

\subsection*{Exit Exercises} \label{exit-functions-polynomial-graphs}

\begin{myExit}
	\begin{enumerate}
		\item What is the difference between a zero and an $x$-intercept of a polynomial function?
		\vfill
		\item How is the factored version of a polynomial function useful when graphing the function?  What does the factored version help us to be able to quickly identify?
		\vfill
		\item How is the expanded (non-factored) version of a polynomial function useful when graphing the function?  What does the non-factored version help us to be able to quickly identify?
		\vfill
		\item If the graph of a polynomial function touches, but doesn't cross, the $x$-axis at a point $(k,0)$, what do we know about the factored form of that function?
		\vfill
	\end{enumerate}
\end{myExit}

\begin{myExit}
Identify a possible formula for the polynomial function $F$ whose graph is in Figure~\ref{fig:poly-ex1}.\\
There is a point on the graph at $(-3,21)$.\\
\begin{minipage}{0.6\linewidth}
~
\end{minipage}
\begin{minipage}{0.4\linewidth}
	\begin{center}
	\captionof{figure}{$y=F(x)$}~\\[-0.8em]
	\begin{tikzpicture}
 		\begin{axis}[
 			framedaxes,
 			height=7cm,
 			width=7cm,
 			xlabel={$x$},
 			ylabel={$y$},
 			xmin=-8,xmax=8,
 			ymin=-32,ymax=32,
		        xtick={-6,-4,...,6},
		       	minor xtick={-11,-9,-7,...,11},
		        ytick={-28,-24,...,28},
%	         	minor ytick={-30,-28,...,30},
		         grid=both
 			]
 		% use TeX as calculator:
			\addplot[first,line width=1.25pt,<->,samples=400]expression[domain=-5.494:5.88]{1/42*x^2*(x+5)*(x-4)^2};
			\addplot[mark=*,first] plot coordinates {(-3,21)	};
 		\end{axis}
 		\end{tikzpicture}
		\label{fig:poly-ex1}
	\end{center}
\end{minipage}

\end{myExit}


\vfill

\exitlikert{polynomial functions}



























