%============================================================
% MTH 111Z Project - Template File
% Part of Section 6.1 from OpenStax OER
%	Updated 202302
%============================================================

\section{Exponential and Logarithmic Models} \label{exponential-and-logarithmic-models}

In this section, we'll investigate a few ways that exponential and logarithmic functions are used to model the world around us, as well as how logarithms help us to quantify sound levels, acidity, and earthquakes.\\[0.5em]
\textbook{6.7}


\subsection*{Preparation Exercises} \label{prep-exponential-and-logarithmic-models}


\begin{myPrep}
Find the formula for an exponential function $f$ that satisfies $f(-2) = \dfrac{16}{27}$ and $f(3)=\dfrac{9}{2}$.	
\vfill
\end{myPrep}

\begin{myPrep}
The substance Einsteinium-253 decays at a continuous rate of about 3.3862\% per day.  If a scientist starts with a 60 mg sample of Einsteinium-253, how long until there are only 30 mg remaining?
\vfill
\end{myPrep}

\begin{myPrep}
The function $V(t) = 32.8(1.093)^t$ represents the value (in thousands of dollars) of a collectable car $t$ years after June 1st, 2015.  How long will it take for the value of the car to reach \$65,600?
\vfill
\end{myPrep}



