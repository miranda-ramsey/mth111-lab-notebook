
%======================================================
\newpage
%======================================================


\subsection*{Practice Exercises} \label{practice-functions-piecewise}

\begin{myPractice}
~\\[-5mm]
	\begin{minipage}{0.5\linewidth}
		Let $f(x)=
			\begin{cases}
				x^2-4	&	\textrm{if}\ 	\ \	-2\leq x<3		\\
				\frac{2}{3}x-1	&	\textrm{if}\ 	\ \	x\geq 3		\\
			\end{cases}
		$
		\begin{enumerate}
			\item Evaluate $f(5)$.\\[12mm]
			\item What is the domain of $f$?\\[12mm]
			\item Graph $y=f(x)$ in Figure~\ref{fig:pw-ex1}.
		\end{enumerate}
	\end{minipage}
	\begin{minipage}{0.5\linewidth}
	\begin{center}
	\captionof{figure}{$y=f(x)$}~\\[-0.8em]
	\begin{tikzpicture}
 		\begin{axis}[
 			framed,
 			height=7cm,
 			width=7cm,
 			xlabel={},
 			ylabel={},
 			xmin=-8,xmax=8,
 			ymin=-8,ymax=8,
		        xtick={-16,16},
		       	minor xtick={-17,...,17},
		        ytick={-16,16},
	         	minor ytick={-17,...,17},
		         grid=both
 			]
 		% use TeX as calculator:
 		\end{axis}
 		\end{tikzpicture}
		\label{fig:pw-ex1}
	\end{center}

	\end{minipage}
\end{myPractice}
~\\[10mm]

\begin{myPractice}
The graph of $y=g(x)$ is in Figure~\ref{fig:pw-ex2}.\\

	\begin{minipage}{0.4\linewidth}
	\begin{center}
	\captionof{figure}{$y=g(x)$}~\\[-0.8em]
	\begin{tikzpicture}
 		\begin{axis}[
 			framedaxes,
 			height=7cm,
 			width=7cm,
 			xlabel={$x$},
 			ylabel={$y$},
 			xmin=-8,xmax=8,
 			ymin=-8,ymax=8,
		        xtick={-6,-4,...,6},
		       	minor xtick={-11,-9,-7,...,11},
		        ytick={-6,-4,...,6},
	         	minor ytick={-7,-5,...,7},
		         grid=both
 			]
 		% use TeX as calculator:
				\addplot[first,line width=1.5pt,samples=400,<-]expression[domain=-8:-3]{2/3*(x+3)+1};
					\addplot[first,smooth,mark=*,line width=1pt,fill=first]coordinates{	(-3,1)	};
					\addplot[first,smooth,mark=*,line width=1pt,fill=white]coordinates{	(-3,-4)	};
				\addplot[first,line width=1.5pt,samples=400,-]expression[domain=-3:1]{-4};
					\addplot[first,smooth,mark=*,line width=1pt,fill=white]coordinates{	(1,-4)	};
					\addplot[first,smooth,mark=*,line width=1pt,fill=first]coordinates{	(2,3)	};
				\addplot[first,line width=1.5pt,samples=400,->]expression[domain=2:7.5]{-2*(x-2)+3};
 		\end{axis}
 		\end{tikzpicture}
		\label{fig:pw-ex2}
	\end{center}
	\end{minipage}
	\begin{minipage}{0.6\linewidth}
	\begin{enumerate}
		\begin{multicols}{2}
			\item Evaluate $g(-2)$.
			\item Solve $g(x)=-1$.
		\end{multicols}
		~\\
		\begin{multicols}{2}
		\item What is the domain of $g$? 
		\item What is the range of $g$? 
		\end{multicols}
		~\\
		\item State the formula for the function $g$.\\[10mm]
	\end{enumerate}
	\end{minipage}
\end{myPractice}
\vfill



%======================================================
\newpage
%======================================================

\subsection*{Definitions} \label{def-functions-piecewise}

\begin{myDefinition}[Piecewise-Defined Function:]~\\[0.5mm]
A {\bf piecewise-defined function} is a function which uses different formulas for calculating the output on different intervals of its domain.  Each formula is used on a distinct interval of the domain.  \\

The notation we use to write a piecewise-defined function is:
\[
	f(x)=
	\begin{cases}
		\text{formula \#1}	&	\textrm{if}\ 	\ \	\text{$x$ is in this part of the domain of $f$}		\\
		\text{formula \#2}	&	\textrm{if}\ 	\ \	\text{$x$ is in this other part of the domain of $f$}		\\
		\text{etc.}			&	\textrm{if}\ 	\ \	\text{etc.}		\\
	\end{cases}
\]

The domain of the function is the union of the intervals used by the separate formulas.\\[0.5em]
\defexample \\
\begin{minipage}{0.5\linewidth}
The function $$f(x)=
			\begin{cases}
				-\frac{1}{2}(x+3)^2+4	&	\textrm{if}\ 	\ \	-5< x\leq1		\\
				\sqrt{x-1}+3	&	\textrm{if}\ 	\ \	x> 1		\\
			\end{cases}
		$$
is a piece-wise defined function. \\[0.75em]
Its graph is in Figure~\ref{fig:pw-def1} to the right. \\[0.75em]
Its domain is $(-5, \infty)$ and its range is $[-4,\infty)$.		
\end{minipage}
\begin{minipage}{0.5\linewidth}
	\begin{center}
			\captionof{figure}{$y=f(x)$}~\\[-0.8em]
			\begin{tikzpicture}
 				\begin{axis}[
 					framedaxes,
 					height=5.5cm,
 					width=5.5cm,
		 			xlabel={$x$},
 					ylabel={$y$},
 					xmin=-8,xmax=8,
 					ymin=-8,ymax=8,
				        xtick={-6,-4,...,6},
				       	minor xtick={-11,-9,-7,...,11},
				        ytick={-6,-4,...,6},
	         			minor ytick={-7,-5,...,7},
		      		   grid=both
 					]
		 		% use TeX as calculator:
				\addplot[first,line width=1.5pt,samples=400,-]expression[domain=-5:1]{-0.5*(x+3)^2+4};
					\addplot[first,smooth,mark=*,line width=1pt,fill=white]coordinates{	(-5,2)	};
					\addplot[first,smooth,mark=*,line width=1pt,fill=first]coordinates{	(1,-4)	};
				\addplot[first,line width=1.5pt,samples=400,->]expression[domain=-1:8]{(x-1)^0.5+3};
					\addplot[first,smooth,mark=*,line width=1pt,fill=white]coordinates{	(1,3)	};
		 		\end{axis}
		 		\end{tikzpicture}
				\label{fig:pw-def1}
			\end{center}
\end{minipage}
\end{myDefinition}

%Note: Dennis prefers ``expression'' instead of ``formula''.








%======================================================
 \newpage
%======================================================

\subsection*{Exit Exercises } \label{exit-functions-piecewise}

%%%%%%%%
\iffalse 
\begin{myExit}
Use $F(x)=
			\begin{cases}
				-\frac{1}{2}x^2+1	&	\textrm{if}\ 	\ \	 x\leq2		\\
				-\frac{1}{2}x+5	&	\textrm{if}\ 	\ \	x> 4		\\
			\end{cases}$ to answer the following.
	\begin{enumerate}
	\begin{minipage}{0.5\linewidth}
		\item Graph $y=F(x)$ in Figure~\ref{fig:pw-exit1} below.
		\begin{center}
			\captionof{figure}{$y=F(x)$}
			\begin{tikzpicture}
 				\begin{axis}[
 					framed,
 					height=7cm,
 					width=7cm,
		 			xlabel={$x$},
 					ylabel={$y$},
 					xmin=-8,xmax=8,
 					ymin=-8,ymax=8,
				        xtick={-6,-4,...,6},
				       	minor xtick={-11,-9,-7,...,11},
				        ytick={-6,-4,...,6},
	         			minor ytick={-7,-5,...,7},
		      		   grid=both
 					]
		 		% use TeX as calculator:
		 		\end{axis}
		 		\end{tikzpicture}
				\label{fig:pw-exit1}
			\end{center}
	\end{minipage}
	\begin{minipage}{0.1\linewidth}
	\end{minipage}
	\begin{minipage}{0.4\linewidth}
		\item Evaluate $F(-3)$.\\[15mm]
		\item Solve $F(x) = 1$ from your graph.\\[15mm]
	\end{minipage}		
	
	\end{enumerate}
\end{myExit}

\fi
%%%%%%%%

\begin{myExit}
Use $F(x)=
			\begin{cases}
				-\frac{1}{2}x^2+1	&	\textrm{if}\ 	\ \	 x\leq2		\\
				-\frac{1}{2}x+5	&	\textrm{if}\ 	\ \	x> 4		\\
			\end{cases}$ to answer the following.
	\begin{enumerate}
	\begin{minipage}{0.5\linewidth}
		\item Evaluate $F(8)$.
	\end{minipage}
	\begin{minipage}{0.5\linewidth}
		\item Evaluate $F(-6)$.
	\end{minipage}			
	\end{enumerate}
\vfill
\vfill
\end{myExit}


\begin{myExit}
In November 2022, Portland General Electric set the rates for basic residential service to be \$0.0642/kWh for up to 1000 kWh used and then \$0.07002/kWh for any usage greater than 1000 kWh.  Find a piecewise-defined function that gives the cost of electricity used (in dollars) as a function of x, the amount of kWh used.
\vfill
\vfill
\vfill
\end{myExit}



\begin{myExit}~\\[-5mm]
	\begin{minipage}{0.5\linewidth}
		$y=G(x)$ is graphed in Figure~\ref{fig:pw-exit2} below.
		\begin{center}
			\captionof{figure}{$y=G(x)$}
			\begin{tikzpicture}
 				\begin{axis}[
 					framedaxes,
 					height=7cm,
 					width=7cm,
		 			xlabel={$x$},
 					ylabel={$y$},
 					xmin=-8,xmax=8,
 					ymin=-8,ymax=8,
				        xtick={-6,-4,...,6},
				       	minor xtick={-11,-9,-7,...,11},
				        ytick={-6,-4,...,6},
	         			minor ytick={-7,-5,...,7},
		      		   grid=both
 					]
		 		% use TeX as calculator:
				\addplot[first,line width=1.5pt,samples=400,<-]expression[domain=-7.666:-4]{-3*(x+4)-3};
					\addplot[first,smooth,mark=*,line width=1pt,fill=first]coordinates{	(-4,-3)	};
					\addplot[first,smooth,mark=*,line width=1pt,fill=white]coordinates{	(-4,1)	};
				\addplot[first,line width=1.5pt,samples=400,-]expression[domain=-4:2]{3/4*(x+4)+1};
					\addplot[first,smooth,mark=*,line width=1pt,fill=white]coordinates{	(2,5.5)	};
					\addplot[first,smooth,mark=*,line width=1pt,fill=first]coordinates{	(3,3)	};
				\addplot[first,line width=1.5pt,samples=400,->]expression[domain=3:8]{-1*(x-3)+3};
		 		\end{axis}
		 		\end{tikzpicture}
				\label{fig:pw-exit2}
			\end{center}
			\vspace{1cm}
	\end{minipage}
	\begin{minipage}{0.1\linewidth}
	\end{minipage}
	\begin{minipage}{0.4\linewidth}
		\begin{enumerate}
			\item State the formula for the function $G$.\\[30mm]
			\item Evaluate $G(-4)$ from the graph.\\[12.5mm]
			\item Solve $G(x) = 0$ from the graph.\\[12.5mm]
		\end{enumerate}
	\end{minipage}		
	
\end{myExit}
\vfill




\exitlikert{piecewise-defined functions}





