


%======================================================
\newpage
%======================================================


\subsection*{Practice Exercises} \label{practice-logarithms-properties}
\begin{myPractice}
Expand $\log_6(5x^3y)$ as much as possible by rewriting the expression as a sum, difference, or product of logs or constant factors.
\end{myPractice}
\vfill

\begin{myPractice}
Rewrite $\frac{1}{2}\ln(x+5)-6\ln(x)$ as a single logarithm.
\end{myPractice}
\vfill

\begin{myPractice}
Find the exact value of $\log_{5}(75)-\log_{5}(3)+\log_2(16)$ without using a calculator.
\end{myPractice}
\vfill

\begin{myPractice}
Saskia and José disagree.  Saskia says that $\log_3(x+2)-\log_3(x+1) = \dfrac{\log_3(x+2)}{\log_3(x+1)}$, but José says that's wrong.  \\
Who is correct and why?
\end{myPractice}
\vfill






%======================================================
\newpage
%======================================================

\subsection*{Properties} \label{def-logarithms-properties}

\begin{myDefinition}[Four Properties of Logarithms]~\\[0.5mm]
Given any real number $x$ and any positive real number $b$, with $b\neq 1$,\\[0.5em]
	\begin{minipage}{0.4\linewidth}
			$$\begin{aligned} 
			\log_b(1)&=0\\[0.5em]
			\log_b(b)&=1
			\end{aligned}$$
	\end{minipage}
	\begin{minipage}{0.4\linewidth}
			$$\begin{aligned} 
			\log_b(b^x)&=x\\[0.6em]
			b^{\log_b(x)}&=x
			\end{aligned}$$

%			$$\log_b(b^x)=x$$
%			$$b^{\log_b(x)}=x, \ x>0$$
	\end{minipage}
\end{myDefinition}


\begin{myDefinition}[Product Rule for Logarithms]~\\[0.5mm]
Given any positive real numbers $M$, $N$, and $b$, with $b\neq 1$,
\[\log_b\left(M\cdot N\right) = \log_b(M)+\log_b(N)\]
\end{myDefinition}

\begin{myDefinition}[Quotient Rule for Logarithms]~\\[0.5mm]
Given any positive real numbers $M$, $N$, and $b$, with $b\neq 1$,
\[\log_b\left(\dfrac{M}{N}\right) = \log_b(M)-\log_b(N)\]
\end{myDefinition}

\begin{myDefinition}[Power Rule for Logarithms]~\\[0.5mm]
Given any real number $n$, positive real numbers $M$ and $b$, with $b\neq 1$,
\[\log_b\left(M^n\right) = n\cdot \log_b(M)\]
\end{myDefinition}

\begin{myDefinition}[Change of Base Formula]~\\[0.5mm]
Given positive real numbers $M$, $n$, and $b$, with $b\neq 1$ and $n\neq1$,
\[\log_b(M) = \dfrac{ \log_n(M)}{\log_n(b)}\]
\end{myDefinition}





%======================================================
 \newpage
%======================================================

\subsection*{Exit Exercises} \label{exit-logarithms-properties}



\begin{myExit}
	\begin{enumerate}
		\item Only one of these is true.  Which one and why? 
		\begin{enumerate}[label=\roman*)]
			\begin{minipage}{0.55\linewidth}
				 \item $ \log_2(8) +\log_2(16) = \log_2(8\cdot16) $
			\end{minipage}
			\begin{minipage}{0.45\linewidth}
			 	\item $ \log_2(8) \cdot \log_2(16)  =\log_2(8+16) $
			\end{minipage}
			\end{enumerate}
		\vfill
		
		\item Only one of these is true.  Which one and why? 
		\begin{enumerate}[label=\roman*)]
			\begin{minipage}{0.55\linewidth}
			 	\item $ \dfrac{\log_3(27)}{ \log_3(9)}  =\log_3(27-9) $
			\end{minipage}
			\begin{minipage}{0.45\linewidth}
				\item $ \log_3(27) -\log_3(9) = \log_3\left(\frac{27}{9}\right) $
			\end{minipage}
			\end{enumerate}
		\vfill
		
		\item Rewrite each of these as a single logarithm.
		\begin{enumerate}[label=\roman*)]
			\begin{minipage}{0.55\linewidth}
				 \item $ \log(10) - \log(2) + \log(3)$
			\end{minipage}
			\begin{minipage}{0.45\linewidth}
				 \item $ \ln(a) - \ln(b) - \ln(c)$
			\end{minipage}
			\end{enumerate}
		\vfill
		
		\item Fully expand each of the following.
		\begin{enumerate}[label=\roman*)]
			\begin{minipage}{0.55\linewidth}
				 \item $ \log_2\left( \dfrac{x^2}{y^3z} \right)$
			\end{minipage}
			\begin{minipage}{0.45\linewidth}
				 \item $ \ln(e^5n^3\sqrt[4]{k})$
			\end{minipage}
			\end{enumerate}
		\vfill
	\end{enumerate}
\end{myExit}


\exitlikert{properties of logarithms}
























